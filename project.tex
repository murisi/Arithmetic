\documentclass[twocolumn]{article}
\usepackage{fullpage}
\usepackage{parskip}
\usepackage{amsmath}
\usepackage{amssymb}
\usepackage{datetime}
\usepackage[nottoc,numbib]{tocbibind}
\usepackage{hyperref}
\hypersetup{
	colorlinks = true,
	linktoc = all,
}
\setcounter{tocdepth}{2}
\DeclareMathOperator{\sgn}{sgn}
\DeclareMathOperator{\mat}{mat}
\DeclareMathOperator{\pol}{pol}
\DeclareMathOperator{\tr}{tr}
\DeclareMathOperator{\disc}{disc}
\DeclareMathOperator{\bdiag}{bdiag}
\DeclareMathOperator{\rcan}{rcan}
\DeclareMathOperator{\mon}{mon}
\DeclareMathOperator{\pows}{pows}
\DeclareMathOperator{\cols}{cols}
\DeclareMathOperator{\rows}{rows}
\DeclareMathOperator{\sel}{sel}
\DeclareMathOperator{\last}{last}
\newcommand{\ul}[1]{\underline{#1}}

\begin{document}
	\title{Strictured Programming}
	\author{Murisi Tarusenga}
	\date{\today{} \currenttime}
	\maketitle
	\section{Introduction}
		\paragraph{What is this?}
			What follows is an experiment where I construct programs according to certain rules. While I do not list what these rules are, the following are a sketch of the sort of rules I have in mind:
			\begin{enumerate}
				\item The instruction "verify that $a=a$" is legal if it occurs after "choose an integer $a$"
				\item The instruction "verify that $b=a$" is legal if it occurs after "verify that $a=b$"
				\item The instruction "verify that $a=c$" is legal if it occurs after "verify that $a=b$" and "verify that $b=c$"
				\item The instruction "verify that $(a+b)+c=a+(b+c)$" is legal if it occurs after "choose integers $a,b,c$"
			\end{enumerate}
		\paragraph{Why was this made?}
			I wanted to see whether programs constructed according to certain rules can serve a similar function to mathematical proofs. For example, let $A$ be the $100\times 100$ matrix containing the multiplication table up to $100$. At least to me, seeing the form of \hyperref[sec:procedure 22]{procedure 22} allows me to be confident enough to bet that $\det(A^2)=\det(A)^2$ without carrying out the necessary computations.
		
		\paragraph{How do I understand this?}
			The task of understanding the following procedures should be the same as that of understanding any codebase. Hence domain specific knowledge is required, which in this case comprises rational, formal polynomial, and matrix arithmetic as well as inequalities. Otherwise, running a debugger, that is, executing the following procedures step by step on some chosen input(s) and observing their control flows and sequences of program states should be equally helpful in making sense of them.
	\tableofcontents
	\section{Matrix Arithmetic}\label{sec:body}
		Let us use the notation $\mathbb{Q}[x_1,x_2,\cdots,x_n]$ as a shorthand for "formal polynomial in the indeterminates $x_1,x_2,\cdots,x_n$".
		
		Let us use the notation $\mathcal{M}_{m,n}(A)$ as a shorthand for "$m\times n$ matrix of $A$s".
		
		Let us use the notation $p\circ q$ as a shorthand for "the sum of products where each product is the coefficient of a monomial in $p$ times the coefficient of the same monomial in $q$".
		\subsection{Procedure 1}\label{sec:procedure 1}
			\subsubsection{Objective}
				Choose a $\mathcal{M}_{m,2}(\mathbb{Q}[x])$, $A$. Let $\deg(0)=\infty$. Let $k=\min(\deg(A_{1,1}),\deg(A_{1,2}))$ and $q=\deg(A_{1,1})$. The objective of the following instructions is to make $A_{1,2}=0$, $\deg(A_{1,1})\le k$, and either leave $A_{*,1}$ unchanged or make $\deg(A_{1,1})<q$ by a sequence of operations whereby, in each step a $\mathbb{Q}[x]$ times either of the columns is added to the other.
			\subsubsection{Implementation}
				\begin{enumerate}
					\item Let $A$ be our working matrix.
					\item While $A_{1,2}\ne 0$, do the following:
					\begin{enumerate}
						\item If $\deg(A_{1,1})\le\deg(A_{1,2})$, then:
						\begin{enumerate}
							\item Subtract $\frac{x^{\deg(A_{1,2})}\circ A_{1,2}}{x^{\deg(A_{1,1})}\circ A_{1,1}}x^{\deg(A_{1,2})-\deg(A_{1,1})}$ times $A_{1,1}$ from $A_{1,2}$.
							\item Now verify that either $A_{1,2}$'s degree has decreased or $A_{1,2}=0$.
						\end{enumerate}
						\item Otherwise, do the following:
						\begin{enumerate}
							\item Let $p=\frac{x^{\deg(A_{1,1})}\circ A_{1,1}}{x^{\deg(A_{1,2})}\circ A_{1,2}}x^{\deg(A_{1,1})-\deg(A_{1,2})}$.
							\item If $A_{1,1}=pA_{1,2}$, then do the following:
							\begin{enumerate}
								\item Add $1-p$ times $A_{1,2}$ to $A_{1,1}$.
								\item Verify that now $A_{1,1}=A_{1,2}$.
							\end{enumerate}
							\item Otherwise, do the following:
							\begin{enumerate}
								\item Verify that $A_{1,1}\ne pA_{1,2}$.
								\item Add $-p$ times $A_{1,2}$ to $A_{1,1}$.
							\end{enumerate}
							\item Therefore verify that $A_{1,1}\ne 0$.
							\item Also verify that $A_{1,1}$'s degree has decreased.
						\end{enumerate}
					\end{enumerate}
					\item \textbf{Verify that $A_{1,2}=0$.}
					\item Verify that the changes to $A_{1,1}$, if any, have decreased its degree.
					\item If sensical, do the following:
					\begin{enumerate}
						\item Verify that all changes to $A_{1,2}$ but the last have decreased its degree.
						\item Verify that $\deg(A_{1,1})\le$ the degree of the penultimate value of $A_{1,2}$.
					\end{enumerate}
					\item \textbf{Therefore verify that $\deg(A_{1,1})\le k$.}
					\item If $A_{*,1}$ was changed, then do the following:
					\begin{enumerate}
						\item Verify that $A_{1,1}$ was also changed.
						\item \textbf{Therefore verify that $\deg(A_{1,1})<q$.}
					\end{enumerate}
					\item \textbf{Yield the tuple $\langle A\rangle$.}
				\end{enumerate}
		Let us use the notation "diagonal" as a shorthand for "matrix positions such that the row index equals the column index".
		
		Let us use the notation $\mathcal{D}_{m,n}(A)$ as a shorthand for "$\mathcal{M}_{m,n}(A)$ with $0$s in all the off-diagonal positions".
		\subsection{Procedure 2}\label{sec:procedure 2}
			\subsubsection{Objective}
				Choose a $\mathcal{M}_{m,n}(\mathbb{Q}[x])$, $A$. The objective of the following instructions is to transform $A$ into a $\mathcal{D}_{m,n}(\mathbb{Q}[x])$ by a sequence of operations whereby either a $\mathbb{Q}[x]$ times any of the columns is added to a different column, or a $\mathbb{Q}[x]$ times any of the rows is added to a different row.
			\subsubsection{Implementation}
				\begin{enumerate}
					\item If $m=0$ or $n=0$, then do the following:
					\begin{enumerate}
						\item \textbf{Verify that $A$ is a $\mathcal{D}_{m,n}(\mathbb{Q}[x])$.}
						\item \textbf{Yield the tuple $\langle A\rangle$.}
					\end{enumerate}
					\item Otherwise do the following:
					\item Verify that $m>0$ and $n>0$.
					\item Let $A$ be our working matrix.
					\item Now do the following:
						\begin{enumerate}
						\item While there are non-zero entries in the top row less its first entry, do the following:
						\begin{enumerate}
							\item In the first row, select the $\mathcal{M}_{m,2}(\mathbb{Q}[x])$ whose top-right entry coincides with the last non-zero entry of the first row
							\item Apply \hyperref[sec:procedure 1]{procedure 1} on this submatrix.
							\item Verify that the top-left and top-right entries of the submatrix are now non-zero and zero respectively.
							\item If the first column of $A$ was modified by (5aii), then do the following:
							\begin{enumerate}
								\item Verify that $\deg(A_{1,1})$ decreased.
								\item Go back to (5).
							\end{enumerate}
						\end{enumerate}
						\item Now do the same operations as in (a), but this time with the operations themselves reflected across the matrix's diagonal.
					\end{enumerate}
					\item Verify that, except for the top-left entry, the first row and the first column are zero.
					\item Apply \hyperref[sec:procedure 2]{procedure 2} on the submatrix $A_{[2:m+1],[2:n+1]}$.
					\item Verify that (7)'s execution leaves the first row and column unchanged.
					\item \textbf{Verify that $A$ is now a $\mathcal{D}_{m,n}(\mathbb{Q}[x])$.}
					\item \textbf{Yield the tuple $\langle A\rangle$.}
				\end{enumerate}
		\subsection{Procedure 3 (Associativity verification)}\label{sec:procedure 3}
			\subsubsection{Objective}
				Choose a $\mathcal{M}_{m,n}(\mathbb{Q}[x])$, A, a $\mathcal{M}_{n,p}(\mathbb{Q}[x])$, B, and a $\mathcal{M}_{p,q}(\mathbb{Q}[x])$, C. The objective of the following instructions is to show that $(AB)C=A(BC)$.
			\subsubsection{Implementation}
				\begin{enumerate}
					\item Verify that $(AB)_{i,l}=\sum_{k=1}^{n} \left(A_{i,k}*B_{k,l}\right)$ for $1\le i\le m$, for $1\le l\le p$.
					\item Verify that $((AB)C)_{i,r}=\sum_{l=1}^{p} \left((AB)_{i,l}*C_{l,r}\right)=\sum_{l=1}^{p} \left(\sum_{k=1}^{n} \left(A_{i,k}*B_{k,l}\right)*C_{l,r}\right)$ for $1\le i\le m$, for $1\le r\le q$.
					\item Verify that $(BC)_{k,r}=\sum_{l=1}^{p}\left(B_{k,l}*C_{l,r}\right)$ for $1\le k\le n$, for $1\le r\le q$.
					\item Verify that $(A(BC))_{i,r}=\sum_{k=1}^{n}\left(A_{i,k}*(BC)_{k,r}\right)=\sum_{k=1}^{n}\left(A_{i,k}*\sum_{l=1}^{p}\left(B_{k,l}*C_{l,r}\right)\right)$ for $1\le i\le m$, for $1\le r\le q$.
					\item Therefore Verify that $(2)=\sum_{l=1}^{p} \left(\sum_{k=1}^{n} \left(A_{i,k}*B_{k,l}*C_{l,r}\right)\right)=\sum_{k=1}^{n} \left(\sum_{l=1}^{p} \left(A_{i,k}*B_{k,l}*C_{l,r}\right)\right)=\sum_{k=1}^{n}\left(A_{i,k}*\sum_{l=1}^{p}\left(B_{k,l}*C_{l,r}\right)\right)=(4)$ for $1\le i\le m$, for $1\le r\le q$.
					\item \textbf{Therefore verify that $(AB)C=A(BC)$.}
				\end{enumerate}
		Let us use the notation $I_n$ as a shorthand for "the $\mathcal{M}_{n,n}(\mathbb{Q})$ with only $1$s on the diagonal and $0$s everywhere else".
		
		Let us use the notation $\mathcal{T}_{m}(\mathbb{Q}[x])$ as a shorthand for "$\mathcal{M}_{m,m}(\mathbb{Q}[x])$ with only $1$s on the diagonal, a single $\mathbb{Q}[x]$ off the diagonal, and $0$s everywhere else".
		
		Let us use the notation $\lvert A\rvert$ as a shorthand for "the number of items in the list $A$".
		\subsection{Procedure 4 (Row and column operation recording)}\label{sec:procedure 4}
			\subsubsection{Objective}
				Choose a procedure, $A$, and two non-negative integers $m,n$. The objective of the following instructions is to construct a list of $\mathcal{T}_{m}(\mathbb{Q}[x])$s, $M$, and a list of $\mathcal{T}_{n}(\mathbb{Q}[x])$s, $N$ such that $M_{\lvert M\rvert +1-i}$ equals $I_m$ after applying the $i^{th}$ row operation carried out by $A$ also on it, and $N_i$ equals $I_n$ after applying the $i^{th}$ row operation carried out by $A$ also on it.
			\subsubsection{Implementation}
				\begin{enumerate}
					\item Make an empty list, $N$.
					\item Augment procedure $A$ so that each time a polynomial $x$ times a column $i$ is added onto column $j$, an $n\times n$ matrix that only has $1$s on its diagonal, and such that the only non-zero entry off its diagonal is $x$ at position $(i,j)$, is appended onto $N$.
					\item Make an empty list, $M$.
					\item Also augment procedure $A$ so that each time a polynomial $x$ times a row $i$ is added onto row $j$, an $n\times n$ matrix that only has $1$s on its diagonal, and such that the only non-zero entry off its diagonal is $x$ at position $(j,i)$, is prepended onto $M$.
					\item Now run procedure $A$.
					\item \textbf{Yield the tuple $\langle M,N\rangle$.}
				\end{enumerate}
		\subsection{Procedure 5 (Multiplication by identity)}\label{sec:procedure 5}
			\subsubsection{Objective}
				Choose a $\mathcal{M}_{m,n}(\mathbb{Q}[x])$, $A$. The objective of the following instructions is to show that $I_mA=A=AI_n$.
			\subsubsection{Implementation}
				\begin{enumerate}
					\item For $1\le r\le m$, do the following:
					\begin{enumerate}
						\item For $1\le t\le n$, do the following:
						\begin{enumerate}
							\item Verify that $(I_mA)_{r,t}=\sum_{u=1}^m (I_m)_{r,u}A_{u,t}=(I_m)_{r,r}A_{r,t}=1*A_{r,t}=A_{r,t}$.
						\end{enumerate}
					\end{enumerate}
					\item \textbf{Therefore verify that $I_mA=A$.}
					\item For $1\le r\le m$, do the following:
					\begin{enumerate}
						\item For $1\le t\le n$, do the following:
						\begin{enumerate}
							\item Verify that $(AI_n)_{r,t}=\sum_{u=1}^m A_{r,u}(I_n)_{u,t}=A_{r,t}(I_n)_{t,t}=A_{r,t}*1=A_{r,t}$.
						\end{enumerate}
					\end{enumerate}
					\item \textbf{Therefore verify that $AI_n=A$.}
				\end{enumerate}
		Let us use the notation $M_*$ as a shorthand for "$\prod_{i=1}^{\lvert M\rvert} M_i$".
		\subsection{Procedure 6 (Matrix list inversion)}\label{sec:procedure 6}
			\subsubsection{Objective}
				Choose a list of $\mathcal{T}_{m}(\mathbb{Q}[x])$, $A$. The objective of the following instructions is to define a list of $\mathcal{T}_{m}(\mathbb{Q}[x])$, $A^{-1}$, such that $A_*{A^{-1}}_*=I_m$.
			\subsubsection{Implementation}
				\begin{enumerate}
					\item Let $A^{-1}$ be $\langle\rangle$.
					\item For $i=1$ to $i=\lvert A\rvert$, do the following:
					\begin{enumerate}
						\item Let $(j,k)$ be the position of the off diagonal entry of $A_i$.
						\item Let $B$ equal $A_i$ but with entry $(j,k)$ negated.
						\item For $1\le r\le m$ and $r\ne j$, do the following:
						\begin{enumerate}
							\item For $1\le t\le m$, do the following:
							\begin{enumerate}
								\item Verify that $(A_iB)_{r,t}=\sum_{u=1}^m (A_i)_{r,u}B_{u,t}=(A_i)_{r,r}B_{r,t}=1*B_{r,t}=[r=t]$.
							\end{enumerate}
						\end{enumerate}
						\item For $1\le t\le m$ and $t\ne k$, do the following:
						\begin{enumerate}
							\item Verify that $(A_iB)_{j,t}=\sum_{u=1}^m (A_i)_{j,u}B_{u,t}=(A_i)_{j,t}B_{t,t}=(A_i)_{j,t}*1=[j=t]$.
						\end{enumerate}
						\item Verify that $(A_iB)_{j,k}=\sum_{u=1}^m (A_i)_{j,u}B_{u,k}=(A_i)_{j,j}B_{j,k}+(A_i)_{j,k}B_{k,k}=1*B_{j,k}+(A_i)_{j,k}*1=B_{j,k}+(A_i)_{j,k}=0$.
						\item Therefore verify that $A_iB=I_m$.
						\item Now prepend $B$ onto $A^{-1}$.
					\end{enumerate}
					\item Verify that $\lvert A\rvert=\lvert A^{-1}\rvert$.
					\item Therefore using \hyperref[sec:procedure 3]{procedure 3} and \hyperref[sec:procedure 5]{procedure 5}, verify that $A_*{A^{-1}}_*$
					\begin{enumerate}
						\item $=A_1\cdots A_{\lvert A\rvert-1}A_{\lvert A\rvert}{A^{-1}}_1{A^{-1}}_2\cdots {A^{-1}}_{\lvert A\rvert}$
						\item $=A_1\cdots A_{\lvert A\rvert-2}A_{\lvert A\rvert-1}I_m{A^{-1}}_2{A^{-1}}_3\cdots {A^{-1}}_{\lvert A\rvert}$
						\item $=A_1\cdots A_{\lvert A\rvert-2}A_{\lvert A\rvert-1}{A^{-1}}_2{A^{-1}}_3\cdots {A^{-1}}_{\lvert A\rvert}$
						\item $\vdots$
						\item $=A_1I_m{A^{-1}}_{\lvert A\rvert}$
						\item $=A_1{A^{-1}}_{\lvert A\rvert}$
						\item $=I_m$.
					\end{enumerate}
				\end{enumerate}
		\subsection{Procedure 7}\label{sec:procedure 7}
			\subsubsection{Objective}
				Choose a list of $\mathcal{T}_{m}(\mathbb{Q}[x])$, $A$. The objective of the following instructions is to show that $(A^{-1})^{-1}=A$ and ${A^{-1}}_*A_*=I_m$.
			\subsubsection{Implementation}
				\begin{enumerate}
					\item \textbf{Verify that $(A^{-1})^{-1}=A$.}
					\item \textbf{Therefore using \hyperref[sec:procedure 6]{procedure 6}, verify that ${A^{-1}}_*A_*={A^{-1}}_*{(A^{-1})^{-1}}_*=I_m$.}
				\end{enumerate}
		\subsection{Procedure 8}\label{sec:procedure 8}
			\subsubsection{Objective}
				Choose a $\mathcal{M}_{m,n}(\mathbb{Q}[x])$, $A$. The objective of the following instructions is to define the polynomials $u_1,u_2,\cdots,u_{\min(m,n)}$ and transform $A$ into a $\mathcal{D}_{m,n}(\mathbb{Q}[x])$ such that $A_{k,k}=u_kA_{1,1}$ for $1\le k\le\min(m,n)$ by a sequence of operations whereby either a $\mathbb{Q}[x]$ times any of the columns is added to a different column, or a $\mathbb{Q}[x]$ times any of the rows is added to a different row.
			\subsubsection{Implementation}
				\begin{enumerate}
					\item Let $u=\langle 1\rangle$.
					\item For $j$ going from $2$ to $\min(m,n)$, do the following:
					\begin{enumerate}
						\item Verify that $A_{k,k}=u_kA_{1,1}$ for $k=1$ to $k=\lvert u\rvert$.
						\item Add row $j$ to row $1$.
						\item Now verify that $A_{1,j}=A_{j,j}$.
						\item Set $A'=A$ and let $A'$ be our working matrix.
						\item Let $\langle M,N\rangle$ receive the results of executing \hyperref[sec:procedure 4]{procedure 4} on the pair $\langle m,n\rangle$ and the following procedure:
						\begin{enumerate}
							\item Execute \hyperref[sec:procedure 1]{procedure 1} on the submatrix of $A'$ formed by selecting row $1$ and columns $1$ and $j$ as if there were nothing in between.
						\end{enumerate}
						\item Now verify that:
						\begin{enumerate}
							\item $M$ is empty.
							\item $AN_*=M_*AN_*=A'$.
							\item $A=AI_n=AN_*{N^{-1}}_*=A'{N^{-1}}_*$.
							\item $A'_{1,j}=0$.
							\item $A_{1,1}=A'_{1,1}{{N^{-1}}_*}_{1,1}+A'_{1,j}{{N^{-1}}_*}_{j,1}=A'_{1,1}{{N^{-1}}_*}_{1,1}$.
							\item $A_{j,j}=A_{1,j}=A'_{1,1}{{N^{-1}}_*}_{1,j}+A'_{1,j}{{N^{-1}}_*}_{j,j}=A'_{1,1}{{N^{-1}}_*}_{1,j}$.
							\item $A_{j,1}=0$.
							\item $A'_{j,1}=A_{j,1}{N_*}_{1,1}+A_{j,j}{N_*}_{j,1}=A_{j,j}{N_*}_{j,1}=A'_{1,1}{{N^{-1}}_*}_{1,j}{N_*}_{j,1}$.
							\item $A'_{j,j}=A_{j,1}{N_*}_{1,j}+A_{j,j}{N_*}_{j,j}=A_{j,j}{N_*}_{j,j}=A'_{1,1}{{N^{-1}}_*}_{1,j}{N_*}_{j,j}$.
						\end{enumerate}
						\item Subtract ${{N^{-1}}_*}_{1,j}{N_*}_{j,1}$ times row $1$ from row $j$.
						\item Now verify that $A'_{j,1}=0$.
						\item For $k=2$ to $k=\lvert u\rvert$, do the following:
						\begin{enumerate}
							\item Verify that $A'_{k,k}=A_{k,k}=u_kA_{1,1}=u_kA'_{1,1}{{N^{-1}}_*}_{1,1}$.
							\item Set $u_k=u_k{{N^{-1}}_*}_{1,1}$.
							\item Hence verify that $A'_{k,k}=u_kA'_{1,1}$.
						\end{enumerate}
						\item Let $u_j={{N^{-1}}_*}_{1,j}{N_*}_{j,j}$.
						\item Hence verify that $A'_{j,j}=u_jA'_{1,1}$.
						\item Now let $A=A'$.
					\end{enumerate}
					\item \textbf{Hence verify that $A'_{k,k}=u_kA'_{1,1}$ for $k=1$ to $k=\min(m,n)$.}
					\item \textbf{Yield $\langle u\rangle$.}
				\end{enumerate}
		Let us use the notation $[a:b]$ as a shorthand for "the list $\langle a,a+1,\cdots b-1\rangle$".
		\subsection{Procedure 9 (Block matrix multiplication)}\label{sec:procedure 9}
			\subsubsection{Objective}
				Choose a $\mathcal{M}_{m,n}(\mathbb{Q}[x])$, $A$, and a $\mathcal{M}_{n,k}(\mathbb{Q}[x])$, $B$. Choose integers $1\le a\le m$, $1\le b\le n$, and $1\le c\le k$. The objective of the following instructions is to show that $(AB)_{[1:a],[1:c]}=A_{[1:a],[1:b]}B_{[1:b],[1:c]}+A_{[1:a],[b:n+1]}B_{[b:n+1],[1:c]}$.
			\subsubsection{Implementation}
				\begin{enumerate}
					\item Multiply matrix $A$ by matrix $B$.
					\item For each $1\le i\le a-1$, do the following:
					\begin{enumerate}
						\item For each $1\le j\le c-1$, do the following:
							\begin{enumerate}
								\item Verify that $(AB)_{i,j}=\sum_{p=1}^n A_{i,p}B_{p,j}=\sum_{p=1}^{b-1} A_{i,p}B_{p,j}+\sum_{p=b}^n A_{i,p}B_{p,j}=\sum_{p=1}^{b-1} (A_{[1:a],[1:b]})_{i,p}(B_{[1:b],[1:c]})_{p,j}+\sum_{p=1}^{1+n-b} (A_{[1:a],[b:n+1]})_{i,p}(B_{[b:n+1],[1:c]})_{p,j}=(A_{[1:a],[1:b]}B_{[1:b],[1:c]})_{i,j}+(A_{[1:a],[b:n+1]}B_{[b:n+1],[1:c]})_{i,j}$.
							\end{enumerate}
					\end{enumerate}
					\item \textbf{Therefore verify that $(AB)_{[1:a],[1:c]}=A_{[1:a],[1:b]}B_{[1:b],[1:c]}+A_{[1:a],[b:n+1]}B_{[b:n+1],[1:c]}$.}
					\item \textbf{Do similar computations to verify that the other three blocks of $AB$ are computed in an analogous way to multiplying two $2\times 2$ matrices.}
				\end{enumerate}
		\subsection{Procedure 10 (Smith normal form construction)}\label{sec:procedure 10}
			\subsubsection{Objective}
				Choose a $\mathcal{M}_{m,n}(\mathbb{Q}[x])$, $A$. Let $A_{0,0}=1$. The objective of the following instructions is to define the polynomials $v_1,v_2,\cdots,v_{\min(m,n)}$ and transform $A$ into a $\mathcal{D}_{m,n}(\mathbb{Q}[x])$ such that $A_{k,k}=v_kA_{k-1,k-1}$ for $1\le k\le\min(m,n)$ by a sequence of operations whereby either a $\mathbb{Q}[x]$ times any of the columns is added to a different column, or a $\mathbb{Q}[x]$ times any of the rows is added to a different row.
			\subsubsection{Implementation}
				\begin{enumerate}
					\item Apply \hyperref[sec:procedure 2]{procedure 2} on matrix $A$.
					\item Let $v=\langle\rangle$.
					\item Let $p=\langle A_{1,1},A_{2,2},\cdots,A_{\min(m,n),\min(m,n)}\rangle$.
					\item For $j$ going from $1$ to $\min(m,n)$, do the following:
					\begin{enumerate}
						\item Set $A'=A$.
						\item Let $\langle M,N\rangle$ receive the results of executing \hyperref[sec:procedure 4]{procedure 4} on the pair $\langle m,n\rangle$ and the following procedure:
						\begin{enumerate}
							\item Apply \hyperref[sec:procedure 8]{procedure 8} on the submatrix of $A'$ containing rows $j$ to $m$ and columns $j$ to $n$, and let $\langle u\rangle$ receive.
						\end{enumerate}
						\item Verify that $A'_{k,k}=u_{k+1-j}A'_{j,j}$ for $k=j$ to $k=\min(m,n)$.
						\item \textbf{Verify that $A'$ is the same as $A$ modulo the submatrix spanning rows $j$ to $m$ and columns $j$ to $n$.}
						\item Verify that $M_i$ is the same as $I_m$ modulo the submatrix spanning rows $j$ to $m$ and columns $j$ to $m$, for $i=1$ to $\lvert M\rvert$.
						\item Therefore verify that $M_*$ is the same as $I_m$ modulo the submatrix spanning rows $j$ to $m$ and columns $j$ to $m$.
						\item Verify that $N_i$ is the same as $I_n$ modulo the submatrix spanning rows $j$ to $n$ and columns $j$ to $n$, for $i=1$ to $\lvert N\rvert$.
						\item Therefore verify that $N_*$ is the same as $I_n$ modulo the submatrix spanning rows $j$ to $n$ and columns $j$ to $n$.
						\item Verify that $A'=M_*AN_*$.
						\item Let $v_j=\sum_{r=j}^{\min(m,n)} (M_*)_{j,r}p_{r+1-j}(N_*)_{r,j}$.
						\item Hence using (f), (h), and (i), verify that $A'_{j,j}$
						\begin{enumerate}
							\item $=(M_*AN_*)_{j,j}$
							\item $=\sum_{r=1}^m (M_*)_{j,r}(AN_*)_{r,j}$
							\item $=\sum_{r=1}^{\min(m,n)} (M_*)_{j,r}(AN_*)_{r,j}$
							\item $=\sum_{r=1}^{\min(m,n)} (M_*)_{j,r}A_{r,r}(N_*)_{r,j}$
							\item $=\sum_{r=j}^{\min(m,n)} (M_*)_{j,r}A_{r,r}(N_*)_{r,j}$
							\item $=\sum_{r=j}^{\min(m,n)} (M_*)_{j,r}A_{j-1,j-1}p_{r+1-j}(N_*)_{r,j}$
							\item $=A_{j-1,j-1}\sum_{r=j}^{\min(m,n)} (M_*)_{j,r}p_{r+1-j}(N_*)_{r,j}$
							\item $=A'_{j-1,j-1}\sum_{r=j}^{\min(m,n)} (M_*)_{j,r}p_{r+1-j}(N_*)_{r,j}$
							\item $=A'_{j-1,j-1}v_j$.
						\end{enumerate}
						\item Set $A$ to $A'$.
						\item Set $p$ to $u_{2:\lvert u\rvert}$.
					\end{enumerate}
					\item \textbf{Yield the tuple $\langle v\rangle$.}
				\end{enumerate}
		\subsection{Procedure 11 (Determinant calculation)}\label{sec:procedure 11}
			\subsubsection{Objective}
				Choose a $\mathcal{M}_{m,m}(\mathbb{Q}[x])$, $A$. The objective of the following instructions is to define the $\mathbb{Q}[x]$ $\det(A)$.
			\subsubsection{Implementation}
				\begin{enumerate}
					\item If $m=0$, then do the following:
					\begin{enumerate}
						\item \textbf{Yield the tuple $\langle 1\rangle$.}
					\end{enumerate}
					\item Otherwise, do the following:
					\begin{enumerate}
						\item \textbf{Yield the tuple $\langle\sum_{r=1}^m (-1)^{r-1}A_{r,1}\det(A_{[1:r][r+1,m+1],[2:m+1]})\rangle$.}
					\end{enumerate}
				\end{enumerate}
		\subsection{Procedure 12 (Multilinearity verification)}\label{sec:procedure 12}
			\subsubsection{Objective}
				Choose a $\mathbb{Q}[x]$ $p$. Choose two $\mathcal{M}_{m,1}(\mathbb{Q}[x])$s, $B$ and $C$. Choose an integer $0<i\le m$. Choose a $\mathcal{M}_{m,m}(\mathbb{Q}[x])$, $A$, such that its $i^{th}$ column is $B+pC$. Let $A'$ be $A$ but with the $i^{th}$ column replaced by $B$ and let $A'''$ be $A$ but with the $i^{th}$ column replaced by $C$. The objective of the following instructions is to show that $\det(A)=\det(A') + p\det(A''')$.
			\subsubsection{Implementation}
				\begin{enumerate}
					\item If $i=1$, then verify that $\det(A)$
					\begin{enumerate}
						\item $=\sum_{r=1}^m (-1)^{r-1}A_{r,1}\cdot\det(A_{[1:r][r+1:m+1],[2:m+1]})$
						\item $=\sum_{r=1}^m (-1)^{r-1}(B+pC)_{r,1}\cdot\det(A_{[1:r][r+1:m+1],[2:m+1]})$
						\item $=\sum_{r=1}^m (-1)^{r-1}(B)_{r,1}\cdot\det(A_{[1:r][r+1:m+1],[2:m+1]})+\sum_{r=1}^m (-1)^{r-1}(pC)_{r,1}\cdot\det(A_{[1:r][r+1:m+1],[2:m+1]})$
						\item $=\sum_{r=1}^m (-1)^{r-1}(B)_{r,1}\cdot\det(A_{[1:r][r+1:m+1],[2:m+1]})+p\sum_{r=1}^m (-1)^{r-1}(C)_{r,1}\cdot\det(A_{[1:r][r+1:m+1],[2:m+1]})$
						\item $=\sum_{r=1}^m (-1)^{r-1}(A')_{r,1}\cdot\det(A'_{[1:r][r+1:m+1],[2:m+1]})+p\sum_{r=1}^m (-1)^{r-1}(A''')_{r,1}\cdot\det(A'''_{[1:r][r+1:m+1],[2:m+1]})$
						\item $=\det(A')+p\det(A''')$
					\end{enumerate}
					\item Otherwise, do the following:
					\begin{enumerate}
						\item For $r=1$ to $r=m$, do the following:
						\begin{enumerate}
							\item Execute \hyperref[sec:procedure 12]{procedure 12} on $\langle p,B_{[1:r][r+1:m+1],1},C_{[1:r][r+1:m+1],1},i-1,\allowbreak A_{[1:r][r+1:m+1],[2:m+1]}\rangle$.
							\item Therefore verify that $\det(A_{[1:r][r+1:m+1],[2:m+1]})=\det(A'_{[1:r][r+1:m+1],[2:m+1]})+p\det(A'''_{[1:r][r+1:m+1],[2:m+1]})$.
						\end{enumerate}
						\item Therefore using (a), verify that $\det(A)$
						\begin{enumerate}
							\item $=\sum_{r=1}^m (-1)^{r-1}A_{r,1}\cdot\det(A_{[1:r][r+1:m+1],[2:m+1]})$
							\item $=\sum_{r=1}^m (-1)^{r-1}A_{r,1}\cdot(\det(A'_{[1:r][r+1:m+1],[2:m+1]})+p\det(A'''_{[1:r][r+1:m+1],[2:m+1]}))$
							\item $=\sum_{r=1}^m (-1)^{r-1}A'_{r,1}\cdot\det(A'_{[1:r][r+1:m+1],[2:m+1]})+\sum_{r=1}^m (-1)^{r-1}A'''_{r,1}\cdot p\det(A'''_{[1:r][r+1:m+1],[2:m+1]})$
							\item $=\det(A')+p\det(A''')$.
						\end{enumerate}
					\end{enumerate}
				\end{enumerate}
			\textbf{Make an analogous procedure for cases when a given row is the sum of two $1\times m$ matrices.}
		\subsection{Procedure 13 (Alternation verification)}\label{sec:procedure 13}
			\subsubsection{Objective}
				Choose a $\mathcal{M}_{m,m}(\mathbb{Q}[x])$, $A$. Choose a row $1<i\le m$. Let $A'$ be $A$ with columns $i-1$ and $i$ swapped. The objective of the following instructions is to show that $\det(A')=-\det(A)$.
			\subsubsection{Implementation}
				\begin{enumerate}
					\item If $i=2$, then verify that $\det(A)$
					\begin{enumerate}
						\item $=\sum_{r=1}^m (-1)^{r-1}A_{r,1}\det(A_{[1:r][r+1:m+1],[2:m+1]})$
						\item $=\sum_{r=1}^m (-1)^{r-1}A_{r,1}\sum_{t=r+1}^m (-1)^{t-2}A_{t,2}\cdot\det(A_{[1:r][r+1:t][t+1:m+1],[3:m+1]})+\sum_{t=1}^m (-1)^{t-1}A_{t,1}\sum_{r=1}^{t-1} (-1)^{r-1}A_{r,2}\cdot\det(A_{[1:r][r+1:t][t+1:m+1],[3:m+1]})$
						\item $=\sum_{t=1}^m (-1)^{t-2}A_{t,2}\sum_{r=1}^{t-1} (-1)^{r-1}A_{r,1}\cdot\det(A_{[1:r][r+1:t][t+1:m+1],[3:m+1]})+\sum_{r=1}^m (-1)^{r-1}A_{r,2}\sum_{t=r+1}^m (-1)^{t-1}A_{t,1}\cdot\det(A_{[1:r][r+1:t][t+1:m+1],[3:m+1]})$
						\item $=\sum_{t=1}^m (-1)^{t-2}A'_{t,1}\sum_{r=1}^{t-1} (-1)^{r-1}A'_{r,2}\cdot\det(A'_{[1:r][r+1:t][t+1:m+1],[3:m+1]})+\sum_{r=1}^m (-1)^{r-1}A'_{r,1}\sum_{t=r+1}^m (-1)^{t-1}A'_{t,2}\cdot\det(A'_{[1:r][r+1:t][t+1:m+1],[3:m+1]})$
						\item $=-(\sum_{r=1}^m (-1)^{r-1}A'_{r,1}\sum_{t=r+1}^m (-1)^{t-2}A'_{t,2}\cdot\det(A'_{[1:r][r+1:t][t+1:m+1],[3:m+1]})+\sum_{t=1}^m (-1)^{t-1}A'_{t,1}\sum_{r=1}^{t-1} (-1)^{r-1}A'_{r,2}\cdot\det(A'_{[1:r][r+1:t][t+1:m+1],[3:m+1]}))$
						\item $=-\det(A')$.
					\end{enumerate}
					\item Otherwise do the following:
					\begin{enumerate}
						\item Verify that $i>2$.
						\item For $r=1$ to $r=m$, do the following:
						\begin{enumerate}
							\item Execute \hyperref[sec:procedure 13]{procedure 13} on $\langle i-1,\allowbreak A_{[1:r][r+1:m+1],[2:m+1]}\rangle$.
							\item Therefore verify that $\det(A_{[1:r][r+1:m+1],[2:m+1]})=-\det(A'_{[1:r][r+1:m+1],[2:m+1]})$.
						\end{enumerate}
						\item Therefore using (b), verify that $\det(A)=\sum_{r=1}^m (-1)^{r-1}A_{r,1}\cdot\det(A_{[1:r][r+1:m+1],[2:m+1]})=\sum_{r=1}^m (-1)^{r-1}A'_{r,1}\cdot(-\det(A'_{[1:r][r+1:m+1],[2:m+1]}))=-\det(A')$.
					\end{enumerate}
				\end{enumerate}
			\textbf{Make an analogous procedure to verify that row swaps cause sign alternations.}
		\subsection{Procedure 14}\label{sec:procedure 14}
			\subsubsection{Objective}
				Choose integers $1<i\le m$. Choose a $\mathcal{M}_{m,m}(\mathbb{Q}[x])$, $A$, such that columns $i-1$ and $i$ are the same. The objective of the following instructions is to show that $\det(A)=0$.
			\subsubsection{Implementation}
				\begin{enumerate}
					\item If $i=2$, then verify that $\det(A)$
					\begin{enumerate}
						\item $=\sum_{r=1}^m (-1)^{r-1}A_{r,1}\det(A_{[1:r][r+1:m+1],[2:m+1]})$
						\item $=\sum_{r=1}^m (-1)^{r-1}A_{r,1}\sum_{t=r+1}^m (-1)^{t-2}A_{t,2}\cdot\det(A_{[1:r][r+1:t][t+1:m+1],[3:m+1]})+\sum_{t=1}^m (-1)^{t-1}A_{t,1}\sum_{r=1}^{t-1} (-1)^{r-1}A_{r,2}\cdot\det(A_{[1:r][r+1:t][t+1:m+1],[3:m+1]})$
						\item $=\sum_{r=1}^m (-1)^{r-1}A_{r,1}\sum_{t=r+1}^m (-1)^{t-2}A_{t,2}\cdot\det(A_{[1:r][r+1:t][t+1:m+1],[3:m+1]})+\sum_{r=1}^m (-1)^{r-1}A_{r,2}\sum_{t=r+1}^m (-1)^{t-1}A_{t,1}\cdot\det(A_{[1:r][r+1:t][t+1:m+1],[3:m+1]})$
						\item $=-\sum_{r=1}^m (-1)^{r-1}A_{r,1}\sum_{t=r+1}^m (-1)^{t-1}A_{t,2}\cdot\det(A_{[1:r][r+1:t][t+1:m+1],[3:m+1]})+\sum_{r=1}^m (-1)^{r-1}A_{r,1}\sum_{t=r+1}^m (-1)^{t-1}A_{t,2}\cdot\det(A_{[1:r][r+1:t][t+1:m+1],[3:m+1]})$
						\item $=0$.
					\end{enumerate}
					\item Otherwise do the following:
					\begin{enumerate}
						\item Verify that $i>2$.
						\item For $r=1$ to $r=m$, do the following:
						\begin{enumerate}
							\item Execute \hyperref[sec:procedure 14]{procedure 14} on $\langle i-1,\allowbreak A_{[1:r][r+1:m+1],[2:m+1]}\rangle$.
							\item Therefore verify that $\det(A_{[1:r][r+1:m+1],[2:m+1]})=0$.
						\end{enumerate}
						\item Therefore using (b), verify that $\det(A)=\sum_{r=1}^m (-1)^{r-1}A_{r,1}\det(A_{[1:r][r+1:m+1],[2:m+1]})=\sum_{r=1}^m (-1)^{r-1}A_{r,1}*0=0$.
					\end{enumerate}
				\end{enumerate}
			\textbf{Make an analogous procedure to verify that matrix choices with repeated rows yield determinants equal to zero.}
		\subsection{Procedure 15}\label{sec:procedure 15}
			\subsubsection{Objective}
				Choose integers $1\le i\le m$. Choose an integer $0<j\le m-i$. Choose a $\mathcal{M}_{m,m}(\mathbb{Q}[x])$, $A$. Let $A'$ be $A$ but with column $i$ moved $j$ places. The objective of the following instructions is to show that $\det(A')=(-1)^j\det(A)$.
			\subsubsection{Implementation}
				\begin{enumerate}
					\item Let $A_i=A$.
					\item For $k=i+1$ to $k=i+j$, do the following:
					\begin{enumerate}
						\item Let $A_k$ be obtained by swapping columns $k-1$ and $k$ of $A_{k-1}$.
						\item Using \hyperref[sec:procedure 13]{procedure 13}, verify that $\det(A_k)=-\det(A_{k-1})$.
					\end{enumerate}
					\item Verify that $A'=A_{i+j}$.
					\item \textbf{Therefore verify that $\det(A')=\det(A_{i+j})=(-1)^1\det(A_{i+j-1})=\cdots=(-1)^j\det(A_{i})=(-1)^j\det(A)$.}
				\end{enumerate}
			\textbf{Make an analogous procedure that verifies that $\det(A')=(-1)^j\det(A)$ when a non-positive integer, $j$, is chosen.}
			
			\textbf{Also make an analogous procedure that does the verification for moved rows.}
		\subsection{Procedure 16 (Compound matrix calculation)}\label{sec:procedure 16}
			\subsubsection{Objective}
				Choose a $\mathcal{M}_{m,n}(\mathbb{Q}[x])$, $A$, and choose an integer $0\le k\le\min(m,n)$. The objective of the following instructions is to define the $\mathcal{M}_{\binom{m}{k},\binom{n}{k}}(\mathbb{Q}[x])$ $C_k(A)$.
			\subsubsection{Implementation}
				\begin{enumerate}
					\item Yield a tuple comprising the $\binom{m}{k}\times\binom{n}{k}$ matrix constructed as follows:
					\begin{enumerate}
						\item The rows are labeled by the colexicographically sorted list of increasing length-$k$ sequences whose elements are picked from the first $m$ positive integers.
						\item The columns are labeled by the colexicographically sorted list of increasing length-$k$ sequences whose elements are picked from the first $n$ positive integers.
						\item For each row label $I$: For each column label $J$: Let the entry at position $(I,J)$ be $\det(A_{I,J})$.
					\end{enumerate}
				\end{enumerate}
			\textbf{We will use the notation $C_k(A)$ to refer to an invocation of \hyperref[sec:procedure 16]{procedure 16} on the matrix $A$.}
			
			\textbf{We will use the notation $A_{\ul{I},\ul{J}}$ to refer to the entry of $A$ with row label $I$ and column label $J$.}
		\subsection{Procedure 17 (Compound matrix of identity calculation)}\label{sec:procedure 17}
			\subsubsection{Objective}
				Choose two integers $0\le k\le m$. The objective of the following instructions is to show that $C_k(I_m)=I_{\binom{m}{k}}$.
			\subsubsection{Implementation}
				\begin{enumerate}
					\item For each row label $I$ of $C_k(I_m)$, for each column label $J$ of $C_k(I_m)$, do the following:
					\begin{enumerate}
						\item If the $I=J$, then do the following:
						\begin{enumerate}
							\item Verify that $((I_m)_{I,J})_{i,j}=((I_m)_{J,J})_{i,j}=(I_m)_{J_i,J_j}=[J_i=J_j]=[i=j]$ for $1\le i\le k$, for $1\le j\le k$.
							\item Therefore verify that $(C_k(I_m))_{\ul{I},\ul{J}}=I_k$.
							\item \textbf{Therefore using \hyperref[sec:procedure 11]{procedure 11}, verify that $(C_k(I_m))_{\ul{I},\ul{J}}=\det((I_m)_{I,J})=\det(I_k)=1$.}
						\end{enumerate}
						\item Otherwise, do the following:
						\begin{enumerate}
							\item Verify that $I\ne J$.
							\item Let $i$ be the index of an element of $I$ that is not an element of $J$.
							\item Now verify that $(I_m)_{I_i,j}=[I_i=j]=0$, for each $j$ in $J$.
							\item Therefore verify that $((I_m)_{I,J})_{i,*}=0_{1\times k}$.
							\item \textbf{Therefore using \hyperref[sec:procedure 11]{procedure 11}, verify that $(C_k(I_m))_{\ul{I},\ul{J}}=\det((I_m)_{I,J})=0$.}
						\end{enumerate}
					\end{enumerate}
					\item \textbf{Therefore verify that $C_k(I_m)=I_{\binom{m}{k}}$.}	
				\end{enumerate}
		\subsection{Procedure 18}\label{sec:procedure 18}
			\subsubsection{Objective}
				Choose an integer $1\le k\le\min(m,n)$. Choose a $\mathcal{T}_{m}(\mathbb{Q}[x])$, $A$, such that the off diagonal entry is the $\mathbb{Q}[x]$ $p$ at $(i,j)$. Also choose a $\mathcal{M}_{m,n}(\mathbb{Q}[x])$, $B$. The objective of the following instructions is to construct a $\mathcal{M}_{\binom{m}{k},\binom{m}{k}}(\mathbb{Q}[x])$ $D$ such that $C_k(AB)=DC_k(B)$.
			\subsubsection{Implementation}
				\begin{enumerate}
					\item Let $D=C_k(I_m)=I_{\binom{m}{k}}$.
					\item Verify that $AB$ equals $B$, but with its row $i$ having $p$ times $B$'s row $j$ added to it.
					\item Go through the row labels, $I$, of $C_k(AB)$ and do the following:
					\begin{enumerate}
						\item If $i\notin I$, then do the following:
						\begin{enumerate}
							\item Verify that $(AB)_{I,[1:n+1]}=B_{I,[1:n+1]}$.
							\item Therefore for each column label $J$, verify that ${C_k(AB)}_{\ul{I},\ul{J}}=\det((AB)_{I,J})=\det(B_{I,J})={C_k(B)}_{\ul{I},\ul{J}}$.
							\item \textbf{Therefore verify that $(C_k(AB))_{\ul{I},*}=(C_k(B))_{\ul{I},*}$.}
						\end{enumerate}
						\item Otherwise, if $i\in I$, then:
						\begin{enumerate}
							\item Let $I'$ be $I$ but with an in-place replacement of $i$ by $j$.
							\item For each column label $J$: Using \hyperref[sec:procedure 12]{procedure 12}, verify that ${C_k(AB)}_{\ul{I},\ul{J}}=\det((AB)_{I,J})=\det(B_{I,J})+p*\det(B_{I',J})$.
							\item If $j\in I$, then do the following:
							\begin{enumerate}
								\item Verify that the sequence $I'$ contains two $j$s.
								\item For each column label $J$: Using \hyperref[sec:procedure 14]{procedure 14} verify that $\det(B_{I',J})=0$.
								\item Therefore for each column label $J$: verify that ${C_k(AB)}_{\ul{I},\ul{J}}=\det(B_{I,J})={C_k(B)}_{\ul{I},\ul{J}}$.
								\item \textbf{Therefore verify that ${C_k(AB)}_{\ul{I},*}={C_k(B)}_{\ul{I},*}$.}
							\end{enumerate}
							\item Otherwise if $j\notin I$, do the following:
							\begin{enumerate}
								\item Let $l$ be the signed number of places that the $j$ introduced above needs to be moved in order to make $I'$ an increasing sequence.
								\item Let $I''$ be obtained from $I'$ by moving the integer $j$ in $I'$ by $l$ places.
								\item For each column label $J$: Using \hyperref[sec:procedure 15]{procedure 15}, verify that $\det(B_{I',J})=(-1)^l\det(B_{I'',J})$.
								\item Therefore for each column label $J$: Verify that ${C_k(AB)}_{\ul{I},\ul{J}}=\det(B_{I,J})+p*\det(B_{I',J})=\det(B_{I,J})+(-1)^lp*\det(B_{I'',J})$.
								\item Verify that $I''$ is a row label of $C_k(B)$.
								\item Therefore for each column label $J$: Verify that ${C_k(AB)}_{\ul{I},\ul{J}}=\det(B_{I,J})+(-1)^lp*\det(B_{I'',J})={C_k(B)}_{\ul{I},\ul{J}}+(-1)^lp*{C_k(B)}_{\ul{I''},\ul{J}}$.
								\item \textbf{Therefore verify that $(C_k(AB))_{\ul{I},*}=(C_k(B))_{\ul{I},*}+(-1)^lp(C_k(B))_{\ul{I''},*}$.}
								\item \textbf{Set $D_{\ul{I},\ul{I''}}$ to $(-1)^lp$.}
							\end{enumerate}
						\end{enumerate}
						\item \textbf{Therefore verify that ${C_k(AB)}_{\ul{I},*}=D_{\ul{I},*}C_k(B)$.}
					\end{enumerate}
					\item \textbf{Therefore verify that $C_k(AB)=DC_k(B)$.}
				\end{enumerate}
		\subsection{Procedure 19}\label{sec:procedure 19}
			\subsubsection{Objective}
				Choose a $\mathcal{D}_{m,n}(\mathbb{Q}[x])$, $A$. Also choose an $\mathcal{M}_{n,n}(\mathbb{Q}[x])$, $B$. Also choose an integer $0\le k\le\min(m,n)$. The objective of the following instructions is to construct a $\mathcal{D}_{\binom{m}{k},\binom{n}{k}}(\mathbb{Q}[x])$ $D$ such that $C_k(AB)=DC_k(B)$.
			\subsubsection{Implementation}
				\begin{enumerate}
					\item Let $D=C_k(0_{m\times n})=0_{\binom{m}{k}\times\binom{n}{k}}$.
					\item Verify that $AB$ equals $B_{[1:\min(m,n)+1],[1:n+1]}$ with each row $i$ multiplied by $A_{i,i}$.
					\item Go through the row labels, $I$, of $C_k(AB)$ and do the following:
					\begin{enumerate}
						\item If $I_k\le \min(m,n)$, then do the following:
						\begin{enumerate}
							\item Using \hyperref[sec:procedure 16]{procedure 16}, verify that every element of $I$ is less than or equal to $\min(m,n)$.
							\item Let $A_0=A$.
							\item For $i=1$ to $i=k$: Let $A_i$ equal $A_{i-1}$ but with position $(I_i,I_i)$ set to $1$.
							\item For each column label $J$: Repeatedly using \hyperref[sec:procedure 12]{procedure 12}, verify that ${C_k(AB)}_{I,J}$
							\begin{enumerate}
								\item $=\det((AB)_{I,J})$
								\item $=\det((A_0B)_{I,J})$
								\item $=A_{I_1,I_1}\det((A_1B)_{I,J})$
								\item $=A_{I_1,I_1}A_{I_2,I_2}\det((A_2B)_{I,J})$
								\item $\vdots$
								\item $=A_{I_1,I_1}A_{I_2,I_2}\cdots A_{I_k,I_k}\det((A_kB)_{I,J})$
								\item $=A_{I_1,I_1}A_{I_2,I_2}\cdots A_{I_k,I_k}\det(B_{I,J})$
								\item $=A_{I_1,I_1}A_{I_2,I_2}\cdots A_{I_k,I_k}{C_k(B)}_{I,J}$.
							\end{enumerate}
							\item \textbf{Therefore verify that $(C_k(AB))_{\ul{I},*}=A_{I_1,I_1}A_{I_1,I_1}\cdots A_{I_k,I_k}*(C_k(B))_{\ul{I},*}$.}
							\item \textbf{Set $D_{\ul{I},\ul{I}}$ to $A_{I_1,I_1}A_{I_1,I_1}\cdots A_{I_k,I_k}$.}
						\end{enumerate}
						\item Otherwise if $I_k>\min(m,n)$, then do the following:
						\begin{enumerate}
							\item Verify that $A_{I_k,*}=0_{1\times n}$.
							\item Therefore verify that $(AB)_{I_k,*}=0_{1\times n}$.
							\item Therefore verify that $((AB)_{I,*})_{k,*}=0_{1\times n}$.
							\item Therefore using \hyperref[sec:procedure 11]{procedure 11}, for each column label $J$: verify that ${C_k(AB)}_{I,J}=\det((AB)_{I,J})=0$.
							\item \textbf{Therefore verify that $(C_k(AB))_{\ul{I},*}$ is zero.}
						\end{enumerate}
						\item \textbf{Therefore verify that ${C_k(AB)}_{\ul{I},*}=D_{\ul{I},*}C_k(B)$.}
					\end{enumerate}
					\item \textbf{Verify that $D$ is diagonal.}
					\item \textbf{Verify that $C_k(AB)=DC_k(B)$.}
				\end{enumerate}
		\subsection{Procedure 20}\label{sec:procedure 20}
			\subsubsection{Objective}
				Choose an integer $1\le k\le\min(m,n)$. Choose a $\mathcal{T}_{m}(\mathbb{Q}[x])$, $A$, such that the off diagonal entry is the $\mathbb{Q}[x]$ $p$ at $(i,j)$. Also choose a $\mathcal{M}_{m,n}(\mathbb{Q}[x])$, $B$. The objective of the following instructions is to show that $C_k(AB)=C_k(A)C_k(B)$.
			\subsubsection{Implementation}
				\begin{enumerate}
					\item Execute \hyperref[sec:procedure 18]{procedure 18} on matrices $A$ and $I_m$. Let $D$ be the matrix constructed.
					\item Using \hyperref[sec:procedure 17]{procedure 17}, verify that $C_k(A)=C_k(AI_m)=DC_k(I_m)=DI_{\binom{m}{k}}=D$.
					\item Execute \hyperref[sec:procedure 18]{procedure 18} on matrices $A$ and $B$. Let $D'$ be the matrix constructed.
					\item Verify that $C_k(AB)=D'C_k(B)$.
					\item Verify that $D'=D=C_k(A)$.
					\item \textbf{Therefore verify that $C_k(AB)=C_k(A)C_k(B)$.}
				\end{enumerate}
			\textbf{Make an analogous procedure to show that $C_k(BA)=C_k(B)C_k(A)$.}
			
			\textbf{Using \hyperref[sec:procedure 19]{procedure 19}, make a procedure similar to the above but that only instead allows for a diagonal matrix of $\mathbb{Q}[x]$s, $A$, to be chosen.}
		\subsection{Procedure 21}\label{sec:procedure 21}
			\subsubsection{Objective}
				Choose a $\mathcal{M}_{m,n}(\mathbb{Q}[x])$, $A$. Let $D_{0,0}=1$. The objective of the following instructions is to construct a list of $\mathcal{T}_{m}(\mathbb{Q}[x])$s, $M$, a $\mathcal{D}_{m,n}(\mathbb{Q}[x])$, $D$, a list of $\mathbb{Q}[x]$s, $v$, and a list of $\mathcal{T}_{n}(\mathbb{Q}[x])$s, $N$, such that $MAN=D$, $A=M^{-1}DN^{-1}$, and $D_{i,i}=v_iD_{i-1,i-1}$ for $i=1$ to $i=\min(m,n)$.
			\subsubsection{Implementation}
				\begin{enumerate}
					\item Let $D$ be a copy of $A$.
					\item Let $\langle M,N\rangle$ receive the results of executing \hyperref[sec:procedure 4]{procedure 4} on the pair $\langle m,n\rangle$ and the following procedure:
						\begin{enumerate}
							\item Execute \hyperref[sec:procedure 10]{procedure 10} on the matrix $D$ and let $\langle v\rangle$ receive.
						\end{enumerate}
					\item \textbf{Verify that $D_{i,i}=v_iD_{i-1,i-1}$ for $i=1$ to $i=\min(m,n)$.}
					\item \textbf{Verify that $M_*AN_*=D$.}
					\item Hence verify that $A=I_mAI_n={M^{-1}}_*M_*AN_*{N^{-1}}_*={M^{-1}}_*D{N^{-1}}_*$.
					\item \textbf{Yield the tuple $\langle M,D,v,N\rangle$.}
				\end{enumerate}
		\subsection{Procedure 22 (Compound matrix of matrix product calculation)}\label{sec:procedure 22}
			\subsubsection{Objective}
				Choose integers $0\le k\le\min(m,n,p)$. Choose a $\mathcal{M}_{m,n}(\mathbb{Q}[x])$, $A$. Also choose a $\mathcal{M}_{n,p}(\mathbb{Q}[x])$, $B$. The objective of the following instructions is to show that $C_k(AB)=C_k(A)C_k(B)$.
			\subsubsection{Implementation}
				\begin{enumerate}
					\item Execute \hyperref[sec:procedure 21]{procedure 21} on $A$ and let $\langle M,D,,N\rangle$ receive.
					\item Using repeated applications of \hyperref[sec:procedure 20]{procedure 20}, verify that $C_k(AB)$
					\begin{enumerate}
						\item $=C_k({M^{-1}}_1\cdots {M^{-1}}_{\lvert M\rvert}D{N^{-1}}_1\cdots {N^{-1}}_{\lvert N\rvert}B)$
						\item $=C_k({M^{-1}}_1)\cdots C_k({M^{-1}}_{\lvert M\rvert})*C_k(D)*C_k({N^{-1}}_1)\cdots C_k({N^{-1}}_{\lvert N\rvert})C_k(B)$
						\item $=C_k({M^{-1}}_1\cdots {M^{-1}}_{\lvert M\rvert}D{N^{-1}}_1\cdots {N^{-1}}_{\lvert N\rvert})C_k(B)$
						\item $=C_k(A)C_k(B)$.
					\end{enumerate}
				\end{enumerate}
		\subsection{Procedure 23 (Determinant equals product of diagonal entries verification)}\label{sec:procedure 23}
			\subsubsection{Objective}
				Choose a $\mathcal{M}_{m,m}(\mathbb{Q}[x])$, $A$. Let $D$ be a copy of $A$. Execute \hyperref[sec:procedure 2]{procedure 2} on $D$. The objective of the following instructions is to show that $\det(A)$ is the product of the diagonal entries of $D$.
			\subsubsection{Implementation}
				\begin{enumerate}
					\item Execute \hyperref[sec:procedure 21]{procedure 21} on $A$ and let $\langle M,D,,N\rangle$ receive.
					\item Using \hyperref[sec:procedure 11]{procedure 11} and \hyperref[sec:procedure 22]{procedure 22}, verify that $\det(A)$
					\begin{enumerate}
						\item $=C_m(A)$
						\item $=C_m({M^{-1}}_1\cdots {M^{-1}}_{\lvert M\rvert}D{N^{-1}}_1\cdots {N^{-1}}_{\lvert N\rvert})$
						\item $=C_m({M^{-1}}_1)\cdots C_m({M^{-1}}_{\lvert M\rvert})C_m(D)C_m({N^{-1}}_1)\cdots\allowbreak C_m({N^{-1}}_{\lvert N\rvert})$
						\item $=1\cdots 1C_m(D)1\cdots 1=C_m(D)$
						\item $=\det(D)$.
					\end{enumerate}
					\item \textbf{Using \hyperref[sec:procedure 11]{procedure 11}, verify that $\det(D)$ is the product of the diagonal entries of $D$.}
				\end{enumerate}
		\subsection{Procedure 24 (Transpose calculation)}\label{sec:procedure 24}
			\subsubsection{Objective}
				Choose a $\mathcal{M}_{m,n}(\mathbb{Q}[x])$, $A$. The objective of the following instructions is to define the $\mathcal{M}_{n,m}(\mathbb{Q}[x])$ $A^T$.
			\subsubsection{Implementation}
				\begin{enumerate}
					\item Make an $n\times m$ matrix, $A^T$.
					\item For $i=1$ to $i=n$:
					\begin{enumerate}
						\item For $j=1$ to $j=m$:
						\begin{enumerate}
							\item Let ${A^T}_{i,j}=A_{j,i}$.
						\end{enumerate}
					\end{enumerate}
					\item \textbf{Yield the tuple $\langle A^T\rangle$.}
				\end{enumerate}
		\subsection{Procedure 25 (Transpose of product verification)}\label{sec:procedure 25}
			\subsubsection{Objective}
				Choose a $\mathcal{M}_{m,n}(\mathbb{Q}[x])$, $A$, and a $\mathcal{M}_{n,k}(\mathbb{Q}[x])$, $B$. The objective of the following instructions is to show that $B^TA^T=(AB)^T$.
			\subsubsection{Implementation}
				\begin{enumerate}
					\item Verify that $B^TA^T$ and $(AB)^T$ have dimensions $k\times m$.
					\item For $i=1$ to $i=k$:
					\begin{enumerate}
						\item For $j=1$ to $j=m$:
						\begin{enumerate}
							\item Using \hyperref[sec:procedure 24]{procedure 24}, verify that $(B^TA^T)_{i,j}=\sum_{l=0}^n B_{l,i}A_{j,l}=\sum_{l=0}^n A_{j,l}B_{l,i}=(AB)_{j,i}=((AB)^T)_{i,j}$.
						\end{enumerate}
					\end{enumerate}
					\item \textbf{Therefore verify that $B^TA^T=(AB)^T$.}
				\end{enumerate}
		\subsection{Procedure 26 (Determinant of transpose verification)}\label{sec:procedure 26}
			\subsubsection{Objective}
				Choose a $\mathcal{M}_{m,m}(\mathbb{Q}[x])$, $A$. The objective of the following instructions is to show that $\det(A^T)=\det(A)$.
			\subsubsection{Implementation}
				\begin{enumerate}
					\item Execute \hyperref[sec:procedure 21]{procedure 21} on $A$ and let $\langle M,D,,N\rangle$ receive.
					\item Therefore using procedures \hyperref[sec:procedure 23]{19} and \hyperref[sec:procedure 24]{20}, verify that $\det(A^T)$
					\begin{enumerate}
						\item $=\det(({M^{-1}}_1\cdots {M^{-1}}_{\lvert M\rvert}D{N^{-1}}_1\cdots {N^{-1}}_{\lvert N\rvert})^T)$
						\item $=\det(({N^{-1}}_{\lvert N\rvert})^T\cdots({N^{-1}}_1)^TD^T({M^{-1}}_{\lvert M\rvert})^T\cdots({M^{-1}}_1)^T)$
						\item $=\det(D^T)$
						\item $=\det(D)$
						\item $=\det({M^{-1}}_1\cdots {M^{-1}}_{\lvert M\rvert}D{N^{-1}}_1\cdots {N^{-1}}_{\lvert N\rvert})$
						\item $=\det(A)$.
					\end{enumerate}
				\end{enumerate}
		\subsection{Procedure 27 (Compound matrix of transpose verification)}\label{sec:procedure 27}
			\subsubsection{Objective}
				Choose a $\mathcal{M}_{m,n}(\mathbb{Q}[x])$, $A$, and an integer $0\le k\le\min(m,n)$. The objective of the following instructions is to show that $C_k(A)^T=C_k(A^T)$.
			\subsubsection{Implementation}
				\begin{enumerate}
					\item For each row label $I$ of $C_k(A^T)$, do the following:
					\begin{enumerate}
						\item For each column label $J$ of $C_k(A^T)$, do the following:
						\begin{enumerate}
							\item Using \hyperref[sec:procedure 26]{procedure 26}, verify that $(C_k(A^T))_{\ul{I},\ul{J}}=\det((A^T)_{I,J})=\det(A_{J,I})=(C_k(A))_{\ul{J},\ul{I}}$.
						\end{enumerate}
					\end{enumerate}
					\item \textbf{Therefore verify that $(C_k(A))^T=(C_k(A^T))$.}
				\end{enumerate}
		\subsection{Procedure 28 (Linear system solution construction)}\label{sec:procedure 28}
			\subsubsection{Objective}
				Choose a $\mathcal{M}_{m,n}(\mathbb{Q})$, $A$, and a $\mathcal{M}_{m,p}(\mathbb{Q})$, $B$. Execute \hyperref[sec:procedure 21]{procedure 21} on $A$ and let $\langle M,D,,N\rangle$ receive the result. If the indices of the rows of $D$ that are entirely zero are also the indices of the rows of $MB$ that are entirely zero, then the objective of the following instructions is to construct a $\mathcal{M}_{n,p}(\mathbb{Q})$ $E$ such that $AE=B$.
			\subsubsection{Implementation}
				\begin{enumerate}
					\item Verify that $A=M^{-1}DN^{-1}$.
					\item Verify that $M^{-1}$, $D$, and $N^{-1}$ are $\mathcal{M}_{*,*}(\mathbb{Q})$s.
					\item Let $C$ be an $n\times p$ matrix with its $i^{th}$ row given as follows:
					\begin{enumerate}
						\item If $D_{i,i}\ne 0$, then do the following:
						\begin{enumerate}
							\item Let row $i$ be row $i$ of $MB$ divided by $D_{i,i}$.
						\end{enumerate}
						\item Otherwise, do the following:
						\begin{enumerate}
							\item \textbf{Choose $p$ rational numbers to fill up the row.}
						\end{enumerate}
					\end{enumerate}
					\item Verify that $DC=MB$.
					\item Let $E$ be $NC$.
					\item \textbf{Therefore using \hyperref[sec:procedure 6]{procedure 6}, verify that $AE=M^{-1}DN^{-1}E=M^{-1}DN^{-1}NC=M^{-1}DI_nC=M^{-1}DC=M^{-1}MB=I_mB=B$.}
					\item \textbf{Yield the tuple $\langle E\rangle$.}
				\end{enumerate}
			\textbf{The notation $A\backslash B$ shall be used to refer to the result, $E$, of invoking \hyperref[sec:procedure 28]{procedure 28} on matrices $A$ and $B$.}
			
			\textbf{Make an analogous procedure to yield an $F$ such that $FA=B$. The notation $B/A$ shall be used to refer to the $F$ yielded by invoking this procedure.}
		\subsection{Procedure 29}\label{sec:procedure 29}
			\subsubsection{Objective}
				Choose a $\mathcal{M}_{m,n}(\mathbb{Q})$, $A$, a $\mathcal{M}_{n,p}(\mathbb{Q})$, $E$, and a $\mathcal{M}_{m,p}(\mathbb{Q})$, $B$ such that $AE=B$. Execute \hyperref[sec:procedure 21]{procedure 21} on $A$ and let $\langle M,D,,N\rangle$ receive the result. If the indices of the rows of $D$ that are entirely zero are not also the indices of the rows of $M_*B$ that are entirely zero, then the objective of the following instructions is to show that $0\ne 0$.
			\subsubsection{Implementation}
				\begin{enumerate}
					\item Verify that ${M^{-1}}_*D{N^{-1}}_*E=AE=B$.
					\item Therefore verify that $D{N^{-1}}_*E=M_*B$.
					\item Let $i$ be an integer such that $D_{i,*}$ is zero and yet $(M_*B)_{i,*}$ is not zero.
					\item Verify that $D_{i,*}=D_{i,*}{N^{-1}}_*E=(D{N^{-1}}_*E)_{i,*}=(M_*B)_{i,*}$.
					\item Let $j$ be an integer such that $(M_*B)_{i,j}\ne 0$.
					\item \textbf{Now verify that $0=D_{i,j}=(M_*B)_{i,j}\ne 0$.}
				\end{enumerate}
		\subsection{Procedure 30}\label{sec:procedure 30}
			\subsubsection{Objective}
				Choose two $\mathcal{M}_{m,m}(\mathbb{Q})$s, $A$ and $B$, such that $AB=I_m$. The objective of the following instructions is to show that either $0=1$ or $BA=I_m$.
			\subsubsection{Implementation}
				\begin{enumerate}
					\item Execute \hyperref[sec:procedure 5]{procedure 5} on $B$ and let $\langle M^{-1},D,N^{-1}\rangle$ receive the result.
					\item Verify that $B={M^{-1}}_*D{N^{-1}}_*$.
					\item If $D$ has a zero on its diagonal, then do the following:
					\begin{enumerate}
						\item Using \hyperref[sec:procedure 23]{procedure 23}, verify that $\det(I_m)=\det(AB)=\det(A)\det(B)=\det(A)\det(D)=\det(A)*0=0$.
						\item Using \hyperref[sec:procedure 11]{procedure 11}, verify that $\det(I_m)=1^m=1$.
						\item Therefore verify that $0=1$.
						\item \textbf{Abort procedure.}
					\end{enumerate}
					\item Otherwise do the following:
					\begin{enumerate}
						\item Verify that $D$ does not have a zero on its diagonal.
						\item Verify that $B\backslash I_m=I_m(B\backslash I_m)=AB(B\backslash I_m)=A(B(B\backslash I_m))=AI_m=A$.
						\item \textbf{Therefore verify that $BA=B(B\backslash I_m)=I_m$.}
					\end{enumerate}
				\end{enumerate}
		\subsection{Procedure 31}\label{sec:procedure 31}
			\subsubsection{Objective}
				Choose an $\mathcal{M}_{m,m}(\mathbb{Q}[x])$, $M$, and an $\mathcal{M}_{m,m}(\mathbb{Q})$, $B$. The objective of the following instructions is to construct a $\mathcal{M}_{m,m}(\mathbb{Q}[x])$, $Q$, and a $\mathcal{M}_{m,m}(\mathbb{Q})$, $R$, such that $M=(xI_m-B)Q+R$.
			\subsubsection{Implementation}
				\begin{enumerate}
					\item Let $M_0x^b+M_1x^{b-1}+\cdots+M_bx^0=M$, where the $M_i$ are $\mathcal{M}_{m,m}(\mathbb{Q})$s.
					\item Now let $R=B^bM_0+B^{b-1}M_1+\cdots+B^0M_b$.
					\item Let $Q=\sum_{k=1}^b (x^{k-1}I_mB^0+x^{k-2}I_mB^1+\cdots+x^0I_mB^{k-1})M_k$.
					\item Verify that $M-R=(xI_m-B)\sum_{k=1}^b (x^{k-1}I_mB^0+x^{k-2}I_mB^1+\cdots+x^0I_mB^{k-1})M_k=(xI_m-B)Q$.
					\item \textbf{Verify that $M=(xI_m-B)Q+R$.}
					\item \textbf{Yield the tuple $\langle Q,R\rangle$.}
				\end{enumerate}
			\textbf{Make an analogous procedure that instead has the objective of constructing a $Q$ and $R$ such that $M=Q(xI_m-B)+R$.}
		\subsection{Procedure 32}\label{sec:procedure 32}
			\subsubsection{Objective}
				Choose two $\mathcal{M}_{m,m}(\mathbb{Q})$s, $B,A$, and two lists of $\mathcal{T}_{m}(\mathbb{Q}[x])$s such that $xI_m-B=M(xI_m-A)N$. The objective of the following instructions is to either show that $0=1$ or to construct $\mathcal{M}_{m,m}(\mathbb{Q})$s $R_1$ and $R_3$ such that $I_m=R_1R_3$ and $B=R_1AR_3$.
			\subsubsection{Implementation}
				\begin{enumerate}
					\item Verify that $(xI_m-B)N^{-1}=M(xI_m-A)NN^{-1}=M(xI_m-A)I_m=M(xI_m-A)$.
					\item Execute \hyperref[sec:procedure 31]{procedure 31} on $\langle M,B\rangle$ and let $\langle Q_1,R_1\rangle$ receive.
					\item Verify that $M=(xI_m-B)Q_1+R_1$.
					\item Execute \hyperref[sec:procedure 31]{procedure 31} on $\langle N^{-1},A\rangle$ and let $\langle Q_2,R_2\rangle$ receive.
					\item Verify that $N^{-1}=Q_2(xI_m-A)+R_2$.
					\item By substituting $M$ and $N^{-1}$ into (2), verify that $(xI_m-B)(Q_2(xI_m-A)+R_2)=((xI_m-B)Q_1+R_1)(xI_m-A)$.
					\item By rearranging both sides, verify that $(xI_m-B)(Q_2-Q_1)(xI_m-A)=R_1(xI_m-A)-(xI_m-B)R_2$.
					\item By equating the coefficients of different powers of $x$ both sides, verify that $Q_2-Q_1=0_{m\times m}$.
					\item Verify that $R_1(xI_m-A)-(xI_m-B)R_2=(xI_m-B)(Q_2-Q_1)(xI_m-A)=(xI_m-B)0_{m\times m}(xI_m-A)=0_{m\times m}$.
					\item Therefore by adding $(xI_m-B)R_2$ to both sides, verify that $xR_1-R_1A=R_1(xI_m-A)=(xI_m-B)R_2=xR_2-BR_2$.
					\item By equating the coefficients of $x$ on both sides, verify that $R_1=R_2$.
					\item Therefore verify that $R_1A=BR_1$.
					\item Execute \hyperref[sec:procedure 31]{procedure 31} on $\langle M^{-1},A\rangle$ and let $\langle Q_3,R_3\rangle$ receive.
					\item Verify that $M^{-1}=(xI_m-A)Q_3+R_3$.
					\item Verify that $I_m=MM^{-1}=((xI_m-B)Q_1+R_1)M^{-1}=(xI_m-B)Q_1M^{-1}+R_1M^{-1}=(xI_m-B)Q_1M^{-1}+R_1(xI-A)Q_3+R_1R_3=(xI_m-B)Q_1M^{-1}+(xI-B)R_1Q_3+R_1R_3=(xI_m-B)(Q_1M^{-1}+R_1Q_3)+R_1R_3$.
					\item By equating the powers of $x$ on both sides, verify that $Q_1M^{-1}+R_1Q_3=0$.
					\item By substituting zero for $Q_1M^{-1}+R_1Q_3$, \textbf{verify that $I_m=(xI_m-B)0_{m\times m}+R_1R_3=R_1R_3$.}
					\item \textbf{Therefore using \hyperref[sec:procedure 30]{procedure 30}, verify that $R_3R_1=I_m$.}
					\item \textbf{Also, verify that $B=BI_m=BR_1R_3=R_1AR_3$.}
					\item \textbf{Yield the pair $(R_1,R_3)$.}
				\end{enumerate}
		\subsection{Procedure 33}\label{sec:procedure 33}
			\subsubsection{Objective}
				Choose a $\mathcal{M}_{m,n}(\mathbb{Q}[x])$, $A$. Choose two integers $1\le i,j\le m$ such that $i\ne j$. The objective of the following instructions is to negate row $i$ and swap it with row $j$ using only elementary row and column operations.
			\subsubsection{Implementation}
				\begin{enumerate}
					\item Let $A$ be our working matrix.
					\item Subtract row $j$ from row $i$.
					\item Add row $i$ to row $j$.
					\item Subtract row $j$ from row $i$.
					\item \textbf{Verify that the $i^{th}$ row has been negated and swapped with the $j^{th}$ row.}
				\end{enumerate}
		\textbf{Make an analogous procedure to negate column $i$ and swap it with column $j$.}
		\subsection{Procedure 34}\label{sec:procedure 34}
			\subsubsection{Objective}
				Choose a $\mathcal{D}_{m,n}(\mathbb{Q}[x])$, $A$. Choose two integers $1\le i,j\le\min(m,n)$ such that $i\ne j$. The objective of the following instructions is to swap $B_{i,i}$ and $B_{j,j}$ using only elementary row and column operations.
			\subsubsection{Implementation}
				\begin{enumerate}
					\item Let $A$ be our working matrix.
					\item Use \hyperref[sec:procedure 33]{procedure 33} to negate the $i^{th}$ row and swap it with the $j^{th}$ row.
					\item Use \hyperref[sec:procedure 33]{procedure 33} to negate the $i^{th}$ column and swap it with the $j^{th}$ column.
					\item \textbf{Therefore, overall verify that $B_{i,i}$ and $B_{j,j}$ have been swapped.}
				\end{enumerate}
		\subsection{Procedure 35}\label{sec:procedure 35}
			\subsubsection{Objective}
				Choose a $\mathcal{D}_{m,n}(\mathbb{Q}[x])$, $A$. Choose two integers $1\le i,j\le\min(m,n)$ such that $i\ne j$. Choose a rational $k\ne 0$. The objective of the following instructions is to multiply $B_{i,i}$ by $k$ and $B_{j,j}$ by $\frac{1}{k}$ using only elementary row and column operations.
			\subsubsection{Implementation}
				\begin{enumerate}
					\item Let $A$ be our working matrix.
					\item Add $k$ times row $i$ to row $j$.
					\item Subtract $\frac{1}{k}$ times row $j$ from row $i$.
					\item Add $k$ times row $i$ to row $j$.
					\item Verify that the $i^{th}$ row has been scaled by $k$, the $j^{th}$ row by $-\frac{1}{k}$, and that both these rows are swapped.
					\item Use \hyperref[sec:procedure 33]{procedure 33} to negate the $i^{th}$ row and swap it with the $j^{th}$ row.
					\item \textbf{Therefore, overall verify that $B_{i,i}$ has been multiplied by $k$, and $B_{j,j}$ by $\frac{1}{k}$.}
				\end{enumerate}
		Let us use the notation "$p$ is monic" as a shorthand for "$x^{\deg(p)}\circ p=1$".
		\subsection{Procedure 36}\label{sec:procedure 36}
			\subsubsection{Objective}
				Choose a $\mathcal{M}_{m,m}(\mathbb{Q})$, $A$. Execute \hyperref[sec:procedure 10]{procedure 10} on the polynomial matrix $xI-A$ and let $\langle B\rangle$ be the result. The objective of the following instructions is to show that either none of the diagonal entries of $B$ are equal to zero, or $1=0$.
			\subsubsection{Implementation}
				\begin{enumerate}
					\item Using \hyperref[sec:procedure 11]{procedure 11}, verify that $\det(xI-A)$ is a monic polynomial of degree $m$.
					\item Therefore verify that $\det(B)=\det(xI-A)$ is a monic polynomial of degree $m$.
					\item If any of the diagonal entries of $B$ equal zero, then do the following:
					\begin{enumerate}
						\item Using \hyperref[sec:procedure 11]{procedure 11}, verify that $\det(B)=B_{1,1}B_{2,2}\cdots B_{m,m}=0$.
						\item Therefore verify that $1=0$.
						\item \textbf{Abort procedure.}
					\end{enumerate}
					\item Otherwise do the following:
					\begin{enumerate}
						\item \textbf{Verify that none of the diagonal entries of $B$ equal zero.}
					\end{enumerate}
				\end{enumerate}
			Let us use the notation $[P]$ as a shorthand for "if $P$, then yield $1$, otherwise yield $0$".
			
			Let us use the notation $\cols(A)$ as a shorthand for "the number of columns of $A$".
			
			Let us use the notation $\rows(A)$ as a shorthand for "the number of rows of $A$".
		\subsection{Procedure 37 (Block diagonal construction)}\label{sec:procedure 37}
			\subsubsection{Objective}
				Choose a list of $\mathcal{M}_{*}(\mathbb{Q})$, $C$. Let $m=\sum_{i=1}^{\lvert C\rvert}\cols(C_i)$. The objective of the following instructions is to define the $\mathcal{M}_{m,m}(\mathbb{Q})$, $\bdiag(C)$.
			\subsubsection{Implementation}
				\begin{enumerate}
					\item Let $E$ be a $0\times 0$ matrices.
					\item \textbf{Now for $i=1$ to $i=\lvert C\rvert$:}
					\begin{enumerate}
						\item Add $\cols(C_i)$ columns filled with zeros to the right end of $E$.
						\item Add $\cols(C_i)$ rows filled with zeros to the bottom end of $E$.
						\item Set the bottom-right corner of $E$ equal to $C_i$.
					\end{enumerate}
					\item Verify that $\cols(E)=\sum_{i=1}^{\lvert C\rvert}\cols(C_i)=m$.
					\item \textbf{Yield the tuple $\langle E\rangle$.}
				\end{enumerate}
		\subsection{Procedure 38}\label{sec:procedure 38}
			\subsubsection{Objective}
				Choose a positive integer $m$ and an $\mathcal{M}_{m,m}(\mathbb{Q})$, $A$. Execute \hyperref[sec:procedure 21]{procedure 21} on the polynomial matrix $xI_m-A$ and let $\langle ,B,v,\rangle$ be the result. The objective of the following instructions is to either show that $0<0$ or to construct an integer $a$ such that $\sum_{i=a}^m\deg(B_{i,i})=m$, $\deg(B_{i,i})>0$ for $i=a$ to $i=m$, and $\deg(B_{i,i})=0$ for $i=1$ to $i=a-1$.
			\subsubsection{Implementation}
				\begin{enumerate}
					\item Execute \hyperref[sec:procedure 36]{procedure 36} on $A$.
					\item If $\deg(B_{i,i})=0$ for $i=1$ to $i=m$, then do the following:
					\begin{enumerate}
						\item Verify that $\det(xI_m-A)=\det(B)=B_{1,1}B_{2,2}\cdots B_{m,m}$.
						\item Therefore verify that $0<m=\deg(\det(xI_m-A))=\deg(B_{1,1}B_{2,2}\cdots B_{m,m})=0+0+\cdots+0=0$.
						\item \textbf{Abort procedure.}
					\end{enumerate}
					\item Otherwise do the following:
					\begin{enumerate}
						\item Let $1\le a\le m$ be the least integer such that $\deg(B_{a,a})>0$.
						\item \textbf{Verify that $\deg(B_{i,i})=0$ for $i=1$ to $i=a-1$.}
						\item \textbf{Verify that $\sum_{i=a}^m\deg(B_{i,i})=\sum_{i=1}^m\deg(B_{i,i})=\deg(B_{1,1}B_{2,2}\cdots B_{m,m})=\deg(\det(B))=\deg(xI_m-A)=m$.}
						\item For $i=a+1$ to $i=m$, do the following:
						\begin{enumerate}
							\item Verify that $B_{i,i}=u_iB_{i-1,i-1}$.
							\item Verify that $B_{i,i}\ne 0$.
							\item Therefore verify that $u_i\ne 0$.
							\item \textbf{Therefore verify that $\deg(B_{i,i})=\deg(u_iB_{i-1,i-1})\ge\deg(B_{i-1,i-1})>0$.}
						\end{enumerate}
						\item \textbf{Yield the tuple $\langle a\rangle$.}
					\end{enumerate}
				\end{enumerate}
		\subsection{Procedure 39 (Rational canonical form construction)}\label{sec:procedure 39}
			\subsubsection{Objective}
				Choose a $\mathbb{Q}[x]$, $p=x^k+p_1x^{k-1}+p_2x^{k-2}+\cdots+p_kx^0$ such that $k>0$. The objective of the following instructions is to define the $\mathcal{M}_{k,k}(\mathbb{Q})$, $\rcan(p)$.
			\subsubsection{Implementation}
				\begin{enumerate}
					\item Make a $k\times k$ matrix $C$.
					\item Let $C$'s first $k-1$ columns be filled with the last $k-1$ columns of $I_k$.
					\item Let $C$'s last column from top to bottom be $-p_k, -p_{k-1},\cdots,-p_1$.
					\item \textbf{Yield the tuple $\langle C\rangle$.}
				\end{enumerate}
		\subsection{Procedure 40}\label{sec:procedure 40}
			\subsubsection{Objective}
				Choose a monic $\mathbb{Q}[x]$, $p$ such that $\deg(p)>0$. Let $k=\deg(p)$. Choose a $\mathcal{M}_{k,k}(\mathbb{Q}[x])$, $D$, such that $D=xI_k-\rcan(p)$. The objective of the following instructions is to transform $D$ into $\bdiag(1,\cdots,1,p)$ by a sequence of elementary operations.
			\subsubsection{Implementation}
				\begin{enumerate}
					\item Let the matrix $D$ be our working matrix.
					\item For $i=k$ going down to $i=2$, add $x$ times row $i$ to row $i-1$.
					\item Verify that $D$'s first $k-1$ columns are now the last $k-1$ columns of $-I_k$.
					\item Verify that $D$'s last column is $p$ followed by some other polynomials.
					\item For $i=2$ going up to $i=k$, subtract $D_{i, k}$ times column $i-1$ from column $k$.
					\item Verify that $D$'s last column is now $p$ followed by zeros.
					\item For $i=2$ going up to $i=k$, negate row $i-1$ and exchange it with row $i$ using \hyperref[sec:procedure 33]{procedure 33}.
					\item \textbf{Therefore verify that $D=\bdiag(1,\cdots,1,p)$.}
				\end{enumerate}
		\textbf{Let us use the notation $\mon(p)$ as a shorthand for "$\frac{p}{x^{\deg(p)}\circ p}$" in what follows.}
		\subsection{Procedure 41}\label{sec:procedure 41}
			\subsubsection{Objective}
				Choose a positive integer $m$ and an $\mathcal{M}_{m,m}(\mathbb{Q})$, $A$. Execute \hyperref[sec:procedure 4]{procedure 4} on the polynomial matrix $xI_m-A$ and let $\langle,B,,\rangle$ receive the result. Execute \hyperref[sec:procedure 38]{procedure 38} on $A$ and let $\langle a\rangle$ receive the result. Let $E_i=\rcan(\mon(B_{a-1+i,a-1+i}))$ for $i=1$ to $i=m+1-a$. The objective of the following instructions is to first show that $\cols(\bdiag(E))=m$, and second to apply a sequence of elementary operations on $xI_m-\bdiag(E)$ to obtain the matrix $B$.
			\subsubsection{Implementation}
				\begin{enumerate}
					\item Verify that the diagonal of $B$ comprises $x-1$ rationals followed by $B_{a,a},B_{a+1,a+1},\cdots,B_{m,m}$.
					\item \textbf{Using \hyperref[sec:procedure 40]{procedure 40}, verify that $\cols(\bdiag(E))=\sum_{i=1}^{\lvert E\rvert}\cols(E_i)=\sum_{i=1}^{\lvert E\rvert}\cols(\rcan(\mon(B_{a-1+i,a-1+i})))=\sum_{i=1}^{\lvert E\rvert}\deg(\mon(B_{a-1+i,a-1+i}))=\sum_{i=1}^{m+1-a}\deg(B_{a-1+i,a-1+i})=\sum_{i=a}^m\deg(B_{i,i})=m$.}
					\item Let $F=xI_m-\bdiag(E)$.
					\item Now for $i=1$ to $i=\lvert E\rvert$:
					\begin{enumerate}
						\item Let $j=1+\sum_{r=1}^{i-1}\cols(E_r)$.
						\item Let $k=j+\cols(E_i)$.
						\item Apply \hyperref[sec:procedure 40]{procedure 40} on the tuple $\langle\mon(B_{a-1+i,a-1+i}),F_{[j:k],[j:k]}\rangle$.
					\end{enumerate}
					\item Now verify that $F$ is a $\mathcal{D}_{m,m}(\mathbb{Q})$.
					\item Also verify that the diagonal of $F$ comprises $\mon(B_{a,a}),\mon(B_{a+1,a+1}),\cdots,\mon(B_{m,m})$ and $a-1$ $1$s.
					\item Rearrange the diagonal of $F$ so that $\mon(B_{i,i})$ is at the $i^{th}$ position on the diagonal for $i=a$ to $i=m$ by doing pairwise swaps. In general, swap the $i^{th}$ and $j^{th}$ diagonal entries using \hyperref[sec:procedure 34]{procedure 34}.
					\item For $i=1$ to $i=m-1$, do the following:
					\begin{enumerate}
						\item Let $k=\frac{x^{\deg(B_{i,i})}\circ B_{i,i}}{x^{\deg(F_{i,i})}\circ F_{i,i}}$.
						\item Scale $B_{i,i}$ by $k$ and $B_{i+1,i+1}$ by $\frac{1}{k}$ using \hyperref[sec:procedure 35]{procedure 35}.
						\item Now verify that $F_{i,i}=B_{i,i}$.
					\end{enumerate}
					\item Now verify that $x^m\circ\det(F)=x^m\circ\det(xI_m-\bdiag(E))=1=x^m\circ\det(xI_m-A)=x^m\circ\det(B)$.
					\item Therefore verify that $x^{\deg(F_{m,m})}\circ F_{m,m}=\frac{x^m\circ\det(F)}{x^{m-\deg(F_{m,m})}\circ(\det(F_{[1:m],[1:m]}))}=\frac{x^m\circ\det(B)}{x^{m-\deg(B_{m,m})}\circ(\det(B_{[1:m],[1:m]}))}=x^{\deg(B_{m,m})}\circ B_{m,m}$.
					\item Therefore verify that $F_{m,m}=B_{m,m}$.
					\item \textbf{Therefore verify that $F=B$.}
				\end{enumerate}
		\subsection{Procedure 42}\label{sec:procedure 42}
			\subsubsection{Objective}
				Choose a $\mathcal{M}_{m,m}(\mathbb{Q})$, $A$. Execute \hyperref[sec:procedure 38]{procedure 38} on $A$ and let $\langle a\rangle$ receive the result. Let $E_i=\rcan(\mon(B_{a-1+i,a-1+i}))$ for $i=1$ to $i=m+1-a$. The objective of the following instructions is to either show that $0=1$ or to construct $\mathcal{M}_{m,m}(\mathbb{Q})$s $R,T$ such that $A=R\bdiag(E)T$, $RT=I_m$, and $TR=I_m$.
			\subsubsection{Implementation}
				\begin{enumerate}
					\item Execute \hyperref[sec:procedure 21]{procedure 21} on the polynomial matrix $xI_m-A$ and let $\langle P,B,,Q\rangle$ be the result.
					\item Verify that $P_*(xI_m-A)Q_*=B$.
					\item Verify that $xI_m-A={P^{-1}}_*B{Q^{-1}}_*$.
					\item Let $Z$ be a variant of \hyperref[sec:procedure 21]{procedure 21} where every occurence of \hyperref[sec:procedure 10]{procedure 10} in its instructions is replaced with \hyperref[sec:procedure 41]{procedure 41}, and where every mention of $v$ is ignored.
					\item Execute procedure $Z$ on the matrix $xI_m-\bdiag(E)$ and let $\langle M,,,N\rangle$ receive the result.
					\item Verify that $M_*(xI_m-\bdiag(E))N_*=B$.
					\item Verify that $xI_m-A={P^{-1}}_*B{Q^{-1}}_*={P^{-1}}_*M(xI_m-\bdiag(E))N{Q^{-1}}_*$.
					\item Execute \hyperref[sec:procedure 32]{procedure 32} on the matrices $\langle A,{P}^{-1}M,\bdiag(E),N{Q}^{-1}\rangle$. Let the tuple $\langle R,T\rangle$ be the result.
					\item \textbf{Verify that $A=R\bdiag(E)T$.}
					\item \textbf{Verify that $RT=I_m$.}
					\item \textbf{Verify that $TR=I_m$.}
					\item \textbf{Yield the tuple $\langle R,E,T\rangle$.}
				\end{enumerate}
		\subsection{Procedure 43}\label{sec:procedure 43}
			\subsubsection{Objective}
				Choose a $\mathbb{Q}[x]$, $r=r_0x^t+r_1x^{t-1}+\cdots+r_tx^0$, and $\mathcal{M}_{m,m}(\mathbb{Q})$s, $R,A,S$ such that $SR=I_m$. The objective of the following instructions is to show that $r(RAS)=Rr(A)S$.
			\subsubsection{Implementation}
				\begin{enumerate}
					\item \textbf{Verify that $r(RAS)=r_0(RAS)^t+r_1(RAS)^{t-1}+\cdots+r_t(RAS)^0=r_0RA^tS+r_1RA^{t-1}S+\cdots+r_tRA^0S=R(r_0A^t+r_1A^{t-1}+\cdots+r_tA^0)S=Rr(A)S$.}
				\end{enumerate}
		\subsection{Procedure 44}\label{sec:procedure 44}
			\subsubsection{Objective}
				Choose a list of $\mathcal{M}_{m,m}(\mathbb{Q})$s, $A$, and a $\mathbb{Q}[x]$, $r=r_0x^t+r_1x^{t-1}+\cdots+r_tx^0$. The objective of the following instructions is to show that $r(\bdiag(A))=\bdiag(r(A))$.
			\subsubsection{Implementation}
				\begin{enumerate}
					\item For $i=0$ up to $i=t$, by repeated applications of \hyperref[sec:procedure 9]{procedure 9}, verify that $\bdiag(A)^i$ evaluates to $\bdiag(A^i)$ (where the exponentiation is done element-wise).
					\item Therefore verify that $r(\bdiag(A))$
					\begin{enumerate}
						\item $=r_0\bdiag(A)^t+r_1\bdiag(A)^{t-1}+\cdots+r_t\bdiag(A)^0$
						\item $=r_0\bdiag(A^t)+r_1\bdiag(A^{t-1})+\cdots+r_t\bdiag(A^0)$
						\item $=\bdiag(r_0A^t)+\bdiag(r_1A^{t-1})+\cdots+\bdiag(r_tA^0)$
						\item \textbf{$=\bdiag(r(A))$ (where $r$ is applied element-wise).}
					\end{enumerate}
				\end{enumerate}
		\subsection{Procedure 45}\label{sec:procedure 45}
			\subsubsection{Objective}
				Choose a $\mathcal{M}_{m,m}(\mathbb{Q})$, $A$, and a $\mathbb{Q}[x]$, $r$. Execute \hyperref[sec:procedure 42]{procedure 42} on the matrix $A$ and let the tuple $\langle R_1,E,R_3\rangle$ receive the result. The objective of the following instructions is to show that $r(A)=R_1\bdiag(r(E))R_3$ (where $r$ is applied element-wise).
			\subsubsection{Implementation}
				\begin{enumerate}
					\item Verify that $R_3R_1=I_m$.
					\item Using \hyperref[sec:procedure 43]{procedure 43}, verify that $r(A)=r(R_1\bdiag(E)R_3)=R_1r(\bdiag(E))R_3$.
					\item Using \hyperref[sec:procedure 44]{procedure 44}, verify that $r(\bdiag(E))=\bdiag(r(E))$ (where $r$ is applied element-wise).
					\item \textbf{Therefore verify that $r(A)=R_1\bdiag(r(E))R_3$ (where $r$ is applied element-wise).}
				\end{enumerate}
		Let us use the notation $e_i$ as a shorthand for "the $\mathcal{M}_{k,1}(\mathbb{Q})$ that is $0$, except for its $i^{th}$ entry which is $1$".
		
		Let us use the notation $0_{m\times n}$ as a shorthand for "the $\mathcal{M}_{m,m}(\mathbb{Q})$ such that every entry is $0$".
		\subsection{Procedure 46}\label{sec:procedure 46}
			\subsubsection{Objective}
				Choose a $\mathbb{Q}[x]$ $p=x^k+p_1x^{k-1}+p_2x^{k-2}+\cdots+p_kx^0$ such that $k>0$. The objective of the following instructions is to show that $p(\rcan(p))=0_{k\times k}$.
			\subsubsection{Implementation}
				\begin{enumerate}
					\item Let $G=\rcan(p)$.
					\item Then by $G$'s construction, for $i=1$ up to $i=k$, verify that $G^{i-1}e_1=G^{i-2}e_2=\cdots=G^{0}e_i=e_i$.
					\item Therefore, for $i=1$ up to $i=k$: Cognizant of the construction of $G$'s last column, verify that $p(G)e_i$
					\begin{enumerate}
						\item $=(G^k+p_1G^{k-1}+p_2G^{k-2}+\cdots+p_kG^0)e_i$
						\item $=(G^k+p_1G^{k-1}+p_2G^{k-2}+\cdots+p_kG^0)G^{i-1}e_1$
						\item $=G^{i-1}(GG^{k-1}+p_1G^{k-1}+p_2G^{k-2}+\cdots+p_kG^0)e_1$
						\item $=G^{i-1}(Ge_k+p_1e_k+p_2e_{k-1}+\cdots+p_ke_1)$
						\item $=G^{i-1}0_{k\times 1}$
						\item $=0_{k\times 1}$.
					\end{enumerate}
					\item Therefore verify that $p(G)=0_{k\times k}$.
				\end{enumerate}
		\subsection{Procedure 47}\label{sec:procedure 47}
			\subsubsection{Objective}
				Choose a $\mathcal{M}_{m,m}(\mathbb{Q})$, $A$. The objective of the following instructions is to define the $\mathbb{Q}[x]$ $\last_A$ and show that either $1=0$ or $\last_A\ne 0$.
			\subsubsection{Implementation}
				\begin{enumerate}
					\item Execute \hyperref[sec:procedure 21]{procedure 21} on the polynomial matrix $xI_m-A$ and let the tuple $\langle,B,,\rangle$ receive the result.
					\item Execute \hyperref[sec:procedure 36]{procedure 36} on $A$.
					\item Verify that $B_{m,m}\ne 0$.
					\item Yield $\langle B_{m,m}\rangle$.
				\end{enumerate}
		\subsection{Procedure 48}\label{sec:procedure 48}
			\subsubsection{Objective}
				Choose a $\mathcal{M}_{m,m}(\mathbb{Q})$, $A$. The objective of the following instructions is to either show that $0<0$ or to show that $\last_A(A)=0_{m\times m}$.
			\subsubsection{Implementation}
				\begin{enumerate}
					\item Execute \hyperref[sec:procedure 21]{procedure 21} on the matrix $A$ and let the tuple $\langle M,B,v,N\rangle$ receive the result.
					\item Execute \hyperref[sec:procedure 38]{procedure 38} on $A$ and let $\langle a\rangle$ receive.
					\item Execute \hyperref[sec:procedure 42]{procedure 42} on $A$ and let $\langle R,E,T\rangle$ receive.
					\item For $j=1$ to $j=\lvert E\rvert$:
					\begin{enumerate}
						\item Verify that $E_j=\rcan(\mon(B_{a-1+j,a-1+j}))$.
						\item Verify that $\last_A=B_{m,m}=B_{a-1+j,a-1+j}v_{a+j}v_{a+j+1}\cdots v_m$.
						\item Let $k=\deg(\mon(B_{a-1+j,a-1+j}))$.
						\item Therefore using \hyperref[sec:procedure 46]{procedure 46} verify that $\last_A(E_j)=B_{m,m}(E_j)=B_{a-1+j,a-1+j}(\rcan(\mon(B_{a-1+j,a-1+j})))\cdot v_{a+j}(E_j)v_{a+j+1}(E_j)\cdots v_m(E_j)=0_{k\times k}v_{a+j}(E_j)v_{a+j+1}(E_j)\cdots v_m(E_j)=0_{k\times k}$.
					\end{enumerate}
					\item \textbf{Therefore using \hyperref[sec:procedure 45]{procedure 45} verify that $\last_A(A)=R\bdiag(\last_A(E))T=R\bdiag(B_{m,m}(E))T=R0_{m\times m}T=0_{m\times m}$.}
				\end{enumerate}
		\subsection{Procedure 49}\label{sec:procedure 49}
			\subsubsection{Objective}
				Choose a monic $\mathbb{Q}[x]$ $p$ such that $\deg(p)>0$. Choose a $\mathbb{Q}[x]$ $g=g_0x^k+g_1x^{k-1}+\cdots+g_kx^0$ such that $g_0\ne 0$ and $k<\deg(p)$. The objective of the following instructions is to show that $g(\rcan(p))\ne 0_{\deg(p)\times \deg(p)}$.
			\subsubsection{Implementation}
				\begin{enumerate}
					\item Let $G=\rcan(p)$.
					\item Therefore cognizant of $G$'s construction, verify that $g(G)e_1=(g_0G^k+g_1G^{k-1}+\cdots+g_kG^0)e_1=g_0e_{k+1}+g_1e_k+\cdots+g_we_1\ne 0_{\deg(p)\times 1}$.
					\item \textbf{Therefore verify that $g(G)\ne 0_{\deg(p)\times \deg(p)}$.}
				\end{enumerate}
		\subsection{Procedure 50}\label{sec:procedure 50}
			\subsubsection{Objective}
				Choose two $\mathbb{Q}[x]$s $g=g_0x^k+g_1x^{k-1}+\cdots+g_kx^0$, $p=x^k+p_1x^{k-1}+p_2x^{k-2}+\cdots+p_kx^0$ such that $\deg(p)=\deg(g)>0$ and $g(\rcan(p))=0_{\deg(p)\times\deg(p)}$. The objective of the following instructions is to show that $g=g_0p$.
			\subsubsection{Implementation}
				\begin{enumerate}
					\item Let $G=\rcan(p)$.
					\item Let $u=\deg(g)$.
					\item Cognizant of $G$'s construction, verify that $0_{u\times 1}=g(G)e_1=(g_0G^u+g_1G^{u-1}+g_2G^{u-2}+\cdots+g_uG^0)e_1=g_0Ge_u+g_1e_u+g_2e_{u-1}+\cdots+g_ue_1$.
					\item Therefore for $i=1$ to $i=u$, do the following:
					\begin{enumerate}
						\item Verify that $0=(g_0Ge_u+g_1e_u+g_2e_{u-1}+\cdots+g_ue_1)_{i,1}$.
						\item Therefore cognizant of $G$'s construction, verify that $-g_0p_{u+1-i}+g_{u+1-i}=0$.
						\item Therefore verify that $g_{u+1-i}=g_0p_{u+1-i}$.
					\end{enumerate}
					\item \textbf{Therefore verify that $g=g_0p$.}
				\end{enumerate}
		\subsection{Procedure 51}\label{sec:procedure 51}
			\subsubsection{Objective}
				Choose a $\mathcal{M}_{m,m}(\mathbb{Q})$, $A$. Choose a $\mathbb{Q}[x]$ $p=p_0x^t+p_1x^{t-1}+p_2x^{t-2}+\cdots+p_tx^0$ where $p_0\ne 0$, such that $p(A)=0_{m\times m}$. The objective of the following instructions is to either show that $0\ne 0$ or to construct a $\mathbb{Q}[x]$ $f$ such that $p=f\last_A$.
			\subsubsection{Implementation}
				\begin{enumerate}
					\item Let $F$ be a $1\times 2$ matrix consisting in-order of $p$ and $\last_A$.
					\item Execute \hyperref[sec:procedure 21]{procedure 21} on $F$ and let $\langle M,D,,N\rangle$ receive the result.
					\item Verify that $D_{1,1}\ne 0$.
					\item Let $g=g_0x^w+g_1x^{w-1}+g_2x^{w-2}+\cdots+g_wx^0=D_{1,1}$ in such a way that $g_0\ne 0$.
					\item Verify that $F=M^{-1}DN^{-1}=DN^{-1}$.
					\item Verify that $\last_A=F_{1,2}=D_{1,1}{N^{-1}}_{1,2}+D_{1,2}{N^{-1}}_{2,2}=D_{1,1}{N^{-1}}_{1,2}=g{N^{-1}}_{1,2}$.
					\item Let $u=\last_A$.
					\item Therefore verify that ${N^{-1}}_{1,2}\ne 0$.
					\item Therefore verify that $u=\deg(\last_A)=\deg(D_{1,1}{N^{-1}}_{1,2})\ge\deg(D_{1,1})=\deg(g)=w$.
					\item Verify that $D=MFN=FN$.
					\item Therefore verify that $g=D_{1,1}=N_{1,1}p+N_{2,1}\last_A$.
					\item Therefore using \hyperref[sec:procedure 46]{procedure 46}, verify that $g(A)=N_{1,1}(A)p(A)+N_{2,1}(A)\last_A(A)=N_{1,1}(A)0_{m\times m}+N_{2,1}(A)0_{m\times m}=0_{m\times m}$.
					\item Execute \hyperref[sec:procedure 42]{procedure 42} on the matrix $A$ and let the tuple $\langle R_1,E,R_3\rangle$ receive the result.
					\item Using \hyperref[sec:procedure 45]{procedure 45}, and \hyperref[sec:procedure 42]{procedure 42}, verify that $\bdiag(g(E))=I_m\bdiag(g(E))I_m=R_3R_1\bdiag(g(E))R_3R_1=R_3g(A)R_1=R_30_{m\times m}R_1=0_{m\times m}$.
					\item Let $G=\rcan(\mon(\last_A))$.
					\item Verify that $g(G)=g(E_{\lvert E\rvert})=\bdiag(g(E))_{[m-u+1:m+1],[m-u+1:m+1]}=0_{u\times u}$.
					\item If $w<u$, then:
					\begin{enumerate}
						\item Using \hyperref[sec:procedure 49]{procedure 49}, verify that $g(G)\ne 0_{u\times u}$.
						\item \textbf{Abort procedure.}
					\end{enumerate}
					\item Otherwise, do the following:
					\begin{enumerate}
						\item Verify that $w=u$.
						\item Using \hyperref[sec:procedure 50]{procedure 50}, verify that $g=g_0\last_A$.
						\item \textbf{Therefore verify that $p=F_{1,1}=D_{1,1}{N^{-1}}_{1,1}+D_{1,2}{N^{-1}}_{2,1}={N^{-1}}_{1,1}g+{N^{-1}}_{2,1}*0={N^{-1}}_{1,1}g={N^{-1}}_{1,1}g_0\last_A$.}
						\item \textbf{Yield the tuple $\langle {N^{-1}}_{1,1}g_0\rangle$.}
					\end{enumerate}
				\end{enumerate}
		\subsection{Procedure 52 (Difference of powers)}\label{sec:procedure 52}
			\subsubsection{Objective}
				Choose an integer $n\ge 0$ and a $\mathbb{Q}[x]$ $p=p_0x^n+p_1x^{n-1}+\cdots+p_n$. Let $y,z$ be indeterminates. The objective of the following instructions is to construct a $\mathbb{Q}[y,z]$ $G$ such that $p(z)-p(y)=(z-y)G(y,z)$.
			\subsubsection{Implementation}
				\begin{enumerate}
					\item Let the $\mathbb{Q}[y,z]$ $G=\sum_{r=1}^n p_{n-r}(z^{r-1}+z^{r-2}y+\cdots+zy^{r-2}+y^{r-1})$.
					\item Verify that $p(z)-p(y)$
					\begin{enumerate}
						\item $=(p_0z^n+p_1z^{n-1}+\cdots+p_n)-(p_0y^n+p_1y^{n-1}+\cdots+p_n)$
						\item $=(\sum_{r=0}^n p_{n-r}z^r)-(\sum_{r=0}^n p_{n-r}y^r)$
						\item $=\sum_{r=1}^n p_{n-r}(z^r-y^r)$
						\item $=\sum_{r=1}^n p_{n-r}(z-y)(z^{r-1}+z^{r-2}y+\cdots+zy^{r-2}+y^{r-1})$
						\item $=(z-y)\sum_{r=1}^n p_{n-r}(z^{r-1}+z^{r-2}y+\cdots+zy^{r-2}+y^{r-1})$
						\item $=(z-y)G(y,z)$.
					\end{enumerate}
					\item \textbf{Yield the tuple $\langle G\rangle$.}
				\end{enumerate}
		\subsection{Procedure 53}\label{sec:procedure 53}
			\subsubsection{Objective}
				Choose a $\mathbb{Q}[x]$ $p=x^n+p_1x^{n-1}+\cdots+p_n$ and $\mathbb{Q}$s $a_1<a_2<\cdots<a_n<a_{n+1}$ in such a way that for $i=1$ to $i=n+1$, $p(a_i)=0$. The objective of the following instructions is to show that $0\ne 0$.
			\subsubsection{Implementation}
				\begin{enumerate}
					\item Write $p$ as $1*p$, so that it has two factors.
					\item For $i=1$ up to $i=n$, do the following:
					\begin{enumerate}
						\item Let $g$ be the rightmost factor of $p$.
						\item If $g(a_i)\ne 0$, do the following:
						\begin{enumerate}
							\item For $k=1$ to $k=i-1$, verify that $(a_i-a_k)\ne 0$.
							\item Verify that $p(a_i)\ne 0$.
							\item Therefore verify that $0\ne 0$.
							\item \textbf{Abort procedure.}
						\end{enumerate}
						\item Otherwise $g(a_i)=0$. Now do the following:
						\begin{enumerate}
							\item Execute \hyperref[sec:procedure 52]{procedure 52} on $g$ and let the tuple $\langle G\rangle$ receive the result.
							\item Let $x$ be an indeterminate.
							\item Let the $\mathbb{Q}[x]$ $q=q(x)=G(a_i,x)$.
							\item Verify that the $\mathbb{Q}[x]$ $g=g(x)=g(x)-g(a_i)=(x-a_i)G(a_i,x)=(x-a_i)q(x)=(x-a_i)q$.
							\item Verify that $p=(x-a_1)(x-a_2)\cdots(x-a_i)q$.
						\end{enumerate}
					\end{enumerate}
					\item Now verify that $p=(x-a_1)(x-a_2)\cdots(x-a_n)1$.
					\item Using (3), verify that $p(a_{n+1})\ne 0$.
					\item Therefore verify that $0\ne 0$.
					\item \textbf{Abort procedure.}
				\end{enumerate}
		\subsection{Procedure 54 (Bisection)}\label{sec:procedure 54}
			\subsubsection{Objective}
				Choose a $\mathbb{Q}[x]$ $f$. Choose $\mathbb{Q}$s $a<b$ such that $\sgn(f(a))=-\sgn(f(b))$. Choose a rational number target $B>0$. The objective of the following instructions is to construct a $\mathbb{Q}$ $d$ such that $a\le d\le b$ and $\lvert f(d)\rvert<B$.
			\subsubsection{Implementation}
				\begin{enumerate}
					\item Execute \hyperref[sec:procedure 52]{procedure 52} on $f$ and let the tuple $\langle G\rangle$ receive the result.
					\item Let $x,y$ be indeterminates.
					\item Verify that the $\mathbb{Q}[x,y]$ $f(y)-f(x)=(y-x)G(x,y)$.
					\item Let $c=a$ and $d=b$.
					\item Until $\lvert d-c\rvert \lvert G\rvert(\lvert a\rvert,\lvert b\rvert)<B$
					\begin{enumerate}
						\item Let $e=\frac{c+d}{2}$.
						\item If $\sgn(f(c))=-\sgn(f(e))$, then:
						\begin{enumerate}
							\item Let $d=e$.
						\end{enumerate}
						\item Otherwise if $\sgn(f(e))=-\sgn(f(d))$, then:
						\begin{enumerate}
							\item Let $c=e$.
						\end{enumerate}
						\item Otherwise if $f(e)=0$, then do the following:
						\begin{enumerate}
							\item \textbf{Verify that $\lvert f(e)\rvert=0<B$.}
							\item Yield the tuple $\langle e\rangle$.
						\end{enumerate}
					\end{enumerate}
					\item \textbf{Verify that $\lvert f(c)\rvert,\lvert f(d)\rvert<\lvert f(d)-f(c)\rvert=\lvert(d-c)G(c,d)\rvert=\lvert d-c\rvert\lvert G(c,d)\rvert\le\lvert d-c\rvert\lvert G\rvert(\lvert c\rvert,\lvert d\rvert)\le\lvert d-c\rvert\lvert G\rvert(\lvert a\rvert,\lvert b\rvert)<B$.}
					\item \textbf{Yield the tuple $\langle c\rangle$.}
				\end{enumerate}
		\subsection{Procedure 55}\label{sec:procedure 55}
			\subsubsection{Objective}
				Choose a $\mathbb{Q}[x]$ $f=x^n+p_1x^{n-1}+\cdots+p_n$ and pairs of $\mathbb{Q}$s $(a_n,b_n),(a_{n-1},b_{n-1}),\cdots,(a_0,b_0)$ in such a way that:
				\begin{enumerate}
					\item $a_n<b_n\le a_{n-1}<b_{n-1}\le\cdots\le a_1<b_1\le a_0<b_0$.
					\item $\sgn(f(a_i))=-\sgn(f(b_i))$ for $i=0$ to $i=n$.
				\end{enumerate}
				The objective of the following instructions is to show that $1=-1$.
			\subsubsection{Implementation}
				\begin{enumerate}
					\item If $n>0$:
					\begin{enumerate}
						\item Let $B=\min_{k=0}^{n-1}\min(\lvert f(a_k)\rvert,\lvert f(b_k)\rvert)$.
						\item For $k=0$ to $k=n-1$, verify that $\lvert f(a_k)\rvert\ge B$.
						\item Execute \hyperref[sec:procedure 54]{procedure 54} on the formal polynomial $f$, interval $(a_n, b_n)$, and target of $B$. Let the tuple $\langle d\rangle$ receive the result.
						\item Verify that $\lvert f(d)\rvert<B$.
						\item Execute \hyperref[sec:procedure 52]{procedure 52} on the formal polynomial $f$ and let the tuple $\langle G\rangle$ receive the result.
						\item Let $x$ be an indeterminate.
						\item Let the formal polynomial $h=G(d,x)$.
						\item Verify that $h$ is a monic $(n-1)^{th}$ degree formal polynomial.
						\item Verify that the formal polynomial $f=f(x)=f(x)-f(d)+f(d)=(x-d)G(d,x)+f(d)=(x-d)h(x)+f(d)=(x-d)h+f(d)$.
						\item For $k=0$ to $k=n-1$, do the following:
						\begin{enumerate}
							\item If $f(a_k)\ge B$, in-order verify that:
							\begin{enumerate}
								\item $f(a_k)\ge B>\lvert f(d)\rvert\ge f(d)$.
								\item $f(a_k)-f(d)>0$.
								\item $(a_k-d)h(a_k)>0$.
								\item \textbf{$h(a_k)>0$.}
								\item $f(b_k)\le-B<-\lvert f(d)\rvert\le f(d)$.
								\item $f(b_k)-f(d)<0$.
								\item $(b_k-d)h(b_k)<0$.
								\item \textbf{$h(b_k)<0$.}
							\end{enumerate}
							\item Otherwise, if $f(a_k)\le -B$, do the following:
							\begin{enumerate}
								\item \textbf{Using steps analogous to (ji), verify that $h(a_k)<0$.}
								\item \textbf{Using steps analogous to (ji), verify that $h(b_k)>0$.}
							\end{enumerate}
						\end{enumerate}
						\item Execute \hyperref[sec:procedure 55]{procedure 55} on $h$ and $a_{n-1}<b_{n-1}\le a_{n-2}<b_{n-2}\le\cdots\le a_1<b_1\le a_0<b_0$.
					\end{enumerate}
					\item Otherwise, do the following:
					\begin{enumerate}
						\item Verify that $n=0$.
						\item Therefore verify that $h=1$.
						\item \textbf{Therefore verify that $1=\sgn(1)=\sgn(f_0(a_0))=-\sgn(f_0(b_0))=-\sgn(1)=-1$.}
						\item \textbf{Abort procedure.}
					\end{enumerate}
				\end{enumerate}
		\subsection{Procedure 56 (Sturm's procedure initialization)}\label{sec:procedure 56}
			\subsubsection{Objective}
				Choose two lists of $\mathbb{Q}[x]$s $s,q$ in such a way that, letting $m=\lvert s\rvert-1$,
				\begin{enumerate}
					\item For $i=0$ to $i=m$, $\deg(s_i)=i$.
					\item For $i=0$ to $i=m$, $\sgn(x^i\circ s_i)=\sgn(x^m\circ s_m)$.
					\item For $i=1$ to $i=m-1$, $s_{i-1}+s_{i+1}=q_is_i$.
				\end{enumerate}
				Let $x,y$ be indeterminates. The objective of the following instructions is to construct lists of $\mathbb{Q}[x]$s $g,h$ such that $g_is_{i+1}+h_is_i=1$ for $i=1$ to $i=m-1$.
			\subsubsection{Implementation}
				\begin{enumerate}
					\item Let $g=h=\langle\rangle$.
					\item If $m>2$, do the following:
					\begin{enumerate}
						\item Verify that $q_{m-1}s_{m-1}-s_{m}=s_{m-2}$.
						\item Execute \hyperref[sec:procedure 56]{procedure 56} on $s_{[0:m]}$ and $q_{[1:m-1]}$ and let the tuple $\langle,,g,h\rangle$ receive.
						\item Verify that $g_{m-2}s_{m-1}+h_{m-2}s_{m-2}=1$.
						\item Let $g_{m-1}=-h_{m-2}$.
						\item Let $h_{m-1}=g_{m-2}+h_{m-2}q_{m-1}$.
						\item \textbf{Therefore verify that $g_{m-1}s_{m}+h_{m-1}s_{m-1}=g_{m-2}s_{m-1}+h_{m-2}(q_{m-1}s_{m-1}-s_{m})=g_{m-2}s_{m-1}+h_{m-2}s_{m-2}=1$.}
					\end{enumerate}
					\item Otherwise, if $m=2$ do the following:
					\begin{enumerate}
						\item Verify that $s_{0}+s_{2}=q_1s_1$.
						\item Let $g_1=-\frac{1}{s_0}$.
						\item Let $h_1=\frac{q_1}{s_0}$.
						\item \textbf{Therefore verify that $g_1s_2+h_1s_1=1$.}
					\end{enumerate}
					\item \textbf{Yield the tuple $\langle s,q,g,h\rangle$.}
				\end{enumerate}
		Let us use the notation $J_{s}(x)$ as a shorthand for "the number of sign changes in the list $s_0(x),s_1(x),\cdots,s_{\lvert s\rvert-1}(x)$".
		\subsection{Procedure 57 (Change in number of sign changes verification)}\label{sec:procedure 57}
			\subsubsection{Objective}
				Execute \hyperref[sec:procedure 56]{procedure 56} and let $\langle s,q,g,h\rangle$ receive. Execute \hyperref[sec:procedure 52]{procedure 52} on $s$ and let $\langle G\rangle$ receive the result. Choose $\mathbb{Q}$s $c$ and $d$ in such a way that:
				\begin{enumerate}
					\item $J_m(c)$ and $J_m(d)$ are well defined.
					\item Letting $B=\max_{i=1}^m \lvert G_i(c,d)\rvert$.
					\item Letting $C=\max_{i=1}^{m-1}\max(\lvert g_i(c)\rvert,\lvert h_i(c)\rvert,\lvert g_i(d)\rvert,\lvert h_i(d))\rvert$.
					\item Letting $D=\max_{i=1}^{m-1}\max(\lvert q_i(c)\rvert,\lvert q_i(d)\rvert,2)$.
					\item $\lvert d-c\rvert\le\frac{1}{BCD}$.
				\end{enumerate}
				The objective of the following instructions is to show that either $0<0$ or $\lvert J_m(d)-J_m(c)\rvert=[\sgn(s_m(c))\ne\sgn(s_m(d))]$.
			\subsubsection{Implementation}
				\begin{enumerate}
					\item Let $i=0$.
					\item Do the following:
					\begin{enumerate}
						\item Verify that $\sgn(s_i(c))=\sgn(s_i(d))$.
						\item Verify that $J_i(c)=J_i(d)$.
						\item If $\sgn(s_{i+1}(c))=\sgn(s_{i+1}(d))$, do the following:
						\begin{enumerate}
							\item Verify that $J_{i+1}(c)=J_{i+1}(d)$.
							\item Set $i$ to $i+1$ and go to (2) if the new $i<m$.
						\end{enumerate}
						\item Otherwise, if $\sgn(s_{i+1}(c))\ne\sgn(s_{i+1}(d))$ and $i+2\le m$, do the following:
						\begin{enumerate}
							\item Execute \hyperref[sec:procedure 57 auxilliary procedure]{procedure 57 auxilliary procedure} on $i$.
							\item If $\sgn(s_{i+2}(c))\ne\sgn(s_{i+2}(d))$, do the following:
							\begin{enumerate}
								\item Verify that $\lvert s_{i+2}(c)\rvert<\lvert s_{i+2}(d)-s_{i+2}(c)\rvert=\lvert (d-c)G_{i+2}(c,d)\rvert\le\frac{1}{BCD}\cdot B=\frac{1}{CD}=\frac{1}{C}\cdot\frac{1}{D}\le\frac{1}{C}(1-\frac{1}{D})$.
								\item Using (A) and (i), verify that $\frac{1}{C}(1-\frac{1}{D})<\lvert s_{i+2}(c)\rvert<\frac{1}{C}(1-\frac{1}{D})$.
								\item \textbf{Abort procedure.}
							\end{enumerate}
							\item Otherwise if $\sgn(s_i(c))=\sgn(s_{i+2}(c))$, do the following:
							\begin{enumerate}
								\item Verify that $2\frac{1}{C}(1-\frac{1}{D})<\lvert s_i(c)\rvert+\lvert s_{i+2}(c)\rvert=\lvert s_i(c)+s_{i+2}(c)\rvert=\lvert q_{i+1}(c)s_{i+1}(c)\rvert<D\frac{1}{CD}$.
								\item Verify that $2(1-\frac{1}{D})<1$.
								\item Using (B) and the construction of $D$, verify that $2\le D<2$.
								\item \textbf{Abort procedure.}
							\end{enumerate}
							\item Otherwise, do the following:
							\begin{enumerate}
								\item Verify that $\sgn(s_i(d))=\sgn(s_i(c))\ne\sgn(s_{i+2}(c))=\sgn(s_{i+2}(d))$.
								\item Therefore verify that $1=J_{i+2}(c)-J_{i}(c)=J_{i+2}(d)-J_{i}(d)$.
								\item Therefore verify that $J_{i}(c)+1=J_{i+2}(c)=J_{i+2}(d)=J_{i}(d)+1$.
								\item Set $i$ to $i+2$ and go to (2).
							\end{enumerate}
						\end{enumerate}
						\item Otherwise, verify the following:
						\begin{enumerate}
							\item $\sgn(s_{i+1}(c))\ne\sgn(s_{i+1}(d))$.
							\item $\lvert J_{i+1}(c)-J_{i+1}(d)\rvert=1$.
							\item $i+1=m$.
						\end{enumerate}
					\end{enumerate}
					\textbf{
						\item If $\sgn(s_m(c))=\sgn(s_m(d))$, then do the following:
						\begin{enumerate}
							\item Verify that $J_m(c)=J_m(d)$.
						\end{enumerate}
						\item Otherwise do the following:
						\begin{enumerate}
							\item Verify that $\lvert J_m(d)-J_m(c)\rvert=1$.
						\end{enumerate}
					}
				\end{enumerate}
			\subsubsection{Auxilliary Procedure}\label{sec:procedure 57 auxilliary procedure}
				\paragraph{Objective}
					Choose a non-negative integer $i<m$ such that $\sgn(s_{i+1}(c))\ne\sgn(s_{i+1}(d))$ and $i+2\le m$. The objective of the following instructions is to show that $\lvert s_{i+1}(c)\rvert<\frac{1}{CD}$, $\lvert s_{i+1}(d)\rvert<\frac{1}{CD}$, $\frac{1}{C}(1-\frac{1}{D})<\lvert s_{i}(c)\rvert$, $\frac{1}{C}(1-\frac{1}{D})<\lvert s_{i}(d)\rvert$, $\frac{1}{C}(1-\frac{1}{D})<\lvert s_{i+2}(c)\rvert$, and $\frac{1}{C}(1-\frac{1}{D})<\lvert s_{i+2}(d)\rvert$.
				\paragraph{Implementation}
					\begin{enumerate}
						\item Verify the following in order:
						\begin{enumerate}
							\item $\lvert s_{i+1}(c)\rvert<\lvert s_{i+1}(c)-s_{i+1}(d)\rvert=\lvert c-d\rvert\lvert G_{i+1}(c,d)\rvert\le\lvert c-d\rvert B\le lB=\frac{1}{CD}$
							\item $\lvert s_{i+1}(d)\rvert<\lvert s_{i+1}(c)-s_{i+1}(d)\rvert\le\frac{1}{CD}$
							\item $1=g_{i}(c)s_{i+1}(c)+h_{i}(c)s_{i}(c)=\lvert g_{i}(c)s_{i+1}(c)+h_{i}(c)s_{i}(c)\rvert\le\lvert g_{i}(c)\rvert\lvert s_{i+1}(c)\rvert+\lvert h_{i}(c)\rvert\lvert s_{i}(c)\rvert<C(\frac{1}{CD}+\lvert s_{i}(c)\rvert)$
							\item $\frac{1}{C}(1-\frac{1}{D})<\lvert s_{i}(c)\rvert$
							\item $1<C(\frac{1}{CD}+\lvert s_{i}(d)\rvert)$
							\item $\frac{1}{C}(1-\frac{1}{D})<\lvert s_{i}(d)\rvert$
							\item $1=g_{i+1}(c)s_{i+2}(c)+h_{i+1}(c)s_{i+1}(c)=\lvert g_{i+1}(c)s_{i+2}(c)+h_{i+1}(c)s_{i+1}(c)\rvert\le\lvert g_{i+1}(c)\rvert\lvert s_{i+2}(c)\rvert+\lvert h_{i+1}(c)\rvert\lvert s_{i+1}(c)\rvert<C(\lvert s_{i+2}(c)\rvert+\frac{1}{CD})$
							\item $\frac{1}{C}(1-\frac{1}{D})<\lvert s_{i+2}(c)\rvert$
							\item $1<C(\lvert s_{i+2}(d)\rvert+\frac{1}{CD})$
							\item $\frac{1}{C}(1-\frac{1}{D})<\lvert s_{i+2}(d)\rvert$
						\end{enumerate}
					\end{enumerate}
		\subsection{Procedure 58 (Cauchy's positive verification)}\label{sec:procedure 58}
			\subsubsection{Objective}
				Choose a $\mathbb{Q}[x]$ $p=p_0x^t+p_1x^{t-1}+p_2x^{t-2}+\cdots+p_tx^0$, where $p_0\ne 0$. Choose a $\mathbb{Q}$ $k>1+\max_{i=1}^t\lvert\frac{p_i}{p_0}\rvert$. The objective of the following instructions is to show that $\sgn(p(k))=\sgn(p_0)$.
			\subsubsection{Implementation}
				\begin{enumerate}
					\item In reverse order verify the following:
					\begin{enumerate}
						\item \textbf{$\sgn(p_0k^n+p_1k^{n-1}+\cdots+p_nk^0)=\sgn(p_0)$}
						\item $\sgn(k^n+\frac{p_1}{p_0}k^{n-1}+\cdots+\frac{p_n}{p_0}k^0)=1$
						\item $k^n+\frac{p_1}{p_0}k^{n-1}+\cdots+\frac{p_n}{p_0}k^0>0$
						\item $k^n>-(\frac{p_1}{p_0}k^{n-1}+\cdots+\frac{p_n}{p_0}k^0)$
						\item $k^n>\lvert \frac{p_1}{p_0}k^{n-1}+\cdots+\frac{p_n}{p_0}k^0\rvert$
						\item $k^n>\lvert\max_{i=1}^t\lvert \frac{p_i}{p_0}\rvert(k^{n-1}+\cdots+k^0)\rvert$
						\item $k^n>\max_{i=1}^t\lvert \frac{p_i}{p_0}\rvert\frac{k^n-1}{k-1}$
						\item $k^{n+1}-k^n>\max_{i=1}^t\lvert \frac{p_i}{p_0}\rvert(k^n-1)$
						\item $k^{n+1}-(1+\max_{i=1}^t\lvert \frac{p_i}{p_0}\rvert)k^n+\max_{i=1}^t\lvert \frac{p_i}{p_0}\rvert>0$
						\item $k>1+\max_{i=1}^t\lvert \frac{p_i}{p_0}\rvert$
					\end{enumerate}
				\end{enumerate}
		\subsection{Procedure 59 (Cauchy's alternation verification)}\label{sec:procedure 59}
			\subsubsection{Objective}
				Choose a $\mathbb{Q}[x]$ $p=p_0x^t+p_1x^{t-1}+p_2x^{t-2}+\cdots+p_tx^0$, where $p_0\ne 0$. Choose a $\mathbb{Q}$ $k<-(1+\max_{i=1}^t\lvert\frac{p_i}{p_0}\rvert)$. The objective of the following instructions is to show that $\sgn(p(k))=(-1)^t\sgn(p_0)$.
			\subsubsection{Implementation}
				\begin{enumerate}
					\item Let $q=q_0x^t+q_1x^{t-1}+q_2x^{t-2}+\cdots+q_tx^0$, where $q_i=(-1)^ip_i$.
					\item Verify that $k<-(1+\max_{i=1}^t\lvert\frac{q_i}{q_0}\rvert)$.
					\item Therefore verify that $-k>1+\max_{i=1}^t\lvert\frac{q_i}{q_0}\rvert$.
					\item Execute \hyperref[sec:procedure 58]{procedure 58} on $q$ and $-k$.
					\item Hence verify that $(-1)^t\sgn(p(k))$
					\begin{enumerate}
						\item $=\sgn((-1)^tp(k))$
						\item $=\sgn((-1)^t\sum_{i=0}^t p_ik^{t-i})$
						\item $=\sgn(\sum_{i=0}^t (-1)^i(-1)^{t-i}p_ik^{t-i})$
						\item $=\sgn(\sum_{i=0}^t q_i(-k)^{t-i})$
						\item $=\sgn(q(-k))$
						\item $=\sgn(q_0)$
						\item $=\sgn(p_0)$.
					\end{enumerate}
					\item \textbf{Therefore verify that $\sgn(p(k))=(-1)^t(-1)^t\sgn(p(k))=(-1)^t\sgn(p_0)$.}
				\end{enumerate}
		\subsection{Procedure 60 (Range subdivision)}\label{sec:procedure 60}
			\subsubsection{Objective}
				Choose a list of $\mathbb{Q}[x]$s, $s$, and $\mathbb{Q}$s $a,l,c$ such that $a<c$ and $l>0$. The objective of the following instructions is to either show that $0<0$ or to construct a list of $\mathbb{Q}$s, $b$, such that $a=b_1<b_2<\cdots<b_{\lvert b\rvert}=c$, $b_{i+1}-b_i\le l$ for $i=1$ to $i=\lvert b\rvert-1$, and $J_s(b)$ is defined for $i=1$ to $i=\lvert b\rvert-1$.
			\subsubsection{Implementation}
				\begin{enumerate}
					\item Let $e=\langle\langle\rangle,\langle\rangle,\cdots,\langle\rangle\rangle$.
					\item Let $f=\sum_{r=1}^{\lvert s\rvert}\deg(s_r)$.
					\item Let $b=\langle a\rangle$.
					\item Let $d=b_1$.
					\item While $d+l<c$, do the following:
					\begin{enumerate}
						\item Let $m=l$.
						\item While $J_s(d+m)$ is not defined and $\lvert e\rvert\le f$, do the following:
						\begin{enumerate}
							\item Let $1\le i\le\lvert s\rvert$ be an integer such that $s_i(d+m)=0$.
							\item Append $d+m$ onto $e_i$.
							\item Set $m=\frac{m}{2}$
						\end{enumerate}
						\item If $\sum\lvert e\rvert>f$, then do the following:
						\begin{enumerate}
							\item If $\lvert e_i\rvert\le\deg(s_i)$ for $1\le i\le\lvert s\rvert$, then do the following:
							\begin{enumerate}
								\item Verify that $\sum\lvert e\rvert\le f$.
								\item Therefore using (c), verify that $\sum\lvert e\rvert\le f<\sum\lvert e\rvert$.
								\item \textbf{Abort procedure.}
							\end{enumerate}
							\item Otherwise, do the following:
							\begin{enumerate}
								\item Let $1\le i\le\lvert s\rvert$ be an integer such that $\lvert e_i\rvert>\deg(s_i)$.
								\item Execute \hyperref[sec:procedure 53]{procedure 53} on $s_i$ and a sorted $e_i$.
								\item \textbf{Abort procedure.}
							\end{enumerate}
						\end{enumerate}
						\item Otherwise, do the following:
						\begin{enumerate}
							\item \textbf{Verify that $J_s(d+m)$ is defined.}
							\item Append $d+m$ onto $b$.
							\item \textbf{Verify that $0<b_{\lvert b\rvert}-b_{\lvert b\rvert-1}=m\le l$.}
							\item Set $d$ to $d+m$.
							\item Using (5), verify that $d<c$.
						\end{enumerate}
					\end{enumerate}
					\item Verify that $d<c$.
					\item Verify that $d+l\ge c$.
					\item \textbf{Therefore verify that $0<c-d\le l$.}
					\item Append $c$ onto $b$.
					\item \textbf{Yield $\langle b\rangle$.}
				\end{enumerate}
		\subsection{Procedure 61 (Sturm's sign change)}\label{sec:procedure 61}
			\subsubsection{Objective}
				Execute \hyperref[sec:procedure 56]{procedure 56} and let $\langle s,q,g,h\rangle$ receive. Let $m=\lvert s\rvert-1$. The objective of the following instructions is to either show that $0<0$ or to construct two lists of rational numbers $c,d$ such that $c_1<d_1\le c_2<d_2\le\cdots\le c_m<d_m$ and $\sgn(s_m(c_i))=-\sgn(s_m(d_i))$ for $i=1$ to $i=m$.
			\subsubsection{Implementation}
				\begin{enumerate}
					\item Let $U=1+\max_{i=0}^m\left(1+\max_{j=1}^i\lvert\frac{x^{i-j}\circ s_i}{x^i\circ s_i}\rvert\right)$
					\item Using \hyperref[sec:procedure 58]{procedure 58}, verify that $J(U)=0$.
					\item Using \hyperref[sec:procedure 59]{procedure 59}, verify that $J(-U)=m$.
					\item Execute \hyperref[sec:procedure 52]{procedure 52} on $s$ and let $\langle G\rangle$ receive the result.
					\item Let the rational $B=\max_{i=1}^m \lvert G_i\rvert(U,U)$.
					\item Let $C=\max_{i=1}^m \max(\lvert g_i\rvert(U),\lvert h_i\rvert(U))$.
					\item Let $D=\max(3, \max_{i=1}^m \lvert q_i\rvert(U))$
					\item Let $l=\frac{1}{BCD}$.
					\item Execute \hyperref[sec:procedure 60]{procedure 60} on $s$ with endpoints $-U,U$ and a step size of $l$ and let $\langle e\rangle$ receive the result.
					\item Let $c=d=\langle\rangle$.
					\item For $i=1$ to $i=\lvert e\rvert-1$:
					\begin{enumerate}
						\item Execute \hyperref[sec:procedure 57]{procedure 57} on the tuple $\langle e_i,e_{i+1}\rangle$.
						\item If $J_m(c)\ne J_m(d)$, then do the following:
						\begin{enumerate}
							\item Append $e_i$ to $c$.
							\item Append $e_{i+1}$ to $d$.
							\item Cognizant of \hyperref[sec:procedure 57]{procedure 57}, verify that $\lvert J_m(d)-J_m(c)\rvert=1$.
							\item Therefore verify that $\sgn(s_m(c_{\lvert c\rvert}))=-\sgn(s_m(d_{\lvert d\rvert}))$.
							\item Also verify that $d_{\lvert d\rvert-1}\le c_{\lvert c\rvert}<d_{\lvert d\rvert}$.
						\end{enumerate}
					\end{enumerate}
					\item If less than $m$ pairs of rational numbers were recorded, then do the following:
					\begin{enumerate}
						\item Verify that each change of $J_m(x)$ over the course of (12) was by $1$.
						\item Verify that $J_m(x)$ changed less than $m$ times over the course of (12).
						\item Therefore verify that $\lvert J_m(U)-J_m(-U)\rvert<m$.
						\item Therefore using (2) and (3), verify that $m=\lvert J_m(U)-J_m(-U)\rvert<m$.
						\item \textbf{Abort procedure.}
					\end{enumerate}
					\item Otherwise, do the following:
					\begin{enumerate}
						\item Verify that $m\le\lvert c\rvert=\lvert d\rvert$.
						\item \textbf{Yield the tuple $\langle\langle c_1,d_1\rangle,\langle c_2,d_2\rangle,\cdots,\langle c_m,d_m\rangle\rangle$.}
					\end{enumerate}
				\end{enumerate}
		\subsection{Procedure 62}\label{sec:procedure 62}
			\subsubsection{Objective}
				Choose a $\mathcal{M}_{m,m}(\mathbb{Q})$, $A$. The objective of the following instructions is to define the $\mathcal{M}_{m^2,*}(\mathbb{Q})$ $\pows(A)$.
			\subsubsection{Implementation}
				\begin{enumerate}
					\item Let $t=\deg(\last_A)$.
					\item Make an $m^2\times t$ matrix, $\pows(A)$, whose $i^{th}$ column is the sequential concatenation of the columns of $A^{t-i}$.
					\item Yield $\langle\pows(A)\rangle$.
				\end{enumerate}
		\subsection{Procedure 63}\label{sec:procedure 63}
			\subsubsection{Objective}
				Choose an $\mathcal{M}_{m,n}(\mathbb{Q})$, $A$, and an $\mathcal{M}_{n,m}(\mathbb{Q})$, $B$, such that $AB=I_m$. The objective of the following instructions is to show that either $0=1$ or every column of $B$ is non-zero.
			\subsubsection{Implementation}
				\begin{enumerate}
					\item If any column $i$ of $B$, $Be_i$, is equal to zero, then:
					\begin{enumerate}
						\item Verify that $0_{n\times 1}=A0_{n\times 1}=A(Be_i)=(AB)e_i=I_me_i=e_i$.
						\item Therefore verify that 0=1.
						\item \textbf{Abort procedure.}
					\end{enumerate}
				\end{enumerate}
		\subsection{Procedure 64}\label{sec:procedure 64}
			\subsubsection{Objective}
				Choose a $\mathcal{M}_{m,m}(\mathbb{Q})$, $A$. Choose a $\mathbb{Q}[x]$ $p$ such that $p\ne 0$, $p(A)=0$, and $\deg(p)<\deg(\last_A)$. The objective of the following instructions is to show that $0<0$.
			\subsubsection{Implementation}
				\begin{enumerate}
					\item Execute \hyperref[sec:procedure 51]{procedure 51} on $A$ and $p$ and let $f$ receive.
					\item Now verify that $p=f\last_A$.
					\item Verify that $f\ne 0$ and $\last_A\ne 0$.
					\item \textbf{Therefore verify that $\deg(\last_A)>\deg(p)=\deg(f\last_A)\ge\deg(\last_A)$.}
					\item \textbf{Abort procedure.}
				\end{enumerate}
		\subsection{Procedure 65}\label{sec:procedure 65}
			\subsubsection{Objective}
				Choose a $\mathcal{M}_{m,m}(\mathbb{Q})$, $A$. Execute \hyperref[sec:procedure 21]{procedure 21} on $\pows(A)$ and let the tuple $\langle M,D,,N\rangle$ receive the result. Let $t=\cols(\pows(A))$. The objective of the following instructions is to show that either $0<0$ or to show that ${C_t(D)}={C_t(D)}_{1,1}e_1\ne 0$.
			\subsubsection{Implementation}
				\begin{enumerate}
					\item Execute \hyperref[sec:procedure 21]{procedure 21} on $\pows(A)$ and let the tuple $\langle M,D,,N\rangle$ receive the result.
					\item Verify that $M_*\pows(A)N_*=D$.
					\item Using \hyperref[sec:procedure 6]{procedure 6}, verify that $M^{-1}MFN=I_{m^2}FN=FN=M^{-1}D$.
					\item If ${C_t(D)}_{1,1}=0$, then:
					\begin{enumerate}
						\item Verify that for some $1\le i\le t$, $D_{i,i}=0$.
						\item Therefore verify that $De_i=0_{m^2\times 1}$.
						\item Therefore verify that $F(Ne_i)=(FN)e_i=(M^{-1}D)e_i=M^{-1}(De_i)=0_{m^2\times 1}$.
						\item Let $p=N_{1,i}x^{t-1}+N_{2,i}x^{t-2}+\cdots+N_{t,i}x^0$.
						\item Therefore verify that $p(A)=0_{m\times m}$.
						\item Execute \hyperref[sec:procedure 63]{procedure 63} on ${N^{-1}}_*$ and $N_*$.
						\item Therefore verify that $p\ne 0$.
						\item Execute \hyperref[sec:procedure 64]{procedure 64} on $A$ and $p$.
						\item \textbf{Abort procedure.}
					\end{enumerate}
					\item Otherwise, do the following:
					\begin{enumerate}
						\item Execute \hyperref[sec:procedure 19]{procedure 19} on $D,I_t,t$ and let $E$ receive.
						\item Verify that $C_t(D)=C_t(DI_t)=EC_t(I_t)=E*1=E$.
						\item Verify that $E$ is a $\mathcal{D}_{\binom{m^2}{t},\binom{t}{t}}(\mathbb{Q}[x])$.
						\item Therefore verify that $C_t(D)$ is a $\mathcal{D}_{\binom{m^2}{t},1}(\mathbb{Q}[x])$.
						\item \textbf{Therefore verify that $C_t(D)={C_t(D)}_{1,1}e_1\ne 0$.}
					\end{enumerate}
				\end{enumerate}
		\subsection{Procedure 66}\label{sec:procedure 66}
			\subsubsection{Objective}
				Choose a $\mathcal{M}_{m,m}(\mathbb{Q})$, $A$. Let $t=\cols(\pows(A))$. The objective of the following instructions is to show that either $0<0$ or to show that ${C_t(\pows(A))}\ne 0$.
			\subsubsection{Implementation}
				\begin{enumerate}
					\item Execute \hyperref[sec:procedure 21]{procedure 21} on $\pows(A)$ and let the tuple $\langle M,D,,N\rangle$ receive the result.
					\item Verify that $\pows(A)={M^{-1}}_*D{N^{-1}}_*$.
					\item Execute \hyperref[sec:procedure 63]{procedure 63} on $C_t(M_*),C_t({M^{-1}}_*)$.
					\item Verify that all columns of $C_t(M^{-1})$ are non-zero.
					\item Let $t=\cols(\pows(A))$.
					\item Execute \hyperref[sec:procedure 65]{procedure 65} on $A$.
					\item Verify that $C_t(D)={C_t(D)}_{1,1}e_1\ne 0$.
					\item Therefore verify that ${C_t(D)}_{1,1}\ne 0$.
					\item Execute \hyperref[sec:procedure 63]{procedure 63} on $C_t(N_*),C_t({N^{-1}}_*)$.
					\item Verify that $C_t(N^{-1})\ne 0$.
					\item \textbf{Verify that $C_t(\pows(A))=C_t(M^{-1}DN^{-1})=C_t(M^{-1})C_t(D)C_t(N^{-1})=C_t(M^{-1}){C_t(D)}_{1,1}e_1C_t(N^{-1})={C_t(D)}_{1,1}C_t(N^{-1})C_t(M^{-1})e_1\ne 0_{\binom{m^2}{t}\times 1}$.}
				\end{enumerate}
		Let $\mat_t(p)$ be a shorthand for "$(x^{t-1}\circ p)e_1+(x^{t-2}\circ p)e_2+\cdots+(x^0\circ p)e_t$" in what follows.
		
		Let $\pol(P)$ be a shorthand for "$P_{1,1}x^{t-1}+P_{2,1}x^{t-2}+\cdots+P_{t,1}$ where $t=\rows(P)$" in what follows.
		\subsection{Procedure 67}\label{sec:procedure 67}
			\subsubsection{Objective}
				Choose an $\mathcal{M}_{m,m}(\mathbb{Q})$, $A$. The objective of the following instructions is to either show that $0<0$ or to define the $\mathbb{Q}[x]$ $\sel_A$.
			\subsubsection{Implementation}
				\begin{enumerate}
					\item Using \hyperref[sec:procedure 27]{procedure 27} and \hyperref[sec:procedure 66]{procedure 66}, verify that $C_t(\pows(A)^T\pows(A))=C_t(\pows(A)^T)C_t(\pows(A))={C_t(\pows(A))}^TC_t(\pows(A))=\lVert C_t(\pows(A))\rVert^2>0$.
					\item Let $H=(\pows(A)^T\pows(A))\backslash e_1$.
					\item Let $t=\deg(\last_A)$.
					\item Let $\sel_A=\frac{\pol(H)}{x^t\circ\last_A}$.
					\item Yield $\langle\sel_A\rangle$.
				\end{enumerate}
		Let us use the notation $\tr(X)$ as a shorthand for "the sum of the diagonal entries of the square matrix $X$" in what follows.
		
		Let us use the notation "$A$ is symmetric" as a shorthand for "$A^T=A$".
		\subsection{Procedure 68}\label{sec:procedure 68}
			\subsubsection{Objective}
				Choose a symmetric $\mathcal{M}_{m,m}(\mathbb{Q})$, $A$. Let $t=\deg(\last_A)$. Choose two $\mathbb{Q}[x]$s $u=u_1x^{t-1}+u_2x^{t-2}+\cdots+u_tx^0,w=w_1x^{t-1}+w_2x^{t-2}+\cdots+w_tx^0$. The objective of the following instructions is to show that $\tr(u(A)w(A))=\mat(u)^T\pows(A)^T\pows(A)\mat_t(w)$.
			\subsubsection{Implementation}
				\begin{enumerate}
					\item Verify that $\tr(u(A)w(A))$
					\begin{enumerate}
						\item $=\tr((\sum_{p=1}^t u_pA^{t-p})(\sum_{q=1}^t w_qA^{t-q}))$
						\item $=\tr(\sum_{p=1}^t\sum_{q=1}^t u_pw_qA^{t-p}A^{t-q})$
						\item $=\sum_{p=1}^t\sum_{q=1}^t u_pw_q\tr(A^{t-p}A^{t-q})$
						\item $=\sum_{p=1}^t\sum_{q=1}^t u_pw_q\sum_{e=1}^m\sum_{f=1}^m{A^{t-p}}_{e,f}{A^{t-q}}_{f,e}$
						\item $=\sum_{p=1}^t\sum_{q=1}^t u_pw_q\sum_{e=1}^m\sum_{f=1}^m{A^{t-p}}_{f,e}{A^{t-q}}_{f,e}$
						\item $=\sum_{p=1}^t\sum_{q=1}^t u_pw_q\sum_{g=1}^{m^2}{\pows(A)}_{g,p}{\pows(A)}_{g,q}$
						\item $=\sum_{p=1}^t\sum_{q=1}^t u_pw_q(\pows(A)^T\pows(A))_{p,q}$
						\item $=\sum_{p=1}^t u_p(\pows(A)^T\pows(A)\mat_t(w))_{p}$
						\item $=\mat_t(u)^T\pows(A)^T\pows(A)\mat_t(w)$
					\end{enumerate}
				\end{enumerate}
		\subsection{Procedure 69}\label{sec:procedure 69}
			\subsubsection{Objective}
				Choose a symmetric $\mathcal{M}_{m,m}(\mathbb{Q})$, $A$. Let $t=\deg(\last_A)$. Choose a $\mathbb{Q}[x]$ $u$ such that $\deg(u)<t$. The objective of the following instructions is to show that $\tr(u(A)\sel_A(A))=\frac{x^{t-1}\circ u}{x^t\circ\last_A}$.
			\subsubsection{Implementation}
				\begin{enumerate}
					\item Using \hyperref[sec:procedure 68]{procedure 68} and \hyperref[sec:procedure 67]{procedure 67}, verify that $\tr(u(A)\sel_A(A))$
					\begin{enumerate}
						\item $=\mat(u)^T\pows(A)^T\pows(A)\mat_t(\sel_A)$
						\item $=\frac{\mat(u)^T\pows(A)^T\pows(A)((\pows(A)^T\pows(A))\backslash e_1)}{x^t\circ\last_A}$
						\item $=\frac{\mat(u)^Te_1}{x^t\circ\last_A}$
						\item $=\frac{\mat(u)_{1,1}}{x^t\circ\last_A}$
						\item $=\frac{x^{t-1}\circ u}{x^t\circ\last_A}$.
					\end{enumerate}
				\end{enumerate}
		\subsection{Procedure 70}\label{sec:procedure 70}
			\subsubsection{Objective}
				Choose a symmetric $\mathcal{M}_{m,m}(\mathbb{Q})$, $A$. The objective of the following instructions is to either show that $0\ne 0$ or construct $\mathbb{Q}[x]$s $u,v$ such that $u\last_A+v\sel_A=1$.
			\subsubsection{Implementation}
				\begin{enumerate}
					\item Let $t=\deg(\last_A)$.
					\item Let $G$ be a $\mathcal{M}_{1,2}(\mathbb{Q}[x])$ where $G_{1,1}=\last_A$ and $G_{1,2}=\sel_A$.
					\item Execute \hyperref[sec:procedure 21]{procedure 21} on $G$ and let the tuple $\langle M,D,,N\rangle$ receive.
					\item Verify that $G={M^{-1}}_*D{N^{-1}}_*$.
					\item Verify that $\last_A\ne 0$.
					\item Therefore verify that $D_{1,1}\ne 0$.
					\item If $\deg(D_{1,1})>0$, then do the following:
					\begin{enumerate}
						\item Let $b={{N^{-1}}_*}_{1,1}$.
						\item Verify that $\last_A=bD_{1,1}$.
						\item Let $z=\deg(b)$.
						\item Verify that $t=\deg(\last_A)=\deg(bD_{1,1})=\deg(b)+\deg(D_{1,1})>\deg(b)=z$.
						\item Let $c={{N^{-1}}_*}_{1,2}$.
						\item Verify that $\sel_A=cD_{1,1}$.
						\item Let $u=x^{t-z-1}b$.
						\item Execute \hyperref[sec:procedure 69]{procedure 69} on $A$ and $u$.
						\item Hence verify that $\tr(u(A)\sel_A(A))=x^{t-1}\circ u=x^z\circ b\ne 0$.
						\item Also verify that $\tr(u(A)\sel_A(A))=\tr(A^{z-1}b(A)c(A)D_{1,1}(A))=\tr(A^{z-1}c(A)b(A)D_{1,1}(A))=\tr(A^{z-1}c(A)\last_A(A))=\tr(A^{z-1}c(A)0_{m\times m})=\tr(0_{m\times m})=0$.
						\item Therefore verify that $0\ne 0$.
						\item \textbf{Abort procedure.}
					\end{enumerate}
					\item Otherwise, do the following:
					\begin{enumerate}
						\item Verify that $\deg(D_{1,1})=0$.
						\item Let $u=\frac{N_{1,1}}{D_{1,1}}$.
						\item Let $v=\frac{N_{2,1}}{D_{1,1}}$.
						\item \textbf{Verify that $u\last_A+v\sel_A=1$.}
						\item \textbf{Yield the tuple $\langle u,v\rangle$.}
					\end{enumerate}
				\end{enumerate}
		\subsection{Procedure 71 (Euclidean division)}\label{sec:procedure 71}
			\subsubsection{Objective}
				Choose two $\mathbb{Q}[x]$s, $\langle a,b\rangle$. The objective of the following instructions is to construct two $\mathbb{Q}[x]$s $u,w$ such that $a=ub+w$ and $\deg(w)<\deg(b)$.
			\subsubsection{Implementation}
				\begin{enumerate}
					\item If $\deg(a)\ge\deg(b)$:
					\begin{enumerate}
						\item Let $y=\frac{x^{\deg(a)}\circ a}{x^{\deg(b)}\circ b}x^{\deg(a)-\deg(b)}$
						\item Let $e=a-yb$.
						\item Verify that $\deg(e)<\deg(a)$.
						\item Execute \hyperref[sec:procedure 71]{procedure 71} on the tuple $\langle e,b\rangle$. Let the tuple $\langle c,d\rangle$ receive the result.
						\item Verify that $cb+d=e$.
						\item Verify that $\deg(d)<\deg(b)$.
						\item Therefore verify that $cb+d=a-yb$
						\item \textbf{Therefore verify that $(y+c)b+d=a$.}
						\item \textbf{Also verify that $\deg(d)<\deg(b)$.}
						\item \textbf{Now yield the tuple $\langle y+c, d\rangle$.}
					\end{enumerate}
					\item Otherwise:
					\begin{enumerate}
						\item \textbf{Verify that $0*b+a=a$.}
						\item \textbf{Verify that $\deg(a)<\deg(b)$.}
						\item \textbf{Yield the tuple $\langle 0,a\rangle$.}
					\end{enumerate}
				\end{enumerate}
		\subsection{Procedure 72}\label{sec:procedure 72}
			\subsubsection{Objective}
				Choose two lists of $\mathbb{Q}[x]$s $s,q$ and a non-negative integer $k$ in such a way that, letting $m=\lvert s\rvert-1$,
				\begin{enumerate}
					\item $k<m$.
					\item For $k\le i\le m$, $\deg(s_i)=i$.
					\item For $k<i<m$, $s_{i-1}+s_{i+1}=q_is_i$.
				\end{enumerate}
				Let $\deg(0)=-1$. The objective of the following instructions is to construct $\mathbb{Q}[x]$s $g,h$ such that $s_k=gs_{m-1}+hs_m$, $\deg(g)=m-1-k$, and $\deg(h)=m-2-k$.
			\subsubsection{Implementation}
				\begin{enumerate}
					\item If $k<m-2$, do the following:
					\begin{enumerate}
						\item Verify that $s_k+s_{k+2}=q_{k+1}s_{k+1}$.
						\item Therefore verify that $s_k=q_{k+1}s_{k+1}-s_{k+2}$.
						\item Execute \hyperref[sec:procedure 72]{procedure 72} on $s,q,k+1$ and let the tuple $\langle g_1,h_1\rangle$ receive.
						\item Verify that $s_{k+1}=g_1s_{m-1}+h_1s_m$.
						\item Verify that $\deg(g_1)=m-1-(k+1)=m-k-2$.
						\item Verify that $\deg(h_1)=m-2-(k+1)=m-k-3$.
						\item Execute \hyperref[sec:procedure 72]{procedure 72} on $s,q,k+2$ and let the tuple $\langle g_2,h_2\rangle$ receive.
						\item Verify that $s_{k+2}=g_2s_{m-1}+h_2s_m$.
						\item Verify that $\deg(g_2)=m-1-(k+2)=m-k-3$.
						\item Verify that $\deg(h_2)=m-2-(k+2)=m-k-4$.
						\item Let $g=q_{k+1}g_1-g_2$.
						\item \textbf{Verify that $\deg(g)=\max(1+(m-k-2),m-k-3)=m-1-k$.}
						\item Let $h=q_{k+1}h_1-h_2$.
						\item \textbf{Verify that $\deg(h)=\max(1+(m-k-3),m-k-4)=m-2-k$.}
						\item \textbf{Verify that $s_k=q_{k+1}(g_1s_{m-1}+h_1s_m)-(g_2s_{m-1}+h_2s_m)=(q_{k+1}g_1-g_2)s_{m-1}+(q_{k+1}h_1-h_2)s_m=gs_{m-1}+hs_m$.}
					\end{enumerate}
					\item Otherwise, if $k=m-2$ do the following:
					\begin{enumerate}
						\item Verify that $s_{m-2}+s_m=q_{m-1}s_{m-1}$.
						\item Let $g=q_{m-1}$.
						\item \textbf{Verify that $\deg(g)=1=m-1-k$.}
						\item Let $h=-1$.
						\item \textbf{Verify that $\deg(h)=0=m-2-k$.}
						\item \textbf{Therefore verify that $s_k=s_{m-2}=q_{m-1}s_{m-1}-s_m=gs_{m-1}+hs_m$.}
					\end{enumerate}
					\item Otherwise, if $k=m-1$ do the following:
					\begin{enumerate}
						\item Let $g=1$.
						\item \textbf{Verify that $\deg(g)=0=m-1-k$.}
						\item Let $h=0$.
						\item \textbf{Verify that $\deg(h)=-1=m-2-k$.}
						\item \textbf{Verify that $s_k=s_{m-1}=gs_{m-1}+hs_m$.}
					\end{enumerate}
					\item \textbf{Yield the tuple $\langle g,h\rangle$.}
				\end{enumerate}
		\subsection{Procedure 73 (Edwards' Sturm chain construction)}\label{sec:procedure 73}
			\subsubsection{Objective}
				Choose a symmetric $\mathcal{M}_{m,m}(\mathbb{Q})$, $A$. Let $t=\deg(\last_A)$. The objective of the following instructions is to either show that $0\ne 0$ or to construct lists of $\mathbb{Q}[x]$s $s,q$ such that
				\begin{enumerate}
					\item For $i=0$ to $i=t$, $\deg(s_i)=i$.
					\item For $i=0$ to $i=t$, $\sgn(x^i\circ s_i)=\sgn(x^t\circ s_t)$.
					\item For $i=1$ to $i=t-1$, $s_{i-1}+s_{i+1}=q_is_i$.
					\item $s_t=\last_A$.
				\end{enumerate}
			\subsubsection{Implementation}
				\begin{enumerate}
					\item Execute \hyperref[sec:procedure 70]{procedure 70} on $A$ and let $\langle u,s_{t+1}\rangle$ receive the result.
					\item Verify that $us_t+s_{t+1}\sel_A=1$.
					\item Execute \hyperref[sec:procedure 71]{procedure 71} on the tuple $\langle s_{t+1},s_t\rangle$. Let the tuple $\langle q_t,s_{t-1}\rangle$ receive the result.
					\item Verify that $s_{t+1}=q_ts_t+s_{t-1}$, where $\deg(s_{t-1})<\deg(s_t)=t$.
					\item Therefore verify that $us_t+(q_ts_t+s_{t-1})\sel_A=1$.
					\item Therefore verify that $s_{t-1}(A)\sel_A(A)=u(A)s_t(A)+(q_t(A)s_t(A)+s_{t-1}(A))\sel_A(A)=I_{m,m}$.
					\item Therefore using \hyperref[sec:procedure 69]{procedure 69}, verify that $\frac{x^{t-1}\circ s_{t-1}}{x^t\circ s_t}=\tr(s_{t-1}(A)\sel_A(A))=\tr(I_{m,m})=m>0$.
					\item For $i=t-1$ down to $i=1$, do the following:
					\begin{enumerate}
						\item Execute \hyperref[sec:procedure 71]{procedure 71} on the tuple $\langle -s_{i+1},-s_i\rangle$. Let the tuple $\langle q_i,s_{i-1}\rangle$ receive the result.
						\item Verify that $\deg(q_i)=1$.
						\item Verify that $x\circ q_i=\frac{x^{i+1}\circ s_{i+1}}{x^i\circ s_i}$.
						\item Also verify that $-s_{i+1}=-q_is_i+s_{i-1}$.
						\item Therefore verify that $q_is_i=s_{i+1}+s_{i-1}$.
						\item Therefore verify that $q_is_i-s_{i+1}=s_{i-1}$.
						\item Execute \hyperref[sec:procedure 72]{procedure 72} on the tuple $\langle s,q,i-1\rangle$ and let $\langle p,j\rangle$ receive.
						\item Verify that $s_{i-1}=ps_{t-1}+q_3s_t$.
						\item Verify that $\deg(p)=t-1-(i-1)=t-i$.
						\item Verify that $\deg(q_3)=t-2-(i-1)=t-1-i$
						\item Therefore verify that $s_{i-1}(A)=p(A)s_{t-1}(A)+j(A)s_t(A)=p(A)s_{t-1}(A)+j(A)0_{m\times m}=p(A)s_{t-1}(A)$.
						\item If $p(A)=0$, then do the following:
						\begin{enumerate}
							\item Execute \hyperref[sec:procedure 64]{procedure 64} on $A$ and $p$.
							\item \textbf{Abort procedure.}
						\end{enumerate}
						\item Otherwise, if $s_{i-1}(A)=0_{m\times m}$, then do the following:
						\begin{enumerate}
							\item Verify that $p(A)s_{t-1}(A)\sel_A(A)=s_{i-1}(A)\sel_A(A)=0_{m\times m}\sel_A(A)=0_{m\times m}$.
							\item Verify that $p(A)s_{t-1}(A)\sel_A(A)=p(A)I_{m,m}=p(A)\ne0_{m\times m}$.
							\item Therefore verify that $0\ne 0$.
							\item \textbf{Abort procedure.}
						\end{enumerate}
						\item Otherwise if $s_{i-1}(A)\sel_A(A)=0_{m\times m}$, then do the following:
						\begin{enumerate}
							\item Verify that $s_{i-1}(A)\sel_A(A)s_{t-1}(A)=0_{m\times m}s_{t-1}(A)=0_{m\times m}$.
							\item Verify that $s_{i-1}(A)\sel_A(A)s_{t-1}(A)=s_{i-1}(A)I_{m,m}=s_{i-1}(A)\ne 0$.
							\item Therefore verify that $0\ne 0$.
							\item \textbf{Abort procedure.}
						\end{enumerate}
						\item Otherwise, do the following:
						\begin{enumerate}
							\item Verify that $\deg(s_{i-1})<i$.
							\item Verify that $s_{i-1}(A)\sel_A(A)\ne 0_{m\times m}$.
							\item Execute the \hyperref[sec:procedure 73 auxilliary procedure]{auxilliary procedure} on the tuple $(i-1, s_{i-1})$.
							\item Hence verify that $\frac{x^{i-1}\circ s_{i-1}}{x^i\circ s_i}=\tr(s_{i-1}(A)^2\sel_A(A)^2)=\tr((s_{i-1}(A)\sel_A(A))^2)=\lVert s_{i-1}(A)\sel_A(A)\rVert^2>0$.
							\item \textbf{Therefore verify that $\sgn(x^{i-1}\circ s_{i-1})=\sgn(x^i\circ s_i)$.}
						\end{enumerate}
					\end{enumerate}
					\item Yield the tuple $\langle s_{[0:t+1]}, q_{[0:t]}\rangle$.
				\end{enumerate}
			\subsubsection{Auxilliary procedure}\label{sec:procedure 73 auxilliary procedure}
				\paragraph{Objective}
					Choose an integer $0\le k\le t$ such that polynomial $s_k$ is defined. Choose a $\mathbb{Q}[x]$ $g$ such that $\deg(g)\le\min(k,t-1)$. The objective of the following instructions is to show that $\tr(g(A)s_k(A)\sel_A(A)^2)=\frac{x^k\circ g}{x^{k+1}\circ s_{k+1}}$.
				\paragraph{Implementation}
					\begin{enumerate}
						\item If $k=t$, then verify that $\tr(g(A)s_k(A)\sel_A(A)^2)$
						\begin{enumerate}
							\item $=\tr(g(A)s_t(A)\sel_A(A)^2)$
							\item $=\tr(g(A)0_{m\times m}\sel_A(A)^2)$
							\item $=0$
							\item $=\frac{x^k\circ g}{x^{k+1}\circ s_{k+1}}$.
						\end{enumerate}
						\item Otherwise if $k=t-1$, then verify that $\tr(g(A)s_k(A)\sel_A(A)^2)$
						\begin{enumerate}
							\item $=\tr(g(A)s_{t-1}(A)\sel_A(A)^2)$.
							\item $=\tr(g(A)I_{m,m}\sel_A(A))$.
							\item $=\tr(g(A)\sel_A(A))$.
							\item \textbf{=$\frac{x^k\circ g}{x^{k+1}\circ s_{k+1}}$.}
						\end{enumerate}
						\item Otherwise if $k<t-1$, then do the following:
						\begin{enumerate}
							\item Verify that $\deg(gq_{k+1})=k+1\le t-1$.
							\item Execute the \hyperref[sec:procedure 73 auxilliary procedure]{auxilliary procedure} on the tuple $\langle k+1, gq_{k+1}\rangle$.
							\item Now verify that $\tr((g(A)q_{k+1}(A))s_{k+1}(A)\sel_A(A)^2)=\frac{\frac{x^{k+2}\circ s_{k+2}}{x^{k+1}\circ s_{k+1}}x^k\circ g}{x^{k+2}\circ s_{k+2}}=\frac{x^k\circ g}{x^{k+1}\circ s_{k+1}}$.
							\item Verify that $\deg(g)\le k\le t-2$.
							\item Execute the \hyperref[sec:procedure 73 auxilliary procedure]{auxilliary procedure} on the tuple $\langle k+2,g\rangle$.
							\item Now verify that $\tr(g(A)s_{k+2}(A)\sel_A(A)^2)=\frac{x^{k+2}\circ g}{x^{k+3}\circ{s_{k+3}}}=\frac{0}{x^{k+3}\circ{s_{k+3}}}=0$.
							\item Therefore verify that $\tr(g(A)s_k(A)\sel_A(A)^2)$
							\begin{enumerate}
								\item $=\tr(g(A)(q_{k+1}(A)s_{k+1}(A)+s_{k+2}(A))\sel_A(A)^2)$
								\item $=\tr(g(A)q_{k+1}(A)s_{k+1}(A)\sel_A(A)^2)+\tr(g(A)s_{k+2}(A)\sel_A(A)^2)$
								\item $=\frac{x^k\circ g}{x^{k+1}\circ s_{k+1}}+0$
								\item $=\frac{x^k\circ g}{x^{k+1}\circ s_{k+1}}$.
							\end{enumerate}
						\end{enumerate}
					\end{enumerate}
		\subsection{Procedure 74}\label{sec:procedure 74}
			\subsubsection{Objective}
				Choose a symmetric $\mathcal{M}_{m,m}(\mathbb{Q})$, $A$. Let $t=\deg(\last_A)$. The objective of the following instructions is to either show that $0<0$ or to construct two lists of rational numbers $c,d$ such that $c_1<d_1\le c_2<d_2\le\cdots\le c_t<d_t$ and $\sgn(\last_A(c_i))=-\sgn(\last_A(d_i))$ for $i=1$ to $i=t$.
			\subsubsection{Implementation}
				\begin{enumerate}
					\item Execute \hyperref[sec:procedure 73]{procedure 73} on the matrix $A$ and let the tuple $\langle s,q\rangle$ receive the result.
					\item Execute \hyperref[sec:procedure 60]{procedure 60} supplying the tuple $\langle s,q\rangle$. Let the tuple $\langle c,d\rangle$ receive the result.
					\item \textbf{Verify that $c_1<d_1\le c_2<d_2\le\cdots\le c_t<d_t$.}
					\item \textbf{Verify that $\sgn(\last_A(c_i))=-\sgn(\last_A(d_i))$ for $i=1$ to $i=t$.}
					\item \textbf{Yield $\langle c,d\rangle$.}
				\end{enumerate}
		\subsection{Procedure 75}\label{sec:procedure 75}
			\subsubsection{Objective}
				Choose a symmetric $\mathcal{M}_{m,m}(\mathbb{Q})$, $A$. Let $t=\deg(\last_A)$. Execute \hyperref[sec:procedure 74]{procedure 74} on $A$ and let the tuple $\langle c,d\rangle$ receive the result. Execute \hyperref[sec:procedure 21]{procedure 21} on $A$ and let the tuple $\langle,,u,\rangle$ receive the result. The objective of the following instructions is to either show that $1=-1$ or to construct a list of non-negative integers $k$ such that $\sgn(u_{k_i}(c_i))=-\sgn(u_{k_i}(d_i))$ for $i=1$ to $i=t$.
			\subsubsection{Implementation}
				\begin{enumerate}
					\item Verify that $\last_A=u_1u_2\cdots u_m$.
					\item For $i=1$ to $i=t$ do the following:
					\begin{enumerate}
						\item If $\sgn(u_1(c_i))=\sgn(u_1(d_i)), \sgn(u_2(c_i))=\sgn(u_2(d_i)), \cdots, \sgn(u_m(c_i))=\sgn(u_m(d_i))$, then do the following:
						\begin{enumerate}
							\item Verify that $\sgn(u_1(c_i))\sgn(u_2(c_i))\cdots\sgn(u_m(c_i))=\sgn(u_1(d_i))\sgn(u_2(d_i))\cdots\sgn(u_m(d_i))$.
							\item Therefore verify that $\sgn(u_1(c_i)u_2(c_i)\cdots u_m(c_i))=\sgn(u_1(d_i)u_2(d_i)\cdots u_m(d_i))$.
							\item Therefore verify that $\sgn(s_t(c_i))=\sgn(s_t(d_i))$.
							\item Cognizant of the execution of \hyperref[sec:procedure 60]{procedure 60}, verify that $\sgn(s_t(c_i))=-\sgn(s_t(d_i))$.
							\item Therefore verify that $\sgn(s_t(c_i))=-\sgn(s_t(c_i))$.
							\item Therefore verify that $1=-1$.
							\item \textbf{Abort procedure.}
						\end{enumerate}
						\item Otherwise do the following:
						\begin{enumerate}
							\item \textbf{Let $j$ be the least integer such that $\sgn(u_j(c_i))=-\sgn(u_j(d_i))$.}
							\item \textbf{Let $k_i=j$.}
						\end{enumerate}
					\end{enumerate}
					\item \textbf{Yield $\langle k\rangle$.}
				\end{enumerate}
		\subsection{Procedure 76}\label{sec:procedure 76}
			\subsubsection{Objective}
				Choose a symmetric $\mathcal{M}_{m,m}(\mathbb{Q})$, $A$. Execute \hyperref[sec:procedure 21]{procedure 21} on $A$ and let the tuple $\langle,,u,\rangle$ receive the result. Execute \hyperref[sec:procedure 55]{procedure 55} on $A$ and let $k$ receive. Let $t=\deg(\last_A)$. Let $n_j=\sum_{i=1}^t [k_i=j]$. The objective of the following instructions is to either show that $0<0$, or to show that $n_i=\deg(u_i)$ for $i=1$ to $i=m$.
			\subsubsection{Implementation}
				\begin{enumerate}
					\item Verify that $\sum_{j=1}^m n_j=\sum_{j=1}^m\sum_{i=1}^t [k_i=j]=\sum_{i=1}^t\sum_{j=1}^m [k_i=j]=\sum_{i=1}^t 1=t$.
					\item If for any $i=1$ to $i=m$, $n_i>\deg(u_i)$, then do the following:
					\begin{enumerate}
						\item Execute \hyperref[sec:procedure 55]{procedure 55} on the polynomial $u_i$ along with $\deg(u_i)+1$ of the distinct pairs $\langle c_l,d_l\rangle$ such that $k_l=i$.
						\item \textbf{Abort procedure.}
					\end{enumerate}
					\item Otherwise if for any $i=1$ to $i=m$, $n_i<\deg(u_i)$, then do the following:
					\begin{enumerate}
						\item Verify that $\sum_{i=1}^m n_j<\sum_{i=1}^m \deg(u_j)=t$.
						\item Therefore verify that $\sum_{i=1}^m n_j<\sum_{i=1}^m n_j$.
						\item \textbf{Abort procedure.}
					\end{enumerate}
					\item Otherwise, do the following:
					\begin{enumerate}
						\item \textbf{For all $i=1$ to $i=m$, verify that $n_i=\deg(u_i)$.}
					\end{enumerate}
				\end{enumerate}
		Let us use the notation "$A$ is upper triangular" as a shorthand for "all the entries of $A$ below the diagonal are zero" in what follows. 
		\subsection{Procedure 77 (Upper triangular matrix multiplication)}\label{sec:procedure 77}
			\subsubsection{Objective}
				Choose two upper triangular $\mathcal{M}_{m,m}(\mathbb{Q}[x])$s, $A$ and $B$. Let $C=AB$. The objective of the following instructions is to show that $C$ is an upper triangular matrix where $C_{i,i}=A_{i,i}B_{i,i}$ for $i=1$ to $i=m$.
			\subsubsection{Implementation}
				\begin{enumerate}
					\item For $i=1$ to $i=m$, do the following:
					\begin{enumerate}
						\item \textbf{Verify that $C_{i,i}=\sum_{k=1}^m (A_{i,k}B_{k,i})=\sum_{k=1}^{i-1} (A_{i,k}B_{k,i})+A_{i,i}B_{i,i}+\sum_{k=i+1}^m (A_{i,k}B_{k,i})=\sum_{k=1}^{i-1} (0*B_{k,i})+A_{i,i}B_{i,i}+\sum_{k=i+1}^m (A_{i,k}*0)=A_{i,i}B_{i,i}$.}
					\end{enumerate}
					\item For $i=2$ to $i=m$, do the following:
					\begin{enumerate}
						\item For $j=1$ to $j=i-1$, do the following:
						\begin{enumerate}
							\item Verify that $C_{i,j}=\sum_{k=1}^m A_{i,k}B_{k,j}=\sum_{k=1}^{i-1} A_{i,k}B_{k,j}+\sum_{k=i}^m A_{i,k}B_{k,j}=\sum_{k=1}^{i-1} 0*B_{k,j}+\sum_{k=i}^m A_{i,k}*0=0$.
						\end{enumerate}
					\end{enumerate}
					\item \textbf{Therefore verify that $C$ is upper triangular.}
				\end{enumerate}
		\subsection{Procedure 78}\label{sec:procedure 78}
			\subsubsection{Objective}
				Choose integers $m\ge n\ge 0$. Choose a $\mathcal{M}_{n,m}(\mathbb{Q}[x])$, $M$, and a $\mathcal{M}_{m,n}(\mathbb{Q}[x])$, $A_0$, such that $MA_0=I_n$. The objective of the following instructions is to either show that $1=0$ or to define the $\mathcal{M}_{m,n}(\mathbb{Q}[x])$s $A_1,A_2,\cdots,A_n$.
			\subsubsection{Implementation}
				\begin{enumerate}
					\item Using \hyperref[sec:procedure 11]{procedure 11}, verify that $C_n(M_0A_0)=C_n(I_n)=1$.
					\item If $C_n(A_0)=0_{\binom{m}{n}\times 1}$, then do the following:
					\begin{enumerate}
						\item Verify that $C_n(M_0A_0)=C_n(M_0)C_n(A_0)=C_n(M_0)0_{\binom{m}{n}\times 1}=0$.
						\item Therefore verify that $1=0$.
						\item \textbf{Abort procedure.}
					\end{enumerate}
					\item Verify that $C_n(A_0)\ne0_{\binom{m}{n}\times 1}$.
					\item For $i=1$ to $i=n$, do the following:
					\begin{enumerate}
						\item If $A_{i-1}e_i=0_{m\times 1}$, then do the following:
						\begin{enumerate}
							\item Verify that $C_n(A_{i-1})=0$.
							\item Cognizant of the execution of the previous iteration, verify that $C_n(A_{i-1})\ne 0$.
							\item Therefore verify that $0\ne 0$.
							\item \textbf{Abort procedure.}
						\end{enumerate}
						\item Verify that $\lVert A_{i-1}e_i\rVert^2\ne 0$.
						\item Let $D_i$ be a $n\times n$ diagonal matrix comprising $i$ $1$s followed by $n-i$ $\lVert A_{i-1}e_i\rVert^2$s.
						\item Verify that $C_n(D_i)=(\lVert A_{i-1}e_i\rVert^2)^{n-i}\ne 0$.
						\item Let $N_i=I_n$ except that its $i^{th}$ row is $i-1$ $0$s followed by a $1$ followed by $-({A_{i-1}}^TA_{i-1})_{i,i+1}$, then $-({A_{i-1}}^TA_{i-1})_{i,i+2}$, all the way up to $-({A_{i-1}}^TA_{i-1})_{i,n}$.
						\item Using \hyperref[sec:procedure 11]{procedure 11}, verify that $C_n(N_i)=1\ne 0$.
						\item Let $A_i=A_{i-1}D_iN_i$.
						\item \textbf{Verify that $C_n(A_i)=C_n(A_{i-1}D_iN_i)=C_n(A_{i-1})C_n(D_i)C_n(N_i)=C_n(A_{i-1})C_n(D_i)\ne 0$}.
					\end{enumerate}
					\item \textbf{Yield the tuple $\langle A_0,A_1,\cdots,A_n\rangle$.}
				\end{enumerate}
		\subsection{Procedure 79}\label{sec:procedure 79}
			\subsubsection{Objective}
				Choose integers $m\ge n\ge 0$. Choose a $\mathcal{M}_{n,m}(\mathbb{Q}[x])$, $M$, and a $\mathcal{M}_{m,n}(\mathbb{Q}[x])$, $A_0$, such that $MA_0=I_n$. Execute \hyperref[sec:procedure 78]{procedure 78} on $M$ and $A_0$ and let the tuple $\langle A_1,\cdots,A_n\rangle$ receive the result. The objective of the following instructions is to either show that $1=0$ or to show that $({A_i}^TA_i)_{[1:i+1],[1:i+1]}$ is a $\mathcal{D}_{i,i}(\mathbb{Q}[x])$ for $i=1$ to $i=n$.
			\subsubsection{Implementation}
				\begin{enumerate}
					\item For $i=1$ to $i=n$, do the following:
					\begin{enumerate}
						\item Let $D_i$ be a $n\times n$ diagonal matrix comprising $i$ $1$s followed by $n-i$ $\lVert A_{i-1}e_i\rVert^2$s.
						\item Let $N_i=I_n$ except that its $i^{th}$ row is $i-1$ $0$s followed by a $1$ followed by $-({A_{i-1}}^TA_{i-1})_{i,i+1}$, then $-({A_{i-1}}^TA_{i-1})_{i,i+2}$, all the way up to $-({A_{i-1}}^TA_{i-1})_{i,n}$.
						\item Verify that $A_i=A_{i-1}D_iN_i$.
						\item Verify that ${A_i}^TA_i=(A_{i-1}D_iN_i)^T(A_{i-1}D_iN_i)={N_i}^T{D_i}^T({A_{i-1}}^TA_{i-1})D_iN_i$.
						\item Now verify that ${A_i}^TA_i$ and ${A_{i-1}}^TA_{i-1}$ are the same modulo the bottom-right $(n-i+1)\times(n-i+1)$ block.
						\item Also verify that $({A_i}^TA_i)_{i,[i+1:n+1]}=0$.
						\item Also verify that $({A_i}^TA_i)_{[i+1:n+1],i}=0$.
						\item \textbf{Therefore verify that $({A_i}^TA_i)_{[1:i+1],[1:i+1]}$ is a $\mathcal{D}_{i,i}(\mathbb{Q}[x])$.}
					\end{enumerate}
				\end{enumerate}
		\subsection{Procedure 80}\label{sec:procedure 80}
			\subsubsection{Objective}
				Choose integers $m\ge n\ge 0$. Choose a $\mathcal{M}_{n,m}(\mathbb{Q}[x])$, $M$, and a $\mathcal{M}_{m,n}(\mathbb{Q}[x])$, $A_0$, such that $MA_0=I_n$. Execute \hyperref[sec:procedure 78]{procedure 78} on $M$ and $A_0$ and let the tuple $\langle A_1,\cdots,A_n\rangle$ receive the result. The objective of the following instructions is to either show that $1=0$ or to show that $A_0MA_i=A_i$ and $({e_j}^TM)(A_ie_j)=\lVert A_0e_1\rVert^2\cdots\lVert A_{\min(i,j-1)-1}e_{\min(i,j-1)}\rVert^2$ for $j=1$ to $j=n$, for $i=1$ to $i=n$.
			\subsubsection{Implementation}
				\begin{enumerate}
					\item For $i=1$ to $i=n$, do the following:
					\begin{enumerate}
						\item Let $D_i$ be a $n\times n$ diagonal matrix comprising $i$ $1$s followed by $n-i$ $\lVert A_{i-1}e_i\rVert^2$s.
						\item \textbf{Verify that $D_i$ is upper triangular.}
						\item Let $N_i=I_n$ except that its $i^{th}$ row is $i-1$ $0$s followed by a $1$ followed by $-({A_{i-1}}^TA_{i-1})_{i,i+1}$, then $-({A_{i-1}}^TA_{i-1})_{i,i+2}$, all the way up to $-({A_{i-1}}^TA_{i-1})_{i,n}$.
						\item \textbf{Verify that $N_i$ is upper triangular.}
						\item Verify that $A_i=A_{i-1}D_iN_i$.
						\item Verify that $A_i=A_0(D_1N_1)\cdots (D_iN_i)$.
						\item Verify that $MA_i=(D_1N_1)\cdots (D_iN_i)$.
						\item \textbf{Therefore verify that $A_0MA_i=A_i$.}
						\item Using \hyperref[sec:procedure 77]{procedure 77}, for $j=1$ to $j=n$, verify that $({e_j}^TM)(A_ie_j)$
						\begin{enumerate}
							\item $={e_j}^T(MA_i)e_j$
							\item $={e_j}^T((D_1N_1)\cdots (D_iN_i))e_j$
							\item $=({D_1}_{j,j}{N_1}_{j,j})\cdots ({D_i}_{j,j}{N_i}_{j,j})$
							\item $={D_1}_{j,j}\cdots {D_i}_{j,j}$
							\item $={D_1}_{j,j}\cdots {D_{\min(i,j-1)}}_{j,j}$
							\item $=\lVert A_0e_1\rVert^2\cdots\lVert A_{\min(i,j-1)-1}e_{\min(i,j-1)}\rVert^2$.
						\end{enumerate}
					\end{enumerate}
				\end{enumerate}
		\subsection{Procedure 81 (Cauchy-Schwarz inequality)}\label{sec:procedure 81}
			\subsubsection{Objective}
				Choose a $\mathcal{M}_{1,m}(\mathbb{Q})$, $A$, and a $\mathcal{M}_{m,1}(\mathbb{Q})$, $B$. The objective of the following instructions is to show that $(AB)^2\le(AA^T)(B^TB)$.
			\subsubsection{Implementation}
				\begin{enumerate}
					\item Verify that $0$
					\begin{enumerate}
						\item $\le\frac{1}{2}\sum_{i=1}^m\sum_{j=1}^m (A_iB_j-A_jB_i)^2$
						\item $=\frac{1}{2}\sum_{i=1}^m\sum_{j=1}^m ({A_i}^2{B_j}^2-2A_iB_jA_jB_i+{A_j}^2{B_i}^2)$
						\item $=\frac{1}{2}\sum_{i=1}^m {A_i}^2\sum_{j=1}^m {B_j}^2+\frac{1}{2}\sum_{i=1}^m {B_i}^2\cdot\allowbreak\sum_{j=1}^m {A_j}^2-\sum_{i=1}^m A_iB_i\sum_{j=1}^m A_jB_j$
						\item $=\frac{1}{2}(AA^T)(B^TB)+\frac{1}{2}(AA^T)(B^TB)-(AB)^2$
						\item $=(AA^T)(B^TB)-(AB)^2$.
					\end{enumerate}
					\item \textbf{Therefore verify that $(AB)^2\le(AA^T)(B^TB)$.}
				\end{enumerate}
		\subsection{Procedure 82}\label{sec:procedure 82}
			\subsubsection{Objective}
				Choose integers $m\ge n>0$. Choose a $\mathcal{M}_{n,m}(\mathbb{Q}[x])$, $M$, and a $\mathcal{M}_{m,n}(\mathbb{Q}[x])$, $A_0$, such that $MA_0=I_n$. Choose a $\mathbb{Q}$, $x$. Let $a=\max(\lVert M(x)\rVert^2,1)$. Choose a column index $1\le j\le n$ such that $\lVert A_n(x)e_j\rVert^2<\frac{1}{a^{(2n+2)!!}}$. The objective of the following instructions is to show that $1<1$.
			\subsubsection{Implementation}
				\begin{enumerate}
					\item Execute \hyperref[sec:procedure 78]{procedure 78} on $M$ and $A_0$ and let the tuple $\langle A_0,A_1,\cdots,A_n\rangle$ receive the result.
					\item Let $i=n$.
					\item Verify that $\lVert A_i(x)e_j\rVert^2<\frac{1}{a^{(2i+2)!!}}$.
					\item Using \hyperref[sec:procedure 81]{procedure 81}, verify that $({e_j}^TM(x)A_i(x)e_j)^2\le\lVert{e_j}^TM(x)\rVert^2\lVert A_i(x)e_j\rVert^2<\lVert M(x)\rVert^2\frac{1}{a^{(2i+2)!!}}\le a\frac{1}{a^{(2i+2)!!}}\le\frac{1}{a^{(2i)!!*2i}}\le 1$.
					\item If $i=0$, then do the following:
					\begin{enumerate}
						\item Verify that $({e_j}^TM(x)A_i(x)e_j)^2=({e_j}^TM(x)A_0(x)e_j)^2=({e_j}^TI_ne_j)^2=1$.
						\item Therefore verify that $1<1$.
						\item \textbf{Abort procedure.}
					\end{enumerate}
					\item Otherwise, do the following:
					\item Using \hyperref[sec:procedure 80]{procedure 80}, verify that $(1\lVert A_0e_1\rVert^2\cdots\lVert A_{\min(i,j-1)-1}e_{\min(i,j-1)}\rVert^2)^2=({e_j}^TM(x)A_i(x)e_j)^2<\frac{1}{a^{(2i)!!*2i}}\le 1$.
					\item If $\min(i,j-1)=0$, then do the following:
					\begin{enumerate}
						\item Verify that $(1\lVert A_0(x)e_1\rVert^2\cdots\allowbreak\lVert A_{\min(i,j-1)-1}(x)e_{\min(i,j-1)}\rVert^2)^2=1^2=1$.
						\item Therefore verify that $1<1$.
						\item \textbf{Abort procedure.}
					\end{enumerate}
					\item Otherwise do the following:
					\begin{enumerate}
						\item Verify that $\min(i,j-1)>0$.
						\item If for all $k=0$ to $k=\min(i,j-1)-1$, $\lVert A_k(x)e_{k+1}\rVert^2\ge\frac{1}{a^{(2i)!!}}$, then do the following:
						\begin{enumerate}
							\item Verify that $({e_j}^TM(x)A_i(x)e_j)^2=(\lVert A_0(x)e_1\rVert^2\cdots\allowbreak\lVert A_{\min(i,j-1)-1}(x)e_{\min(i,j-1)}\rVert^2)^2\ge(\frac{1}{a^{(2i)!!}})^{2\min(i,j-1)}\ge(\frac{1}{a^{(2i)!!}})^{2i}=\frac{1}{a^{(2i)!!*2i}}$.
							\item Therefore verify that $({e_j}^TM(x)A_i(x)e_j)^2<\frac{1}{a^{(2i)!!*2i}}\le ({e_j}^TM(x)A_i(x)e_j)^2$.
							\item \textbf{Abort procedure.}
						\end{enumerate}
						\item Otherwise, do the following:
						\begin{enumerate}
							\item \textbf{Let $k$, where $0\le k<i$, be one of the integers for which $\lVert A_k(x)e_{k+1}\rVert^2<\frac{1}{a^{(2i)!!}}$.}
							\item \textbf{Verify that $\lVert A_k(x)e_{k+1}\rVert^2<\frac{1}{a^{(2i)!!}}\le\frac{1}{a^{(2k+2)!!}}$.}
							\item \textbf{Simultaneously set $i$ to $k$ and $j$ to $k+1$.}
							\item \textbf{Go to (3).}
						\end{enumerate}
					\end{enumerate}
				\end{enumerate}
		\subsection{Procedure 83}\label{sec:procedure 83}
			\subsubsection{Objective}
				Choose a symmetric $\mathcal{M}_{m,m}(\mathbb{Q})$, $A$. Execute \hyperref[sec:procedure 75]{procedure 75} on the matrix $A$ and let the tuple $\langle k\rangle$ receive the result. The objective of the following instructions is to either show that $0<0$ or to show that $\sum_{i=1}^t(m+1-k_i)=m$.
			\subsubsection{Implementation}
				\begin{enumerate}
					\item Execute \hyperref[sec:procedure 21]{procedure 21} on the matrix $A$ and let the tuple $\langle,D,u,\rangle$.
					\item Using \hyperref[sec:procedure 76]{procedure 76}, verify that $\sum_{i=1}^t(m+1-k_i)$
					\begin{enumerate}
						\item $=\sum_{i=1}^t\sum_{j=1}^m [k_i\le j]$
						\item $=\sum_{j=1}^m\sum_{i=1}^t [k_i\le j]$
						\item $=\sum_{j=1}^m\sum_{i=1}^t [k_i\le j]\sum_{l=1}^m [k_i=l]$
						\item $=\sum_{j=1}^m\sum_{l=1}^m\sum_{i=1}^t [k_i\le j][k_i=l]$
						\item $=\sum_{j=1}^m\sum_{l=1}^m\sum_{i=1}^t [l\le j][k_i=l]$
						\item $=\sum_{j=1}^m\sum_{l=1}^m [l\le j]\sum_{i=1}^t [k_i=l]$
						\item $=\sum_{j=1}^m\sum_{l=1}^m [l\le j]\deg u_l$
						\item $=\sum_{j=1}^m\sum_{l=1}^j \deg u_l$
						\item $=\sum_{j=1}^m \deg D_{j,j}$
						\item $=m$
					\end{enumerate}
				\end{enumerate}
		\subsection{Procedure 84}\label{sec:procedure 84}
			\subsubsection{Objective}
				Choose a symmetric $\mathcal{M}_{m,m}(\mathbb{Q})$, $A$. The objective of the following instructions is to either show that $0<0$ or to define the rational $\disc(A)$.
			\subsubsection{Implementation}
				\begin{enumerate}
					\item Execute \hyperref[sec:procedure 74]{procedure 74} on the matrix $A$ and let the tuple $\langle c,d\rangle$ receive the result.
					\item Execute \hyperref[sec:procedure 5]{procedure 5} with $xI_m-A$ as the choice matrix. Let the tuple $\langle M,D,,N\rangle$ receive the result.
					\item Let $L=\lvert(\lVert N^{-1}\rVert^2)^{(2m+2)!!}\rvert$.
					\item Let $\disc(A)=\frac{1}{\max(1,L(\lvert c_1\rvert),L(\lvert d_t\rvert))}$.
					\item \textbf{Verify that $\disc(A)>0$.}
					\item \textbf{Yield the tuple $\langle\disc(A)\rangle$.}
				\end{enumerate}
		Let us use the notation $\disc(A)$ to refer to the invocation of \hyperref[sec:procedure 84]{procedure 84} on the matrix $A$.
		\subsection{Procedure 85}\label{sec:procedure 85}
			\subsubsection{Objective}
				Choose integers $0<k\le m$ and a list of $\mathcal{T}_m(\mathbb{Q}[x])$, $N$. Let $Q=(I_m)_{*,[k:m]}$. The objective of the following instructions is to either show that $1=0$ or to construct an $\mathcal{M}_{m,m+1-k}(\mathbb{Q}[x])$, $K$, and an $\mathcal{M}_{m+1-k,m+1-k}(\mathbb{Q}[x])$, $E$, such that $K_i=NQE$ and $K^TK$ is a $\mathcal{D}_{m+1-k,m+1-k}(\mathbb{Q}[x])$.
			\subsubsection{Implementation}
				\begin{enumerate}
					\item Verify that $(Q^TN^{-1})(NQ)=Q^T(N^{-1}N)Q=Q^TI_mQ=Q^TQ=I_{m+1-k}$.
					\item Execute \hyperref[sec:procedure 78]{procedure 78} on the matrices $Q^TN^{-1}$ and $NQ$. Let the tuple $\langle ,,\cdots,,K\rangle$ receive the result.
					\item \textbf{Verify that $K$ is a $\mathcal{M}_{m,m+1-k}(\mathbb{Q}[x])$.}
					\item \textbf{Using \hyperref[sec:procedure 79]{procedure 79}, verify that $K^TK$ is a $\mathcal{D}_{m+1-k,m+1-k}(\mathbb{Q}[x])$.}
					\item Let $E=Q^TN^{-1}K$.
					\item \textbf{Verify that $E$ is a $\mathcal{M}_{m+1-k,m+1-k}(\mathbb{Q}[x])$.}
					\item Execute \hyperref[sec:procedure 80]{procedure 80} on the matrices $Q^TN^{-1}$ and $NQ$.
					\item \textbf{Now verify that $K=NQE$.}
					\item \textbf{Yield $\langle K,E\rangle$.}
				\end{enumerate}
		\subsection{Procedure 86 (Symmetric matrix spectral procedure initialization)}\label{sec:procedure 86}
			\subsubsection{Objective}
				Choose a symmetric $\mathcal{M}_{m,m}(\mathbb{Q})$, $A$. Choose a $\mathbb{Q}$ $\epsilon>0$. Execute \hyperref[sec:procedure 75]{procedure 75} on the matrix $A$ and let the tuple $\langle k\rangle$ receive the result. The objective of the following instructions is to either show that $1<1$ or to construct $\mathbb{Q}$s, $0<\delta\le 1\le K'$, a list of $\mathcal{M}_{m,*}(\mathbb{Q})$s, $K$, and a list of $\mathbb{Q}$s, $g$, such that for $1\le i\le\lvert k\rvert$:
				\begin{enumerate}
					\item $\cols(K_i)=m+1-k_i$.
					\item $(K_i)_{p,q}\le K'm$, for $1\le p\le m$, for $1\le q\le\cols(K_i)$.
					\item ${K_i}^T{K_i}$ is a $\mathcal{D}_{*,*}(\mathbb{Q})$.
					\item $({K_i}^T{K_i})_{j,j}\ge\disc(A)$ for $1\le j\le\cols(K_i)$.
					\item $\lvert(g_iK_i-AK_i)_{p,q}\rvert<\frac{\epsilon\delta}{K'm^2}$, for $1\le p\le m$, for $1\le q\le\cols(K_i)$.
					\item $\delta\le\min_{1\le i\ne j\le\lvert g\rvert}\lvert g_j-g_i\rvert$.
				\end{enumerate}
			\subsubsection{Implementation}
				\begin{enumerate}
					\item Execute \hyperref[sec:procedure 74]{procedure 74} on the matrix $A$ and let the tuple $\langle c,d\rangle$ receive the result.
					\item Execute \hyperref[sec:procedure 21]{procedure 21} with $xI_m-A$ as the choice matrix. Let the tuple $\langle M,D,u,N\rangle$ receive the result.
					\item Let $M'=1+\max_{i=1}^m\max_{j=1}^m\lvert {{M^{-1}}_*}_{i,j}\rvert(\max(\lvert c_1\rvert,\lvert d_{\lvert d\rvert}\rvert))$.
					\item Let $N'=1+\max_{i=1}^m\max_{j=1}^m\lvert {N_*}_{i,j}\rvert(\max(\lvert c_1\rvert,\lvert d_{\lvert d\rvert}\rvert))$.
					\item Let $\delta=\min(1,\min_{i=1}^{\lvert d\rvert-1}(d_{i+1}-c_i))$.
					\item Execute \hyperref[sec:procedure 85]{procedure 85} on $\langle k,m,N\rangle$ and let the tuple $\langle\langle K_1,E_1\rangle,\langle K_2,E_2\rangle,\cdots,\langle K_{\lvert k\rvert},E_{\lvert k\rvert}\rangle\rangle$ receive.
					\item \textbf{Using \hyperref[sec:procedure 83]{procedure 83}, verify that $\sum_{p=1}^{\lvert k\rvert}\cols(K_p)=\sum_{p=1}^{\lvert k\rvert} m+1-k_p=m$.}
					\item Let $E'=1+\max_{i=1}^t\max_{j=1}^{m+1-k_i}\max_{l=1}^{m+1-k_i}\lvert E_{j,l}\rvert(\max(\lvert c_1\rvert,\lvert d_{\lvert d\rvert}\rvert))$.
					\item Let $U=(1+\lvert u_1\rvert)(1+\lvert u_2\rvert)\cdots(1+\lvert u_m\rvert)$.
					\item Let $U'=U(\max(\lvert c_1\rvert,\lvert d_{\lvert d\rvert}\rvert))$.
					\item Let $b=\frac{\epsilon\delta}{M'N'E'^2m^3}$.
					\item For $i=1$ to $i=\lvert k\rvert$, do the following:
					\begin{enumerate}
						\item Verify that $\sgn(u_{k_i}(c_i))\ne\sgn(u_{k_i}(d_i))$.
						\item Execute \hyperref[sec:procedure 54]{procedure 54} on the formal polynomial $u_{k_i}$, interval $(c_i, d_i)$, and target of $\frac{b}{U'}$. Let $\langle g_i\rangle$ receive the result.
						\item Now verify that $\lvert u_{k_i}(g_i)\rvert<\frac{b}{U'}$.
						\item Also verify that $c_i\le g_i\le d_i$.
						\item For $j=k_i$ to $j=m$, do the following:
						\begin{enumerate}
							\item Verify that $\lvert D_{j,j}(g_i)\rvert=\lvert u_1(g_i)\rvert\lvert u_2(g_i)\rvert\cdots\lvert u_m(g_i)\rvert\le\lvert u_{k_i}(g_i)\rvert\lvert u_1\rvert(\lvert g_i\rvert)\cdots\lvert u_{k_i-1}\rvert(\lvert g_i\rvert)\cdot\lvert u_{k_i+1}\rvert(\lvert g_i\rvert)\cdots\lvert u_m\rvert(\lvert g_i\rvert)<\frac{b}{U'}U(\lvert g_i\rvert)=\frac{b}{U'}U'=b$.
						\end{enumerate}
						\item Let $Q=(I_m)_{*,[k_i:m]}$.
						\item If a diagonal entry of ${K_i(g_i)}^TK_i(g_i)$ is less than $\disc(A)$, then do the following:
						\begin{enumerate}
							\item Let $z$ be the column index of the diagonal entry less than $\disc(A)$.
							\item Verify that $\disc(A)\le\frac{1}{\max(\lVert (Q^TN^{-1})(g_i)\rVert^2,1)^{(2(m+1-k_i)+2)!!}}$.
							\item Execute \hyperref[sec:procedure 82]{procedure 82} with matrices $Q^TN^{-1}$ and $NQ$, rational number $g_i$, and column index $z$.
							\item \textbf{Abort procedure.}
						\end{enumerate}
						\item Otherwise, do the following:
						\begin{enumerate}
							\item \textbf{For $j=1$ to $j=m+1-k_i$, verify that $({K_i(g_i)}^TK_i(g_i))_{j,j}\ge \disc(A)>0$.}
							\item Verify that $xK_i-AK_i=(xI_m-A)K_i=M^{-1}DN^{-1}K_i=M^{-1}DN^{-1}NQE_i=M^{-1}DQE_i$.
							\item \textbf{Verify that $(g_iK_i(g_i)-AK_i(g_i))_{p,q}=(M^{-1}(g_i)D(g_i)QE_i(g_i))_{p,q}<M'b(m+1-k_i)E'=M'\frac{\epsilon\delta}{M'N'E'^2m^3}(m+1-k_i)E'\le\frac{\epsilon\delta}{N'E'm^2}$ for $1\le p\le m$, for $1\le q\le m+1-k_i$.}
							\item \textbf{Verify that $K_i(g_i)_{p,q}=(N(g_i)QE_i(g_i))_{p,q}=N'(m+1-k_i)E'\le N'E'm$.}
						\end{enumerate}
					\end{enumerate}
					\item \textbf{Yield the tuple $\langle\delta,N'E',\langle K_1(g_1),\cdots,K_t(g_t)\rangle,g\rangle$.}
				\end{enumerate}
		\subsection{Procedure 87 (Symmetric matrix spectral)}\label{sec:procedure 87}
			\subsubsection{Objective}
				Choose a symmetric $\mathcal{M}_{m,m}(\mathbb{Q})$, $A$. Choose a $\mathbb{Q}$ $\epsilon>0$. The objective of the following instructions is to either show that $1<1$ or to construct an $\mathcal{M}_{m,m}(\mathbb{Q})$, $K$, and a $\mathcal{D}_{m,m}(\mathbb{Q})$, $C$, such that:
				\begin{enumerate}
					\item $\sum_{p=1}^m\sum_{q=1}^m\lvert(KC-AK)_{p,q}\rvert<\epsilon$.
					\item $\lvert(K^TK)_{i,j}\rvert\le 2\epsilon$ for $1\le i\ne j\le m$.
					\item $(K^TK)_{j,j}\ge\disc(A)>0$ for $1\le j\le m$.
				\end{enumerate}
			\subsubsection{Implementation}
				\begin{enumerate}
					\item Execute \hyperref[sec:procedure 86]{procedure 86} on matrix $A$ and rational $\epsilon$. Let the tuple $\langle\delta,K',K,g\rangle$ receive the result.
					\item Let $C$ be a diagonal matrix whose $i^{th}$, where $1\le i\le t$, group of entries are $m+1-k_i$ $g_i$s.
					\item \textbf{Using \hyperref[sec:procedure 83]{procedure 83}, verify that $C$ is $m\times m$.}
					\item Let $K$ be a matrix whose columns are the in-order concatenation of those of $K_1,K_2,\cdots,K_t$.
					\item \textbf{Using \hyperref[sec:procedure 83]{procedure 83}, verify that $K$ is $m\times m$.}
					\item \textbf{Using (1), verify that $\sum_{p=1}^m\sum_{q=1}^m\lvert(KC-AK)_{p,q}\rvert<\sum_{p=1}^m\sum_{q=1}^m \frac{\epsilon\delta}{K'm^2}=\frac{\epsilon\delta}{K'}\le\epsilon$.}
					\item For $i=1$ to $i=m$, do the following: For $j=1$ to $j=m$, do the following:
					\begin{enumerate}
						\item Let $a,c$ be such that $Ke_i$ came from $K_ae_c$.
						\item Let $b,d$ be such that $Ke_j$ came from $K_be_d$.
						\item If $a\ne b$, then do the following:
						\begin{enumerate}
							\item Using (1), verify that $\lvert(g_b-g_a)(Ke_i)^T(Ke_j)\rvert$
							\item $=\lvert g_b(Ke_i)^T(Ke_j)-g_a(Ke_i)^T(Ke_j)\rvert$
							\item $=\lvert(Ke_i)^T(g_bKe_j)-(g_aKe_i)^T(Ke_j)\rvert$
							\item $=\lvert(Ke_i)^T(AKe_j+g_bKe_j-AKe_j)-(AKe_i+g_aKe_i-AKe_i)^T(Ke_j)\rvert$
							\item $\le\lvert(Ke_i)^T(AKe_j)-(AKe_i)^T(Ke_j)\rvert+\lvert(Ke_i)^T(g_bKe_j-AKe_j)\rvert+\lvert(g_aKe_i-AKe_i)^T(Ke_j)\rvert$
							\item $\le\lvert(Ke_i)^TA(Ke_j)-(Ke_i)^TA^T(Ke_j)\rvert+\lvert mK'J_{1\times m}\frac{\epsilon\delta}{K'm^2}J_{m\times 1}\rvert+\lvert\frac{\epsilon\delta}{K'm^2}J_{1\times m}mK'J_{m\times 1}\rvert$
							\item $=2\epsilon\delta$.
							\item \textbf{Therefore using (1) and (vii), verify that $\lvert {e_i}^T(K^TK)e_j\rvert=\lvert(Ke_i)^T(Ke_j)\rvert\le\frac{2\epsilon\delta}{\lvert g_b-g_a\rvert}\le 2\epsilon$.}
						\end{enumerate}
						\item Otherwise if $c\ne d$, do the following:
						\begin{enumerate}
							\item Using (1), verify that ${K_a}^TK_b={K_a}^TK_a$ is a $\mathcal{D}_{*,*}(\mathbb{Q})$.
							\item \textbf{Therefore verify that $(Ke_i)^T(Ke_j)=(K_ae_c)^T(K_be_d)={e_c}^T{K_a}^TK_be_d=0\le 2\epsilon$.}
						\end{enumerate}
					\end{enumerate}
					\item \textbf{Therefore using (7), verify that $\lvert(K^TK)_{i,j}\rvert\le 2\epsilon$ for $1\le i\ne j\le m$.}
					\item \textbf{Using (1), verify that $(K^TK)_{j,j}\ge\disc(A)>0$ for $1\le j\le m$.}
					\item \textbf{Yield the tuple $\langle K,C\rangle$.}
				\end{enumerate}
	\begin{thebibliography}{9}
		\bibitem{linearalgebra} 
			Harold Edwards.
			\textit{Linear Algebra}. 
			Springer Science+Business Media, 1995.
		\bibitem{philosophicalgrammar}
			Ludwig Wittgenstein.
			\textit{Philosophical Grammar}.
			Edited by Rush Rhees.
			Translated by Anthony Kenny.
			Basil Blackwell, Oxford, 1974.
	\end{thebibliography}
\end{document}
\grid
