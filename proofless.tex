\documentclass[twocolumn]{article}
\usepackage{fullpage}
\usepackage{parskip}
\usepackage{amsmath}
\usepackage{amssymb}
\usepackage{datetime}
\usepackage{hyperref}
\hypersetup{
	colorlinks = true,
	linktoc = all,
}

\DeclareMathOperator{\sgn}{sgn}
\DeclareMathOperator{\mat}{mat}
\DeclareMathOperator{\pol}{pol}
\DeclareMathOperator{\tr}{tr}

\begin{document}
	\title{A Rigorous Proofless Approach to Linear Algebra}
	\author{Murisi Tarusenga}
	\date{\today{} \currenttime}
	\maketitle
	\section{Introduction}
		\textbf{Only the first 14 pages have been revised. Think of the last 14 pages as a roadmap.}
		
		What follows is an attempt to present linear algebra as a system of algorithms in a manner devoid of argumentation. The correctness of these algorithms hopefully is made apparent by stating expected variable values at certain points of execution. Mathematical argumentation is circumvented by letting algorithms \textit{show} what a proposition in an argument might have \textit{said}.
		
		The subset of linear algebra I intend to tackle consists of the Smith normal form, determinants, compound matrices, general solutions to linear systems, characteristic matrices, the rational canonical form, block matrix multiplication, minimum polynomials, orthogonalization, and the spectral theorem for symmetric matrices. To get the above done, I also have to extract the algorithms from Sturm's theorem, Cauchy's bound, the Cauchy-Schwarz inequality, the Euclidean division algorithm, and the factor theorem.
		
		\textbf{Here, I would like to note that this project is a highly modified version of the textbook Linear Algebra by Harold Edwards. In summary, I have taken the text; removed all the exercises and exposition; removed all the proofs inessential to proving the spectral theorem; extracted and purified the algorithms used in each proof; and conjured up my own algorithms in the cases where I could not extract an algorithm.}
		
		The following are some things to take into consideration whilst reading the \hyperref[sec:body]{body} of this project:
		\begin{enumerate}
			\item Where a proof expresses generality using universal quantification, the algorithms below show generality by allowing one to choose inputs. See \hyperref[sec:algorithm 24]{algorithm 24}.
			\item Where a proof expresses that a proposition implies contradiction, the algorithms below merely show how certain control flows lead to absurdity. See algorithms \hyperref[sec:algorithm 11]{11}, \hyperref[sec:algorithm 28]{28}, \hyperref[sec:algorithm 35]{34}, and \hyperref[sec:algorithm 52]{51}.
			\item Where a proof uses the principle of mathematical induction, the algorithms below merely use iteration/recursion and loop invariants. See algorithms \hyperref[sec:algorithm 47]{46}, \hyperref[sec:algorithm 47]{46 auxilliary}, and \hyperref[sec:algorithm 52]{51}.
		\end{enumerate}
		
		Anyway, to see most easily what I am trying to do, see algorithms \hyperref[sec:algorithm 34]{33}, \hyperref[sec:algorithm 40]{39}, \hyperref[sec:algorithm 49]{48}, and \hyperref[sec:algorithm 51]{50}.
	\tableofcontents
	\section{Body}\label{sec:body}
		\subsection{Algorithm 1}\label{sec:algorithm 1}
			\textbf{Choose a $1\times 2$ matrix, $A$, whose entries are polynomials} and do the following:
			\begin{enumerate}
				\item Let $A$ be our working matrix.
				\item If the $A_{1,1}=0$, then add $A_{1,2}$ to it.
				\item Then while $A_{1,2}\ne 0$, do the following:
				\begin{enumerate}
					\item If $\deg(A_{1,1})=\deg(A_{1,2})$ and $A_{1,1}$ is monic, then:
					\begin{enumerate}
						\item Subtract $a$ times $A_{1,1}$ from $A_{1,2}$ where $a$ is the leading coefficient of $A_{1,2}$.
					\end{enumerate}
					\item Otherwise, if $\deg(A_{1,1})=\deg(A_{1,2})$ but $A_{1,1}$ is not monic, then:
					\begin{enumerate}
						\item Add $1-b/a$ times $A_{1,2}$ to $A_{1,1}$ where $b$ and $a$ are the leading coefficients of $A_{1,1}$ and $A_{1,2}$ respectively.
						\item Go to (3a) again.
					\end{enumerate}
					\item Otherwise, if $\deg(A_{1,1})\ne\deg(A_{1,2})$, then:
					\begin{enumerate}
						\item Let $p$ and $q$ be locations of the polynomials with the lower and higher degree respectively.
						\item Add $-b/ax^{\deg(p)-\deg(q)}$ times $p$ to $q$ where $b$ and $a$ are the leading coefficients of $q$ and $p$ respectively.
						\item If $q$ becomes zero, then:
						\begin{enumerate}
							\item Undo operation (3cii) and do the same thing again, only this time using the fraction $1-b/ax^{\deg(p)-\deg(q)}$.
						\end{enumerate}
					\end{enumerate}
					\item \textbf{Verify that the degree of only one entry changed.}
					\item \textbf{Verify that the changed entry's degree decreased.}
				\end{enumerate}
				\item \textbf{Verify that $A_{1,2}=0$.}
				\item If the original $A$ was not zero, then do the following:
				\begin{enumerate}
					\item Verify that the body of loop (3) executed at least once.
					\item Verify that the last instructions executed were an instance of (3ai), (3d), then (3e).
					\item \textbf{Therefore verify that $A_{1,1}$ is a monic polynomial.}
				\end{enumerate}
				\item Otherwise, do the following:
				\begin{enumerate}
					\item Verify that both entries were originally zero.
					\item \textbf{Therefore, now verify that $A_{1,1}=0$.}
				\end{enumerate}
				\item \textbf{Yield the tuple $\langle A\rangle$.}
			\end{enumerate}
		\subsection{Algorithm 2}\label{sec:algorithm 2}
			\textbf{Choose an $m\times n$ matrix, $A$, whose entries are polynomials} and do the following:
			\begin{enumerate}
				\item Let $A$ be our working matrix.
				\item If the top-left entry of the matrix is zero, then do the following:
					\begin{enumerate}
						\item While there are non-zero entries in the top row less its first entry, do the following:
						\begin{enumerate}
							\item In the first row, select the $1\times 2$ matrix whose right entry coincides with the last non-zero entry of the first row
							\item Apply \hyperref[sec:algorithm 1]{algorithm 1} on this submatrix but this time adding and subtracting the entire columns instead of merely just the entries in the submatrix.
							\item Verify that the left and right entries of the submatrix are now non-zero and zero respectively.
							\item If the left entry of the submatrix coincides with the top-left entry of the matrix,
							\begin{enumerate}
								\item Verify that the top-left entry is now non-zero.
								\item Go to operation (2).
							\end{enumerate}
						\end{enumerate}
						\item Now do the same operations as in (a), but this time with the operations themselves reflected across the matrix's diagonal. I.e. making $2\times 1$ submatrices starting from the bottom-most non-zero row of the first column and working your way upwards.
						\item Verify that, except for the top-left entry, the first row and the first column are zero. Now skip to operation (3).
					\end{enumerate}
				\item While the top-left entry is non-zero, do the following:
					\begin{enumerate}
						\item Call operation (1a) except that at (1aivA), we instead expect that the top-left entry has a lower degree than before.
						\item Call operation (1b) except that at (1bivA), we instead expect that the top-left entry has a lower degree than before.
						\item Call operation (1c).
					\end{enumerate}
				\item Apply \hyperref[sec:algorithm 2]{algorithm 2} to the $(m-1)\times(n-1)$ submatrix formed by removing the first row and first column from the matrix under consideration.
				\item Verify that (3)'s execution leaves the first row and column unchanged.
				\item \textbf{Verify that $A$ is now a diagonal matrix.}
				\item \textbf{Yield the tuple $\langle A\rangle$.}
			\end{enumerate}
		\subsection{Algorithm 3 (Smith normal form construction)}\label{sec:algorithm 3}
			\textbf{Choose an $m\times n$ matrix, $A$, whose entries are polynomials} and do the following:
			\begin{enumerate}
				\item Apply \hyperref[sec:algorithm 2]{algorithm 2} on $A$.
				\item If $m>0$ and $n>0$, then do the following:
				\begin{enumerate}
					\item For $j$ going from $2$ to $\min(m,n)$, do the following:
					\begin{enumerate}
						\item Add row $j$ to row $1$ and let $A'$, a copy of $A$, be our working matrix.
						\item Apply \hyperref[sec:algorithm 1]{algorithm 1} on the submatrix of $A'$ formed by selecting row $1$ and columns $1$ and $j$ as if there were nothing in between.
						\item Verify that the execution of \hyperref[sec:algorithm 1]{algorithm 1} in (ii) manifested itself in a sequence of column operations to make $A'_{1,j}$ zero.
						\item Temporarily stepwise undo these column operations and do the following:
						\begin{enumerate}
							\item Verify that $A_{1,1}=p*A'_{1,1}$, where $p$ is some implicitly constructed polynomial.
							\item Verify that $A_{1,1}$ is a factor of all $A_{2,2},\cdots,A_{j-1,j-1}$.
							\item Therefore verify that $A'_{1,1}$ is also a factor of all $A_{2,2},\cdots,A_{j-1,j-1}$.
							\item Verify that $A_{j,j}=A_{1,j}=q*A'_{1,1}$, where $q$ is some implicitly constructed polynomial.
							\item Verify that $A'_{j,1}=r*A_{j,j}=r*A_{1,j}=rq*A'_{1,1}$, where $r$ is some implicitly constructed polynomial.
							\item Verify that $A'_{j,j}=t*A_{j,j}=t*A_{1,j}=tq*A'_{1,1}$, where $t$ is some implicitly constructed polynomial.
						\end{enumerate}
						\item Subtract $rq$ times row $1$ from row $j$.
						\item Now verify that $A'_{j,1}=0$.
						\item Now let $A$ be equal to our working matrix.
					\end{enumerate}
					\item Call (2) in-place on the submatrix formed by removing the first row and column.
					\item Verify that each entry on the diagonal after $A_{1,1}$ is a linear combination of multiples of $A_{1,1}$.
					\item Therefore verify that each entry on the diagonal is still be a multiple of $A_{1,1}$.
					\item \textbf{Let $A_{0,0}=1$.}
					\item \textbf{Verify that for all $1\le i\le\min(m,n)$, $A_{i,i}=u_iA_{i-1,i-1}$, where $u_i$ is a polynomial implicitly constructed above.}
				\end{enumerate}
				\item \textbf{Yield the tuple $\langle A\rangle$.}
			\end{enumerate}
		\subsection{Algorithm 4 (Associativity verification)}\label{sec:algorithm 4}
			\textbf{Choose an $m\times n$ matrix, A, an $n\times p$ matrix, B, and a $p\times q$ matrix C, all of whose entries are polynomials.} Now do the following:
			\begin{enumerate}
				\item Verify that $(AB)_{i,l}=\sum_{k=0}^{n-1} \left(A_{i,k}*B_{k,l}\right)$.
				\item Verify that $((AB)C)_{i,r}=\sum_{l=0}^{p-1} \left((AB)_{i,l}*C_{l,r}\right)=\sum_{l=0}^{p-1} \left(\sum_{k=0}^{n-1} \left(A_{i,k}*B_{k,l}\right)*C_{l,r}\right)$.
				\item Verify that $(BC)_{k,r}=\sum_{l=0}^{p-1}\left(B_{k,l}*C_{l,r}\right)$.
				\item Verify that $(A(BC))_{i,r}=\sum_{k=0}^{n-1}\left(A_{i,k}*(BC)_{k,r}\right)=\sum_{k=0}^{n-1}\left(A_{i,k}*\sum_{l=0}^{p-1}\left(B_{k,l}*C_{l,r}\right)\right)$.
				\item Verify that $(2)=\sum_{l=0}^{p-1} \left(\sum_{k=0}^{n-1} \left(A_{i,k}*B_{k,l}*C_{l,r}\right)\right)=\sum_{k=0}^{n-1} \left(\sum_{l=0}^{p-1} \left(A_{i,k}*B_{k,l}*C_{l,r}\right)\right)=\sum_{k=0}^{n-1}\left(A_{i,k}*\sum_{l=0}^{p-1}\left(B_{k,l}*C_{l,r}\right)\right)=(4)$.
				\item \textbf{Therefore verify that $(AB)C=A(BC)$.}
			\end{enumerate}
		\subsection{Algorithm 5 (Extended Smith normal form construction)}\label{sec:algorithm 5}
			\textbf{Choose an $m\times n$ matrix, $A$, whose entries are polynomials} and do the following:
			\begin{enumerate}
				\item Make a singleton list containing one item, the chosen matrix $A$.
				\item Augment \hyperref[sec:algorithm 3]{algorithm 3} so that each time a polynomial $x$ times a column $i$ is added onto column $j$, an $n\times n$ matrix that only has $1$s on its diagonal, and such that the only non-zero entry off its diagonal is $x$ at position $(i,j)$, is appended onto the list.
				\item Also augment \hyperref[sec:algorithm 3]{algorithm 3} so that each time a polynomial $x$ times a row $i$ is added onto row $j$, an $n\times n$ matrix that only has $1$s on its diagonal, and such that the only non-zero entry off its diagonal is $x$ at position $(j,i)$, is prepended onto the list.
				\item Now run \hyperref[sec:algorithm 3]{algorithm 3} on the matrix $A$.
				\item Let $D$ be the diagonal matrix produced by \hyperref[sec:algorithm 3]{algorithm 3}.
				\item\textbf{Verify that $D$ equals the product, in-order, of the matrices in the list.}
				\item Let $M$ be the product of the sublist whose entries are all those that precede $A$.
				\item Let $N$ be the product of the sublist whose entries are all those that follow the original $A$.
				\item\textbf{Verify that $D=MAN$.}
				\item \textbf{Yield the tuple $\langle D,M,N\rangle$.}
			\end{enumerate}
		\subsection{Algorithm 6 (Inverse extended Smith normal form construction)}\label{sec:algorithm 6}
			\textbf{Choose an $m\times n$ matrix, $A$, whose entries are polynomials} and do the following:
			\begin{enumerate}
				\item Run \hyperref[sec:algorithm 3]{algorithm 3} on matrix $A$ to get the matrix $D$.
				\item Now, starting work with the matrix $D$, iterate through the row and column operations of \hyperref[sec:algorithm 3]{algorithm 3} starting from the last and ending with the first:
					\begin{enumerate}
						\item If the current operation adds polynomial $x$ times column $i$ to column $j$, then add $-x$ times column $i$ from column $j$ from our working matrix.
						\item If the current operation adds polynomial $x$ times row $i$ to row $j$, then add $-x$ times row $i$ from row $j$ from our working matrix.
					\end{enumerate}
				\item Our working matrix should now be equal to $A$.
				\item Make a singleton list containing one item, the original matrix $D$.
				\item Augment (2) in a way analogous to how \hyperref[sec:algorithm 5]{algorithm 5} augmented \hyperref[sec:algorithm 3]{algorithm 3}, but this time with the list provided in (4).
				\item Now run augmented (2).
				\item \textbf{Verify that $A$ equals the product, in-order, of the matrices in the list.}
				\item Let $M^{-1}$ be the product of the sublist whose entries are all those that precede $D$.
				\item Let $N^{-1}$ be the product of the sublist whose entries are all those that follow $D$.
				\item \textbf{Verify that $A=M^{-1}DN^{-1}$.}
				\item \textbf{Yield the tuple $\langle M^{-1}, D, N^{-1}\rangle$.}
			\end{enumerate}
		\subsection{Algorithm 7}\label{sec:algorithm 7}
			\textbf{Choose an $m\times n$ matrix, $A$, whose entries are polynomials} and do the following:
			\begin{enumerate}
				\item Run \hyperref[sec:algorithm 6]{algorithm 6} with $A$ as the choice matrix.
				\item Now, starting work with the matrix $I_n$, the $n\times n$ matrix with only $1$s on the diagonal, do the following operations:
					\begin{enumerate}
						\item Apply in-order the column operations that were applied on $A$
						\item Then apply in-order the column operations that were applied on $D$
					\end{enumerate}
				\item Verify that the application of operation (2) leaves $I_n$ unchanged.
				\item Verify that $(I_nN)N^{-1}$ evaluates to the same matrix produced by (2).
				\item \textbf{Therefore verify that $(I_nN)N^{-1}=I_n$.}
				\item \textbf{Therefore, using \hyperref[sec:algorithm 4]{algorithm 4}, verify that $I_n=(I_nN)N^{-1}=I_n(NN^{-1})=NN^{-1}$.}
			\end{enumerate}
			\textbf{Using similar computations, verify that $N^{-1}N=I_n$, and that $MM^{-1}=M^{-1}M=I_m$.}
		\subsection{Algorithm 8 (Determinant calculation)}\label{sec:algorithm 8}
			\textbf{Choose an $m\times m$ matrix, $A$, whose entries are polynomials} and do the following:
			\begin{enumerate}
				\item If $m$ equals $0$, then do the following:
				\begin{enumerate}
					\item \textbf{Yield the tuple $\langle 1\rangle$.}
				\end{enumerate}
				\item Otherwise, do the following:
				\begin{enumerate}
					\item Let $a$ be the sum of $m$ terms where, counting from $i=0$, the $i^{th}$ term is ($(-1)^{i}A_{i,1}$ times the result of applying \hyperref[sec:algorithm 8]{algorithm 8} on the submatrix formed by removing the first column and $i^{th}$ row from A).
					\item \textbf{Yield the tuple $\langle a\rangle$.}
				\end{enumerate}
			\end{enumerate}
			\textbf{We will use the notation $\det_{I,J}(A)$ to refer to the invocation of \hyperref[sec:algorithm 8]{algorithm 8} on the square submatrix created by selecting the sequence of rows $I$ and the sequence of columns $J$ from $A$. If neither $I$ nor $J$ are specified, then the invocation shall happen directly on $A$.}
		\subsection{Algorithm 9 (Multilinearity verification)}\label{sec:algorithm 9}
			\textbf{Choose a polynomial $p$. Choose two $m\times 1$ matrices, $B$ and $C$, whose entries are polynomials. Choose an $m\times m$ matrix, $A$, whose entries are polynomials and that is such that its $i^{th}$ column is $B+pC$. The value $\det(A)$ can be evaluated as follows:}
			\begin{enumerate}
				\item Considering every element of $A$ to be a unit, verify that fully expanding out $\det(A)$ yields an alternating sum of $m!$ terms, where each term is the product of $m$ entries of $A$, where each entry has a distinct row and distinct column.
				\item Distribute out the entries of the $i^{th}$ column occuring in this alternating sum.
				\item Verify that the outcome of (2) is $m!*2$ terms.
				\item Reorder the terms so that the currently odd ones come first and the currently even ones come last.
				\item Verify that $\det(A)=\det(A') + \det(A'')$ where $A'$ is $A$ with the $i^{th}$ column replaced by $B$ and $A''$ is $A$ with the $i^{th}$ column replaced by $pC$.
				\item Reorder the factors of each term in $\det(A'')$ to bring $p$ to the front.
				\item Now verify that the $\det(A'')=p\det(A''')$, where $A'''$ is $A$ with the $i^{th}$ column replaced by $C$.
				\item \textbf{Therefore verify that $\det(A)=\det(A') + p\det(A''')$.}
			\end{enumerate}
			\textbf{Make an analogous algorithm for cases when a given row is the sum of two $1\times m$ matrices.}
		\subsection{Algorithm 10 (Alternation verification)}\label{sec:algorithm 10}
			\textbf{Choose an $m\times m$ matrix, $A$, whose entries are polynomials. Choose a row $1<i\le m$. To evaluate $\det(A')$ where $A'$ is $A$ with rows $i-1$ and $i$ swapped, do the following:}
			\begin{enumerate}
				\item Fully expand out $\det A$ into $m!$ terms and then do the same for $\det A'$.
				\item For each of the $m!$ ways to select $m$ rows from $A$, let $r=(r_1, r_2, \cdots, r_m)$ be the rows selected corresponding to the columns $1, 2, \cdots, m$ respectively, and do the following:
				\begin{enumerate}
					\item Verify that the values selected by $r$ in $A$ are the same as the values selected by $r'$ in $A'$, where $r'$ is obtained by swapping the values $i-1$ and $i$ in the sequence $r$.
					\item Execute \hyperref[sec:algorithm 8]{algorithm 8} on $A$, and consider the execution path that produces the term corresponding to the row selections $r$. Ditto for $A'$ and $r'$.
					\item Let $k$ be the lesser of the indicies of the values $i-1$ and $i$ in the sequence $r$.
					\item Verify that the signs attached to $A_{r_1,1},\cdots,A_{r_{k-1},k-1}$ are the same as the signs attached to $A'_{r'_1,1},\cdots,A'_{r'_{k-1},k-1}$.
					\item Verify that indices $r_k$ and $r'_k$ identify adjacent rows in the remaining respective submatrices of $A$ and $A'$.
					\item Therefore verify that the signs then attached to $A_{r_k}$ and $A'_{r'_k}$ are opposite.
					\item Verify that after the removal of $r_k$ and $r'_k$ from their respective submatrices, the submatrices left are identical.
					\item Therefore verify that the signs attached to $A_{r_{k+1},k+1},\cdots,A_{r_{m-1},m-1}$ are the same as the signs attached to $A'_{r'_{k+1},k+1},\cdots,A'_{r'_{m-1},m-1}$.
					\item Therefore verify that the term corresponding to the row selections $r'$ in the full expansion of $\det(A')$ has the opposite sign to the term corresponding to the row selections $r$ in the full expansion of $\det(A)$.
				\end{enumerate}
				\item Therefore verify that every term in the full expansion of $\det(A)$ corresponds to a unique negated version of itself in the full expansion of $\det(A')$.
				\item\textbf{Therefore verify that $\det(A')=-\det(A)$.}
			\end{enumerate}
			\textbf{Make a simpler algorithm to verify that column swaps cause sign alternations.}
		\subsection{Algorithm 11}\label{sec:algorithm 11}
			\textbf{Choose integers $1<i\le m$. Choose an $m\times m$ matrix, $A$, whose entries are polynomials, and such that columns $i-1$ and $i$ are the same. To evaluate $\det(A)$, do the following:}
			\begin{enumerate}
				\item If $\det(A)\ne 0$, then do the following:
				\begin{enumerate}
					\item Let $A'$ be $A$ with columns $i$ and $i-1$ swapped.
					\item Verify that $A'$ equals $A$.
					\item Therefore verify that $\det(A')=\det(A)$.
					\item Using \hyperref[sec:algorithm 10]{algorithm 10}, also verify that $\det(A')=-\det(A)$.
					\item \textbf{Abort algorithm.}
				\end{enumerate}
				\item Otherwise, do the following:
				\begin{enumerate}
					\item \textbf{Verify that $\det(A)=0$.}
				\end{enumerate}
			\end{enumerate}
			\textbf{Make an analogous algorithm to verify that matrix choices with repeated rows yield determinants equal to zero.}
		\subsection{Algorithm 12}\label{sec:algorithm 12}
			\textbf{Choose a column index $1\le i\le m$. Choose a positive integer $j$. Choose an $m\times m$ matrix, $A$, whose entries are polynomials. To evaluate $\det(A')$ where $A'$ is $A$ but with column $i$ moved $j$ places, do the following:}
			\begin{enumerate}
				\item Let $A_i=A$.
				\item For $k=i+1$ to $k=i+j$, do the following:
				\begin{enumerate}
					\item Let $A_k$ be obtained by swapping columns $k-1$ and $k$ of $A_{k-1}$.
					\item Using \hyperref[sec:algorithm 10]{algorithm 10}, verify that $\det(A_k)=-\det(A_{k-1})$.
				\end{enumerate}
				\item Verify that $A'=A_{i+j}$.
				\item \textbf{Therefore verify that $\det(A')=\det(A_{i+j})=(-1)^1\det(A_{i+j-1})=\cdots=(-1)^j\det(A_{i})=(-1)^j\det(A)$.}
			\end{enumerate}
			\textbf{Make an analogous algorithm that verifies that $\det(A')=(-1)^j\det(A)$ when a non-positive integer, $j$, is chosen.}
			
			\textbf{Also make an analogous algorithm that does the verification for moved rows.}
		\subsection{Algorithm 13 (Compound matrix calculation)}\label{sec:algorithm 13}
			\textbf{Choose an $m\times n$ matrix, $A$, of polynomials and choose an integer $1\le k\le\min(m,n)$.} Yield a tuple comprising the $\binom{m}{k}\times\binom{n}{k}$ matrix constructed as follows:
			\begin{enumerate}
				\item The rows are labeled by the colexicographically sorted list of increasing length-$k$ sequences whose elements are picked from the first $m$ positive integers.
				\item The columns are labeled by the colexicographically sorted list of increasing length-$k$ sequences whose elements are picked from the first $n$ positive integers.
				\item For each row label $I$: For each column label $J$: Let the entry at position $(I,J)$ be $\det_{I,J}(A)$.
			\end{enumerate}
			\textbf{We will use the notation $C_k(A)$ to refer to an invocation of \hyperref[sec:algorithm 13]{algorithm 13} on the matrix $A$.}
		\subsection{Algorithm 14 (Compound matrix of identity calculation)}\label{sec:algorithm 14}
			\textbf{Choose two integers $0\le k\le m$.} To evaluate $C_k(I_m)$, iterate through all its entries and do the following:
			\begin{enumerate}
				\item \hyperref[sec:algorithm 13]{Algorithm 13} requires us to form a submatrix $B$ of $A$ using the rows listed in the row label, and the columns listed in the column label.
				\item If the entry is diagonal, then do the following:
				\begin{enumerate}
					\item Verify that the $k$ column indices we just selected from $I_m$ are the same as the $k$ rows indices we just selected from $I_m$.
					\item Therefore verify that the diagonal positions of $B$ correspond to a diagonal entry of $I_m$.
					\item Therefore verify that the diagonal positions of $B$ are $1$.
					\item Also verify that the non-diagonal positions of $B$ correspond to a non-diagonal entry of $I_m$.
					\item Therefore verify that the non-diagonal positions of $B$ are $0$.
					\item Therefore the submatrix $B$ should equal $I_k$.
					\item \textbf{Therefore using \hyperref[sec:algorithm 8]{algorithm 8}, verify that $\lvert B\rvert=1$.}
				\end{enumerate}
				\item Otherwise, do the following:
				\begin{enumerate}
					\item Verify that the $k$ column indices we just selected from $I_m$ are different from the $k$ row indices that we just selected from $I_m$.
					\item Let $i$ be a selected row index that is not also a column index.
					\item Iterate through the columns of the row in the submatrix $B$ that corresponds to row $i$ of $I_m$ and do the following:
					\begin{enumerate}
						\item Let $j$ be the column index of $I_m$ to which this column corresponds.
						\item Using (3b), verify that $i\ne j$.
						\item Verify that $(I_m)_{i,j}=0$.
						\item Therefore verify that this entry is $0$.
					\end{enumerate}
					\item Verify that the row in the submatrix $B$ that corresponds to row $i$ of $I_m$ is entirely zero.
					\item \textbf{Therefore using \hyperref[sec:algorithm 8]{algorithm 8}, verify that $\lvert B\rvert=0$.}
				\end{enumerate}
				\item \textbf{Therefore verify that $C_k(I_m)=I_{\binom{m}{k}}$.}	
			\end{enumerate}
		\subsection{Algorithm 15}\label{sec:algorithm 15}
			\textbf{Choose an integer $1\le k\le\min(m,n)$. Choose an $m\times m$ matrix, $A$, whose diagonal entries are $1$s, and such that the only entry off the diagonal is the polynomial $p$ at $(i,j)$. Also choose an $m\times n$ matrix, $B$, whose entries are polynomials.} To evaluate $C_k(AB)$, do the following:
			\begin{enumerate}
				\item Verify that $AB$ equals $B$, but with its row $i$ having $p$ times $B$'s row $j$ added to it.
				\item Go through the row labels, $I$, of $C_k(AB)$ and do the following:
				\begin{enumerate}
					\item If $i\notin I$, then do the following:
					\begin{enumerate}
						\item For $l\in I$: For $j=1$ to $j=n$: Verify that $(AB)_{l,j}=B_{l,j}$.
						\item Therefore for each column label $J$, verify that ${C_k(AB)}_{I,J}=\det_{I,J}(AB)=\det_{I,J}(B)={C_k(B)}_{I,J}$.
						\item \textbf{Therefore verify that row $I$ of $C_k(AB)$ equals row $I$ of $C_k(B)$.}
					\end{enumerate}
					\item Otherwise, if $i\in I$, then:
					\begin{enumerate}
						\item Let $I'$ be $I$ but with an in-place replacement of $i$ by $j$.
						\item For each column label $J$: Using \hyperref[sec:algorithm 9]{algorithm 9}, verify that ${C_k(AB)}_{I,J}=\det_{I,J}(AB)=\det_{I,J}(B)+p*\det_{I',J}(B)$.
						\item Let $l$ be the signed number of places that the $j$ introduced above needs to be moved in order to make $I'$ a non-increasing sequence.
						\item Let $I''$ be obtained from $I'$ by moving the integer $j$ in $I'$ by $l$ places.
						\item For each column label $J$: Using \hyperref[sec:algorithm 12]{algorithm 12}, verify that $\det_{I',J}(B)=(-1)^l\det_{I'',J}(B)$.
						\item Therefore for each column label $J$: Verify that ${C_k(AB)}_{I,J}=\det_{I,J}(B)+p*\det_{I',J}(B)=\det_{I,J}(B)+(-1)^lp*\det_{I'',J}(B)$.
						\item If $j\notin I$, then do the following:
						\begin{enumerate}
							\item Verify that $I''$ is a decreasing sequence.
							\item Verify that $I''$ is a row label of $C_k(B)$.
							\item Therefore for each column label $J$: Verify that ${C_k(AB)}_{I,J}=\det_{I,J}(B)+(-1)^lp*\det_{I'',J}(B)={C_k(B)}_{I,J}+(-1)^lp*{C_k(B)}_{I'',J}$.
							\item \textbf{Therefore verify that row $I$ of $C_k(AB)$ is row $I$ of $C_k(B)$ plus $(-1)^lp$ times row $I''$ added to it.}
						\end{enumerate}
						\item Otherwise if $j\in I$, do the following:
						\begin{enumerate}
							\item Verify that the sequence $I''$ contains two consecutive $j$s.
							\item Using \hyperref[sec:algorithm 11]{algorithm 11}, verify that $\det_{I'',J}(B)=0$.
							\item Therefore verify that ${C_k(AB)}_{I,J}=\det_{I,J}(B)={C_k(B)}_{I,J}$.
							\item \textbf{Therefore verify that row $I$ of $C_k(AB)$ is row $I$ of $C_k(B)$.}
						\end{enumerate}
					\end{enumerate}
				\end{enumerate}
				\item Let $D$ be the matrix of row operations implicitly applied in (2).
				\item \textbf{Verify that $C_k(AB)=DC_k(B)$.}
			\end{enumerate}
		\subsection{Algorithm 16}\label{sec:algorithm 16}
			\textbf{Choose an $m\times n$ matrix, $A$, whose only entries are polynomials on the diagonal positions. Also choose an $n\times n$ matrix, $B$, whose entries are polynomials. Also choose an integer $0\le k\le\min(m,n)$.} To evaluate $C_k(AB)$, do the following:
			\begin{enumerate}
				\item Verify that $AB$ equals the top $m\times n$ submatrix of $B$ with each row $i$ multiplied by $A_{i,i}$.
				\item Go through the row labels, $I$, of $C_k(AB)$ and do the following:
				\begin{enumerate}
					\item Let $(r_1, r_2, \cdots, r_k)=I$.
					\item If $r_k\le n$, then do the following:
					\begin{enumerate}
						\item Using \hyperref[sec:algorithm 13]{algorithm 13}, verify that every element of $I$ is less than or equal to $n$.
						\item Let $A_0=A$.
						\item For $i=1$ to $i=k$: Let $A_i$ equal $A_{i-1}$ but with position $(r_i,r_i)$ set to $1$.
						\item For each column label $J$: Repeatedly using \hyperref[sec:algorithm 9]{algorithm 9}, verify that ${C_k(AB)}_{I,J}=\det_{I,J}(AB)=\det_{I,J}(A_0B)=A_{r_1,r_1}\det_{I,J}(A_1B)=A_{r_1,r_1}A_{r_2,r_2}\det_{I,J}(A_2B)=\cdots=A_{r_1,r_1}A_{r_2,r_2}\cdots A_{r_k,r_k}\det_{I,J}(A_kB)=A_{r_1,r_1}A_{r_2,r_2}\cdots A_{r_k,r_k}\det_{I,J}(B)=A_{r_1,r_1}A_{r_2,r_2}\cdots A_{r_k,r_k}{C_k(B)}_{I,J}$.
						\item \textbf{Therefore verify that row $I$ of $C_k(AB)$ is $A_{r_1,r_1}A_{r_1,r_1}\cdots A_{r_k,r_k}$ times row $I$ of $C_k(B)$.}
					\end{enumerate}
					\item Otherwise if $r_k>n$, then do the following:
					\begin{enumerate}
						\item Verify that row $r_k$ of $A$ is zero.
						\item Therefore verify that row $r_k$ of $AB$ is zero.
						\item Therefore verify that the last row of the submatrix obtained by selecting rows $I$ from $AB$ is zero.
						\item Therefore using \hyperref[sec:algorithm 8]{algorithm 8}, for each column label $J$: verify that ${C_k(AB)}_{I,J}=0$.
						\item \textbf{Therefore verify that row $I$ of $C_k(AB)$ is zero.}
					\end{enumerate}
				\end{enumerate}
				\item Let $D$ be the matrix of row operations implicitly applied in (2).
				\item \textbf{Verify that $D$ is diagonal.}
				\item \textbf{Verify that $C_k(AB)=DC_k(B)$.}
			\end{enumerate}
		\subsection{Algorithm 17}\label{sec:algorithm 17}
			\textbf{Choose an integer $1\le k\le\min(m,n)$. Choose an $m\times m$ matrix, $A$, whose diagonal entries are $1$s, and such that the only entry off the diagonal is the polynomial $p$ at $(i,j)$. Also choose an $m\times n$ matrix, $B$, whose entries are polynomials.} To evaluate $C_k(AB)$, do the following:
			\begin{enumerate}
				\item Execute \hyperref[sec:algorithm 15]{algorithm 15} on matrices $A$ and $I_m$. Let $D$ be the matrix constructed.
				\item Verify that $C_k(AI_m)=DC_k(I_m)$.
				\item Using \hyperref[sec:algorithm 14]{algorithm 14}, verify that $C_k(A)=C_k(AI_m)=DC_k(I_m)=DI_{\binom{m}{k}}=D$.
				\item Execute \hyperref[sec:algorithm 15]{algorithm 15} on matrices $A$ and $B$. Let $D'$ be the matrix constructed.
				\item Verify that $C_k(AB)=D'C_k(B)$.
				\item Verify that $D'=D=C_k(A)$.
				\item \textbf{Therefore verify that $C_k(AB)=C_k(A)C_k(B)$.}
			\end{enumerate}
			\textbf{Make an analogous algorithm to verify that $C_k(BA)=C_k(B)C_k(A)$.}
			
			\textbf{Using \hyperref[sec:algorithm 16]{algorithm 16}, make an algorithm similar to above but that works when a diagonal matrix of polynomials, $A$, is instead chosen.}
		\subsection{Algorithm 18 (Compound matrix of matrix product calculation)}\label{sec:algorithm 18}
			\textbf{Choose an integer $0\le k\le\min(m,n,p)$. Choose an $m\times n$ matrix, $A$, whose only entries are polynomials. Also choose an $n\times p$ matrix, $B$, whose entries are polynomials.} To evaluate $C_k(AB)$, do the following:
			\begin{enumerate}
				\item Execute \hyperref[sec:algorithm 6]{algorithm 6} on $A$. Let ${M^{-1}}_i$ be the $i^{th}$ element of the sublist corresponding to $M$ and ${N^{-1}}_i$ be the $i^{th}$ element of the sublist corresponding to $N$.
				\item Verify that $A={M^{-1}}_1\cdots {M^{-1}}_mD{N^{-1}}_1\cdots {N^{-1}}_n$.
				\item \textbf{Using repeated applications of \hyperref[sec:algorithm 17]{algorithm 17}, verify that $C_k(AB)=C_k({M^{-1}}_1\cdots {M^{-1}}_mD{N^{-1}}_1\cdots {N^{-1}}_nB)=C_k({M^{-1}}_1)\cdots C_k({M^{-1}}_m)*C_k(D)*C_k({N^{-1}}_1)\cdots C_k({N^{-1}}_n)C_k(B)=C_k({M^{-1}}_1\cdots {M^{-1}}_mD{N^{-1}}_1\cdots {N^{-1}}_n)C_k(B)=C_k(A)C_k(B)$.}
			\end{enumerate}
		\subsection{Algorithm 19 (Determinant equals product of diagonal entries verification)}\label{sec:algorithm 19}
			\textbf{Choose an $m\times m$ matrix, $A$, whose entries are polynomials.} To evaluate $\det(A)$, do the following:
			\begin{enumerate}
				\item Execute \hyperref[sec:algorithm 6]{algorithm 6} on $A$. Let ${M^{-1}}_i$ be the $i^{th}$ element of the sublist corresponding to $M$ and ${N^{-1}}_i$ be the $i^{th}$ element of the sublist corresponding to $N$.
				\item Verify that $A={M^{-1}}_1\cdots {M^{-1}}_mD{N^{-1}}_1\cdots {N^{-1}}_n$.
				\item \textbf{Using \hyperref[sec:algorithm 8]{algorithm 8} and \hyperref[sec:algorithm 18]{algorithm 18}, verify that $\det(A)=C_m(A)=C_m({M^{-1}}_0\cdots {M^{-1}}_mD{N^{-1}}_0\cdots {N^{-1}}_n)=C_m({M^{-1}}_0)\cdots C_m({M^{-1}}_m)C_m(D)C_m({N^{-1}}_0)\cdots C_m({N^{-1}}_n)=1\cdots 1C_m(D)1\cdots 1=C_m(D)=\det(D)$.}
				\item \textbf{Using \hyperref[sec:algorithm 8]{algorithm 8}, verify that $\det(D)$ is the product of the diagonal entries of $D$.}
			\end{enumerate}
		\subsection{Algorithm 20 (Transpose calculation)}\label{sec:algorithm 20}
			\textbf{Choose an $m\times n$ matrix, $A$, whose entries are polynomials} and do the following:
			\begin{enumerate}
				\item Make an $n\times m$ matrix, $A^T$.
				\item For $i=1$ to $i=n$:
				\begin{enumerate}
					\item For $j=1$ to $j=m$:
					\begin{enumerate}
						\item Let ${A^T}_{i,j}=A_{j,i}$.
					\end{enumerate}
				\end{enumerate}
				\item \textbf{Yield the tuple $\langle A^T\rangle$.}
			\end{enumerate}
			\textbf{The notation $A^T$ shall be used to refer to the result of invoking \hyperref[sec:algorithm 20]{algorithm 20} on a matrix $A$.}
		\subsection{Algorithm 21 (Transpose of product verification)}\label{sec:algorithm 21}
			\textbf{Choose an $m\times n$ matrix, $A$, and an $n\times k$ matrix, $B$, both of whose entries are polynomials.} Now do the following:
			\begin{enumerate}
				\item Verify that $B^TA^T$ and $(AB)^T$ have dimensions $k\times m$.
				\item For $i=1$ to $i=k$:
				\begin{enumerate}
					\item For $j=1$ to $j=m$:
					\begin{enumerate}
						\item Using \hyperref[sec:algorithm 20]{algorithm 20}, verify that $(B^TA^T)_{i,j}=\sum_{l=0}^n B_{l,i}A_{j,l}=\sum_{l=0}^n A_{j,l}B_{l,i}=(AB)_{j,i}=((AB)^T)_{i,j}$.
					\end{enumerate}
				\end{enumerate}
				\item \textbf{Therefore verify that $B^TA^T=(AB)^T$.}
			\end{enumerate}
		\subsection{Algorithm 22 (Determinant of transpose verification)}\label{sec:algorithm 22}
			\textbf{Choose an $m\times m$ matrix, $A$, whose entries are polynomials.} To evaluate $\det(A^T)$, do the following:
			\begin{enumerate}
				\item Execute \hyperref[sec:algorithm 6]{algorithm 6} on $A$. Let ${M^{-1}}_i$ be the $i^{th}$ element of the sublist corresponding to $M$ and ${N^{-1}}_i$ be the $i^{th}$ element of the sublist corresponding to $N$.
				\item Verify that $A={M^{-1}}_1\cdots {M^{-1}}_mD{N^{-1}}_1\cdots {N^{-1}}_n$.
				\item \textbf{Therefore using \hyperref[sec:algorithm 19]{algorithms 19} and \hyperref[sec:algorithm 20]{20}, verify that $\det(A^T)=\det(({M^{-1}}_0\cdots {M^{-1}}_mD{N^{-1}}_0\cdots {N^{-1}}_n)^T)=\det(({N^{-1}}_n)^T\cdots({N^{-1}}_0)^TD^T({M^{-1}}_m)^T\cdots({M^{-1}}_0)^T)=\det(D^T)=\det(D)=\det({M^{-1}}_1\cdots {M^{-1}}_mD{N^{-1}}_1\cdots {N^{-1}}_n)=\det(A)$.}
			\end{enumerate}
		\subsection{Algorithm 23 (Compound matrix of transpose verification)}\label{sec:algorithm 23}
			\textbf{Choose an $m\times n$ matrix, $A$, whose entries are polynomials and an integer $0\le k\le\min(m,n)$.} Now do the following:
			\begin{enumerate}
				\item Compute the value $C_k(A^T)$.
				\item For each row label $I$ of $C_k(A^T)$, do the following:
				\begin{enumerate}
					\item For each column label $J$ of $C_k(A^T)$, do the following:
					\begin{enumerate}
						\item Let $B$ be a submatrix of $A^T$ formed by selecting the rows $I$ and the columns $J$.
						\item Using \hyperref[sec:algorithm 22]{algorithm 22}, verify that $(C_k(A^T))_{I,J}=\det_{I,J}(A^T)=\det(B)=\det(B^T)=\det_{J,I}(A)=(C_k(A))_{J,I}$.
					\end{enumerate}
				\end{enumerate}
				\item \textbf{Therefore verify that $(C_k(A))^T=(C_k(A^T))$.}
			\end{enumerate}
		\subsection{Algorithm 24 (Linear system solution construction)}\label{sec:algorithm 24}
			\textbf{Choose an $m\times n$ matrix, $A$, and an $m\times p$ matrix, $B$, both of whose entries are only rationals} and do the following:
			\begin{enumerate}
				\item Execute \hyperref[sec:algorithm 6]{algorithm 6} on $A$ and let $\langle M^{-1},D,N^{-1}\rangle$ receive the result.
				\item Verify that $A=M^{-1}DN^{-1}$.
				\item Verify that $M^{-1}$, $D$, and $N^{-1}$ contain only rational entries.
				\item If the indices of the rows of $D$ that are entirely zero are also the indices of the rows of $MB$ that are entirely zero, then:
				\begin{enumerate}
					\item Let $C$ be an $n\times p$ matrix with its $i^{th}$ row given as follows:
					\begin{enumerate}
						\item If $D_{i,i}\ne 0$, then do the following:
						\begin{enumerate}
							\item Let row $i$ be row $i$ of $MB$ divided by $D_{i,i}$.
						\end{enumerate}
						\item Otherwise, do the following:
						\begin{enumerate}
							\item \textbf{Choose $p$ rational numbers to fill up the row.}
						\end{enumerate}
					\end{enumerate}
					\item Verify that $DC=MB$.
					\item Let $E$ be $NC$.
					\item \textbf{Therefore using \hyperref[sec:algorithm 7]{algorithm 7}, verify that $AE=M^{-1}DN^{-1}E=M^{-1}DN^{-1}NC=M^{-1}DI_nC=M^{-1}DC=M^{-1}MB=I_mB=B$.}
					\item \textbf{Yield the tuple $\langle E\rangle$.}
				\end{enumerate} 
			\end{enumerate}
			\textbf{The notation $A\backslash B$ shall be used to refer to the result, $E$, of invoking \hyperref[sec:algorithm 24]{algorithm 24} on matrices $A$ and $B$.}
			
			\textbf{Make an analogous algorithm that yields an $F$ such that $FA=B$. The notation $B/A$ shall be used to refer to the $F$ yielded by invoking this algorithm.}
		\subsection{Algorithm 25}\label{sec:algorithm 25}
			\textbf{Choose two $m\times m$ matrices, $A$ and $B$, both of whose entries are only rationals, such that $AB=I_m$} and do the following:
			\begin{enumerate}
				\item Execute \hyperref[sec:algorithm 6]{algorithm 6} on $B$ and let $\langle M^{-1},D,N^{-1}\rangle$ receive the result.
				\item Verify that $B=M^{-1}DN^{-1}$.
				\item If $D$ has a zero on its diagonal, then do the following:
				\begin{enumerate}
					\item Using \hyperref[sec:algorithm 19]{algorithm 19}, verify that $\det(I_m)=\det(AB)=\det(A)\det(B)=\det(A)\det(D)=\det(A)*0=0$.
					\item Using \hyperref[sec:algorithm 8]{algorithm 8}, verify that $\det(I_m)=1^m=1$.
					\item Verify that $0=1$.
					\item \textbf{Abort algorithm.}
				\end{enumerate}
				\item Otherwise do the following:
				\begin{enumerate}
					\item Verify that $D$ does not have a zero on its diagonal.
					\item Verify that $B\backslash I_m=I_m(B\backslash I_m)=AB(B\backslash I_m)=A(B(B\backslash I_m))=AI_m=A$.
					\item \textbf{Therefore verify that $BA=B(B\backslash I_m)=I_m$.}
				\end{enumerate}
			\end{enumerate}
		\subsection{Algorithm 26}\label{sec:algorithm 26}
			\textbf{Choose $m\times m$ polynomial matrices $\langle B,M,M^{-1},A,N^{-1},N\rangle$ such that:
			\begin{enumerate}
				\item $B$ and $A$ are rational matrices.
				\item $MM^{-1}=I_m$.
				\item $N^{-1}N=I_m$.
				\item $xI_m-B=M(xI_m-A)N$.
			\end{enumerate}} and do the following:
			\begin{enumerate}
				\item Post-multiply both sides of (3) by $N^{-1}$.
				\item Verify that $(xI_m-B)N^{-1}=M(xI_m-A)NN^{-1}=M(xI_m-A)I_m=M(xI_m-A)$.
				\item Let $M_0x^b+M_1x^{b-1}+\cdots+M_bx^0=M$, where the $M_i$ are rational matrices.
				\item Now let $R_1=B^bM_0+B^{b-1}M_1+\cdots+B^0M_b$.
				\item Verify that $M-R_1=(xI_m-B)\sum_{k=1}^b (x^{k-1}I_mB^0+x^{k-2}I_mB^1+\cdots+x^0I_mB^{k-1})M_k$.
				\item Let $Q_1=\sum_{k=1}^b (x^{k-1}I_mB^0+x^{k-2}I_mB^1+\cdots+x^0I_mB^{k-1})M_k$.
				\item Verify that $M=(xI_m-B)Q_1+R_1$.
				\item Let ${N^{-1}}_0x^a+{N^{-1}}_1x^{a-1}+\cdots+{N^{-1}}_ax^0=N^{-1}$ where the ${N^{-1}}_j$ are rational matrices.
				\item Let $R_2={N^{-1}}_0A^a+{N^{-1}}_1A^{a-1}+\cdots+{N^{-1}}_aA^0$.
				\item Verify that $N^{-1}-R_2=\sum_{k=1}^a {N^{-1}}_k(x^{k-1}I_mA^0+x^{k-2}I_mA^1+\cdots+x^0I_mA^{k-1})(xI_m-A)$.
				\item Let $Q_2=\sum_{k=1}^a {N^{-1}}_k(x^{k-1}I_mA^0+x^{k-2}I_mA^1+\cdots+x^0I_mA^{k-1})$.
				\item Verify that $N^{-1}=Q_2(xI_m-A)+R_2$.
				\item By substituting $M$ and $N^{-1}$ into (2), verify that $(xI_m-B)(Q_2(xI_m-A)+R_2)=((xI_m-B)Q_1+R_1)(xI_m-A)$.
				\item By rearranging both sides, verify that $(xI_m-B)(Q_2-Q_1)(xI_m-A)=R_1(xI_m-A)-(xI_m-B)R_2$.
				\item By equating the coefficients of different powers of $x$ both sides, verify that $Q_2-Q_1=0_{m\times m}$.
				\item Verify that $R_1(xI_m-A)-(xI_m-B)R_2=(xI_m-B)(Q_2-Q_1)(xI_m-A)=(xI_m-B)0_{m\times m}(xI_m-A)=0_{m\times m}$.
				\item Therefore by adding $(xI_m-B)R_2$ to both sides, verify that $xR_1-R_1A=R_1(xI_m-A)=(xI_m-B)R_2=xR_2-BR_2$.
				\item By equating the coefficients of $x$ on both sides, verify that $R_1=R_2$.
				\item Therefore verify that $xR_1-R_1A=R_1(xI_m-A)=(xI_m-B)R_2=(xI_m-B)R_1=xR_1-BR_1$.
				\item Therefore by adding $-xR_1$ to both sides, verify that $-R_1A=-BR_1$
				\item Therefore by negating both sides, verify that $R_1A=BR_1$.
				\item In a similar way to (8), construct a polynomial matrix $Q_3$, and a rational matrix $R_3$ such that $M^{-1}=(xI-A)Q_3+R_3$.
				\item Verify that $I_m=MM^{-1}=((xI_m-B)Q_1+R_1)M^{-1}=(xI_m-B)Q_1M^{-1}+R_1M^{-1}=(xI_m-B)Q_1M^{-1}+R_1(xI-A)Q_3+R_1R_3=(xI_m-B)Q_1M^{-1}+(xI-B)R_1Q_3+R_1R_3=(xI_m-B)(Q_1M^{-1}+R_1Q_3)+R_1R_3$.
				\item By equating the powers of $x$ on both sides, verify that $Q_1M^{-1}+R_1Q_3=0$.
				\item By substituting zero for $Q_1M^{-1}+R_1Q_3$, \textbf{verify that $I_m=(xI_m-B)0_{m\times m}+R_1R_3=R_1R_3$.}
				\item \textbf{Therefore, verify that $B=BI_m=BR_1R_3=R_1AR_3$.}
				\item \textbf{Yield the pair $(R_1,R_3)$.}
			\end{enumerate}
		\subsection{Algorithm 27}\label{sec:algorithm 27}
			\textbf{Choose an $m\times n$ matrix, $A$, whose entries are polynomials. Choose two integers $1\le i,j\le m$ such that $i\ne j$} and do the following:
			\begin{enumerate}
				\item Subtract row $j$ from row $i$.
				\item Add row $i$ to row $j$.
				\item Subtract row $j$ from row $i$.
				\item \textbf{Verify that the matrix obtained is the same as the original $A$ with its $i^{th}$ row negated and swapped with the $j^{th}$ row.}
			\end{enumerate}
		\subsection{Algorithm 28}\label{sec:algorithm 28}
			\textbf{Choose an $m\times m$ matrix, $A$, whose entries are only rationals} and do the following:
			\begin{enumerate}
				\item Using \hyperref[sec:algorithm 8]{algorithm 8}, verify that $\det(xI-A)$ is a monic polynomial of degree $m$.
				\item Execute \hyperref[sec:algorithm 3]{algorithm 3} on the polynomial matrix $xI-A$ and let $\langle B\rangle$ be the result.
				\item If any of the diagonal entries of $B$ equal zero, then do the following:
				\begin{enumerate}
					\item Using \hyperref[sec:algorithm 8]{algorithm 8}, verify that $\det(B)=0$.
					\item Therefore using \hyperref[sec:algorithm 19]{algorithm 19}, verify that $\det(xI-A)=0$.
					\item \textbf{Abort algorithm.}
				\end{enumerate}
				\item Otherwise do the following:
				\begin{enumerate}
					\item Verify that none of the diagonal entries of $B$ equal zero.
					\item If the last diagonal entry of $B$ is not monic, then do the following:
					\begin{enumerate}
						\item Cognizant of the execution of \hyperref[sec:algorithm 3]{algorithm 3}, verify that the first $m-1$ diagonal entries of $B$ are monic.
						\item Using \hyperref[sec:algorithm 8]{algorithm 8}, verify that $\det(B)$ equals the product of the diagonal entries of $B$.
						\item Therefore verify that $\det(B)$ is not monic
						\item Therefore by \hyperref[sec:algorithm 19]{algorithm 19}, verify that $\det(xI-A)$ is not monic
						\item \textbf{Abort algorithm.}
					\end{enumerate}
					\item Otherwise do the following:
					\begin{enumerate}
						\item Verify that the last diagonal entry of $B$ is monic.
						\item \textbf{Verify that none of the diagonal entries of $B$ equal zero.}
						\item \textbf{Verify that all the diagonal entries of $B$ are monic.}
					\end{enumerate}
				\end{enumerate}
			\end{enumerate}
		\subsection{Algorithm 29 (Rational canonical form construction)}
			\textbf{Choose an $m\times m$ matrix, $A$, whose entries are only rationals} and do the following:
			\begin{enumerate}
				\item Execute \hyperref[sec:algorithm 5]{algorithm 5} on the polynomial matrix $xI_m-A$ and let $\langle B,,\rangle$ be the result.
				\item Execute \hyperref[sec:algorithm 28]{algorithm 28} on $A$.
				\item Let $E$ and $F$ be a $0\times 0$ matrices.
				\item \textbf{Now iterate through the diagonal entries $p=x^k+p_1x^{k-1}+p_2x^{k-2}+\cdots+p_kx^0$ of $B$ and:}
				\begin{enumerate}
					\item \textbf{If $k>0$:}
					\begin{enumerate}
						\item Make a $k\times k$ matrix $C$.
						\item Let $C$'s first $k-1$ columns be filled with the last $k-1$ columns of $I_k$.
						\item Let $C$'s last column from top to bottom be $-p_k, -p_{k-1},\cdots,-p_1$.
						\item Add $k$ columns filled with zeros to the right end of $E$.
						\item Add $k$ rows filled with zeros to the bottom end of $E$.
						\item Set the bottom-right corner of $E$ equal to $C$.
						\item Let the matrix $D=xI_k-C$ be our working matrix.
						\item For $i=k$ going down to $i=2$, add $x$ times row $i$ to row $i-1$.
						\item Verify that $D$'s first $k-1$ columns are now the last $k-1$ columns of $-I_k$.
						\item Verify that $D$'s last column is $p$ followed by some other polynomials.
						\item For $i=2$ going up to $i=k$, subtract $D_{i, k}$ times column $i-1$ from column $k$.
						\item Verify that $D$'s last column is now $p$ followed by zeros.
						\item For $i=2$ going up to $i=k$, negate row $i-1$ and exchange it with row $i$ using \hyperref[sec:algorithm 27]{algorithm 27}.
						\item Verify that $D$ is now a diagonal matrix whose first $k-1$ diagonal entries are $1$ and whose last diagonal entry is $p$.
						\item Add $k$ columns filled with zeros to the right end of $F$.
						\item Add k rows filled with zeros to the bottom end of $F$.
						\item Set the bottom-right corner of $F$ equal to $D$.
					\end{enumerate}
					\item Otherwise if $k=0$, then do the following:
					\begin{enumerate}
						\item Verify that $p$ is monic.
						\item Verify that $p=1$.
					\end{enumerate}
					\item Otherwise do the following:
					\begin{enumerate}
						\item \textbf{Abort algorithm.}
					\end{enumerate}
				\end{enumerate}
				\item Let $n$ be the sum of the positive degrees of the polynomials on the diagonal of $B$.
				\item Verify that $E$ and $F$ are $n\times n$ matrices.
				\item Using \hyperref[sec:algorithm 8]{algorithm 8}, verify that $n=\sum_{i=1}^{i=m}\deg(B_{i,i})=\deg(\det(B))=\deg(\det(xI-A))=m$.
				\item Therefore verify that $E$ and $F$ are $m\times m$ matrices.
				\item \textbf{Based on operations (4aviii) to (4axiii), use row and column operations to transform $xI_m-E$ into $F$.}
				\item \textbf{Yield the tuple $\langle E, F\rangle$.}
			\end{enumerate}
		\subsection{Algorithm 30}\label{sec:algorithm 30}
			\textbf{Choose an $m\times m$ matrix, $A$, whose entries are only rationals} and do the following:
			\begin{enumerate}
				\item Execute \hyperref[sec:algorithm 5]{algorithm 5} on the polynomial matrix $xI_m-A$ and let $\langle B,{M_3}^{-1},{N_3}^{-1}\rangle$ be the result.
				\item Verify that ${M_3}^{-1}(xI_m-A){N_3}^{-1}=B$.
				\item Execute \hyperref[sec:algorithm 6]{algorithm 6} on the polynomial matrix $xI_m-A$ and let $\langle M_3,B,N_3\rangle$ be the result.
				\item Verify that $xI_m-A=M_3BN_3$.
				\item Using \hyperref[sec:algorithm 7]{algorithm 7}, verify that $M_3{M_3}^{-1}=I_m$.
				\item Using \hyperref[sec:algorithm 7]{algorithm 7}, verify that ${N_3}^{-1}N_3=I_m$.
				\item Augment (9) in an analogous way to how \hyperref[sec:algorithm 5]{algorithm 5} augments \hyperref[sec:algorithm 3]{algorithm 3}. Let $\langle F,M_1,N_1\rangle$ receive the result once (9) has executed.
				\item Augment (9) in an analogous way to how \hyperref[sec:algorithm 6]{algorithm 6} augments \hyperref[sec:algorithm 3]{algorithm 3}. Let $\langle {M_1}^{-1},F,{N_1}^{-1}\rangle$ receive the result once (9) has executed.
				\item Execute \hyperref[sec:algorithm 29]{algorithm 29} on the matrix $A$ and let $\langle E,F\rangle$ receive the result.
				\item Verify that $M_1(xI_m-E)N_1=F$.
				\item Using \hyperref[sec:algorithm 7]{algorithm 7}, verify that $M_1{M_1}^{-1}=I_m$.
				\item Using \hyperref[sec:algorithm 7]{algorithm 7}, verify that ${N_1}^{-1}N_1=I_m$.
				\item Augment (18) in an analogous way to how \hyperref[sec:algorithm 5]{algorithm 5} augments \hyperref[sec:algorithm 3]{algorithm 3}. Let $\langle B,M_2,N_2\rangle$ receive the result once (18) has executed.
				\item Augment (18) in an analogous way to how \hyperref[sec:algorithm 6]{algorithm 6} augments \hyperref[sec:algorithm 3]{algorithm 3}. Let $\langle {M_2}^{-1},B,{N_2}^{-1}\rangle$ receive the result once (18) has executed.
				\item Verify that $F$ and $B$ have the same positive degree polynomials on their diagonals.
				\item Verify that the rest of the diagonals of $F$ and $B$ are $1$s.
				\item Verify that $F$ is the same as $B$ up to rearrangement of the diagonal entries.
				\item Rearrange a copy of $F$ to be $B$ by using diagonal entry swaps. In general, swap the $i^{th}$ and $j^{th}$ diagonal entries as follows:
				\begin{enumerate}
					\item Use \hyperref[sec:algorithm 27]{algorithm 27} to negate row $i$ and swap it with row $j$.
					\item Use an algorithm analogous to \hyperref[sec:algorithm 27]{algorithm 27} to negate column $i$ and swap it with column $j$.
				\end{enumerate}
				\item Verify that $M_2FN_2=B$.
				\item Using \hyperref[sec:algorithm 7]{algorithm 7}, verify that $M_2{M_2}^{-1}=I_m$.
				\item Using \hyperref[sec:algorithm 7]{algorithm 7}, verify that ${N_2}^{-1}N_2=I_m$.
				\item Let $M=M_3M_2M_1$.
				\item Let $N=N_1N_2N_3$.
				\item Verify that $xI_m-A=M_3BN_3=M_3M_2FN_2N_3=M_3M_2M_1(xI_m-E)N_1N_2N_3=M(xI_m-E)N$.
				\item Let $M^{-1}={M_1}^{-1}{M_2}^{-1}{M_3}^{-1}$.
				\item Let $N^{-1}={N_3}^{-1}{N_2}^{-1}{N_1}^{-1}$.
				\item Verify that $MM^{-1}=M_3M_2M_1{M_1}^{-1}{M_2}^{-1}{M_3}^{-1}=I_m$.
				\item Verify that $N^{-1}N={N_3}^{-1}{N_2}^{-1}{N_1}^{-1}N_1N_2N_3=I_m$.
				\item \textbf{Execute \hyperref[sec:algorithm 26]{algorithm 26} on the matrices $\langle A,M,M^{-1},E,N^{-1},N\rangle$. Let the tuple $\langle R_1,R_3\rangle$ be the result.}
				\item Verify that $A=R_1ER_3$.
				\item Verify that $R_1R_3=I_m$.
				\item \textbf{Yield the tuple $\langle R_1,E,R_3\rangle$.}
			\end{enumerate}
		\subsection{Algorithm 31 (Block matrix multiplication)}\label{sec:algorithm 31}
			\textbf{Choose an $m\times n$ matrix, $A$, and an $n\times k$ matrix, $B$, whose entries are polynomials. Choose integers $1\le a\le m$, $1\le b\le n$, and $1\le c\le k$.} Now do the following:
			\begin{enumerate}
				\item Let $C$ be the submatrix of $A$ that spans rows $1$ to $a-1$ and columns $1$ to $b-1$.
				\item Let $D$ be the submatrix of $A$ that spans rows $1$ to $a-1$ and columns $b$ to $n$.
				\item Let $E$ be the submatrix of $B$ that spans rows $1$ to $b-1$ and columns $1$ to $c-1$.
				\item Let $F$ be the submatrix of $A$ that spans rows $b$ to $n$ and columns $1$ to $c-1$.
				\item Multiply matrix $A$ by matrix $B$.
				\item For each $1\le i\le a-1$, do the following:
				\begin{enumerate}
					\item For each $1\le j\le c-1$, do the following:
						\begin{enumerate}
							\item Verify that $(AB)_{i,j}=\sum_{p=1}^n A_{i,p}B_{p,j}=\sum_{p=1}^{b-1} A_{i,p}B_{p,j}+\sum_{p=b}^n A_{i,p}B_{p,j}=\sum_{p=1}^{b-1} C_{i,p}E_{p,j}+\sum_{p=1}^{1+n-b} D_{i,p}F_{p,j}=(CE)_{i,j}+(DF)_{i,j}$.
						\end{enumerate}
				\end{enumerate}
				\item \textbf{Therefore verify that the top left $(a-1)\times(c-1)$ block of $AB$ equals $CE+DF$.}
				\item \textbf{Do similar computations to verify that the other three blocks of $AB$ are computed in an analogous way to multiplying two $2\times 2$ matrices.}
			\end{enumerate}
		\subsection{Algorithm 32}\label{sec:algorithm 32}
			\textbf{Choose an $m\times m$ matrix, $A$, whose entries are only rationals} and do the following:
			\begin{enumerate}
				\item Execute \hyperref[sec:algorithm 3]{algorithm 3} on the polynomial matrix $xI-A$ and let $B$ be the result.
				\item Execute \hyperref[sec:algorithm 28]{algorithm 28} on $A$.
				\item Let $r=x^t+r_1x^{t-1}+r_2x^{t-2}+\cdots+r_tx^0=B_{m,m}$.
				\item Execute \hyperref[sec:algorithm 30]{algorithm 30} on the matrix $A$ to obtain the matrix $E$.
				\item Verify that $R_3R_1=I_m$.
				\item Using \hyperref[sec:algorithm 30]{algorithm 30}, verify that $r(A)=A^t+r_1A^{t-1}+r_2A^{t-2}+\cdots+r_tA^0=(R_1ER_3)^t+r_1(R_1ER_3)^{t-1}+r_2(R_1ER_3)^{t-2}+\cdots+r_t(R_1ER_3)^0=R_1(E^t+r_1E^{t-1}+r_2E^{t-2}+\cdots+r_tE^0)R_3=R_1r(E)R_3$.
				\item For $i=0$ up to $i=t$, by repeated applications of \hyperref[sec:algorithm 31]{algorithm 31}, verify that $E^i$ evaluates to $E$ with all its diagonal blocks exponentiated to $i$.
				\item Therefore verify that $r(E)$ evaluates to $m\times m$ matrix whose diagonal blocks are the application of $r$ on the corresponding diagonal blocks of $E$.
				\item For $j=1$ to $j=m$:
				\begin{enumerate}
					\item If $\deg(B_{j,j})>0$, then do the following:
					\item Let $p=x^k+p_1x^{k-1}+p_2x^{k-2}+\cdots+p_kx^0=B_{j,j}$.
					\item Let $G$ be the corresponding $k\times k$ block on the diagonal of $E$.
					\item Let $e_i$ denote a $k\times 1$ matrix that is $0$, except for its $i^{th}$ entry which is $1$.
					\item Then by $G$'s construction, for $i=1$ up to $i=k$, verify that $G^{i-1}e_1=G^{i-2}e_2=\cdots=G^{0}e_i=e_i$.
					\item Let $0_{m\times n}$ denote an $m\times n$ matrix of zeros.
					\item Therefore, for $i=1$ up to $i=k$: Cognizant of the construction of $G$'s last column, verify that $p(G)e_i=(G^k+p_1G^{k-1}+p_2G^{k-2}+\cdots+p_kG^0)e_i=(G^k+p_1G^{k-1}+p_2G^{k-2}+\cdots+p_kG^0)G^{i-1}e_1=G^{i-1}(GG^{k-1}+p_1G^{k-1}+p_2G^{k-2}+\cdots+p_kG^0)e_1=G^{i-1}(Ge_k+p_1e_k+p_2e_{k-1}+\cdots+p_ke_1)=G^{i-1}0_{k\times 1}=0_{k\times 1}$.
					\item Therefore verify that $p(G)=0_{k\times k}$.
					\item Using the execution of \hyperref[sec:algorithm 3]{algorithm 3} in (1), verify that $r=B_{m,m}=B_{j,j}u_{j+1}u_{j+2}\cdots u_m=B_{j,j}q=pq$, where $q=u_{j+1}u_{j+2}\cdots u_m$.
					\item Therefore verify that $r(G)=p(G)q(G)=0_{k\times k}q(G)=0_{k\times k}$.
				\end{enumerate}
				\item Therefore verify that the diagonal blocks of $r(E)$ each equal $0_{i\times i}$, where $i$ is the size of each diagonal block.
				\item Therefore verify that $r(E)=0_{m\times m}$.
				\item \textbf{Therefore verify that $r(A)=R_1r(E)R_3=R_10_{m\times m}R_3=0_{m\times m}$.}
			\end{enumerate}
		\subsection{Algorithm 33}\label{sec:algorithm 33}
			\textbf{Choose an $m\times m$ matrix, $A$, whose entries are only rationals. Choose a non-zero polynomial $p=x^t+p_1x^{t-1}+p_2x^{t-2}+\cdots+p_tx^0$ such that $p(A)=0$.}
			\begin{enumerate}
				\item Execute \hyperref[sec:algorithm 3]{algorithm 3} on the polynomial matrix $xI-A$ and let $B$ be the result.
				\item Execute \hyperref[sec:algorithm 28]{algorithm 28} on $A$.
				\item Let $r=x^u+r_1x^{u-1}+r_2x^{u-2}+\cdots+r_ux^0=B_{m,m}$.
				\item Execute \hyperref[sec:algorithm 30]{algorithm 30} on the matrix $A$ to obtain the matrix $E$.
				\item Let $F$ be a $1\times 2$ matrix consisting in-order of $p$ and $r$.
				\item Execute \hyperref[sec:algorithm 5]{algorithm 5} on $F$ and let $\langle D,M,N\rangle$ receive the result.
				\item Execute \hyperref[sec:algorithm 6]{algorithm 6} on $F$ and let $\langle M^{-1},D,N^{-1}\rangle$ receive the result.
				\item Let $g=x^w+g_1x^{w-1}+g_2x^{w-2}+\cdots+g_wx^0=D_{1,1}$.
				\item If $\deg(p)\ne\deg(r)$, do the following:
				\begin{enumerate}
					\item Verify that the execution of the inner \hyperref[sec:algorithm 1]{algorithm 1} begun with repeated executions of (2c) to bring the degree of the higher degree polynomial down to that of the lower.
					\item Verify that afterwords, each iteration of (2) lowered the degree of either entry.
				\end{enumerate}
				\item Therefore verify that $\deg(g)\le\min(\deg(p),\deg(r))$.
				\item Therefore verify that $w\le u$.
				\item Verify that $D=MFN$.
				\item Therefore verify that $g=D_{1,1}=N_{1,1}p+N_{2,1}r$.
				\item Therefore using \hyperref[sec:algorithm 32]{algorithm 32}, verify that $g(A)=N_{1,1}(A)p(A)+N_{2,1}(A)r(A)=N_{1,1}(A)0_{m\times m}+N_{2,1}(A)0_{m\times m}=0_{m\times m}$.
				\item Using steps of \hyperref[sec:algorithm 32]{algorithm 32}, \hyperref[sec:algorithm 25]{algorithm 25}, and \hyperref[sec:algorithm 30]{algorithm 30}, verify that $g(E)=I_mg(E)I_m=R_3R_1g(E)R_3R_1=R_3g(A)R_1=R_30_{m\times m}R_1=0_{m\times m}$.
				\item Let $G$ be the $u\times u$ block in $E$ corresponding to the polynomial $r$ in $B$.
				\item Using steps of \hyperref[sec:algorithm 32]{algorithm 32}, verify that $g(G)=0_{u\times u}$.
				\item Therefore verify, using steps of \hyperref[sec:algorithm 32]{algorithm 32}, that $g(G)e_1=(G^w+g_1G^{w-1}+g_2G^{w-2}+\cdots+g_wG^0)e_1=Ge_w+g_1e_w+g_2e_{w-1}+\cdots+g_we_1$.
				\item If $w<u$, then:
				\begin{enumerate}
					\item Verify that $Ge_w+g_1e_w+g_2e_{w-1}+\cdots+g_we_1=e_{w+1}+g_1e_w+g_2e_{w-1}+\cdots+g_we_1=0_{u\times 1}$.
					\item Therefore verify that $1=0$.
					\item \textbf{Abort algorithm.}
				\end{enumerate}
				\item Otherwise it should be that $w=u$. Now:
				\begin{enumerate}
					\item Verify that $Ge_w+g_1e_w+g_2e_{w-1}+\cdots+g_we_1=Ge_u+g_1e_u+g_2e_{u-1}+\cdots+g_ue_1=0_{u\times 1}$.
					\item Therefore for $i=1$ to $i=u$, do the following:
					\begin{enumerate}
						\item Verify that $-r_i+g_i=0$.
						\item Therefore verify that $g_i=r_i$.
					\end{enumerate}
					\item \textbf{Therefore verify that $g=r$.}
					\item Verify that $F=M^{-1}DN^{-1}$.
					\item \textbf{Therefore verify that $p=F_{1,1}=D_{1,1}{N^{-1}}_{1,1}+D_{1,2}{N^{-1}}_{2,1}={N^{-1}}_{1,1}g+{N^{-1}}_{2,1}*0={N^{-1}}_{1,1}g={N^{-1}}_{1,1}r$.}
				\end{enumerate}
			\end{enumerate}
		\subsection{Algorithm 34 (Difference of powers)}\label{sec:algorithm 34}
			\textbf{Choose an integer $n>0$ and a formal polynomial $p=p_0x^n+p_1x^{n-1}+\cdots+p_n$.}
			\begin{enumerate}
				\item Let the formal polynomial $G(y,z)=\sum_{r=1}^n p_{n-r}(z^{r-1}+z^{r-2}y+\cdots+zy^{r-2}+y^{r-1})$.
				\item \textbf{Verify that the formal polynomial $p(z)-p(y)=(p_0z^n+p_1z^{n-1}+\cdots+p_n)-(p_0y^n+p_1y^{n-1}+\cdots+p_n)=(\sum_{r=0}^n p_{n-r}z^r)-(\sum_{r=0}^n p_{n-r}y^r)=\sum_{r=1}^n p_{n-r}(z^r-y^r)=\sum_{r=1}^n p_{n-r}(z-y)(z^{r-1}+z^{r-2}y+\cdots+zy^{r-2}+y^{r-1})=(z-y)\sum_{r=1}^n p_{n-r}(z^{r-1}+z^{r-2}y+\cdots+zy^{r-2}+y^{r-1})=(z-y)G(y,z)$.}
				\item \textbf{Yield the tuple $\langle G(y,z)\rangle$.}
			\end{enumerate}
		\subsection{Algorithm 35}\label{sec:algorithm 35}
			\textbf{Choose a formal polynomial $p=x^n+p_1x^{n-1}+\cdots+p_n$ and rationals $a_1<a_2<\cdots<a_n<a_{n+1}$ in such a way that for $i=1$ to $i=n+1$, $p(a_i)=0$.} Now do the following:
			\begin{enumerate}
				\item Write $p$ as $1*p$, so that it has two factors.
				\item For $i=1$ up to $i=n$, do the following:
				\begin{enumerate}
					\item Let $g$ be the rightmost factor of $p$.
					\item If $g(a_i)\ne 0$, do the following:
					\begin{enumerate}
						\item For $k=1$ to $k=i-1$, verify that $(a_i-a_k)\ne 0$.
						\item Verify that $p(a_i)\ne 0$.
						\item \textbf{Abort algorithm.}
					\end{enumerate}
					\item Otherwise $g(a_i)=0$. Now do the following:
					\begin{enumerate}
						\item Execute \hyperref[sec:algorithm 34]{algorithm 34} on $g$ and let $(G(x,y))$ be the result.
						\item Let the formal polynomial $q=q(x)=G(a_i,x)$.
						\item Verify that the formal polynomial $g=g(x)=g(x)-g(a_i)=(x-a_i)G(a_i,x)=(x-a_i)q(x)=(x-a_i)q$.
						\item Verify that $p=(x-a_1)(x-a_2)\cdots(x-a_i)q$.
					\end{enumerate}
				\end{enumerate}
				\item Now verify that $p=(x-a_1)(x-a_2)\cdots(x-a_n)1$.
				\item Using (3), verify that $p(a_{n+1})\ne 0$.
				\item \textbf{Abort algorithm.}
			\end{enumerate}
		\subsection{Algorithm 36 (Bisection)}\label{sec:algorithm 36}
			\textbf{Choose a formal polynomial $f$. Choose rational numbers $a<b$ such that $\sgn(f(a))=-\sgn(f(b))$. Choose a rational number target $B>0$.} Now do the following:
			\begin{enumerate}
				\item Execute \hyperref[sec:algorithm 34]{algorithm 34} on $f$ and let $(G(x,y))$ be the result.
				\item Verify that the formal polynomial $f(y)-f(x)=(y-x)G(x,y)$.
				\item Let $G'$ be $G$ but with all negative signs replaced with positive signs.
				\item Let $U=\max(G'(\lvert a\rvert,\lvert a\rvert), G'(\lvert b\rvert,\lvert b\rvert))$.
				\item Until $\lvert b-a\rvert U<B$
				\begin{enumerate}
					\item Let $c=\frac{a+b}{2}$.
					\item If $\sgn(f(a))=-\sgn(f(c))$, then:
					\begin{enumerate}
						\item Let $b=c$.
					\end{enumerate}
					\item Otherwise if $\sgn(f(c))=-\sgn(f(b))$, then:
					\begin{enumerate}
						\item Let $a=c$.
					\end{enumerate}
					\item Otherwise if $f(c)=0$, then do the following:
					\begin{enumerate}
						\item Record the rational number $c$.
						\item If less than or equal to $i$ rational numbers have been recorded, then:
						\begin{enumerate}
							\item Let $a=c$.
							\item Go to (1bi).
						\end{enumerate}
						\item Otherwise the $i+1$ numbers $c_k$ such that $c_1<c_2<\cdots<c_i<c_{i+1}$ have been recorded. Now do the following:
						\begin{enumerate}
							\item Execute \hyperref[sec:algorithm 35]{algorithm 35} on the formal polynomial $f_i$ and the rationals $c_1<c_2<\cdots<c_i<c_{i+1}$.
							\item \textbf{Abort algorithm.}
						\end{enumerate}
					\end{enumerate}
					\item Otherwise, do the following:
					\begin{enumerate}
						\item Verify that $f(a)\nless 0$.
						\item Verify that $f(a)\ne 0$.
						\item Verify that $f(a)\ngtr 0$.
						\item \textbf{Abort algorithm.}
					\end{enumerate}
				\end{enumerate}
				\item \textbf{Verify that $\lvert f(a)\rvert,\lvert f(b)\rvert<\lvert f(b)-f(a)\rvert=\lvert(b-a)G(a,b)\rvert<\lvert(b-a)G'(a,b)\rvert<\lvert b-a\rvert U<B$.}
			\end{enumerate}
		\subsection{Algorithm 37}\label{sec:algorithm 37}
			\textbf{Choose a polynomial $f_n=x^n+p_1x^{n-1}+\cdots+p_n$ and pairs of rationals $(a_n,b_n),(a_{n-1},b_{n-1}),\cdots,(a_0,b_0)$ in such a way that:
			\begin{enumerate}
				\item $a_n<b_n\le a_{n-1}<b_{n-1}\le\cdots\le a_1<b_1\le a_0<b_0$.
				\item For $i=0$ to $i=n$, $\sgn(f_n(a_i))=-\sgn(f_n(b_i))$.
			\end{enumerate}}
			Now do the following:
			\begin{enumerate}
				\item For $i=n$ to $i=1$:
				\begin{enumerate}
					\item Let $B=\min_{k=0}^{i-1}\min(\lvert f_i(a_i)\rvert,\lvert f_i(b_i)\rvert)$.
					\item Execute \hyperref[sec:algorithm 36]{algorithm 36} on the formal polynomial $f_i$, interval $(a_i, b_i)$, and target of $B$. Let $a_i$ and $b_i$ receive their updates.
					\item Verify that $\lvert f_i(b_i)\rvert<B$.
					\item $f$ should evaluate to $f-f_i(b_i)+f_i(b_i)$ which should evaluate to $(x-b_i)f_{i-1}+f_i(b_i)$, where $f_{i-1}$ is a monic $(i-1)^{th}$ degree formal polynomial.
					\item For $k=0$ to $k=i-1$, do the following:
					\begin{enumerate}
						\item If $f_i(a_k)>B$, verify that:
						\begin{enumerate}
							\item $f_i(a_k)>B>\lvert f_i(b_i)\rvert\ge f_i(b_i)$.
							\item $f_i(a_k)-f_i(b_i)>0$.
							\item $(a_k-b_i)f_{i-1}(a_k)>0$.
							\item $f_{i-1}(a_k)>0$.
							\item $f_i(b_k)<-B<-\lvert f_i(b_i)\rvert\le f_i(b_i)$.
							\item $f_i(b_k)-f_i(b_i)<0$.
							\item $(b_k-b_i)f_{i-1}(b_k)<0$.
							\item $f_{i-1}(b_k)<0$.
						\end{enumerate}
						\item Otherwise, if $f_i(a_k)<-B$, verify that:
						\begin{enumerate}
							\item $f_{i-1}(a_k)<0$.
							\item $f_{i-1}(b_k)>0$.
						\end{enumerate}
						\item Otherwise:
						\begin{enumerate}
							\item \textbf{Verify that 0=1.}
						\end{enumerate}
					\end{enumerate}
				\end{enumerate}
				\item By (1d), $f_0$ should equal $1$.
				\item By (1eiD) and (1eiH) or (1eiiA) and (1eiiB), $\sgn(f_0(a_0))=-\sgn(f_0(b_0))$.
				\item \textbf{Verify that 0=1.}
			\end{enumerate}
		\subsection{Algorithm 38 (Sturm's algorithm initialization)}\label{sec:algorithm 38}
			\textbf{Choose a sequence of polynomials $s_0={s_0}_0,s_1={s_1}_0x^1+{s_1}_1,\cdots,s_m={s_m}_0x^m+{s_m}_1x^{m-1}+\cdots+{s_m}_m$ and another sequence of polynomials $q_1,q_2,\cdots,q_{m-1}$ in such a way that
			\begin{enumerate}
				\item For $i=0$ to $i=m$, ${s_i}_0>0$
				\item For $i=1$ to $i=m-1$, $s_{i-1}+s_{i+1}=q_is_i$.
			\end{enumerate}}
			Now do the following:
			\begin{enumerate}
				\item Let $J_i(x)$ be a shorthand for the number of sign changes in the sequence $s_0(x),s_1(x),\cdots,s_i(x)$.
				\item For $i=1$ to $i=m$, do the following:
				\begin{enumerate}
					\item Execute \hyperref[sec:algorithm 34]{algorithm 34} on $s_i$ and let $(G_i(x,y))$ be the result.
					\item \textbf{Verify that the formal polynomial $s_i(y)-s_i(x)=(y-x)G_i(x,y)$.}
				\end{enumerate}
				\item For $i=1$ to $i=m-1$, do the following:
				\begin{enumerate}
					\item It should be that $q_is_i-s_{i+1}=s_{i-1}$. Make this our working equation.
					\item If $i>1$:
					\begin{enumerate}
						\item Substitute this $s_{i-1}$ into the equation mentioned in (7c) of the previous iteration. Let the outcome be our working equation.
					\end{enumerate}
					\item Our working equation should now be in terms of $s_i$, $s_{i+1}$, and $s_0$.
					\item \textbf{Factorize this equation and divide it through by $s_0$ to obtain the equation $g_is_{i+1}+h_is_i=1$, where $g_i$ and $h_i$ are polynomials.}
				\end{enumerate}
			\end{enumerate}
		\subsection{Algorithm 39 (Change in number of sign changes verification)}\label{sec:algorithm 39}
			\begin{enumerate}
				\item \textbf{Execute \hyperref[sec:algorithm 38]{algorithm 38}.}
				\item \textbf{Choose rational numbers $c$ and $d$ in such a way that:
				\begin{enumerate}
					\item $J_m(c)$ and $J_m(d)$ are well defined.
					\item Letting $B=\max_{i=1}^m \lvert G_i(c,d)\rvert$.
					\item Letting $C=\max_{i=1}^{m-1}\max(\lvert g_i(c)\rvert,\lvert h_i(c)\rvert,\lvert g_i(d)\rvert,\lvert h_i(d))\rvert$.
					\item Letting $D=\max_{i=1}^{m-1}\max(\lvert q_i(c)\rvert,\lvert q_i(d)\rvert)$.
					\item $\lvert d-c\rvert<\frac{1}{BCD}$.
				\end{enumerate}}
				\item Let $i=0$.
				\item Do the following:
				\begin{enumerate}
					\item $\sgn(s_i(c))$ should equal $\sgn(s_i(d))$.
					\item $J_i(c)$ should equal $J_i(d)$.
					\item If $\sgn(s_{i+1}(c))=\sgn(s_{i+1}(d))$, verify the following:
					\begin{enumerate}
						\item $J_{i+1}(c)=J_{i+1}(d)$.
						\item Set $i$ to $i+1$ and go to (4) if the new $i<m$.
					\end{enumerate}
					\item Otherwise, if $\sgn(s_{i+1}(c))\ne\sgn(s_{i+1}(d))$ and $i+2\le m$, verify the following:
					\begin{enumerate}
						\item $\lvert s_{i+1}(c)\rvert<\lvert s_{i+1}(c)-s_{i+1}(d)\rvert=\lvert c-d\rvert\lvert G_{i+1}(c,d)\rvert<\lvert c-d\rvert B<lB=\frac{1}{CD}$
						\item $\lvert s_{i+1}(d)\rvert<\frac{1}{CD}$
						\item $1=g_{i+1}(c)s_{i+2}(c)+h_{i+1}(c)s_{i+1}(c)=\lvert g_{i+1}(c)s_{i+2}(c)+h_{i+1}(c)s_{i+1}(c)\rvert\le\lvert g_{i+1}(c)\rvert\lvert s_{i+2}(c)\rvert+\lvert h_{i+1}(c)\rvert\lvert s_{i+1}(c)\rvert<C(\lvert s_{i+2}(c)\rvert+\frac{1}{CD})$
						\item $\frac{1}{C}(1-\frac{1}{D})<\lvert s_{i+2}(c)\rvert$
						\item $\frac{1}{C}(1-\frac{1}{D})<\lvert s_{i+2}(d)\rvert$
						\item If $\sgn(s_{i+2}(c))\ne\sgn(s_{i+2}(d))$, verify the following:
						\begin{enumerate}
							\item $\lvert s_{i+2}(c)\rvert<\frac{1}{CD}=\frac{1}{C}\cdot\frac{1}{D}<\frac{1}{C}(1-\frac{1}{D})$
							\item \textbf{Verify that 0=1.}
						\end{enumerate}
						\item Otherwise if $\sgn(s_i(c))=\sgn(s_{i+2}(c))$, verify the following:
						\begin{enumerate}
							\item $2\frac{1}{C}(1-\frac{1}{D})\le\lvert s_i(c)\rvert+\lvert s_{i+2}(c)\rvert=\lvert s_i(c)+s_{i+2}(c)\rvert=\lvert q_{i+1}(c)s_{i+1}(c)\rvert<D\frac{1}{CD}$.
							\item $2(1-\frac{1}{D})<1$.
							\item $D<2$.
							\item \textbf{Verify that 0=1.}
						\end{enumerate}
						\item Otherwise $\sgn(s_{i+2}(c))=\sgn(s_{i+2}(d))$ and $\sgn(s_i(c))\ne\sgn(s_{i+2}(c))$. Now verify the following:
						\begin{enumerate}
							\item $J_{i+2}(c)=J_{i+2}(d)$.
							\item Set $i$ to $i+2$ and go to (4).
						\end{enumerate}
					\end{enumerate}
					\item Otherwise $\sgn(s_{i+1}(c))\ne\sgn(s_{i+1}(d))$ and $i+1=m$. Now verify the following:
					\begin{enumerate}
						\item $\lvert s_{i+1}(c)\rvert<\frac{1}{CD}$.
						\item $\lvert s_{i+1}(d)\rvert<\frac{1}{CD}$.
						\item $\lvert J_{i+1}(c)-J_{i+1}(d)\rvert=1$.
					\end{enumerate}
				\end{enumerate}
				\item \textbf{If $\sgn(s_m(c))=\sgn(s_m(d))$, then $J_m(c)$ should equal $J_m(d)$. Otherwise $\lvert J_m(d)-J_m(c)\rvert$ should equal $1$.}
			\end{enumerate}
		\subsection{Algorithm 40 (Cauchy's positive verification)}\label{sec:algorithm 40}
			\textbf{Choose a non-zero polynomial $p=p_0x^t+p_1x^{t-1}+p_2x^{t-2}+\cdots+p_tx^0$, where $p_0>0$. Choose a rational $k>1+\max_{i=1}^t\lvert\frac{p_i}{p_0}\rvert$.} In reverse order verify the following:
			\begin{enumerate}
				\item \textbf{$p(k)>0$}
				\item $p_0k^n+p_1k^{n-1}+\cdots+p_nk^0>0$
				\item $k^n+\frac{p_1}{p_0}k^{n-1}+\cdots+\frac{p_n}{p_0}k^0>0$
				\item $k^n>-(\frac{p_1}{p_0}k^{n-1}+\cdots+\frac{p_n}{p_0}k^0)$
				\item $k^n>\lvert \frac{p_1}{p_0}k^{n-1}+\cdots+\frac{p_n}{p_0}k^0\rvert$
				\item $k^n>\lvert\max_{i=1}^t\lvert \frac{p_i}{p_0}\rvert(k^{n-1}+\cdots+k^0)\rvert$
				\item $k^n>\max_{i=1}^t\lvert \frac{p_i}{p_0}\rvert\frac{k^n-1}{k-1}$
				\item $k^{n+1}-k^n>\max_{i=1}^t\lvert \frac{p_i}{p_0}\rvert(k^n-1)$
				\item $k^{n+1}-(1+\max_{i=1}^t\lvert \frac{p_i}{p_0}\rvert)k^n+\max_{i=1}^t\lvert \frac{p_i}{p_0}\rvert>0$
				\item $k>1+\max_{i=1}^t\lvert \frac{p_i}{p_0}\rvert$
			\end{enumerate}
		\subsection{Algorithm 41 (Cauchy's alternation verification)}\label{sec:algorithm 41}
			\textbf{Choose a non-zero polynomial $p=p_0x^t+p_1x^{t-1}+p_2x^{t-2}+\cdots+p_tx^0$, where $p_0>0$. Choose a rational $k<-(1+\max_{i=1}^t\lvert\frac{p_i}{p_0}\rvert)$.} Now do the following:
			\begin{enumerate}
				\item Let $q=q_0x^t+q_1x^{t-1}+q_2x^{t-2}+\cdots+q_tx^0$, where $q_i=(-1)^ip_i$.
				\item It should be that $k<-(1+\max_{i=1}^t\lvert\frac{q_i}{q_0}\rvert)$.
				\item Therefore it should be that $-k>1+\max_{i=1}^t\lvert\frac{q_i}{q_0}\rvert$.
				\item Now execute \hyperref[sec:algorithm 38]{algorithm 38} on $q$ and $-k$.
				\item It should be that $q(-k)>0$, that is, $\sum_{i=0}^t q_i(-k)^{t-i}>0$.
				\item Therefore, it should be that $\sum_{i=0}^t (-1)^i(-1)^{t-i}p_ik^{t-i}>0$.
				\item Therefore, it should be that $(-1)^t\sum_{i=0}^t p_ik^{t-i}>0$.
				\item \textbf{Therefore it should be that $(-1)^tp(k)>0$.}
			\end{enumerate}
		\subsection{Algorithm 42 (Sturm's sign change)}\label{sec:algorithm 42}
			\begin{enumerate}
				\item \textbf{Execute \hyperref[sec:algorithm 38]{algorithm 38}.}
				\item Let $U=1+\max_{i=0}^m\left(1+\max_{j=1}^i\lvert\frac{{s_i}_j}{{s_i}_0}\rvert\right)$
				\item By \hyperref[sec:algorithm 40]{algorithm 40}, $J(U)$ should evaluate to $0$.
				\item By \hyperref[sec:algorithm 41]{algorithm 41}, $J(-U)$ should evaluate to $m$.
				\item For $i=1$ to $i=m$: Let $G_i'(x,y)$ be $G_i(x,y)$ with all negative signs replaced with positive signs.
				\item Let the rational $B=\max_{i=1}^m G_i'(U,U)$.
				\item For $i=1$ to $i=m-1$: Let the polynomial $g_i'$ be $g_i$ with all negative signs replaced with positive signs. Let the polynomial $h_i'$ be $h_i$ with all negative signs replaced with positive signs.
				\item Let $C=C=\max_{i=1}^m \max(g_i'(U),h_i'(U))$.
				\item For $i=1$ to $i=m-1$: Let the polynomial $q_i'$ be $q_i$ with all negative signs replaced with positive signs.
				\item Let $D=\max(3, \max_{i=1}^m q_i'(U))$
				\item Let $l=\frac{1}{BCD}$ and $c=-U$.
				\item Do the following:
				\begin{enumerate}
					\item If $c+l\le U$, then:
					\begin{enumerate}
						\item Choose a number $d$ in a way such that $c<d<c+l$ and $J(d)$ is well-defined.
					\end{enumerate}
					\item Otherwise let $d=U$.
					\item Execute \hyperref[sec:algorithm 39]{algorithm 39}.
					\item \textbf{If $J_m(c)\ne J_m(d)$, then record the pair of rational numbers $(c,d)$.}
					\item Now let $c=d$ and go to (12) if the new $c\ne U$.
				\end{enumerate}
				\item If less than $m$ pairs of rational numbers were recorded, then verify the following:
				\begin{enumerate}
					\item Each change of $J_m(x)$ over the course of (12) was by $1$.
					\item $J_m(x)$ changed less than $m$ times over the course of (12).
					\item Therefore it should be that $\lvert J_m(U)-J_m(-U)\rvert<m$.
					\item \textbf{Verify that 0=1.}
				\end{enumerate}
				\item If more than $m$ pairs of rational numbers were recorded, then verify the following:
				\begin{enumerate}
					\item \textbf{Execute \hyperref[sec:algorithm 37]{algorithm 37} on the formal polynomial $s_m(x)$ along with the first $m+1$ recorded pairs.}
				\end{enumerate}
				\textbf{\item Otherwise verify that:
				\begin{enumerate}
					\item Exactly $m$ pairs of rational numbers $(c_1,d_1),(c_2,d_2),\cdots,(c_m,d_m)$ were recorded.
					\item $c_1<d_1\le c_2<d_2\le\cdots\le c_m<d_m$.
					\item For $i=1$ to $i=m$, $\sgn(s_m(c_i))=-\sgn(s_m(d_i))$.
				\end{enumerate}}
			\end{enumerate}
		\subsection{Algorithm 43}\label{sec:algorithm 43}
			\textbf{Choose an $m\times m$ matrix, $A$, whose entries are only rationals. Execute \hyperref[sec:algorithm 3]{algorithm 3} on the polynomial matrix $xI-A$ and let $B$ be the result. Choose a polynomial $p$ such that $\deg(p)<\deg(B_{m,m})$.} Now do the following:
			\begin{enumerate}
				\item Execute \hyperref[sec:algorithm 33]{algorithm 33} on matrix $A$ and polynomial $p$.
				\item Since the inner execution of \hyperref[sec:algorithm 1]{algorithm 1} at \hyperref[sec:algorithm 33]{algorithm 33}'s (5) was decreasing the degree of an entry on every iteration, verify for \hyperref[sec:algorithm 33]{algorithm 33}'s $g$ that $\deg(g)\le\deg(B_{m,m})-1$.
				\item Since $B_{m,m}=g$, verify that $\deg(B_{m,m})\le\deg(B_{m,m})-1$.
				\item \textbf{Abort algorithm.}
			\end{enumerate}
		\subsection{Algorithm 44}\label{sec:algorithm 44}
			\textbf{Choose an $m\times m$ matrix, $A$, whose entries are only rationals} and do the following:
			\begin{enumerate}
				\item Execute \hyperref[sec:algorithm 3]{algorithm 3} on the polynomial matrix $xI-A$ and let $B$ be the result.
				\item Execute \hyperref[sec:algorithm 28]{algorithm 28} on $A$.
				\item Let $r=x^t+r_1x^{t-1}+r_2x^{t-2}+\cdots+r_tx^0$ be the last diagonal entry of $B$.
				\item Make an $m^2*t$ matrix, $F$, whose $i^{th}$ column is the sequential concatenation of the columns of $A^{t-i}$.
				\item Execute \hyperref[sec:algorithm 5]{algorithm 5} on $F$ to obtain matrices $M$, $N$, and $D$ such that $MFN=D$.
				\item Execute \hyperref[sec:algorithm 6]{algorithm 6} on $F$ to obtain matrices $M$, $N$, and $D$ such that $F=M^{-1}DN^{-1}$.
				\item Using \hyperref[sec:algorithm 7]{algorithm 7}, it should be that $M^{-1}MFN=I_{m^2}FN=FN=M^{-1}D$.
				\item If any column $i$ of $N$, $Ne_i$, is equal to zero, then:
				\begin{enumerate}
					\item Column $i$ of $N^{-1}N$ should equal zero.
					\item \textbf{Verify that 0=1.}
				\end{enumerate}
				\item Otherwise if the first entry of $C_t(D)$ is zero, then:
				\begin{enumerate}
					\item Some diagonal entry, $D_{i,i}$, of $D$ should be zero.
					\item Column $i$ of $D$ should be zero.
					\item Column $i$ of $M^{-1}D$ should be zero.
					\item Column $i$ of $FN$ should be zero.
					\item $F(Ne_i)$ should equal zero.
					\item Therefore $N_{1,i}A^{t-1}+N_{2,i}A^{t-2}+\cdots+N_{t,i}A^0$ should equal zero.
					\item Execute \hyperref[sec:algorithm 43]{algorithm 43} on matrix $A$ and polynomial $p=N_{1,i}x^{t-1}+N_{2,i}x^{t-2}+\cdots+N_{t,i}x^0$.
					\item \textbf{Verify that 0=1.}
				\end{enumerate}
				\item Otherwise, if any column $i$ of $C_t(M^{-1})$ is equal to zero, then:
				\begin{enumerate}
					\item Column $i$ of $C_t(M)C_t(M^{-1})$ should equal zero.
					\item $C_t(M)C_t(M^{-1})$ should, by \hyperref[sec:algorithm 18]{algorithm 18}, equal $C_t(MM^{-1})$ which should, by \hyperref[sec:algorithm 7]{algorithm 7}, equal $C_t(I_{m^2})$ which, by \hyperref[sec:algorithm 14]{algorithm 14}, should equal $I_{\binom{m^2}{t}}$.
					\item Therefore column $i$ of $I_{\binom{m^2}{t}}$ should equal zero.
					\item \textbf{Verify that 0=1.}
				\end{enumerate}
				\item Otherwise, if column $i$ of $C_t(N^{-1})$ is equal to zero, then:
				\begin{enumerate}
					\item \textbf{Verify that 0=1.}
				\end{enumerate}
				\item Otherwise all columns of $C_t(M^{-1})$ are non-zero, the first entry of $C_t(D)$ is non-zero, and $C_t(N^{-1})$ is non-zero. Now verify the following in order:
				\begin{enumerate}
					\item $C_t(F)=C_t(M^{-1}DN^{-1})$
					\item $C_t(F)=C_t(M^{-1})C_t(D)C_t(N^{-1})$
					\item $C_t(F)=C_t(M^{-1})ke_1C_t(N^{-1})$, where $k$ is a non-zero rational number
					\item $C_t(F)=kC_t(N^{-1})C_t(M^{-1})e_1$
					\item \textbf{$C_t(F)\ne 0_{\binom{m^2}{t}\times 1}$}
				\end{enumerate}
			\end{enumerate}
		\subsection{Algorithm 45}\label{sec:algorithm 45}
			\textbf{Choose an $m\times m$ matrix, $A$, whose entries are only rationals and is such that $A^T=A$.} Now do the following:
			\begin{enumerate}
				\item \textbf{Execute \hyperref[sec:algorithm 44]{algorithm 44} on the matrix $A$.}
				\item $C_t(F^TF)$ should evaluate to $C_t(F^T)C_t(F)$ which should evaluate to the sum of squares of the entries of $C_t(F)$ which should be more than zero.
				\item Using \hyperref[sec:algorithm 6]{algorithm 6}, it should be possible to evaluate $F^TF$ as $M^{-1}DN^{-1}$.
				\item If $D$ has a zero on its diagonal, then:
				\begin{enumerate}
					\item $C_t(F^TF)$ should evaluate to $C_t(M^{-1}DN^{-1})$ which should evaluate to $C_t(D)$ which should evaluate to $0$.
					\item \textbf{Verify that 0=1.}
				\end{enumerate}
				\item Otherwise $D$ should not have a zero on its diagonal. So do the following:
				\begin{enumerate}
					\item Let $\tr(X)$ be a shorthand for the sum of the diagonal entries of the square matrix $X$.
					\item Let $\mat(p_1x^{t-1}+p_1x^{t-2}+\cdots+p_t)$ be a shorthand for $p_1e_1+p_2e_2+\cdots+p_te_t$.
					\item Let $\pol(p_1e_1+p_2e_2+\cdots+p_te_t)$ be a shorthand for $p_1x^{t-1}+p_1x^{t-2}+\cdots+p_t$.
					\item Let $H=(F^TF)\backslash e_1$.
					\item Separately apply \hyperref[sec:algorithm 5]{algorithm 5} and \hyperref[sec:algorithm 6]{algorithm 6} on a $1\times 2$ matrix comprising $r$ followed by $\pol(H)$.
					\item If $d$ is a monic polynomial of degree is more than $0$, then do the following:
					\begin{enumerate}
						\item From the matrices constructed by \hyperref[sec:algorithm 6]{algorithm 6}, let $d=D_{1,1}$, $b={N^{-1}}_{1,1}$ and $c={N^{-1}}_{1,2}$.
						\item It should be that $r=bd$, where $b$ is a monic polynomial with degree $z<t$.
						\item It should be that $\pol(H)=cd$, where $c$ is some polynomial.
						\item Let $u=x^{t-z-1}b$.
						\item $\tr(u(A)h(A))$ should evaluate to $\mat(u)^TF^TFH$ which should evaluate to $\mat(u)^TF^TF((F^TF)\backslash e_1)$ which should evaluate to $\mat(u)^Te_1$ which should evaluate to $\mat(u)_{1,1}$ which should evaluate to $1$.
						\item $\tr(u(A)h(A))$ should also evaluate to $\tr(A^{z-1}b(A)c(A)d(A))$ which should evaluate to $\tr(A^{z-1}c(A)b(A)d(A))$ which should evaluate to $\tr(A^{z-1}c(A)f(A))$ which should evaluate to $\tr(A^{z-1}c(A)0_{m\times m})$ which should evaluate to $\tr(0_{m\times m})$ which should evaluate to $0$.
						\item \textbf{Verify that 0=1.}
					\end{enumerate}
					\item Otherwise if $d=1$, then do the following:
					\begin{enumerate}
						\item From the matrices constructed by \hyperref[sec:algorithm 6]{algorithm 6}, let $d=D_{1,1}$, $u=N_{1,1}$ and $s_{t+1}=N_{2,1}$.
						\item \textbf{It should be that $uf+s_{t+1}h=1$.}
					\end{enumerate}
					\item Otherwise:
					\begin{enumerate}
						\item \textbf{Verify that 0=1.}
					\end{enumerate}
				\end{enumerate}
			\end{enumerate}
		\subsection{Algorithm 46 (Euclidean division)}\label{sec:algorithm 46}
			\textbf{Choose two polynomials in $x$, $a$ and $b$} and do the following:
			\begin{enumerate}
				\item If the degree of $a$ is more than or equal to the degree of $b$:
				\begin{enumerate}
					\item Let $y$ be $\frac{\text{$a$'s leading coefficient}}{\text{$b$'s leading coefficient}}x^{\text{$a$'s degree - $b$'s degree}}$
					\item Let $e$ be $a-yb$. The degree of $e$ should be less than that of the degree of $a$.
					\item Apply \hyperref[sec:algorithm 46]{algorithm 46} on the ordered pair of polynomials $e$ and $b$. Let $c$ and $d$ be the ordered pair of polynomials yielded by this application.
					\item It should be that $cb+d=e$ and that $d$ has degree less than $b$.
					\item Therefore it should be that $cb+d=a-yb$
					\item \textbf{Therefore it should be that $(y+c)b+d=a$ and that $d$ has degree less than $b$.}
					\item \textbf{Now yield the ordered pair of polynomials consisting of $y+c$  and $d$.}
				\end{enumerate}
				\item Otherwise:
				\begin{enumerate}
					\item \textbf{It should be that $0b+a=a$ and that $a$ has degree less than $b$.}
					\item \textbf{Yield the ordered pair consisting of $0$ then $a$.}
				\end{enumerate}
			\end{enumerate}
		\subsection{Algorithm 47 (Edwards' Sturm chain construction)}\label{sec:algorithm 47}
			\textbf{Choose an $m\times m$ matrix, $A$, whose entries are only rationals and is such that $A^T=A$.} Now do the following:
			\begin{enumerate}
				\item \textbf{Execute \hyperref[sec:algorithm 45]{algorithm 45} on the matrix $A$.}
				\item If polynomials $u$ and $s_{t+1}$ such that $us_t+s_{t+1}h=1$ were successfully created, then:
				\item Execute \hyperref[sec:algorithm 46]{algorithm 46} on the ordered pair $s_{t+1}$ and $s_t$. Let $q_t$ and $s_{t-1}$ be the polynomials yielded by this execution.
				\item Verify that $s_{t+1}=q_ts_t+s_{t-1}$, where $\deg(s_{t-1})<t$.
				\item Therefore verify that $us_t+(q_ts_t+s_{t-1})h=1$.
				\item Therefore verify that $u(A)s_t(A)+(q_t(A)s_t(A)+s_{t-1}(A))h(A)=I_{m,m}$.
				\item Therefore verify that $s_{t-1}(A)h(A)=I_{m,m}$.
				\item Therefore verify that ${s_{t-1}}_0=\tr(s_{t-1}(A)h(A))=\tr(I_{m,m})=m>0$.
				\item For $i=t-1$ down to $i=1$, do the following:
				\begin{enumerate}
					\item If $i<t-1$, then do the following:
					\begin{enumerate}
						\item Verify that $s_i=p_1s_{t-1}+j_1s_t$ where $\deg(s_i)=i$, ${s_i}_0>0$, $\deg(p_1)=t-1-i$, and $\deg(j_1)=t-2-i$.
						\item Verify that $s_{i+1}=p_2s_{t-1}+j_2s_t$ where $\deg(s_{i+1})=i$, $\deg(p_2)=t-2-i$, and $\deg(j_2)=t-3-i$.
					\end{enumerate}
					\item Execute \hyperref[sec:algorithm 46]{algorithm 46} on the ordered pair $-s_{i+1}$ and $-s_i$. Let $q_i$ and $s_{i-1}$ be the polynomials yielded by this execution.
					\item Verify that $\deg(q_i)=1$ and that ${q_i}_0=\frac{{s_{i+1}}_0}{{s_i}_0}$.
					\item Also verify that $-s_{i+1}=-q_is_i+s_{i-1}$.
					\item Therefore verify that $q_is_i=s_{i+1}+s_{i-1}$.
					\item Therefore verify that $q_is_i-s_{i+1}=s_{i-1}$.
					\item If $i<t-1$, then do the following:
					\begin{enumerate}
						\item \textbf{It should be that $s_{i-1}=q_i(p_1s_{t-1}+j_1s_t)-(p_2s_{t-1}+j_2s_t)=p_3s_{t-1}+j_3s_t$, where $p_3=q_ip_1-p_2$ is of degree $t-i$ and $j_3=q_ij_1-j_2$ is of degree $t-1-i$.}
					\end{enumerate}
					\item Otherwise:
					\begin{enumerate}
						\item \textbf{Verify that $s_{i-1}=s_{t-2}=q_{t-1}s_{t-1}-s_t=p_3s_{t-1}+j_3s_t$, where $p_3=q_{t-1}$ of degree $1=t-1$ and $j_3=-1$ is of degree $0=t-1-i$.}
					\end{enumerate}
					\item Therefore verify that $s_{i-1}(A)=p_3(A)s_{t-1}(A)+j_3(A)s_t(A)=p_3(A)s_{t-1}(A)+j_3(A)0_{m\times m}=p_3(A)s_{t-1}(A)$.
					\item If $p_3(A)=0$, then do the following:
					\begin{enumerate}
						\item Execute \hyperref[sec:algorithm 43]{algorithm 43} on the matrix $A$ and polynomial $p_3$.
						\item \textbf{Abort algorithm.}
					\end{enumerate}
					\item Otherwise, if $s_{i-1}(A)=0_{m\times m}$, then do the following:
					\begin{enumerate}
						\item Verify that $p_3(A)s_{t-1}(A)h(A)=s_{i-1}(A)h(A)=0_{m\times m}h(A)=0_{m\times m}$.
						\item Verify that $p_3(A)s_{t-1}(A)h(A)=p_3(A)I_{m,m}=p_3(A)\ne0_{m\times m}$.
						\item \textbf{Abort algorithm.}
					\end{enumerate}
					\item Otherwise if $s_{i-1}(A)h(A)=0_{m\times m}$, then do the following:
					\begin{enumerate}
						\item Verify that $s_{i-1}(A)h(A)s_{t-1}(A)=0_{m\times m}s_{t-1}(A)=0_{m\times m}$.
						\item Verify that $s_{i-1}(A)h(A)s_{t-1}(A)=s_{i-1}(A)I_{m,m}=s_{i-1}(A)\ne 0$.
						\item \textbf{Abort algorithm.}
					\end{enumerate}
					\item Otherwise, do the following:
					\begin{enumerate}
						\item Verify that $\deg(s_{i-1})<i$.
						\item Execute the \hyperref[sec:algorithm 47 auxilliary algorithm]{auxilliary algorithm} on the pair $(s_{i-1}, s_{i-1})$.
						\item Now verify that $\tr(s_{i-1}(A)^2h(A)^2)=\frac{{s_{i-1}}_0}{{s_i}_0}$.
						\item Verify that $s_{i-1}(A)h(A)\ne 0_{m\times m}$.
						\item Therefore verify that $\tr(s_{i-1}(A)^2h(A)^2)=\tr((s_{i-1}(A)h(A))^2)=\lVert s_{i-1}(A)h(A)\rVert^2>0$.
						\item Therefore verify that $\frac{{s_{i-1}}_0}{{s_i}_0}>0$.
						\item \textbf{Therefore verify that ${s_{i-1}}_0>0$.}
					\end{enumerate}
				\end{enumerate}
			\end{enumerate}
			\subsubsection{Auxilliary algorithm}\label{sec:algorithm 47 auxilliary algorithm}
				\textbf{Choose an integer $0\le k\le t$ such that polynomial $s_k$ is defined. Choose a polynomial $g=g_0x^k+g_1x^{k-1}+\cdots+g_k.$}
				\begin{enumerate}
					\item If $k=t$, then verify that $\tr(g(A)s_k(A)h(A)^2)$
					\begin{enumerate}
						\item $=\tr(g(A)s_t(A)h(A)^2)$.
						\item $=\tr(g(A)0_{m\times m}h(A)^2)$.
						\item $=\tr(0_{m\times m})$.
						\item $=0$.
					\end{enumerate}
					\item Otherwise if $k=t-1$, then verify that $\tr(g(A)s_k(A)h(A)^2)$
					\begin{enumerate}
						\item $=\tr(g(A)s_{t-1}(A)h(A)^2)$.
						\item $=\tr(g(A)I_{m,m}h(A))$.
						\item $=\tr(g(A)h(A))$.
						\item $=g_0$.
						\item \textbf{=$\frac{g_0}{{s_{k+1}}_0}$.}
					\end{enumerate}
					\item Otherwise if $k<t-1$, then do the following:
					\begin{enumerate}
						\item Verify that $\tr(g(A)s_k(A)h(A)^2)=\tr(g(A)(q_{k+1}(A)s_{k+1}(A)+s_{k+2}(A))h(A)^2)$.
						\item Verify that $\tr(g(A)s_k(A)h(A)^2)=\tr(g(A)q_{k+1}(A)s_{k+1}(A)h(A)^2)+\tr(g(A)s_{k+2}(A)h(A)^2)$.
						\item Execute the \hyperref[sec:algorithm 47 auxilliary algorithm]{auxilliary algorithm} supplying the integer $k+1$ and $(k+1)^{th}$ degree polynomial $gq$.
						\item Since it should be that $k+1<t$, the execution of the \hyperref[sec:algorithm 47 auxilliary algorithm]{auxilliary algorithm} should have verified that $\tr((g(A)q_{k+1}(A))s_{k+1}(A)h(A)^2)=\frac{({s_{k+2}}_0/{s_{k+1}}_0)g_0}{{s_{k+2}}_0}=\frac{g_0}{s_{k+1}}$.
						\item Execute the \hyperref[sec:algorithm 47 auxilliary algorithm]{auxilliary algorithm} supplying the integer $k+2$ and $k^{th}$ degree polynomial $g$.
						\item If $k+2<t$, then:
						\begin{enumerate}
							\item The execution of the \hyperref[sec:algorithm 47 auxilliary algorithm]{auxilliary algorithm} should have verified that $\tr(g(A)s_{k+2}(A)h(A)^2)=\frac{0}{{s_{k+2}}_0}=0$.
						\end{enumerate}
						\item Otherwise if $k+2=t$, then:
						\begin{enumerate}
							\item The execution of the \hyperref[sec:algorithm 47 auxilliary algorithm]{auxilliary algorithm} should have verified that $\tr(g(A)s_{k+2}(A)h(A)^2)=\tr(g(A)s_t(A)h(A)^2)=0$.
						\end{enumerate}
						\item \textbf{Therefore verify that $\tr(g(A)s_k(A)h(A)^2)=\frac{g_0}{{s_{k+1}}_0}+0=\frac{g_0}{{s_{k+1}}_0}$.}
					\end{enumerate}
				\end{enumerate}
		\subsection{Algorithm 48}\label{sec:algorithm 48}
			\textbf{Choose an $m\times m$ matrix, $A$, whose entries are only rationals and is such that $A^T=A$.} Now do the following:
			\begin{enumerate}
				\item Execute \hyperref[sec:algorithm 3]{algorithm 3} on the polynomial matrix $xI-A$ and let $D$ be the result.
				\item The last diagonal entry of $D$ should be a product of $m$ factors, $u_1u_2\cdots u_m$.
				\item Execute \hyperref[sec:algorithm 47]{algorithm 47} on the matrix $A$.
				\item Execute \hyperref[sec:algorithm 42]{algorithm 42} supplying the sequences of polynomials $s_0,s_1,\cdots,s_t$ and $q_1,q_2,\cdots,q_{t-1}$ constructed above.
				\item For $i=1$ to $i=t$ do the following:
				\begin{enumerate}
					\item If $\sgn(u_1(c_i))=\sgn(u_1(d_i)), \sgn(u_2(c_i))=\sgn(u_2(d_i)), \cdots, \sgn(u_m(c_i))=\sgn(u_m(d_i))$, then do the following:
					\begin{enumerate}
						\item Verify that $\sgn(u_1(c_i))\sgn(u_2(c_i))\cdots\sgn(u_m(c_i))=\sgn(u_1(d_i))\sgn(u_2(d_i))\cdots\sgn(u_m(d_i))$.
						\item Verify that $\sgn(u_1(c_i)u_2(c_i)\cdots u_m(c_i))=\sgn(u_1(d_i)u_2(d_i)\cdots u_m(d_i))$.
						\item Verify that $\sgn(s_t(c_i))=\sgn(s_t(d_i))$.
						\item \textbf{Abort algorithm.}
					\end{enumerate}
					\item Otherwise do the following:
					\begin{enumerate}
						\item Assign $(c_i,d_i)$ to one of the $u_j$s for which $\sgn(u_j(c_i))=-\sgn(u_j(d_i))$.
					\end{enumerate}
				\end{enumerate}
				\item Let $n_i$ be the number of pairs assigned to the polynomial $u_i$.
				\item Verify that $\sum_{i=1}^m n_j=t$.
				\item If for any $i=1$ to $i=m$, $n_i>\deg(u_i)$, then do the following:
				\begin{enumerate}
					\item Execute \hyperref[sec:algorithm 37]{algorithm 37} on the polynomial $u_i$ along with $\deg(u_i)+1$ of the rational number pairs assigned to it.
					\item \textbf{Abort algorithm.}
				\end{enumerate}
				\item Otherwise if for any $i=1$ to $i=m$, $n_i<\deg(u_i)$, then do the following:
				\begin{enumerate}
					\item Verify that $\sum_{i=1}^m n_j<t$.
					\item Verify that $\sum_{i=1}^m n_j<\sum_{i=1}^m n_j$.
					\item \textbf{Abort algorithm.}
				\end{enumerate}
				\item Otherwise if for any $i=1$ to $i=m$, $n_i\ne\deg(u_i)$, then do the following:
				\begin{enumerate}
					\item Verify that $n_i\ne\deg(u_i)$.
					\item Verify that $n_i\nless\deg(u_i)$.
					\item Verify that $n_i\ngtr\deg(u_i)$.
					\item \textbf{Abort algorithm.}
				\end{enumerate}
				\item \textbf{For all $i=1$ to $i=m$, verify that $n_i=\deg(u_i)$.}
			\end{enumerate}
		\subsection{Algorithm 49 (Upper triangular matrix multiplication)}\label{sec:algorithm 49}
			\textbf{Choose two upper triangular $m\times m$ matrices, $A$ and $B$.} Now do the following:
			\begin{enumerate}
				\item Multiply $A$ by $B$ and let $C$ be the result.
				\item For $i=1$ to $i=m$, do the following:
				\begin{enumerate}
					\item \textbf{Verify that $C_{i,i}=\sum_{k=1}^m (A_{i,k}B_{k,i})=\sum_{k=1}^{i-1} (A_{i,k}B_{k,i})+A_{i,i}B_{i,i}+\sum_{k=i+1}^m (A_{i,k}B_{k,i})=\sum_{k=1}^{i-1} (0*B_{k,i})+A_{i,i}B_{i,i}+\sum_{k=i+1}^m (A_{i,k}*0)=A_{i,i}B_{i,i}$.}
				\end{enumerate}
				\item For $i=2$ to $i=m$, do the following:
				\begin{enumerate}
					\item For $j=1$ to $j=i-1$, do the following:
					\begin{enumerate}
						\item Verify that $C_{i,j}=\sum_{k=1}^m A_{i,k}B_{k,j}=\sum_{k=1}^{i-1} A_{i,k}B_{k,j}+\sum_{k=i}^m A_{i,k}B_{k,j}=\sum_{k=1}^{i-1} 0*B_{k,j}+\sum_{k=i}^m A_{i,k}*0=0$.
					\end{enumerate}
				\end{enumerate}
				\item \textbf{Therefore verify that $C$ is upper triangular.}
			\end{enumerate}
		\subsection{Algorithm 50 (Orthogonalization)}\label{sec:algorithm 50}
			\textbf{Choose integers $m\ge n\ge 0$. Choose an $n\times m$ matrix of polynomials $M$ and an $m\times n$ matrix of polynomials $A_0$ such that $MA_0=I_n$.} Now do the following:
			\begin{enumerate}
				\item Using \hyperref[sec:algorithm 8]{algorithm 8}, verify that $C_n(M_0A_0)=C_n(I_n)=1$.
				\item If $C_n(A_0)=0_{\binom{m}{n}\times 1}$, then do the following:
				\begin{enumerate}
					\item Verify that $C_n(M_0A_0)=C_n(M_0)C_n(A_0)=C_n(M_0)0_{\binom{m}{n}\times 1}=0$.
					\item \textbf{Abort algorithm.}
				\end{enumerate}
				\item Otherwise, do the following:
				\item Verify that $C_n(A_0)\ne0_{\binom{m}{n}\times 1}$.
				\item For $i=1$ to $i=n$, do the following:
				\begin{enumerate}
					\item If $A_{i-1}e_i=0_{m\times 1}$, then do the following:
					\begin{enumerate}
						\item Verify that $C_n(A_{i-1})=0$.
						\item \textbf{Abort algorithm.}
					\end{enumerate}
					\item Otherwise, do the following:
					\item Verify that $\lVert A_{i-1}e_i\rVert^2\ne 0$.
					\item Let $D_i$ be a $n\times n$ diagonal matrix comprising $i$ $1$s followed by $n-i$ $\lVert A_{i-1}e_i\rVert^2$s.
					\item \textbf{Verify that $D_i$ is upper triangular.}
					\item Verify that $C_n(D_i)=(\lVert A_{i-1}e_i\rVert^2)^{n-i}\ne 0$.
					\item Let $N_i=I_n$ except that its $i^{th}$ row is $i-1$ $0$s followed by a $1$ followed by $-({A_{i-1}}^TA_{i-1})_{i,i+1}$, then $-({A_{i-1}}^TA_{i-1})_{i,i+2}$, all the way up to $-({A_{i-1}}^TA_{i-1})_{i,n}$.
					\item \textbf{Verify that $N_i$ is upper triangular.}
					\item Using \hyperref[sec:algorithm 8]{algorithm 8}, verify that $C_n(N_i)=1\ne 0$.
					\item Let $A_i=A_{i-1}D_iN_i$.
					\item \textbf{Verify that $C_n(A_i)=C_n(A_{i-1}D_iN_i)=C_n(A_{i-1})C_n(D_i)C_n(N_i)=C_n(A_{i-1})C_n(D_i)\ne 0$}.
					\item Verify that ${A_{i-1}}^TA_i=({A_{i-1}}^TA_{i-1})D_iN_i$ is a matrix with $0$s from position $(i,i+1)$ to $(i,n)$.
					\item \textbf{Verify that ${A_i}^TA_i=(A_{i-1}D_iN_i)^T(A_{i-1}D_iN_i)={N_i}^T{D_i}^T({A_{i-1}}^TA_{i-1})D_iN_i$ is a matrix with $0$s from position $(i,i+1)$ to $(i,n)$ and from position $(i+1,i)$ to $(n,i)$.}
					\item Verify that $A_i=A_0(D_1N_1)\cdots (D_iN_i)$.
					\item Verify that $MA_i=(D_1N_1)\cdots (D_iN_i)$.
					\item For $j=1$ to $j=n$, do the following:
					\begin{enumerate}
						\item Using \hyperref[sec:algorithm 49]{algorithm 49}, verify that $({e_j}^TM)(A_ie_j)={e_j}^T(MA_i)e_j={e_j}^T((D_1N_1)\cdots (D_iN_i))e_j=({D_1}_{j,j}{N_1}_{j,j})\cdots ({D_i}_{j,j}{N_i}_{j,j})$.
						\item \textbf{Therefore using (5d) verify that $({e_j}^TM)(A_ie_j)={D_1}_{j,j}\cdots {D_i}_{j,j}={D_1}_{j,j}\cdots {D_{\min(i,j-1)}}_{j,j}=\lVert A_0e_1\rVert^2\cdots\lVert A_{\min(i,j-1)-1}e_{\min(i,j-1)}\rVert^2$.}
					\end{enumerate}
				\end{enumerate}
				\item Let $E=(D_1N_1)\cdots (D_nN_n)$.
				\item \textbf{Verify that ${A_n}^TA_n=(A_0(D_1N_1)\cdots (D_nN_n))^T(A_0(D_1N_1)\cdots (D_nN_n))=((D_1N_1)\cdots (D_nN_n))^T({A_0}^TA_0)((D_1N_1)\cdots (D_nN_n))=E^T({A_0}^TA_0)E$ is a diagonal matrix.}
				\item \textbf{Yield the matrix $E$.}
			\end{enumerate}
		\subsection{Algorithm 51 (Cauchy-Schwarz inequality)}\label{sec:algorithm 51}
			\textbf{Choose a $1*m$ matrix $A$ and an $m*1$ matrix $B$.} Now do the following:
			\begin{enumerate}
				\item Verify that $0$
				\begin{enumerate}
					\item $\le\frac{1}{2}\sum_{i=1}^m\sum_{j=1}^m (A_iB_j-A_jB_i)^2$
					\item $=\frac{1}{2}\sum_{i=1}^m\sum_{j=1}^m ({A_i}^2{B_j}^2-2A_iB_jA_jB_i+{A_j}^2{B_i}^2)$
					\item $=\frac{1}{2}\sum_{i=1}^m {A_i}^2\sum_{j=1}^m {B_j}^2+\frac{1}{2}\sum_{i=1}^m {B_i}^2\cdot\allowbreak\sum_{j=1}^m {A_j}^2-\sum_{i=1}^m A_iB_i\sum_{j=1}^m A_jB_j$
					\item $=\frac{1}{2}(AA^T)(B^TB)+\frac{1}{2}(AA^T)(B^TB)-(AB)^2$
					\item $=(AA^T)(B^TB)-(AB)^2$.
				\end{enumerate}
				\item \textbf{Therefore verify that $(AB)^2\le(AA^T)(B^TB)$.}
			\end{enumerate}
		\subsection{Algorithm 52}\label{sec:algorithm 52}
			\textbf{Choose integers $m\ge n>0$. Choose an $n\times m$ matrix of polynomials $M$ and an $m\times n$ matrix of polynomials $A_0$ such that $MA_0=I_n$. Choose a rational number $x$.} Now do the following:
			\begin{enumerate}
				\item Execute \hyperref[sec:algorithm 50]{algorithm 50} on $M$ and $A_0$.
				\item Let $a=\max(\lVert M(x)\rVert^2,1)$.
				\item \textbf{Choose a column index $1\le j\le n$ such that $\lVert A_n(x)e_j\rVert^2<\frac{1}{a^{(2n+2)!!}}$.}
				\item Let $i=n$.
				\item Verify that $\lVert A_i(x)e_j\rVert^2<\frac{1}{a^{(2i+2)!!}}$.
				\item Using \hyperref[sec:algorithm 51]{algorithm 51}, verify that $({e_j}^TM(x)A_i(x)e_j)^2\le\lVert{e_j}^TM(x)\rVert^2\lVert A_i(x)e_j\rVert^2<\lVert M(x)\rVert^2\frac{1}{a^{(2i+2)!!}}\le a\frac{1}{a^{(2i+2)!!}}\le\frac{1}{a^{(2i)!!*2i}}\le 1$.
				\item If $i=0$, then do the following:
				\begin{enumerate}
					\item Verify that $({e_j}^TM(x)A_i(x)e_j)^2=({e_j}^TM(x)A_0(x)e_j)^2=({e_j}^TI_ne_j)^2=1$.
					\item \textbf{Abort algorithm.}
				\end{enumerate}
				\item Otherwise, do the following:
				\item Using \hyperref[sec:algorithm 50]{algorithm 50}, verify that $(1\lVert A_0e_1\rVert^2\cdots\lVert A_{\min(i,j-1)-1}e_{\min(i,j-1)}\rVert^2)^2=({e_j}^TM(x)A_i(x)e_j)^2<\frac{1}{a^{(2i)!!*2i}}\le 1$.
				\item If $\min(i,j-1)=0$, then do the following:
				\begin{enumerate}
					\item Verify that $(1\lVert A_0(x)e_1\rVert^2\cdots\lVert A_{\min(i,j-1)-1}(x)e_{\min(i,j-1)}\rVert^2)^2=1^2=1$.
					\item \textbf{Abort algorithm.}
				\end{enumerate}
				\item Otherwise do the following:
				\begin{enumerate}
					\item Verify that $\min(i,j-1)>0$.
					\item If for all $k=0$ to $k=\min(i,j-1)-1$, $\lVert A_k(x)e_{k+1}\rVert^2\ge\frac{1}{a^{(2i)!!}}$, then do the following:
					\begin{enumerate}
						\item Verify that $({e_j}^TM(x)A_i(x)e_j)^2=(\lVert A_0(x)e_1\rVert^2\cdots\lVert A_{\min(i,j-1)-1}(x)e_{\min(i,j-1)}\rVert^2)^2\ge(\frac{1}{a^{(2i)!!}})^{2\min(i,j-1)}\ge(\frac{1}{a^{(2i)!!}})^{2i}=\frac{1}{a^{(2i)!!*2i}}$.
						\item \textbf{Abort algorithm.}
					\end{enumerate}
					\item Otherwise, do the following:
					\begin{enumerate}
						\item \textbf{Let $k$, where $0\le k<i$, be one of the integers for which $\lVert A_k(x)e_{k+1}\rVert^2<\frac{1}{a^{(2i)!!}}$.}
						\item \textbf{Verify that $\lVert A_k(x)e_{k+1}\rVert^2<\frac{1}{a^{(2i)!!}}\le\frac{1}{a^{(2k+2)!!}}$.}
						\item \textbf{Simultaneously set $i$ to $k$ and $j$ to $k+1$.}
						\item \textbf{Go to (4).}
					\end{enumerate}
				\end{enumerate}
				\item \textbf{Abort algorithm.}
			\end{enumerate}
		\subsection{Algorithm 53}\label{sec:algorithm 53}
			\textbf{Choose an $m\times m$ matrix, $A$, whose entries are only rationals and is such that $A^T=A$.} Now do the following:
			\begin{enumerate}
				\item Execute \hyperref[sec:algorithm 48]{algorithm 48} on the matrix $A$.
				\item For $i=1$ to $i=t$, let $k_i$ be the index of the polynomial to which $(c_i, d_i)$ was associated.
				\item Let the macro $[P]$ expand to "(if $P$, then yield $1$, otherwise yield $0$)".
				\item Verify that $\sum_{i=1}^t(m+1-k_i)$
				\begin{enumerate}
					\item $=\sum_{i=1}^t\sum_{j=1}^m [k_i\le j]$
					\item $=\sum_{j=1}^m\sum_{i=1}^t [k_i\le j]$
					\item $=\sum_{j=1}^m\sum_{i=1}^t [k_i\le j]\sum_{l=1}^m [k_i=l]$
					\item $=\sum_{j=1}^m\sum_{l=1}^m\sum_{i=1}^t [k_i\le j][k_i=l]$
					\item $=\sum_{j=1}^m\sum_{l=1}^m\sum_{i=1}^t [l\le j][k_i=l]$
					\item $=\sum_{j=1}^m\sum_{l=1}^m [l\le j]\sum_{i=1}^t [k_i=l]$
					\item $=\sum_{j=1}^m\sum_{l=1}^m [l\le j]n_l$
					\item $=\sum_{j=1}^m\sum_{l=1}^m [l\le j]\deg u_l$
					\item $=\sum_{j=1}^m\sum_{l=1}^j \deg u_l$
					\item $=\sum_{j=1}^m \deg D_{j,j}$
					\item $=m$
				\end{enumerate}
				\item \textbf{Verify that $\sum_{i=1}^t(m+1-k_i)=m$.}
			\end{enumerate}
		\subsection{Algorithm 54 (Spectral algorithm initialization)}\label{sec:algorithm 54}
			\textbf{Choose an $m\times m$ matrix, $A$, whose entries are only rationals and is such that $A^T=A$. Choose a rational number $\epsilon>0$.} Now do the following:
			\begin{enumerate}
				\item Execute \hyperref[sec:algorithm 48]{algorithm 48} on the matrix $A$.
				\item Execute \hyperref[sec:algorithm 6]{algorithm 6} with $xI_m-A$ as the choice matrix. Take note of $M^{-1}$, $D$, and $N^{-1}$.
				\item Let $M'$ be the matrix obtained by replacing all the negative signs in $M^{-1}$ with positive signs.
				\item Let $M''=\max_{i=1}^m\max_{j=1}^mM'(\max(\lvert c_1\rvert,\lvert d_t\rvert))_{i,j}$.
				\item Let $N'$ be the matrix obtained by replacing all the negative signs in $N$ with positive signs.
				\item Let $N''=1+\max_{i=1}^m\max_{j=1}^mN'(\max(\lvert c_1\rvert,\lvert d_t\rvert))_{i,j}$.
				\item Let $L$ be the formal polynomial obtained by replacing all the negative signs in $(\lVert N^{-1}\rVert^2)^{(2m+2)!!}$ with positive signs.
				\item Let $L'=\frac{1}{\max(1,L(\lvert c_1\rvert),L(\lvert d_t\rvert))}$.
				\item Let $\delta=\min(1,\min_{i=1}^{t-1}(c_{i+1}-c_i))$.
				\item For $i=1$ to $i=t$, do the following:
				\begin{enumerate}
					\item Let $k_i$ be the index of the polynomial to which $(c_i, d_i)$ was associated.
					\item Verify that $\sgn(u_{k_i}(c_i))\ne\sgn(u_{k_i}(d_i))$.
					\item Let $Q$ be the last $m+1-k_i$ columns of $I_m$.
					\item Execute \hyperref[sec:algorithm 50]{algorithm 50} on the matrix $NQ$. Let $E$ be the $(m+1-k_i)\times (m+1-k_i)$ matrix yielded from this.
					\item Let $K_i=NQE$, an $m\times(m+1-k_i)$ matrix.
					\item \textbf{Verify that ${K_i}^T{K_i}$ is a diagonal matrix.}
					\item Let $E'$ be the matrix obtained by replacing all the negative signs in $E$ with positive signs.
					\item Let $E''_i=\max_{j=1}^m\max_{l=1}^mE'(\max(\lvert c_1\rvert,\lvert d_t\rvert))_{j,l}$.
				\end{enumerate}
				\item Let $E''=1+\max_{i=1}^t E''_i$.
				\item For $i=1$ to $i=t$, do the following with the symbols $Q$, $E$, and $F$ retaining their values from the corresponding iteration of the loop at (7).
				\begin{enumerate}
					\item Let $b=\frac{\epsilon\delta}{M''N''E''^2m^2(m+1-k_i)}$.
					\item For $j=k$ to $j=m$, do the following:
					\begin{enumerate}
						\item Execute \hyperref[sec:algorithm 36]{algorithm 36} on the formal polynomial $D_{j,j}$, interval $(c_i, d_i)$, and target of $b$. Let $c_i$ and $d_i$ receive their updates.
					\end{enumerate}
					\item If a diagonal entry of ${K_i(c_i)}^TK_i(c_i)$ is less than $L'$, then do the following:
					\begin{enumerate}
						\item Let $z$ be the index of the column of the entry.
						\item Verify that $(Q^TN^{-1})(NQ)=Q^T(N^{-1}N)Q=Q^TI_mQ=Q^TQ=I_{m+1-k_i}$.
						\item Verify that $L'\le\frac{1}{\max(\lVert (Q^TN^{-1})(c_i)\rVert^2,1)^{(2(m+1-k_i)+2)!!}}$.
						\item Execute \hyperref[sec:algorithm 52]{algorithm 52} with matrices $Q^TN^{-1}$ and $NQ$, rational number $c_i$, and column index $z$.
						\item \textbf{Abort algorithm.}
					\end{enumerate}
					\item Otherwise, do the following:
					\begin{enumerate}
						\item \textbf{For $j=1$ to $j=m+1-k_i$, verify that $({K_i(c_i)}^TK_i(c_i))_{j,j}\ge L'$.}
						\item Verify that $xK_i-AK_i=(xI_m-A)K_i=M^{-1}DN^{-1}K_i=M^{-1}DN^{-1}NQE=M^{-1}DQE$.
						\item Verify that $\lVert c_iK_i(c_i)-AK_i(c_i)\rVert^2=\lVert M^{-1}(c_i)D(c_i)QE(c_i)\rVert^2\le\lVert M''J_m\frac{\epsilon\delta}{M''N''E''^2m^2(m+1-k_i)}QE''J_{m+1-k}\rVert^2=\lVert J_m\frac{\epsilon\delta}{N''E''m^2(m+1-k_i)}QJ_{m+1-k}\rVert^2=\lVert \frac{\epsilon\delta}{N''E''m^2}J_{m\times (m+1-k_i)}\rVert^2\le\lVert \frac{\epsilon\delta}{N''E''m^2}J_{m\times (m+1-k_i)}\rVert^2=\frac{m+1-k_i}{m^3}\cdot\frac{\epsilon^2\delta^2}{(N''E'')^2}$.
					\item \textbf{Therefore verify that $\lVert c_iK_i(c_i)-AK_i(c_i)\rVert^2\le\frac{m+1-k_i}{m^3}\cdot\frac{\epsilon^2\delta^2}{(N''E'')^2}\le\frac{m+1-k_i}{m}\epsilon^2$.}
					\end{enumerate}
				\end{enumerate}
			\end{enumerate}
		\subsection{Algorithm 55 (Spectral algorithm)}\label{sec:algorithm 55}
			\textbf{Choose an $m\times m$ matrix, $A$, whose entries are only rationals and is such that $A^T=A$. Choose a rational number $\epsilon>0$.} Now do the following:
			\begin{enumerate}
				\item Execute \hyperref[sec:algorithm 54]{algorithm 54} on matrix $A$ and rational $\epsilon$.
				\item Let $C$ be a diagonal matrix whose $i^{th}$, where $1\le i\le t$, group of entries are $m+1-k_i$ $c_i$s.
				\item Using \hyperref[sec:algorithm 53]{algorithm 53}, verify that $C$ is $m\times m$.
				\item Let $K$ be a matrix whose columns are the in-order concatenation of those of $K_1(c_1),K_2(c_2),\cdots,K_t(c_t)$.
				\item Using \hyperref[sec:algorithm 53]{algorithm 53} and (9e), verify that $K$ is $m\times m$.
				\item \textbf{Using \hyperref[sec:algorithm 53]{algorithm 53} and \hyperref[sec:algorithm 54]{algorithm 54}, verify that $\lVert KC-AK\rVert^2\le\sum_{i=1}^t\frac{m+1-k_i}{m}\epsilon^2=\frac{\sum_{i=1}^t (m+1-k_i)}{m}\epsilon^2=\frac{m}{m}\epsilon^2=\epsilon^2$.}
				\item For $i=1$ to $i=m$, do the following: For $j=1$ to $j=m$, do the following:
				\begin{enumerate}
					\item If $Ke_i$ was constructed during iteration $i=a$ and $Ke_j$ during iteration $i=b$ of (11), and if $a\ne b$, then do the following:
					\item Verify that $\lvert(c_b-c_a)(Ke_i)^T(Ke_j)\rvert$
					\begin{enumerate}
						\item $=\lvert c_b(Ke_i)^T(Ke_j)-c_a(Ke_i)^T(Ke_j)\rvert$
						\item $=\lvert(Ke_i)^T(c_bKe_j)-(c_aKe_i)^T(Ke_j)\rvert$
						\item $=\lvert(Ke_i)^T(AKe_j+c_bKe_j-AKe_j)-(AKe_i+c_aKe_i-AKe_i)^T(Ke_j)\rvert$
						\item $\le\lvert(Ke_i)^T(AKe_j)-(AKe_i)^T(Ke_j)\rvert+\lvert(Ke_i)^T(c_bKe_j-AKe_j)\rvert+\lvert(c_aKe_i-AKe_i)^T(Ke_j)\rvert$
						\item $\le\lvert(Ke_i)^TA(Ke_j)-(Ke_i)^TA^T(Ke_j)\rvert+\lvert mN''E''J_{1\times m}\frac{\epsilon\delta}{N''E''m^2}J_{m\times 1}\rvert+\lvert\frac{\epsilon\delta}{N''E''m^2}J_{1\times m}mN''E''J_{m\times 1}\rvert$
						\item $=2\epsilon\delta$.
					\end{enumerate}
					\item \textbf{Therefore verify that $\lvert {e_i}^T(K^TK)e_j\rvert=\lvert(Ke_i)^T(Ke_j)\rvert\le\frac{2\epsilon\delta}{c_b-c_a}\le 2\epsilon$.}
				\end{enumerate}
				\item \textbf{Using (7c) and \hyperref[sec:algorithm 54]{algorithm 54}, verify that the absolute values of all the non-diagonal entries $K^TK$ are less than or equal to $2\epsilon$.}
				\item \textbf{Using \hyperref[sec:algorithm 54]{algorithm 54}, verify that all the diagonal entries of $K^TK$ are more than or equal to $L'$.}
			\end{enumerate}
\end{document}
\grid
