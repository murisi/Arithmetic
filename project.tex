\documentclass[twocolumn]{article}
\usepackage{fullpage}
\usepackage{parskip}
\usepackage{amsmath}
\usepackage{amssymb}
\usepackage{datetime}
\usepackage[nottoc]{tocbibind}
\usepackage{hyperref}
\usepackage{enumitem}
\usepackage{glossaries}
\hypersetup{
	colorlinks = true,
	linktoc = all,
}

\setcounter{page}{0}
\setcounter{section}{-1}
\setlist[enumerate,2,3,4]{leftmargin=0.3cm}
\setlength{\columnsep}{0.7cm}

\DeclareMathOperator*{\diff}{\Delta}
\DeclareMathOperator{\sgn}{sgn}
\DeclareMathOperator{\mat}{mat}
\DeclareMathOperator{\tr}{tr}
\DeclareMathOperator{\diag}{diag}
\DeclareMathOperator{\comp}{comp}
\DeclareMathOperator{\mon}{mon}
\DeclareMathOperator{\pows}{pows}
\DeclareMathOperator{\cols}{cols}
\DeclareMathOperator{\rows}{rows}
\DeclareMathOperator{\sel}{sel}
\DeclareMathOperator{\last}{last}
\DeclareMathOperator{\Nu}{nu}
\DeclareMathOperator{\De}{de}
\DeclareMathOperator{\Po}{po}
\DeclareMathOperator{\Ne}{ne}
\let\div\relax\DeclareMathOperator{\div}{div}
\let\mod\relax\DeclareMathOperator{\mod}{mod}
\let\Re\relax\DeclareMathOperator{\Re}{re}
\let\Im\relax\DeclareMathOperator{\Im}{im}

\newcommand{\for}{\text{ for }}
\newcommand{\ul}[1]{\protect\underline{#1}}
\newcommand{\ol}[1]{\overline{#1}}
\newcommand{\conj}[1]{\overline{#1}}
\newcounter{declaration}[part]
\newcommand{\declaration}[1]{\subsubsection*{Declaration \thepart:\thedeclaration}\label{sec:declaration #1}\expandafter\edef\csname declaration#1\endcsname{\thepart:\thedeclaration}\addtocounter{declaration}{1}}
\newcommand{\declarationhr}[1]{\hyperref[sec:declaration #1]{declaration \expandafter\csname declaration#1\endcsname}}
\newcounter{procedure}[part]
\newcommand{\procedure}[1]{\subsection*{Procedure \thepart:\theprocedure}\label{sec:procedure #1}\expandafter\edef\csname procedure#1\endcsname{\thepart:\theprocedure}\addtocounter{procedure}{1}}
\newcommand{\objective}{\subsubsection*{Objective}}
\newcommand{\implementation}{\subsubsection*{Implementation}}
\newcommand{\procedurehr}[1]{\hyperref[sec:procedure #1]{procedure \expandafter\csname procedure#1\endcsname}}

\makenoidxglossaries
\newglossaryentry{integer}{name={integer},description={}}
\newglossaryentry{intPo}{name=$\Po(a)$,description={positive part of $a$}}
\newglossaryentry{intNe}{name=$\Ne(a)$,description={negative part of $a$}}
\newglossaryentry{inta=b}{name=\ensuremath{a=b},description={integer equality}}
\newglossaryentry{inta+b}{name=\ensuremath{a+b},description={integer addition}}
\newglossaryentry{inta}{name=\ensuremath{a},description={}}
\newglossaryentry{int-a}{name=\ensuremath{-a},description={integer negation}}
\newglossaryentry{intab}{name=\ensuremath{ab},description={integer multiplication}}
\newglossaryentry{inta<b}{name=\ensuremath{a<b},description={integer less than}}
\newglossaryentry{int||a||}{name=\ensuremath{\lVert a\rVert},description={absolute value}}
\newglossaryentry{intsgn(a)}{name=\ensuremath{\sgn(a)},description={sign function}}
\newglossaryentry{intadivb}{name=\ensuremath{a\div b},description={integer division}}
\newglossaryentry{intamodb}{name=\ensuremath{a\mod b},description={integer modulus}}
\newglossaryentry{inta=bmodc}{name=\ensuremath{a\equiv b(\mod c)},description={modular equality}}
\newglossaryentry{int(a,b)}{name=\ensuremath{(a,b)},description={}}
\newglossaryentry{int(a_0,a_1,...,a_n-1)}{name=\ensuremath{(a_0,a_1,\cdots,a_{n-1})},description={}}
\newglossaryentry{intprime}{name={prime number},description={}}
\newglossaryentry{int|a|}{name=\ensuremath{\lvert a\rvert},description={length of list}}
\newglossaryentry{inta^b}{name=\ensuremath{a^{\frown}b},description={list concatenation}}
\newglossaryentry{inta_*}{name=\ensuremath{a_*},description={product of list}}
\newglossaryentry{intpif(r)}{name=\ensuremath{\prod_r^Rf(r)},description={pi product notation}}
\newglossaryentry{int[a:b]}{name=\ensuremath{[a:b]},description={integer range}}
\newglossaryentry{int[a,b]}{name=\ensuremath{[a,b]},description={}}
\newglossaryentry{int[a_0,a_1,...,a_n-1]}{name=\ensuremath{[a_0,a_1,\cdots,a_{n-1}]},description={}}
\newglossaryentry{intchibdac}{name=\ensuremath{\chi_{b,d}(a,c)},description={}}
\newglossaryentry{intchib_0,b_1,...,b_n-1a_0a_1,...,a_n-1}{name=\ensuremath{\chi_{b_0,b_1,\cdots,b_{n-1}}(a_0,a_1,\cdots,a_{n-1})},description={}}
\newglossaryentry{intphi(n)}{name=\ensuremath{\phi(n)},description={Euler's phi function}}
\newglossaryentry{intatimesb}{name=\ensuremath{a\times b},description={Cartesian product}}
\newglossaryentry{int[P]}{name=\ensuremath{[P]},description={Iverson bracket}}
\newglossaryentry{inta_+}{name=\ensuremath{a_+},description={sum of list}}
\newglossaryentry{intsigmaf(r)}{name=\ensuremath{\sum_r^Rf(r)},description={sigma summation notation}}
\newglossaryentry{inta^ulb}{name=\ensuremath{a^{\ul{b}}},description={falling power}}
\newglossaryentry{inta^olb}{name=\ensuremath{a^{\ol{b}}},description={rising power}}
\newglossaryentry{intnchooser}{name=\ensuremath{\binom{n}{r}},description={binomial coefficient}}
\newglossaryentry{rational}{name={rational number},description={}}
\newglossaryentry{ratNu}{name=$\Nu(a)$,description={numerator of $a$}}
\newglossaryentry{ratDe}{name=$\De(a)$,description={denominator of $a$}}
\newglossaryentry{rata=b}{name=\ensuremath{a=b},description={rational equality}}
\newglossaryentry{rata+b}{name=\ensuremath{a+b},description={rational addition}}
\newglossaryentry{rata}{name=\ensuremath{a},description={}}
\newglossaryentry{rat-a}{name=\ensuremath{-a},description={rational negation}}
\newglossaryentry{ratab}{name=\ensuremath{ab},description={rational multiplication}}
\newglossaryentry{rat1/a}{name=\ensuremath{\frac{1}{a}},description={rational reciprocal}}
\newglossaryentry{rata<b}{name=\ensuremath{a<b},description={rational less than}}
\newglossaryentry{ratfloor}{name=\ensuremath{\lfloor a\rfloor},description={floor function}}
\newglossaryentry{ratceil}{name=\ensuremath{\lceil a\rceil},description={ceiling function}}
\newglossaryentry{ratminc}{name=\ensuremath{\min(c)},description={minimum of list}}
\newglossaryentry{ratmincr}{name=\ensuremath{\min_r^Rc(r)},description={minimum notation}}
\newglossaryentry{ratmaxc}{name=\ensuremath{\max(c)},description={maximum of list}}
\newglossaryentry{ratmaxcr}{name=\ensuremath{\max_r^Rc(r)},description={maximum notation}}
\newglossaryentry{ratpolynomial}{name={polynomial},description={}}
\newglossaryentry{ratpolya_i}{name=\ensuremath{a_i},description={polynomial coefficient}}
\newglossaryentry{ratpolya=b}{name=\ensuremath{a=b},description={polynomial equality}}
\newglossaryentry{ratpolyLambda}{name=\ensuremath{\Lambda(a,b)},description={polynomial evaluation}}
\newglossaryentry{ratlistcomp}{name=\ensuremath{\langle f(j)\for j\in R\rangle},description={list comprehension}}
\newglossaryentry{ratpolya+b}{name=\ensuremath{a+b},description={polynomial addition}}
\newglossaryentry{ratpolya}{name=\ensuremath{a},description={}}
\newglossaryentry{ratpoly-a}{name=\ensuremath{-a},description={polynomial negation}}
\newglossaryentry{ratpolyab}{name=\ensuremath{ab},description={polynomial multiplication}}
\newglossaryentry{ratpolylambda}{name=\ensuremath{\lambda},description={}}
\newglossaryentry{ratpolydeg}{name=\ensuremath{\deg(a)},description={polynomial degree}}
\newglossaryentry{ratpolymonic}{name={monic polynomial},description={}}
\newglossaryentry{ratpolymon}{name=\ensuremath{\mon(p)},description={}}
\newglossaryentry{ratpolydiv}{name=\ensuremath{a\div b},description={polynomial division}}
\newglossaryentry{ratpolymod}{name=\ensuremath{a\mod b},description={polynomial modulus}}
\newglossaryentry{ratpolyJ}{name=\ensuremath{J_s(x)},description={}}
\newglossaryentry{complex}{name={complex number},description={}}
\newglossaryentry{complexRe}{name=$\Re(a)$,description={real part of $a$}}
\newglossaryentry{complexIm}{name=$\Im(a)$,description={imaginary part of $a$}}
\newglossaryentry{complexa=b}{name=\ensuremath{a=b},description={complex equality}}
\newglossaryentry{complexa+b}{name=\ensuremath{a+b},description={complex addition}}
\newglossaryentry{complexa}{name=\ensuremath{a},description={}}
\newglossaryentry{complex-a}{name=\ensuremath{-a},description={complex negation}}
\newglossaryentry{complexab}{name=\ensuremath{ab},description={complex multiplication}}
\newglossaryentry{complexconj}{name=\ensuremath{\conj{a}},description={complex conjugate}}
\newglossaryentry{complexabs^2}{name=\ensuremath{\lVert a\rVert^2},description={absolute value squared}}
\newglossaryentry{complex1/a}{name=\ensuremath{\frac{1}{a}},description={complex reciprocal}}
\newglossaryentry{complexi}{name=\ensuremath{i},description={imaginary number}}
\newglossaryentry{complexexp}{name=\ensuremath{\exp_n(a)},description={complex exponential function}}
\newglossaryentry{complexdiff}{name=\ensuremath{\diff_{x=y}^zf(x)},description={differentiation}}
\newglossaryentry{complexisqrt}{name=\ensuremath{\lceil\sqrt{x}\rceil},description={integer square root}}
\newglossaryentry{complexcos}{name=\ensuremath{\cos_n(z)},description={cosine}}
\newglossaryentry{complexsin}{name=\ensuremath{\sin_n(z)},description={sine}}
\newglossaryentry{complexintegral}{name=\ensuremath{\int_r^Rf(\#r,r,dr)},description={complex integral}}
\newglossaryentry{complexDeltaX}{name=\ensuremath{\Delta X},description={first difference}}
\newglossaryentry{complexmu}{name=\ensuremath{\mu(a,b,n)},description={arithmetic progression}}
\newglossaryentry{complextau}{name=\ensuremath{\tau_n},description={tau}}
\newglossaryentry{complexpolynomial}{name={complex polynomial},description={}}
\newglossaryentry{matrix}{name={matrix},description={}}
\newglossaryentry{matrixentry}{name=\ensuremath{A_{I,J}},description={submatrix}}
\newglossaryentry{matrixA=B}{name=\ensuremath{A=B},description={matrix equality}}
\newglossaryentry{matrixA+B}{name=\ensuremath{A+B},description={matrix addition}}
\newglossaryentry{matrix0mn}{name=\ensuremath{0_{m\times n}},description={$m\times n$ zero matrix}}
\newglossaryentry{matrix-A}{name=\ensuremath{-A},description={matrix negation}}
\newglossaryentry{matrixAB}{name=\ensuremath{AB},description={matrix multiplication}}
\newglossaryentry{matrixamm}{name=\ensuremath{a_{m\times m}},description={scalar matrix}}
\newglossaryentry{matrixrow}{name=\ensuremath{A_{i,*}},description={matrix row}}
\newglossaryentry{matrixcol}{name=\ensuremath{A_{*,i}},description={matrix column}}
\newglossaryentry{matrixdiag}{name={matrix diagonal},description={}}
\newglossaryentry{matrixdiagmatrix}{name={diagonal matrix},description={}}
\newglossaryentry{matrixtilt}{name={tilt matrix},description={}}
\newglossaryentry{matrixA^-1}{name=\ensuremath{A^{-1}},description={}}
\newglossaryentry{matrixrows}{name=\ensuremath{\rows(A)},description={number of rows of $A$}}
\newglossaryentry{matrixcols}{name=\ensuremath{\cols(A)},description={number of columns of $A$}}
\newglossaryentry{matrixbdiag}{name=\ensuremath{\diag(C)},description={block diagonal matrix}}
\newglossaryentry{matrixdet}{name=\ensuremath{\det(A)},description={matrix determinant}}
\newglossaryentry{matrixC_k(A)}{name=\ensuremath{C_k(A)},description={$k$\textsuperscript{th} compound matrix}}
\newglossaryentry{matrixlabelledentry}{name=\ensuremath{A_{\ul{I},\ul{J}}},description={labelled matrix entry}}
\newglossaryentry{matrixtranspose}{name=\ensuremath{A^T},description={matrix transpose}}
\newglossaryentry{matrixleftdiv}{name=\ensuremath{A\backslash B},description={matrix left division}}
\newglossaryentry{matrixrightdiv}{name=\ensuremath{A/B},description={matrix right division}}
\newglossaryentry{matrixe}{name=\ensuremath{(e_i)_{k\times 1}},description={standard unit vector}}
\newglossaryentry{matrixmattp}{name=\ensuremath{\mat_t(p)},description={}}
\newglossaryentry{matrixcompmatrix}{name=\ensuremath{\comp(p)},description={companion matrix}}
\newglossaryentry{matrixlast}{name=\ensuremath{\last_A},description={last polynomial}}
\newglossaryentry{matrixpows}{name=\ensuremath{\pows(A)},description={}}
\newglossaryentry{matrixtr}{name=\ensuremath{\tr(A)},description={matrix trace}}
\newglossaryentry{matrixsymmetric}{name={symmetric matrix},description={}}
\newglossaryentry{matrixsel}{name=\ensuremath{\sel_A},description={selector polynomial}}

\begin{document}
	\title{Arithmetic: A Programmatic Approach}
	\author{Murisi Tarusenga}
	\date{\today{} \currenttime}
	\maketitle
	\section*{Introduction}
		What follows is a reformulation of the elementary parts of number theory, hard analysis, and linear algebra in terms of a system of procedures for achieving particular objectives, objectives like showing that modular exponentiation of a specific integer with specific properties yields a stated result. So, while formal mathematics usually takes the format of definition-theorem-proof, this project has the format of declaration-procedure objective-procedure implementation. So where there usually would have been a statement and proof of Euler's totient theorem, \procedurehr{1.50} is provided, and where there would have been a definition of Euler's totient function, \declarationhr{1.16} is provided.
		
		At this point the natural question is that of how we are to know that the following procedure implementations achieve the corresponding procedure objectives for all inputs. Well, strictly speaking, the only way to know that the following procedures always achieve their respective objectives is to actually execute them on all possible inputs and actually verify that the objective is met. However, when the input can be any integer, this proposal is in-principle not possible. So the actual question is that of how we can see the \textit{potential} of the following procedure implementations to achieve their respective objectives on different inputs, that is, of how we can get that feeling that if the input is changed to this or that, the objective should still be achieved. And the answer to this question is that we can see the potential of the following procedure implementations by simply looking at their (purposefully chosen) syntax in the same way that we can simply see from the syntax of the code fragment, "if $a=b$ and $b=c$, then verify that $a=c$, otherwise verify that $a\ne c$", the potential of the instructions to be carried out successfully on different integer inputs.
		
		For the purposes of storage and transmission of knowledge pertaining to the elementary parts of number theory and linear algebra, the following procedures are interchangeable with their analogous proofs in the sense that, assuming equal competence in programming and proving, if you have the procedure objective and implementation, you can trivially generate the analogous theorem and proof, and if you are in possession of the theorem and proof, then you can trivially generate the analogous procedure objective and implementation.
	\tableofcontents
	\printnoidxglossary[title={Declarations},sort=def]
	\twocolumn[\part{Integer Arithmetic}]
		\declaration{1.22}
			The phrase "\gls{integer}" will be used as a shorthand for an ordered pair of natural numbers.
		\declaration{1.23}
			The phrase "the positive part of $a$" and the notation \gls{intPo}, where $a$ is an integer, will be used as a shorthand for the first entry of $a$.
		\declaration{1.24}
			The phrase "the negative part of $a$" and the notation \gls{intNe}, where $a$ is an integer, will be used as a shorthand for the second entry of $a$.
		\declaration{1.25}
			The phrase "\gls{inta=b}", where $a,b$ are integers, will be used as a shorthand for "$\Po(a)+\Ne(b)=\Ne(a)+\Po(b)$".
		\procedure{1.65}
			\objective
				Choose an integer $a$. The objective of the following instructions is to show that $a=a$.
			\implementation
				\begin{enumerate}
					\item Verify that $\Po(a)+\Ne(a)=\Ne(a)+\Po(a)$.
					\item \textbf{Hence verify that $a=a$.}
				\end{enumerate}
		\procedure{1.66}
			\objective
				Choose two integers $a,b$ such that $a=b$. The objective of the following instructions is to show that $b=a$.
			\implementation
				\begin{enumerate}
					\item Verify that $\Po(a)+\Ne(b)=\Ne(a)+\Po(b)$.
					\item Hence verify that $\Po(b)+\Ne(a)=\Ne(b)+\Po(a)$.
					\item \textbf{Hence verify that $b=a$.}
				\end{enumerate}
		\procedure{1.67}
			\objective
				Choose three integers $a,b,c$ such that $a=b$ and $b=c$. The objective of the following instructions is to show that $a=c$.
			\implementation
				\begin{enumerate}
					\item Using \declarationhr{1.25}, verify that $\Po(a)+\Ne(b)=\Ne(a)+\Po(b)$.
					\item Using \declarationhr{1.25}, verify that $\Po(b)+\Ne(c)=\Ne(b)+\Po(c)$.
					\item Hence verify that $\Po(a)+\Ne(b)+\Po(b)+\Ne(c)=\Ne(a)+\Po(b)+\Ne(b)+\Po(c)$.
					\item Hence verify that $\Po(a)+\Ne(c)=\Ne(a)+\Po(c)$.
					\item \textbf{Hence verify that $a=c$.}
				\end{enumerate}
		\declaration{1.26}
			The notation \gls{inta+b}, where $a,b$ are integers, will be used as a shorthand for the pair $\langle\Po(a)+\Po(b),\Ne(a)+\Ne(b)\rangle$.
		\procedure{1.68}
			\objective
				Choose four integers $a,b,c,d$ such that $a=c$ and $b=d$. The objective of the following instructions is to show that $a+b=c+d$.
			\implementation
				\begin{enumerate}
					\item Using \declarationhr{1.25}, verify that $\Po(a)+\Ne(c)=\Ne(a)+\Po(c)$.
					\item Using \declarationhr{1.25}, verify that $\Po(b)+\Ne(d)=\Ne(b)+\Po(d)$.
					\item Hence verify that $a+b$
					\begin{enumerate}
						\item $=\langle\Po(a),\Ne(a)\rangle+\langle\Po(b),\Ne(b)\rangle$
						\item $=\langle\Po(a)+\Po(b),\Ne(a)+\Ne(b)\rangle$
						\item $=\langle\Po(a)+\Po(b)+\Ne(c)+\Ne(d),\Ne(a)+\Ne(b)+\Ne(c)+\Ne(d)\rangle$
						\item $=\langle(\Po(a)+\Ne(c))+(\Po(b)+\Ne(d)),\Ne(a)+\Ne(b)+\Ne(c)+\Ne(d)\rangle$
						\item $=\langle(\Ne(a)+\Po(c))+(\Ne(b)+\Po(d)),\Ne(a)+\Ne(b)+\Ne(c)+\Ne(d)\rangle$
						\item $=\langle\Ne(a)+\Ne(b)+\Po(c)+\Po(d),\Ne(a)+\Ne(b)+\Ne(c)+\Ne(d)\rangle$
						\item $=\langle\Po(c)+\Po(d),\Ne(c)+\Ne(d)\rangle$
						\item $=\langle\Po(c),\Ne(c)\rangle+\langle\Po(d),\Ne(d)\rangle$
						\item $=c+d$.
					\end{enumerate}
				\end{enumerate}
		\procedure{1.69}
			\objective
				Choose three integers $a,b,c$. The objective of the following instructions is to show that $(a+b)+c=a+(b+c)$.
			\implementation
				\begin{enumerate}
					\item Verify that $(a+b)+c$
					\begin{enumerate}
						\item $=\langle\Po(a)+\Po(b),\Ne(a)+\Ne(b)\rangle+\langle\Po(c),\Ne(c)\rangle$
						\item $=\langle(\Po(a)+\Po(b))+\Po(c),(\Ne(a)+\Ne(b))+\Ne(c)\rangle$
						\item $=\langle\Po(a)+(\Po(b)+\Po(c)),\Ne(a)+(\Ne(b)+\Ne(c))\rangle$
						\item $=\langle\Po(a),\Ne(a)\rangle+\langle\Po(b)+\Po(c),\Ne(b)+\Ne(c)\rangle$
						\item $=a+(b+c)$.
					\end{enumerate}
				\end{enumerate}
		\procedure{1.70}
			\objective
				Choose two integers $a,b$. The objective of the following instructions is to show that $a+b=b+a$.
			\implementation
				\begin{enumerate}
					\item $a+b$
					\begin{enumerate}
						\item $=\langle\Po(a)+\Po(b),\Ne(a)+\Ne(b)\rangle$
						\item $=\langle\Po(b)+\Po(a),\Ne(b)+\Ne(a)\rangle$
						\item $=b+a$.
					\end{enumerate}
				\end{enumerate}
		\declaration{1.27}
			The notation \gls{inta}, where $a$ is a natural number, will contextually be used as a shorthand for the pair $\langle a,0\rangle$.
		\procedure{1.71}
			\objective
				Choose an integer $a$. The objective of the following instructions is to show that $0+a=a$.
			\implementation
				\begin{enumerate}
					\item Verify that $0+a$
					\begin{enumerate}
						\item $=\langle 0,0\rangle+\langle\Po(a),\Ne(a)\rangle$
						\item $=\langle 0+\Po(a),0+\Ne(a)\rangle$
						\item $=\langle\Po(a),\Ne(a)\rangle$
						\item $=a$.
					\end{enumerate}
				\end{enumerate}
		\declaration{1.28}
			The notation \gls{int-a}, where $a$ is an integer, will be used as a shorthand for the pair $\langle\Ne(a),\Po(a)\rangle$.
		\procedure{1.72}
			\objective
				Choose two integers $a,b$ such that $a=b$. The objective of the following instructions is to show that $-a=-b$.
			\implementation
				\begin{enumerate}
					\item Using \declarationhr{1.25}, verify that $\Po(a)+\Ne(b)=\Ne(a)+\Po(b)$.
					\item Hence verify that $-a$
					\begin{enumerate}
						\item $=\langle\Ne(a),\Po(a)\rangle$
						\item $=\langle\Ne(a)+\Po(b),\Po(a)+\Po(b)\rangle$
						\item $=\langle\Po(a)+\Ne(b),\Po(a)+\Po(b)\rangle$
						\item $=\langle\Ne(b),\Po(b)\rangle$
						\item $=-b$.
					\end{enumerate}
				\end{enumerate}
		\procedure{1.73}
			\objective
				Choose an integer $a$. The objective of the following instructions is to show that $-a+a=0$.
			\implementation
				\begin{enumerate}
					\item Verify that $-a+a$
					\begin{enumerate}
						\item $=(-a)+a$
						\item $=\langle\Ne(a),\Po(a)\rangle+\langle\Po(a),\Ne(a)\rangle$
						\item $=\langle\Ne(a)+\Po(a),\Po(a)+\Ne(a)\rangle$
						\item $=\langle 0,0\rangle$
						\item $=0$.
					\end{enumerate}
				\end{enumerate}
		\declaration{1.29}
			The notation \gls{intab}, where $a,b$ are integers, will be used as a shorthand for the pair $\langle\Po(a)\Po(b)+\Ne(a)\Ne(b),\Po(a)\Ne(b)+\Ne(a)\Po(b)\rangle$.
		\procedure{1.74}
			\objective
				Choose four integers $a,b,c,d$ such that $a=c$ and $b=d$. The objective of the following instructions is to show that $ab=cd$.
			\implementation
				\begin{enumerate}
					\item Using \declarationhr{1.25}, verify that $\Po(a)+\Ne(c)=\Ne(a)+\Po(c)$.
					\item Using \declarationhr{1.25}, verify that $\Po(b)+\Ne(d)=\Ne(b)+\Po(d)$.
					\item Hence verify that $ab$
					\begin{enumerate}
						\item $=\langle\Po(a)\Po(b)+\Ne(a)\Ne(b),\Po(a)\Ne(b)+\Ne(a)\Po(b)\rangle$
						\item $=\langle\Po(a)\Po(b)+\Ne(a)\Ne(b)+\Po(a)\Ne(d)+\Ne(c)\Po(d)+\Po(c)\Ne(d),\Po(a)\Ne(b)+\Ne(a)\Po(b)+\Po(a)\Ne(d)+\Ne(c)\Po(d)+\Po(c)\Ne(d)\rangle$
						\item $=\langle\Po(a)(\Po(b)+\Ne(d))+\Ne(a)\Ne(b)+\Ne(c)\Po(d)+\Po(c)\Ne(d),\Po(a)\Ne(b)+\Ne(a)\Po(b)+\Po(a)\Ne(d)+\Ne(c)\Po(d)+\Po(c)\Ne(d)\rangle$
						\item $=\langle\Po(a)(\Ne(b)+\Po(d))+\Ne(a)\Ne(b)+\Ne(c)\Po(d)+\Po(c)\Ne(d),\Po(a)\Ne(b)+\Ne(a)\Po(b)+\Po(a)\Ne(d)+\Ne(c)\Po(d)+\Po(c)\Ne(d)\rangle$
						\item $=\langle(\Po(a)+\Ne(c))\Po(d)+\Ne(a)\Ne(b)+\Po(c)\Ne(d),\Ne(a)\Po(b)+\Po(a)\Ne(d)+\Ne(c)\Po(d)+\Po(c)\Ne(d)\rangle$
						\item $=\langle(\Ne(a)+\Po(c))\Po(d)+\Ne(a)\Ne(b)+\Po(c)\Ne(d),\Ne(a)\Po(b)+\Po(a)\Ne(d)+\Ne(c)\Po(d)+\Po(c)\Ne(d)\rangle$
						\item $=\langle\Ne(a)(\Po(d)+\Ne(b))+\Po(c)\Po(d)+\Po(c)\Ne(d),\Ne(a)\Po(b)+\Po(a)\Ne(d)+\Ne(c)\Po(d)+\Po(c)\Ne(d)\rangle$
						\item $=\langle\Ne(a)(\Po(b)+\Ne(d))+\Po(c)\Po(d)+\Po(c)\Ne(d),\Ne(a)\Po(b)+\Po(a)\Ne(d)+\Ne(c)\Po(d)+\Po(c)\Ne(d)\rangle$
						\item $=\langle(\Ne(a)+\Po(c))\Ne(d)+\Po(c)\Po(d),\Po(a)\Ne(d)+\Ne(c)\Po(d)+\Po(c)\Ne(d)\rangle$
						\item $=\langle(\Po(a)+\Ne(c))\Ne(d)+\Po(c)\Po(d),\Po(a)\Ne(d)+\Ne(c)\Po(d)+\Po(c)\Ne(d)\rangle$
						\item $=\langle\Ne(c)\Ne(d)+\Po(c)\Po(d),\Ne(c)\Po(d)+\Po(c)\Ne(d)\rangle$
						\item $=cd$.
					\end{enumerate}
				\end{enumerate}
		\procedure{1.75}
			\objective
				Choose three integers $a,b,c$. The objective of the following instructions is to show that $(ab)c=a(bc)$.
			\implementation
				\begin{enumerate}
					\item Verify that $(ab)c$
					\begin{enumerate}
						\item $=\langle\Po(a)\Po(b)+\Ne(a)\Ne(b),\Po(a)\Ne(b)+\Ne(a)\Po(b)\rangle\langle\Po(c),\Ne(c)\rangle$
						\item $=\langle(\Po(a)\Po(b)+\Ne(a)\Ne(b))\Po(c)+(\Po(a)\Ne(b)+\Ne(a)\Po(b))\Ne(c),(\Po(a)\Po(b)+\Ne(a)\Ne(b))\Ne(c)+(\Po(a)\Ne(b)+\Ne(a)\Po(b))\Po(c)\rangle$
						\item $=\langle\Po(a)(\Po(b)\Po(c)+\Ne(b)\Ne(c))+\Ne(a)(\Po(b)\Ne(c)+\Ne(b)\Po(c)),\Po(a)(\Po(b)\Ne(c)+\Ne(b)\Po(c))+\Ne(a)(\Po(b)\Po(c)+\Ne(b)\Ne(c))\rangle$
						\item $=\langle\Po(a),\Ne(a)\rangle\langle\Po(b)\Po(c)+\Ne(b)\Ne(c),\Po(b)\Ne(c)+\Ne(b)\Po(c)\rangle$
						\item $=a(bc)$.
					\end{enumerate}
				\end{enumerate}
		\procedure{1.76}
			\objective
				Choose two integers $a,b$. The objective of the following instructions is to show that $ab=ba$.
			\implementation
				\begin{enumerate}
					\item $ab$
					\begin{enumerate}
						\item $=\langle\Po(a)\Po(b)+\Ne(a)\Ne(b),\Po(a)\Ne(b)+\Ne(a)\Po(b)\rangle$
						\item $=\langle\Po(b)\Po(a)+\Ne(b)\Ne(a),\Po(b)\Ne(a)+\Ne(b)\Po(a)\rangle$
						\item $=ba$.
					\end{enumerate}
				\end{enumerate}
		\procedure{1.77}
			\objective
				Choose an integer $a$. The objective of the following instructions is to show that $1a=a$.
			\implementation
				\begin{enumerate}
					\item Verify that $1a$
					\begin{enumerate}
						\item $=\langle 1,0\rangle\langle\Po(a),\Ne(a)\rangle$
						\item $=\langle 1\Po(a)+0\Ne(a),1\Ne(a)+0\Po(a)\rangle$
						\item $=\langle\Po(a),\Ne(a)\rangle$
						\item $=a$.
					\end{enumerate}
				\end{enumerate}
		\procedure{1.78}
			\objective
				Choose three integers $a,b,c$. The objective of the following instructions is to show that $a(b+c)=ab+ac$.
			\implementation
				\begin{enumerate}
					\item $a(b+c)$
					\begin{enumerate}
						\item $=\langle\Po(a),\Ne(a)\rangle\langle\Po(b)+\Po(c),\Ne(b)+\Ne(c)\rangle$
						\item $=\langle\Po(a)(\Po(b)+\Po(c))+\Ne(a)(\Ne(b)+\Ne(c)),\Po(a)(\Ne(b)+\Ne(c))+\Ne(a)(\Po(b)+\Po(c))\rangle$
						\item $=\langle(\Po(a)\Po(b)+\Ne(a)\Ne(b))+(\Po(a)\Po(c)+\Ne(a)\Ne(c)),(\Po(a)\Ne(b)+\Ne(a)\Po(b))+(\Po(a)\Ne(c)+\Ne(a)\Po(c))\rangle$
						\item $=\langle\Po(a)\Po(b)+\Ne(a)\Ne(b),\Po(a)\Ne(b)+\Ne(a)\Po(b)\rangle+\langle\Po(a)\Po(c)+\Ne(a)\Ne(c),\Po(a)\Ne(c)+\Ne(a)\Po(c)\rangle$
						\item $=ab+ac$.
					\end{enumerate}
				\end{enumerate}
		\declaration{1.30}
			The phrase "\gls{inta<b}", where $a,b$ are rational numbers, will be used as a shorthand for "$\Po(a)+\Ne(b)<\Ne(a)+\Po(b)$".
		\procedure{1.79}
			\objective
				Choose four integers $a,b,c,d$ such that $a<b$, $a=c$ and $b=d$. The objective of the following instructions is to show that $c<d$.
			\implementation
				\begin{enumerate}
					\item Using \declarationhr{1.25}, verify that $\Po(a)+\Ne(c)=\Ne(a)+\Po(c)$.
					\item Using \declarationhr{1.25}, verify that $\Po(b)+\Ne(d)=\Ne(b)+\Po(d)$.
					\item Using \declarationhr{1.30}, verify that $\Po(a)+\Ne(b)<\Ne(a)+\Po(b)$.
					\item Hence verify that $\Po(c)+\Ne(d)$
					\begin{enumerate}
						\item $=(\Ne(a)+\Po(c))+(\Po(b)+\Ne(d))-\Ne(a)-\Po(b)$
						\item $=(\Po(a)+\Ne(c))+(\Ne(b)+\Po(d))-\Ne(a)-\Po(b)$
						\item $=(\Po(a)+\Ne(b))+\Ne(c)+\Po(d)-\Ne(a)-\Po(b)$
						\item $<(\Ne(a)+\Po(b))+\Ne(c)+\Po(d)-\Ne(a)-\Po(b)$
						\item $=\Ne(c)+\Po(d)$.
					\end{enumerate}
					\item \textbf{Hence verify that $c<d$.}
				\end{enumerate}
		\procedure{1.80}
			\objective
				Choose three integers $a,b,c$ such that $a<b$. The objective of the following instructions is to show that $a+c<b+c$.
			\implementation
				\begin{enumerate}
					\item Using \declarationhr{1.30}, verify that $\Po(a)+\Ne(b)<\Ne(a)+\Po(b)$.
					\item Hence verify that $\Po(a+c)+\Ne(b+c)$
					\begin{enumerate}
						\item $=\Po(a)+\Po(c)+\Ne(b)+\Ne(c)$
						\item $=(\Po(a)+\Ne(b))+\Po(c)+\Ne(c)$
						\item $=(\Ne(a)+\Po(b))+\Po(c)+\Ne(c)$
						\item $=\Ne(a)+\Ne(c)+\Po(b)+\Po(c)$
						\item $=\Ne(a+c)+\Po(b+c)$.
					\end{enumerate}
					\item \textbf{Hence verify that $a+c<b+c$.}
				\end{enumerate}
		\procedure{1.81}
			\objective
				Choose two integers $a,b$ such that $a<b$. The objective of the following instructions is to show that $a\ne b$ and $b\not<a$.
			\implementation
				\begin{enumerate}
					\item Verify that $\Po(a)+\Ne(b)<\Ne(a)+\Po(b)$.
					\item Hence verify that $\Po(a)+\Ne(b)\ne\Ne(a)+\Po(b)$.
					\item \textbf{Hence verify that $a\ne b$.}
					\item Also verify that $\Ne(a)+\Po(b)\not<\Po(a)+\Ne(b)$.
					\item \textbf{Hence verify that $b\not<a$.}
				\end{enumerate}
		\procedure{1.82}
			\objective
				Choose two integers $a,b$ such that $a=b$. The objective of the following instructions is to show that $a\not<b$ and $b\not<a$.
			\implementation
				Implementation is analogous to that of \procedurehr{1.81}.
		\procedure{1.83}
			\objective
				Choose two integers $a,b$ such that $a\ne b$. The objective of the following instructions is to show that $a<b$ or $b<a$.
			\implementation
				\begin{enumerate}
					\item Verify that $\Po(a)+\Ne(b)\ne\Ne(a)+\Po(b)$.
					\item If $\Po(a)+\Ne(b)<\Ne(a)+\Po(b)$, then do the following:
					\begin{enumerate}
						\item \textbf{Verify that $a<b$.}
					\end{enumerate}
					\item Otherwise do the following:
					\begin{enumerate}
						\item Verify that $\Ne(a)+\Po(b)<\Po(a)+\Ne(b)$.
						\item \textbf{Hence verify that $b<a$.}
					\end{enumerate}
				\end{enumerate}
		\procedure{1.84}
			\objective
				Choose two integers $a,b$ such that $a\not<b$. The objective of the following instructions is to show that $a=b$ or $b<a$.
			\implementation
				Implementation is analogous to that of \procedurehr{1.83}.
		\procedure{1.85}
			\objective
				Choose two integers $a,b$ such that $0<a$ and $0<b$. The objective of the following instructions is to show that $0<a+b$.
			\implementation
				\begin{enumerate}
					\item Using \declarationhr{1.30}, verify that $\Ne(a)=\Po(0)+\Ne(a)<\Ne(0)+\Po(a)=\Po(a)$.
					\item Using \declarationhr{1.30}, verify that $\Ne(b)=\Po(0)+\Ne(b)<\Ne(0)+\Po(b)=\Po(b)$.
					\item Hence verify that $\Po(0)+\Ne(a+b)=\Ne(a+b)=\Ne(a)+\Ne(b)<\Po(a)+\Po(b)=\Po(a+b)=\Ne(0)+\Po(a+b)$.
					\item \textbf{Hence verify that $0<a+b$.}
				\end{enumerate}
		\procedure{1.86}
			\objective
				Choose two integers $a,b$ such that $0<a$ and $0<b$. The objective of the following instructions is to show that $0<ab$.
			\implementation
				\begin{enumerate}
					\item Using \declarationhr{1.30}, verify that $\Ne(a)=\Po(0)+\Ne(a)<\Ne(0)+\Po(a)=\Po(a)$.
					\item Hence verify that $0<\Po(a)-\Ne(a)$.
					\item Using \declarationhr{1.30}, verify that $\Ne(b)=\Po(0)+\Ne(b)<\Ne(0)+\Po(b)=\Po(b)$.
					\item Hence verify that $0<\Po(b)-\Ne(b)$.
					\item Hence verify that $0<(\Po(a)-\Ne(a))(\Po(b)-\Ne(b))$.
					\item Hence verify that $\Ne(a)(\Po(b)-\Ne(b))<\Po(a)(\Po(b)-\Ne(b))$.
					\item Hence verify that $\Po(0)+\Ne(ab)=\Ne(a)\Po(b)+\Po(a)\Ne(b)<\Po(a)\Po(b)+\Ne(a)\Ne(b)=\Ne(0)+\Po(ab)$.
					\item \textbf{Hence verify that $0<ab$.}
				\end{enumerate}
		\declaration{1.34}
			The notation \gls{int||a||} will be used as a shorthand for the following expression:
			\begin{enumerate}
				\item $-a$ if $a<0$
				\item $a$ if $a\ge 0$
			\end{enumerate}
		\procedure{1.87}
			\objective
				Choose two integers $a,b$. The objective of the following instructions is to show that $\lVert ab\rVert=\lVert a\rVert\lVert b\rVert$.
			\implementation
				\begin{enumerate}
					\item If $a\ge 0$ and $b\ge 0$, then do the following:
					\begin{enumerate}
						\item Verify that $ab\ge 0$.
						\item \textbf{Hence verify that $\lVert ab\rVert=ab=\lVert a\rVert\lVert b\rVert$.}
					\end{enumerate}
					\item Otherwise if $a<0$ and $b\ge 0$, then do the following:
					\begin{enumerate}
						\item Verify that $ab<0$.
						\item \textbf{Hence verify that $\lVert ab\rVert=-(ab)=(-a)b=\lVert a\rVert\lVert b\rVert$.}
					\end{enumerate}
					\item Otherwise if $a\ge 0$ and $b<0$, then do the following:
					\begin{enumerate}
						\item Verify that $ab<0$.
						\item \textbf{Hence verify that $\lVert ab\rVert=-(ab)=a(-b)=\lVert a\rVert\lVert b\rVert$.}
					\end{enumerate}
					\item Otherwise do the following:
					\begin{enumerate}
						\item Verify that $a<0$ and $b<0$.
						\item Hence verify that $ab>0$.
						\item \textbf{Hence verify that $\lVert ab\rVert=ab=(-a)(-b)=\lVert a\rVert\lVert b\rVert$.}
					\end{enumerate}
				\end{enumerate}
		\procedure{1.88}
			\objective
				Choose two integers $a,b$. The objective of the following instructions is to show that $\lVert a+b\rVert\le\lVert a\rVert+\lVert b\rVert$.
			\implementation
				\begin{enumerate}
					\item If $a+b\ge 0$, then do the following:
					\begin{enumerate}
						\item Verify that $a\le\lVert a\rVert$.
						\item Verify that $b\le\lVert b\rVert$.
						\item \textbf{Hence verify that $\lVert a+b\rVert=a+b\le\lVert a\rVert+\lVert b\rVert$.}
					\end{enumerate}
					\item Otherwise do the following:
					\begin{enumerate}
						\item Verify that $-a\le\lVert a\rVert$.
						\item Verify that $-b\le\lVert b\rVert$.
						\item Verify that $a+b<0$.
						\item \textbf{Hence verify that $\lVert a+b\rVert=-(a+b)=(-a)+(-b)\le\lVert a\rVert+\lVert b\rVert$.}
					\end{enumerate}
				\end{enumerate}
		\procedure{1.89}
			\objective
				Choose two integers $a,b$. The objective of the following instructions is to show that $\lVert a\rVert-\lVert b\rVert\le\lVert a-b\rVert$.
			\implementation
				\begin{enumerate}
					\item Execute \procedurehr{1.88} on $\langle b,a-b\rangle$.
					\item Hence verify that $\lVert a\rVert=\lVert b+(a-b)\rVert\le\lVert b\rVert+\lVert a-b\rVert$.
					\item \textbf{Hence verify that $\lVert a\rVert-\lVert b\rVert\le\lVert a-b\rVert$.}
				\end{enumerate}
		\declaration{1.03}
			The notation \gls{intsgn(a)} will be used as a shorthand for the following expression:
			\begin{enumerate}
				\item $-1$ if $a<0$
				\item $0$ if $a=0$
				\item $1$ if $a>0$
			\end{enumerate}
		\procedure{1.90}
			\objective
				Choose an integer $a$. The objective of the following instructions is to show that $a=\sgn(a)\lVert a\rVert$.
			\implementation
				\begin{enumerate}
					\item If $a>0$, then do the following:
					\begin{enumerate}
						\item Verify that $\lVert a\rVert=a$.
						\item Verify that $\sgn(a)=1$.
						\item \textbf{Hence verify that $a=1a=\sgn(a)\lVert a\rVert$.}
					\end{enumerate}
					\item If $a=0$, then do the following:
					\begin{enumerate}
						\item Verify that $\lVert a\rVert=a=0$.
						\item \textbf{Hence verify that $a=0=\sgn(a)0=\sgn(a)\lVert a\rVert$.}
					\end{enumerate}
					\item Otherwise if $a<0$, then do the following:
					\begin{enumerate}
						\item Verify that $\lVert a\rVert=-a$.
						\item Verify that $\sgn(a)=-1$.
						\item \textbf{Hence verify that $a=(-1)(-a)=\sgn(a)\lVert a\rVert$.}
					\end{enumerate}
				\end{enumerate}
		\procedure{1.00}
			\objective
				Choose an integer $a$ and a positive integer $b$. The objective of the following instructions is to construct integers $n$ and $m$ such that $a=nb+m$ and $0\le m<b$.
			\implementation
				\begin{enumerate}
					\item Let $n=0$.
					\item While $(n+1)b\le a$, do the following:
					\begin{enumerate}
						\item Let $n$ receive $n+1$.
						\item Verify that $nb\le a$.
					\end{enumerate}
					\item While $nb>a$, do the following:
					\begin{enumerate}
						\item Let $n$ receive $n-1$.
						\item Verify that $(n+1)b>a$.
					\end{enumerate}
					\item Therefore verify that $nb\le a$.
					\item Also verify that $(n+1)b>a$.
					\item Let $m=a-nb$.
					\item \textbf{Now verify that $b>a-nb=m\ge 0$.}
					\item \textbf{Also verify that $a=bn+a-nb=nb+m$.}
					\item \textbf{Yield $\langle n,m\rangle$.}
				\end{enumerate}
		\declaration{1.00}
			The notation \gls{intadivb} will be used to refer to the first part of the pair yielded by executing \procedurehr{1.00} on $\langle a,b\rangle$.
		\declaration{1.01}
			The notation \gls{intamodb} will be used to refer to the second part of the pair yielded by executing \procedurehr{1.00} on $\langle a,b\rangle$.
		\declaration{1.02}
			The notation \gls{inta=bmodc} will be used as a shorthand for "$a\mod c=b\mod c$".
		\procedure{1.01}
			\objective
				Choose four integers $a,b,c,d$ and a positive integer $e$ in such a way that $a\equiv c(\mod e)$ and $b\equiv d(\mod e)$. The objective of the following instructions is to show that $a+b\equiv c+d(\mod e)$.
			\implementation
				\begin{enumerate}
					\item Verify that $a+b$
					\begin{enumerate}
						\item $\equiv(a\div e)e+(a\mod e)+(b\div e)e+(b\mod e)$
						\item $\equiv(a\mod e)+(b\mod e)$
						\item $\equiv(c\mod e)+(d\mod e)$
						\item $\equiv(c\div e)e+(c\mod e)+(d\div e)e+(d\mod e)$
						\item \textbf{$\equiv c+d(\mod e)$.}
					\end{enumerate}
				\end{enumerate}
		\procedure{1.02}
			\objective
				Choose four integers $a,b,c,d$ and a positive integer $e$ in such a way that $a\equiv c(\mod e)$ and $b\equiv d(\mod e)$. The objective of the following instructions is to show that $ab\equiv cd(\mod e)$.
			\implementation
				\begin{enumerate}
					\item Verify that $ab$
					\begin{enumerate}
						\item $\equiv((a\div e)e+(a\mod e))((b\div e)e+(b\mod e))$
						\item $\equiv(a\div e)(b\div e)e^2+(a\div e)(b\mod e)e+(a\mod e)(b\div e)e+(a\mod e)(b\mod e)$
						\item $\equiv(a\mod e)(b\mod e)$
						\item $\equiv(c\mod e)(d\mod e)$
						\item $\equiv(c\div e)(d\div e)e^2+(c\div e)(d\mod e)e+(c\mod e)(d\div e)e+(c\mod e)(d\mod e)$
						\item \textbf{$\equiv cd(\mod e)$.}
					\end{enumerate}
				\end{enumerate}
		\procedure{1.03}
			\objective
				Choose an integer $a$ and two positive integers $b,c$. The objective of the following instructions is to show that $(a\mod bc)\mod b=a\mod b$.
			\implementation
				\begin{enumerate}
					\item \textbf{Verify that $(a\mod bc)\mod b=(a-(a\div bc)bc)\mod b=a\mod b$.}
				\end{enumerate}
		\procedure{1.04}
			\objective
				Choose a positive integer $a$ and four integers $b_1,b_0,c_1,c_0$ such that $0\le b_0<a$, $0\le c_0<a$, and $b_1a+b_0=c_1a+c_0$. The objective of the following instructions is to show that $b_1=c_1$ and $b_0=c_0$.
			\implementation
				\begin{enumerate}
					\item \textbf{Verify that $b_0=b_0\mod a=(b_1a+b_0)\mod a=(c_1a+c_0)\mod a=c_0\mod a=c_0$.}
					\item Therefore verify that $b_1a=c_1a$.
					\item \textbf{Therefore verify that $b_1=c_1$.}
				\end{enumerate}
		\procedure{1.05}
			\objective
				Choose an integer $a$ and two positive integers $b,c$. The objective of the following instructions is to show that $ca\mod cb=c(a\mod b)$ and that $ca\div cb=a\div b$.
			\implementation
				\begin{enumerate}
					\item Verify that $bc(a\div b)+c(a\mod b)=c(b(a\div b)+a\mod b)=ca=cb(ca\div cb)+ca\mod cb$.
					\item Now verify that $0\le a\mod b<b$.
					\item Therefore verify that $0\le c(a\mod b)<cb$.
					\item Now verify that $0\le ca\mod cb<cb$.
					\item Execute \procedurehr{1.04} on $\langle bc,a\div b,c(a\mod b),ca\div cb,ca\mod cb\rangle$.
					\item \textbf{Therefore verify that $c(a\mod b)=ca\mod cb$.}
					\item \textbf{Also verify that $a\div b=ca\div cb$.}
				\end{enumerate}
		\procedure{1.06}
			\objective
				Choose two integers $a,b$ and a positive integer $c$ such that $a\mod c+b\mod c<c$. The objective of the following instructions is to show that $a\div c+b\div c=(a+b)\div c$ and $a\mod c+b\mod c=(a+b)\mod c$.
			\implementation
				\begin{enumerate}
					\item Verify that $a=c(a\div c)+a\mod c$.
					\item Verify that $b=c(b\div c)+b\mod c$.
					\item Therefore verify that $a+b=c(a\div c+b\div c)+(a\mod c+b\mod c)$.
					\item Verify that $0\le a\mod c+b\mod c<c$.
					\item Also verify that $a+b=((a+b)\div c)c+(a+b)\mod c$.
					\item Verify that $0\le(a+b)\mod c<c$.
					\item Execute \procedurehr{1.04} on $\langle c,a\div c+b\div c,a\mod c+b\mod c,(a+b)\div c,(a+b)\mod c\rangle$.
					\item \textbf{Therefore verify that $a\div c+b\div c=(a+b)\div c$.}
					\item \textbf{Also verify that $a\mod c+b\mod c=(a+b)\mod c$.}
				\end{enumerate}
		\procedure{1.07}
			\objective
				Choose two integers $a,b$ and a positive integer $c$ such that $a\mod c+b\mod c\ge c$. The objective of the following instructions is to show that $1+a\div c+b\div c=(a+b)\div c$ and $a\mod c+b\mod c-c=(a+b)\mod c$.
			\implementation
				\begin{enumerate}
					\item Verify that $a=c(a\div c)+a\mod c$.
					\item Verify that $b=c(b\div c)+b\mod c$.
					\item Therefore verify that $a+b=c(a\div c+b\div c)+a\mod c+b\mod c=c(1+a\div c+b\div c)+(a\mod c+b\mod c-c)$.
					\item Verify that $c\le a\mod c+b\mod c<2c$.
					\item Therefore verify that $0\le a\mod c+b\mod c-c<c$.
					\item Also verify that $a+b=c((a+b)\div c)+(a+b)\mod c$.
					\item Verify that $0\le (a+b)\mod c<c$.
					\item Execute \procedurehr{1.04} on $\langle c,1+a\div c+b\div c,a\mod c+b\mod c-c,(a+b)\div c,(a+b)\mod c\rangle$.
					\item \textbf{Therefore verify that $1+a\div c+b\div c=(a+b)\div c$.}
					\item \textbf{Therefore verify that $a\mod c+b\mod c-c=(a+b)\mod c$.}
				\end{enumerate}
		\procedure{1.08}
			\objective
				Choose an integer $a$ and two positive integers $b,c$. The objective of the following instructions is to show that $a\div bc=(a\div b)\div c$ and $a\mod bc=((a\div b)\mod c)b+a\mod b$.
			\implementation
				\begin{enumerate}
					\item Verify that $a=(a\div b)b+a\mod b$.
					\item Verify that $a\div b=((a\div b)\div c)c+(a\div b)\mod c$.
					\item Therefore verify that $a=(((a\div b)\div c)c+(a\div b)\mod c)b+a\mod b=((a\div b)\div c)bc+((a\div b)\mod c)b+a\mod b$.
					\item Verify that $0\le(a\div b)\mod c\le c-1$.
					\item Therefore verify that $0\le((a\div b)\mod c)b\le cb-b$.
					\item Verify that $0\le a\mod b<b$.
					\item Therefore verify that $0\le ((a\div b)\mod c)b+a\mod b<cb$.
					\item Now verify that $a=(a\div bc)bc+a\mod bc$.
					\item Verify that $0\le a\mod bc<bc$.
					\item Execute \procedurehr{1.04} on $\langle bc,(a\div b)\div c,((a\div b)\mod c)b+a\mod b,a\div bc,a\mod bc\rangle$.
					\item \textbf{Therefore verify that $(a\div b)\div c=a\div bc$.}
					\item \textbf{Also verify that $((a\div b)\mod c)b+a\mod b=a\mod bc$.}
				\end{enumerate}
		\procedure{1.09}
			\objective
				Choose an integer $a$ and a non-negative integer $b$. The objective of the following instructions is to consruct integers $c,d,e,f,g$ such that $a=cd$, $b=ce$, $fa+gb=c$, and if $b=0$, then $c=\lvert a\rvert$, otherwise $0<c\le b$.
			\implementation
				\begin{enumerate}
					\item If $b=0$, then do the following:
					\begin{enumerate}
						\item \textbf{Verify that $a=\sgn(a)\lvert a\rvert$.}
						\item \textbf{Verify that $b=0\lvert a\rvert$.}
						\item \textbf{Verify that $\lvert a\rvert=\sgn(a)a+0b$.}
						\item \textbf{Yield $\langle\lvert a\rvert,\sgn(a),0,\sgn(a),0\rangle$.}
					\end{enumerate}
					\item Otherwise do the following:
					\begin{enumerate}
						\item Verify that $0\le a\mod b<b$.
						\item Execute \procedurehr{1.09} on $\langle b,a\mod b\rangle$ and let $\langle c,d,e,f,g\rangle$ receive.
						\item \textbf{Now verify that $b=cd$.}
						\item Also verify that $a\mod b=ce$.
						\item \textbf{Therefore verify that $a=(a\div b)b+(a\mod b)=c(d(a\div b)+e)$.}
						\item \textbf{Also verify that $(f-g(a\div b))b+ga=fb+g(a-(a\div b)b)=fb+g(a\mod b)=c$.}
						\item If $a\mod b=0$, then do the following:
						\begin{enumerate}
							\item \textbf{Using (2) and (b), verify that $0<b=c\le b$.}
						\end{enumerate}
						\item Otherwise do the following:
						\begin{enumerate}
							\item \textbf{Using (b), verify that $0<c\le a\mod b<b$.}
						\end{enumerate}
						\item \textbf{Therefore yield $\langle c,d(a\div b)+e,d,g,f-g(a\div b)\rangle$.}
					\end{enumerate}
				\end{enumerate}
		\declaration{1.04}
			The notation \gls{int(a,b)} will be used to refer to the first part of the quintuple yielded by executing \procedurehr{1.09} on the pair $\langle a,b\rangle$.
		\procedure{1.10}
			\objective
				Choose an integer $a$ and a positive integer $b$. Let $1\le c\le b$ be the largest integer such that $a\mod c=0$ and $b\mod c=0$. The objective of the following instructions is to either show that $0\ne 0$ or $(a,b)=c$.
			\implementation
				\begin{enumerate}
					\item Execute \procedurehr{1.09} on $\langle a,b\rangle$ and let $\langle d,e,f,g,h\rangle$ receive.
					\item Verify that $0<d\le b$.
					\item If $d>c$, then do the following:
					\begin{enumerate}
						\item Using the precondition, verify that $a\mod d\ne 0$ or $b\mod d\ne 0$.
						\item If $a\mod d\ne 0$, then do the following:
						\begin{enumerate}
							\item Using (1), verify that $a=ed$.
							\item Therefore verify that $a\mod d=0$.
							\item \textbf{Therefore using (3b) and (3bii), verify that $0\ne 0$.}
							\item \textbf{Abort procedure.}
						\end{enumerate}
						\item Otherwise if $b\mod d\ne 0$, then do the following:
						\begin{enumerate}
							\item Using (1), verify that $b=fd$.
							\item Therefore verify that $b\mod d=0$.
							\item \textbf{Therefore using (3c) and (3cii), verify that $0\ne 0$.}
							\item \textbf{Abort procedure.}
						\end{enumerate}
					\end{enumerate}
					\item Otherwise if $d<c$, then do the following:
					\begin{enumerate}
						\item Verify that $ga+hb=d$.
						\item Therefore verify that $0\equiv gc(a\div c)+hc(b\div c)=g(c(a\div c)+a\mod c)+h(c(b\div c)+b\mod c)=ga+hb=d\not\equiv 0(\mod c)$.
						\item \textbf{Therefore verify that $0\ne 0$.}
						\item \textbf{Abort procedure.}
					\end{enumerate}
					\item \textbf{Otherwise verify that $(a,b)=d=c$.}
				\end{enumerate}
		\procedure{1.11}
			\objective
				Choose integers $a,c,d,j$ and a non-negative integer $b$. Execute \procedurehr{1.09} on $\langle a,b\rangle$ and let $\langle e,f,g,h,i\rangle$ receive. The objective of the following instructions is to show that $ca+db=(c+gj)a+(d-fj)b$.
			\implementation
				\begin{enumerate}
					\item \textbf{Verify that $(c+gj)a+(d-fj)b=ca+db+gja-fjb=ca+db+gjef-fjeg=ca+db$.}
				\end{enumerate}
		\procedure{1.12}
			\objective
				Choose integers $a,c,d$ and a non-negative integer $b$ such that $ca+db=(a,b)$. Execute \procedurehr{1.09} on $\langle a,b\rangle$ and let $\langle e,f,g,h,i\rangle$ receive. The objective of the following instructions is to construct a $j$ such that $c=h+gj$ and $d=i-fj$.
			\implementation
				\begin{enumerate}
					\item Verify that $cef+deg=ca+db=(a,b)=e$.
					\item Therefore verify that $cf+dg=1$.
					\item Now verify that $hef+ieg=ha+ib=e$.
					\item Therefore verify that $hf+ig=1$.
					\item Let $j=ci-hd$.
					\item Now verify that $cf=1-dg$.
					\item Therefore verify that $c-cig=c(1-ig)=chf=h(1-dg)=h-hdg$.
					\item \textbf{Therefore verify that $c=h+cig-hdg=h+g(ci-hd)=h+gj$.}
					\item Now verify that $dg=1-cf$.
					\item Therefore verify that $d-dhf=d(1-hf)=dig=i(1-cf)=i-icf$.
					\item \textbf{Therefore verify that $d=i-icf+dhf=i-f(ic-dh)=i-fj$.}
					\item \textbf{Yield $\langle j\rangle$.}
				\end{enumerate}
		\procedure{1.13}
			\objective
				Choose an integer $a$ and a positive integer $b$ such that $0<(a,b)<b$. The objective of the following instructions is to show that $0\ne 0$ or $a\mod b\ne 0$.
			\implementation
				\begin{enumerate}
					\item If $a\mod b=0$, then do the following:
					\begin{enumerate}
						\item Using (1), verify that $af\equiv 0f\equiv 0(\mod b)$.
						\item Execute \procedurehr{1.09} on $\langle a,b\rangle$ and let $\langle c,d,e,f,g\rangle$ receive.
						\item Verify that $0<(a,b)=c=fa+gb<b$.
						\item Therefore verify $fa\equiv(a,b)\not\equiv 0(\mod b)$.
						\item Therefore using (1a) and (1d), verify that $0\ne 0$.
						\item \textbf{Abort ptocedure.}
					\end{enumerate}
					\item \textbf{Otherwise verify that $a\mod b\ne 0$.}
				\end{enumerate}
		\procedure{1.14}
			\objective
				Choose five integers $a,d,e,f,g$ and two non-negative integers $b,c$ such that $a=cd$, $b=ce$, and $fa+gb=c$. The objective of the following instructions is to show that $0<0$ or $(a,b)=c$.
			\implementation
				\begin{enumerate}
					\item Execute \procedurehr{1.09} on $\langle a,b\rangle$ and let $\langle u,v,x,y,z\rangle$ receive.
					\item Verify that $u\ge 0$.
					\item Verify that $a=uv$.
					\item Verify that $b=xu$.
					\item Therefore verify that $c=fa+gb=(fv+gx)u$.
					\item If $u=0$, then do the following:
					\begin{enumerate}
						\item \textbf{Verify that $c=(fv+gx)u=0=u=(a,b)$.}
						\item \textbf{Yield.}
					\end{enumerate}
					\item Also using (1) and the precondition, verify that $u=ya+zb=(yd+ze)c$.
					\item If $c=0$, then do the following:
					\begin{enumerate}
						\item \textbf{Verify that $(a,b)=u=(yd+ze)c=0=c$.}
						\item \textbf{Yield.}
					\end{enumerate}
					\item Verify that $c>0$.
					\item Now verify that $c=(fv+gx)u=(fv+gx)(yd+ze)c$.
					\item Therefore verify that $(fv+gx)(yd+ze)=1$.
					\item Therefore verify that $fv+gx=yd+ze=\pm 1$.
					\item If $fv+gx=yd+ze=-1$, then do the following:
					\begin{enumerate}
						\item Using (7) and (9), verify that $u=(yd+ze)c=-c<0$.
						\item \textbf{Therefore using (2) and (13a), verify that $0\le u<0$.}
						\item \textbf{Abort procedure.}
					\end{enumerate}
					\item Otherwise, do the following:
					\begin{enumerate}
						\item Verify that $fv+gx=yd+ze=1$.
						\item \textbf{Therefore verify that $c=(fv+gx)u=u=(a,b)$.}
					\end{enumerate}
				\end{enumerate}
		\procedure{1.15}
			\objective
				Choose an integer $a$ and a non-negative integer $b$. The objective of the following instructions is to show that $0<0$ or $(a,b)=(-a,b)$.
			\implementation
				\begin{enumerate}
					\item Execute \procedurehr{1.09} on $\langle a,b\rangle$ and let $\langle c,d,e,f,g\rangle$ receive.
					\item Verify that $a=dc$.
					\item Therefore verify that $-a=(-d)c$.
					\item Verify that $b=ec$.
					\item Verify that $fa+gb=c$.
					\item Therefore verify that $(-f)(-a)+gb=c$.
					\item Execute \procedurehr{1.14} on $\langle -a,b,c,-d,e,-f,g\rangle$.
					\item \textbf{Therefore verify that $(-a,b)=c=(a,b)$.}
				\end{enumerate}
		\procedure{1.16}
			\objective
				Choose two non-negative integers $a,b$. The objective of the following instructions is to show that $0<0$ or $(a,b)=(b,a)$.
			\implementation
				\begin{enumerate}
					\item Execute \procedurehr{1.09} on $\langle a,b\rangle$ and let $\langle c,d,e,f,g\rangle$ receive.
					\item Verify that $b=ec$.
					\item Verify that $a=dc$.
					\item Verify that $gb+fa=c$.
					\item Execute \procedurehr{1.14} on $\langle b,a,c,e,d,g,f\rangle$.
					\item \textbf{Therefore verify that $(b,a)=c=(a,b)$.}
				\end{enumerate}
		\procedure{1.17}
			\objective
				Choose two integers $a,b$ and a positive integer $c$ such that $a\equiv b(\mod c)$. The objective of the following instructions is to show that $0<0$ or $(a,c)=(b,c)$.
			\implementation
				\begin{enumerate}
					\item Execute \procedurehr{1.09} on $\langle a,c\rangle$ and let $\langle d,e,f,g,h\rangle$ receive.
					\item Verify that $a=ed$.
					\item Verify that $c=fd$.
					\item Let $j=b\div c-a\div c$.
					\item Therefore verify that $b=a+jc=ed+jfd=(e+jf)d$.
					\item Verify that $gb+(h-gj)c=g(a+jc)+(h-gj)c=ga+hc=d$.
					\item Now execute \procedurehr{1.14} on $\langle b,c,d,e+jf,f,g,h-gj\rangle$.
					\item \textbf{Therefore verify that $(b,c)=d=(a,c)$.}
				\end{enumerate}
		\procedure{1.18}
			\objective
				Choose an integer $a$ and two non-negative integers $b,c$. The objective of the following instructions is to show that either $0<0$ or $(ca,cb)=c(a,b)$.
			\implementation
				\begin{enumerate}
					\item Execute \procedurehr{1.09} on $\langle a,b\rangle$ and let $\langle d,e,f,g,h\rangle$ receive.
					\item Verify that $a=ed$.
					\item Therefore verify that $ca=e(cd)$.
					\item Verify that $b=df$.
					\item Therefore verify that $cb=f(cd)$.
					\item Verify that $ga+hb=d$.
					\item Therefore verify that $g(ca)+h(cb)=cd$.
					\item Now execute \procedurehr{1.14} on $\langle ca,cb,cd,e,f,g,h\rangle$.
					\item Therefore verify that $(ca,cb)=cd=c(a,b)$.
				\end{enumerate}
		\procedure{1.19}
			\objective
				Choose an integer $a$ and two non-negative integers $b,c$. The objective of the following instructions is to show that either $0<0$ or $(a,(b,c))=((a,b),c)$.
			\implementation
				\begin{enumerate}
					\item Execute \procedurehr{1.09} on $\langle a,b\rangle$ and let $\langle d_0,e_0,f_0,g_0,h_0\rangle$ receive.
					\item Execute \procedurehr{1.09} on $\langle b,c\rangle$ and let $\langle d_1,e_1,f_1,g_1,h_1\rangle$ receive.
					\item Execute \procedurehr{1.09} on $\langle (a,b),c\rangle$ and let $\langle d_2,e_2,f_2,g_2,h_2\rangle$ receive.
					\item Verify that $a=d_0e_0=e_0(a,b)=e_0d_2e_2=e_0e_2((a,b),c)$.
					\item Verify that $(b,c)$
					\begin{enumerate}
						\item $=g_1b+h_1c$
						\item $=g_1d_0f_0+h_1d_2f_2$
						\item $=g_1f_0(a,b)+h_1f_2((a,b),c)$
						\item $=g_1f_0d_2e_2+h_1f_2((a,b),c)$
						\item $=g_1f_0e_2((a,b),c)+h_1f_2((a,b),c)$
						\item $=(g_1f_0e_2+h_1f_2)((a,b),c)$.
					\end{enumerate}
					\item Verify that $((a,b),c)$
					\begin{enumerate}
						\item $=d_2$
						\item $=g_2(a,b)+h_2c$
						\item $=g_2d_0+h_2d_1f_1$
						\item $=g_2(g_0a+h_0b)+h_2f_1(b,c)$
						\item $=g_2g_0a+g_2h_0d_1e_1+h_2f_1(b,c)$
						\item $=g_2g_0a+g_2h_0e_1(b,c)+h_2f_1(b,c)$
						\item $=g_2g_0a+(g_2h_0e_1+h_2f_1)(b,c)$.
					\end{enumerate}
					\item Execute \procedurehr{1.14} on $\langle a,(b,c),((a,b),c),e_0e_2,g_1f_0e_2+h_1f_2,g_2g_0,g_2h_0e_1+h_2f_1\rangle$.
					\item \textbf{Therefore verify that $((a,b),c)=(a,(b,c))$.}
				\end{enumerate}
		\declaration{1.05}
			The notation \gls{int(a_0,a_1,...,a_n-1)} will be used to contextually refer to one of the following integers:
			\begin{enumerate}
				\item $((a_0),(a_1,a_2,\cdots,a_{n-1}))$
				\item $((a_0,a_1),(a_2,a_3,\cdots,a_{n-1}))$
				\item $\vdots$
				\item $((a_0,a_1,\cdots,a_{n-2}),(a_{n-1}))$
			\end{enumerate}
		\procedure{1.20}
			\objective
				Choose two integers $a,b$ and a non-negative integer $c$ such that $(a,c)=1$ and $(b,c)=1$. The objective of the following instructions is to show that either $0<0$ or $(ab,c)=1$.
			\implementation
				\begin{enumerate}
					\item Execute \procedurehr{1.09} on $\langle a,c\rangle$ and let $\langle d,e,f,g,h\rangle$ receive.
					\item Verify that $ga+hc=d=(a,c)=1$.
					\item Execute \procedurehr{1.09} on $\langle b,c\rangle$ and let $\langle t,u,v,w,x\rangle$ receive.
					\item Verify that $wb+xc=t=(b,c)=1$.
					\item Therefore verify that $(gw)(ab)+(gax+wbh+hxc)c=(ga+hc)(wb+xc)=1$.
					\item Now execute \procedurehr{1.14} on $\langle ab,c,1,ab,c,gw,gax+wbh+hxc\rangle$.
					\item \textbf{Therefore verify that $(ab,c)=1$.}
				\end{enumerate}
		\procedure{1.21}
			\objective
				Choose an integer $a$ and two non-negative integers $b,c$ such that $(a,bc)=1$. The objective of the following instructions is to show that either $0<0$ or $(a,b)=1$.
			\implementation
				\begin{enumerate}
					\item Execute \procedurehr{1.09} on $\langle a,bc\rangle$ and let $\langle d,e,f,g,h\rangle$ receive.
					\item Verify that $d=(a,bc)=1$.
					\item Verify that $ga+(hc)b=ga+h(bc)=d=1$.
					\item Now execute \procedurehr{1.14} on $\langle a,b,1,a,b,g,hc\rangle$.
					\item \textbf{Therefore verify that $(a,b)=1$.}
				\end{enumerate}
		\declaration{1.06}
			The phrase "\gls{intprime}" will be used to refer to integers $a$ such that $a>1$ and $a\mod k\ne 0$ for $1<k<a$.
		\procedure{1.22}
			\objective
				Choose an integer $a$ and a prime $b$ such that $a\mod b\ne 0$. The objective of the following instructions is to show that either $0\ne 0$ or $(a,b)=1$.
			\implementation
				\begin{enumerate}
					\item Execute \procedurehr{1.09} on $\langle a,b\rangle$ and let $\langle c,d,e,f,g\rangle$ receive.
					\item Verify that $0<c\le b$.
					\item If $c=b$, then do the following:
					\begin{enumerate}
						\item Verify that $a=cd=bd$.
						\item Therefore verify that $a\mod b=0$.
						\item \textbf{Therefore using the precondition and (3b), verify that $0\ne 0$.}
						\item \textbf{Abort procedure.}
					\end{enumerate}
					\item Otherwise if $1<c<b$, then do the following:
					\begin{enumerate}
						\item Verify that $b=ce$.
						\item Therefore verify that $b\mod c=0$.
						\item \textbf{Therefore using the precondition and (4b), verify that $0\ne 0$.}
						\item \textbf{Abort procedure.}
					\end{enumerate}
					\item Otherwise, do the following:
					\begin{enumerate}
						\item \textbf{Verify that $(a,b)=c=1$.}
					\end{enumerate}
				\end{enumerate}
		\procedure{1.23}
			\objective
				Choose two integers $a,b$ and a prime $c$ such that $a\mod c\ne 0$ and $b\mod c\ne 0$. The objective of the following instructions is to show that either $0\ne 0$ or $ab\mod c\ne 0$.
			\implementation
				\begin{enumerate}
					\item Execute \procedurehr{1.22} on $\langle a,c\rangle$.
					\item Verify that $(a,c)=1$.
					\item Execute \procedurehr{1.22} on $\langle b,c\rangle$.
					\item Verify that $(b,c)=1$.
					\item Execute \procedurehr{1.20} on $\langle a,b,c\rangle$.
					\item Now verify that $0<(ab,c)=1<c$.
					\item Execute \procedurehr{1.13} on $\langle ab,c\rangle$.
					\item \textbf{Now verify that $ab\mod c\ne 0$.}
				\end{enumerate}
		\declaration{1.07}
			The notation \gls{int|a|} will be used to refer to the number of items in the list $a$.
		\declaration{1.10}
			The notation \gls{inta^b} will be used to refer to the list formed by concatenating $a$ and $b$.
		\declaration{1.31}
			The notation $f(R)$, where $R$ is a list and $f[r]$ is a function of $r$, will contextually be used as a shorthand for the list $\langle f(R_0),f(R_1),\cdots,f(R_{\lvert R\rvert-1})\rangle$.
		\declaration{1.09}
			The notation \gls{inta_*}, where $a$ is a list, will be used as a shorthand for $1$ if $a$ is empty, otherwise it will be a shorthand for the product of the entries of $a$.
		\declaration{1.08}
			The notation \gls{intpif(r)}, where $R$ is a list and $f[r]$ is a function of $r$, will be used as a shorthand for $f(R)_*$.
		\procedure{1.24}
			\objective
				Choose a positive integer $a$. The objective of the following instructions is to construct a list of prime numbers $b$ such that $a=b_*$.
			\implementation
				\begin{enumerate}
					\item If $a=1$, then do the following:
					\begin{enumerate}
						\item Verify that $a=1=\langle\rangle_*$.
						\item Therefore yield $\langle\rangle$.
					\end{enumerate}
					\item Otherwsie, do the following:
					\begin{enumerate}
						\item Verify that $a>1$.
						\item For $c=2$ up to $c=a-1$, do the following:
						\begin{enumerate}
							\item If $a\mod c=0$, then do the following:
							\begin{enumerate}
								\item Verify that $a=(a\div c)c$.
								\item Therefore verify that $1<a\div c<a$.
								\item Execute \procedurehr{1.24} on $\langle a\div c\rangle$ and let $\langle d\rangle$ receive.
								\item Using (B) and (C), verify that $\lvert d\rvert>0$.
								\item Verify that every element of $d$ is prime.
								\item Verify that $a\div c=d_*$.
								\item Execute \procedurehr{1.24} on $\langle c\rangle$ and let $\langle e\rangle$ receive.
								\item Using (b) and (G), verify that $\lvert e\rvert>0$.
								\item Verify that every element of $e$ is prime.
								\item Verify that $c=e_*$.
								\item \textbf{Therefore verify that $\lvert d^\frown e\rvert>0$.}
								\item \textbf{Also verify that every element of $d^\frown e$ is prime.}
								\item \textbf{Also verify that $a=(a\div c)c=d_*e_*=(d^\frown e)_*$.}
								\item \textbf{Yield $\langle d^\frown e\rangle$.}
							\end{enumerate}
						\end{enumerate}
						\item Otherwise do the following:
						\begin{enumerate}
							\item \textbf{Verify that $a$ is prime.}
							\item \textbf{Yield $\langle a\rangle$.}
						\end{enumerate}
					\end{enumerate}
				\end{enumerate}
		\procedure{1.25}
			\objective
				Choose a prime $a$ and a list of primes $b$ such that $b_*\equiv 0(\mod a)$. The objective of the following instructions is to either show that $0=1$ or to construct a $k$ such that $a=b_k$.
			\implementation
				\begin{enumerate}
					\item Using \declarationhr{1.06}, verify that $a>1$.
					\item If $\lvert b\rvert=0$, then do the following:
					\begin{enumerate}
						\item Verify that $1=b_*\equiv 0(\mod a)$.
						\item \textbf{Therefore using (1) and (a), verify that $0=1$.}
						\item \textbf{Abort procedure.}
					\end{enumerate}
					\item Otherwise if $0\not\in b\mod a$, then do the following:
					\begin{enumerate}
						\item Using \procedurehr{1.23}, verify that $b_*\not\equiv 0 (\mod a)$.
						\item \textbf{Therefore using the precondition and (a), verify that $0\ne 0$.}
						\item \textbf{Abort procedure.}
					\end{enumerate}
					\item Otherwise do the following:
					\begin{enumerate}
						\item Let $k$ be such that $b_k\mod a=0$.
						\item Verify that $b_k=(b_k\div a)a$.
						\item Verify that $b_k\div a\ge 1$.
						\item If $b_k\div a>1$, then do the following:
						\begin{enumerate}
							\item Using (1),(b), and (d), verify that $1<a<b_k$.
							\item Now, using \declarationhr{1.06} verify that $b_k\mod a\ne 0$.
							\item \textbf{Hence using (a) and (ii), verify that $0\ne b_k\mod a=0$.}
							\item \textbf{Abort procedure.}
						\end{enumerate}
						\item Otherwise do the following:
						\begin{enumerate}
							\item Verify that $b_k\div a=1$.
							\item \textbf{Therefore verify that $b_k=a$.}
							\item \textbf{Yield $\langle k\rangle$.}
						\end{enumerate}
					\end{enumerate}
				\end{enumerate}
		\declaration{1.11}
			The notation \gls{int[a:b]} will be used as a shorthand for the list:
			\begin{enumerate}
				\item $\langle a,a+1,\cdots,b-1\rangle$, if $b>a$
				\item $\langle\rangle$, if $b=a$
				\item $\langle a-1,a-2,\cdots,b\rangle$, if $b<a$
			\end{enumerate}
		\procedure{1.26}
			\objective
				Choose two lists of primes $a,b$ such that $a_*=b_*$. The objective of the following instructions is to show that either $1>1$ or $a$ is included in $b$.
			\implementation
				\begin{enumerate}
					\item If $\lvert a\rvert=0$, then do the following:
					\begin{enumerate}
						\item \textbf{Verify that $a$ is included in $b$.}
					\end{enumerate}
					\item Otherwise, do the following:
					\begin{enumerate}
						\item Verify that $\lvert a\rvert>0$.
						\item Verify that $b_*\equiv a_*\equiv 0 (\mod a_0)$.
						\item Execute \procedurehr{1.25} on $\langle a_0, b\rangle$ and let $\langle k\rangle$ receive.
						\item Therefore verify that $b_k=a_0$.
						\item Now verify $(a_{[1:\lvert a\rvert]})_*=(b_{[0:k]^\frown[k+1:\lvert b\rvert]})_*$.
						\item Now execute \procedurehr{1.26} on $\langle a_{[1:\lvert a\rvert]},b_{[0:k]^\frown[k+1:\lvert b\rvert]}\rangle$.
						\item Now verify that $a_{[1:\lvert a\rvert]}$ is included in $b_{[0:k]^\frown[k+1:\lvert b\rvert]}\rangle$.
						\item \textbf{Therefore verify that $a$ is included in $b$.}
					\end{enumerate}
				\end{enumerate}
		\procedure{1.27}
			\objective
				Choose two lists of primes $a,b$ such that $a_*=b_*$. The objective of the following instructions is to show that either $1>1$ or $a$ is a rearrangement of $b$.
			\implementation
				\begin{enumerate}
					\item Execute \procedurehr{1.26} on $\langle a,b\rangle$.
					\item Verify that $a$ is included in $b$.
					\item Execute \procedurehr{1.26} on $\langle b,a\rangle$.
					\item Verify that $b$ is included in $a$.
					\item \textbf{Therefore verify that $a$ is a rearrangement of $b$.}
				\end{enumerate}
		\procedure{1.28}
			\objective
				Choose a positive integer $a$. The objective of the following instructions is to either show that $0=1$ or to construct a prime $b$ such that $b>a$ and $[a+1:b]$ does not contain a prime.
			\implementation
				\begin{enumerate}
					\item Verify that $a!+1>1$.
					\item Execute \procedurehr{1.24} on $\langle a!+1\rangle$ and let $\langle d\rangle$ receive.
					\item Therefore using (1) and (2), verify that $\lvert d\rvert>0$.
					\item Now verify that $(a!+1)\mod d_0=0$.
					\item For $e=2$ up to $e=a$, do the following:
					\begin{enumerate}
						\item Verify that $a!+1\equiv 1 (\mod e)$.
						\item If $e=d_0$, then do the following:
						\begin{enumerate}
							\item Using (4) and (a), verify that $0\equiv a!+1\equiv 1(\mod e=d_0)$.
							\item \textbf{Therefore verify that $0=1$.}
							\item \textbf{Abort procedure.}
						\end{enumerate}
					\end{enumerate}
					\item Otherwise do the following:
					\begin{enumerate}
						\item \textbf{Using (2), verify that $d_0$ is prime.}
						\item Using (a), verify that $d_0>1$.
						\item \textbf{Using (a) and (5), verify that $d_0>a$.}
						\item \textbf{Let $b$ be the least prime between $a+1$ and $d_0$.}
						\item \textbf{Yield $\langle b\rangle$.}
					\end{enumerate}
				\end{enumerate}
		\procedure{1.29}
			\objective
				Choose a positive integer $a$. The objective of the following instructions is to construct a positive integer $b$ such that $[b+1:b+a]$ does not contain a prime.
			\implementation
				\begin{enumerate}
					\item Let $b=a!+1$.
					\item For $i=1$ up to $i=a-1$, do the following:
					\begin{enumerate}
						\item Verify that $b+i=a!+1+i=i!(i+1)(i+2)\cdots(a)+1+i=(1+i)(i!(i+2)(i+3)\cdots(a) +1)$.
						\item Therefore verify that $b+i\equiv 0 (\mod i+1)$.
						\item Also verify that $b+i=a!+1+i>a!\ge a\ge i+1>1$.
						\item \textbf{Therefore verify that $b+i$ is not prime.}
					\end{enumerate}
					\item \textbf{Yield $\langle b\rangle$.}
				\end{enumerate}
		\procedure{1.30}
			\objective
				Choose two lists of primes $a,b$ in such a way that their intersection is empty. The objective of the following instructions is to show that $0=1$ or $(a_*,b_*)=1$.
			\implementation
				\begin{enumerate}
					\item Execute \procedurehr{1.09} on $\langle a_*,b_*\rangle$ and let $\langle c,d,e,f,g\rangle$.
					\item Verify that $0<c\le b\rangle$.
					\item If $c>1$, then do the following:
					\begin{enumerate}
						\item Execute \procedurehr{1.24} on $\langle c\rangle$ and let $\langle h\rangle$ receive.
						\item Using (3) and (a), verify that $\lvert h\rvert>0$.
						\item Now verify that $a_*=dc=dh_*=dh_0(h_{[1:\vert h\rvert]})_*\equiv 0(\mod h_0)$.
						\item Execute \procedurehr{1.25} on $\langle h_0,a\rangle$ and let $\langle k\rangle$ receive.
						\item Now verify that $b_*=ec=eh_*=eh_0(h_{[1:\vert h\rvert]})_*\equiv 0(\mod h_0)$.
						\item Execute \procedurehr{1.25} on $\langle h_0,b\rangle$ and let $\langle m\rangle$ receive.
						\item \textbf{Therefore verify that $a_k=h_0=b_m$.}
						\item \textbf{Abort procedure.}
					\end{enumerate}
					\item Otherwise do the following:
					\begin{enumerate}
						\item \textbf{Verify that $(a_*,b_*)=c=1$.}
					\end{enumerate}
				\end{enumerate}
		\procedure{1.31}
			\objective
				Choose two lists of primes $a,b$. Let $c$ be the common sublist with multiplicity of $a$ and $b$. The objective of the following instructions is to show that either $0<0$ or $(a_*,b_*)=c_*$.
			\implementation
				\begin{enumerate}
					\item Let $d$ be the result of removing with multiplicity elements of $c$ from $a$.
					\item Verify that $a_*=c_*d_*$.
					\item Let $e$ be the result of removing with multiplicity elements of $c$ from $b$.
					\item Verify that $b_*=c_*e_*$.
					\item Verify that $d$ and $e$ share no common elements.
					\item \textbf{Therefore using \procedurehr{1.18} and \procedurehr{1.30}, verify that $(a_*,b_*)=(c_*d_*,c_*e_*)=c_*(d_*,e_*)=c_*$.}
				\end{enumerate}
		\procedure{1.32}
			\objective
				Choose an integer $a$ and a positive integer $b$. The objective of the following instructions is to construct integers $c,f,e$ such that $c=af$, $c=be$, $c(a,b)=ab$, and $\lvert a\rvert\le c\sgn(a)\le\lvert a\rvert b$.
			\implementation
				\begin{enumerate}
					\item Execute \procedurehr{1.09} on $\langle a,b\rangle$ and let $\langle d,e,f,g,h\rangle$ receive.
					\item \textbf{Let $c=af$.}
					\item \textbf{Verify that $c(a,b)=cd=afd=ab$.}
					\item Verify that $d>0$.
					\item Verify that $b=fd$.
					\item Therefore verify that $1\le f\le b$.
					\item Therefore verify that $\lvert a\rvert\le\lvert a\rvert f\le\lvert a\rvert b$.
					\item \textbf{Therefore verify that $\lvert a\rvert\le c\sgn(a)\le\lvert a\rvert b$.}
					\item \textbf{Verify that $c=af=def=be$.}
					\item \textbf{Yield the tuple $\langle c,f,e\rangle$.}
				\end{enumerate}
		\declaration{1.12}
			The notation \gls{int[a,b]} will be used to refer to the first part of the triple yielded by executing \procedurehr{1.32} on $\langle a,b\rangle$.
		\procedure{1.33}
			\objective
				Choose two positive integers $a,b$. The objective of the following instructions is to show that either $0<0$ or $[a,b]=[b,a]$.
			\implementation
				\begin{enumerate}
					\item Verify that $(a,b)>0$.
					\item Using \procedurehr{1.16}, verify that $[a,b](a,b)=ab=ba=[b,a](b,a)=[b,a](a,b)$.
					\item \textbf{Therefore verify that $[a,b]=[b,a]$.}
				\end{enumerate}
		\procedure{1.34}
			\objective
				Choose an integer $a$ and two positive integers $b,c$. The objective of the following instructions is to show that either $0<0$ or $[ca,cb]=c[a,b]$.
			\implementation
				\begin{enumerate}
					\item Verify that $(ca,cb)>0$.
					\item Using \procedurehr{1.18}, verify that $[ca,cb](ca,cb)=cacb=c^2ab=c^2[a,b](a,b)=c[a,b](ca,cb)$.
					\item \textbf{Therefore verify that $[ca,cb]=c[a,b]$.}
				\end{enumerate}
		\procedure{1.35}
			\objective
				Choose an integer $a$ and two positive integers $b,c$. The objective of the following instructions is to show that either $0<0$ or $[[a,b],c]=[a,[b,c]]$.
			\implementation
				\begin{enumerate}
					\item Using \procedurehr{1.19}, verify that $(a,b)(ab,(ac,bc))(b,c)[[a,b],c]$
					\begin{enumerate}
						\item $=(ab,(ac,bc))(b,c)[(a,b)[a,b],(a,b)c]$
						\item $=(ab,(ac,bc))(b,c)[ab,(ac,bc)]$
						\item $=ab(ac,bc)(b,c)$
						\item $=abc(a,b)(b,c)$
						\item $=bc(a,b)(ab,ac)$
						\item $=(a,b)((ab,ac),bc)[(ab,ac),bc]$
						\item $=(a,b)(ab,(ac,bc))[(ab,ac),bc]$
						\item $=(a,b)(ab,(ac,bc))[a(b,c),[b,c](b,c)]$
						\item $=(a,b)(ab,(ac,bc))(b,c)[a,[b,c]]$.
					\end{enumerate}
					\item Verify that $(a,b)(ab,(ac,bc))(b,c)>0$.
					\item \textbf{Therefore verify that $[[a,b],c]=[a,[b,c]]$.}
				\end{enumerate}
		\declaration{1.13}
			The notation \gls{int[a_0,a_1,...,a_n-1]} will be used to contextually refer to one of the following integers:
			\begin{enumerate}
				\item $[[a_0],[a_1,a_2,\cdots,a_{n-1}]]$
				\item $[[a_0,a_1],[a_2,a_3,\cdots,a_{n-1}]]$
				\item $\vdots$
				\item $[[a_0,a_1,\cdots,a_{n-2}],[a_{n-1}]]$
			\end{enumerate}
		\procedure{1.36}
			\objective
				Choose three positive integers $a,b,c$. The objective of the following instructions is to show that either $0<0$ or $([a,b],c)=[(a,c),(b,c)]$.
			\implementation
				\begin{enumerate}
					\item Using \procedurehr{1.32}, \procedurehr{1.18}, \procedurehr{1.19}, \procedurehr{1.16}, and \procedurehr{1.10}, verify that $(a,b)((a,c),(b,c))([a,b],c)$
					\begin{enumerate}
						\item $=((a,c),(b,c))((a,b)[a,b],(a,b)c)$
						\item $=((a,c),(b,c))(ab,(ac,bc))$
						\item $=(a^2b,a^2c,c^2a,c^2b,b^2a,bac,b^2c)$
						\item $=(a,b)(ab,ac,bc,c^2)$
						\item $=(a,b)(a,c)(b,c)$
						\item $=(a,b)((a,c),(b,c))[(a,c),(b,c)]$.
					\end{enumerate}
					\item Verify that $(a,b)((a,c),(b,c))>0$.
					\item \textbf{Therefore verify that $([a,b],c)=[(a,c),(b,c)]$.}
				\end{enumerate}
		\procedure{1.37}
			\objective
				Choose three positive integers $a,b,c$. The objective of the following instructions is to show that either $0<0$ or $[(a,b),c]=([a,c],[b,c])$.
			\implementation
				\begin{enumerate}
					\item Using \procedurehr{1.32}, \procedurehr{1.18}, \procedurehr{1.19}, \procedurehr{1.16}, and \procedurehr{1.10}, verify that $((a,b),c)(a,c)(b,c)[(a,b),c]$
					\begin{enumerate}
						\item $=(a,c)(b,c)(a,b)c$
						\item $=(ab,ac,cb,c^2)(a,b)c$
						\item $=(a^2b,a^2c,ac^2,ab^2,abc,cb^2,bc^2)c$
						\item $=(a,b,c)(ab,ac,bc)c$
						\item $=((a,b),c)(ac(b,c),bc(a,c))$
						\item $=((a,b),c)(a,c)(b,c)([a,c],[b,c])$.
					\end{enumerate}
					\item Verify that $((a,b),c)(a,c)(b,c)>0$.
					\item \textbf{Therefore verify that $[(a,b),c]=([a,c],[b,c])$.}
				\end{enumerate}
		\declaration{1.14}
			The notation \gls{intchibdac}, where $a,c$ are two integers and $b,d$ are two positive integers such that $a\equiv c(\mod(b,d))$,  will be used to refer to the result yielded by executing the following instructions:
			\begin{enumerate}
				\item Execute \procedurehr{1.09} on $\langle b,d\rangle$ and let $\langle f,g,h,i,j\rangle$ receive.
				\item \textbf{Yield the tuple $\langle (a+((c-a)\div(b,d))ib)\mod[b,d]\rangle$.}
			\end{enumerate}
		\procedure{1.39}
			\objective
				Choose three integers $x,a,c$ and two positive integers $b,d$ such that $x\equiv a(\mod b)$ and $x\equiv c(\mod d)$. The objective of the following instructions is to show that $0\ne 0$ if $a\not\equiv d(\mod(b,d))$, otherwise $x\equiv\chi_{b,d}(a,c)(\mod[b,d])$.
			\implementation
				\begin{enumerate}
					\item Execute \procedurehr{1.09} on $\langle b,d\rangle$ and let $\langle e,f,g,h,i\rangle$ receive.
					\item Let $j=x\div b-a\div b$.
					\item Verify that $x=a+jb$.
					\item Let $k=x\div d-c\div d$.
					\item Verify that $x=c+kd$.
					\item Therefore verify that $c-a=jb-kd$.
					\item If $a\not\equiv c(\mod(b,d))$, then do the following:
					\begin{enumerate}
						\item Verify that $0\not\equiv d-a=jb-kd=jef-keg\equiv 0(\mod e)$.
						\item \textbf{Therefore verify that $0\ne 0$.}
						\item \textbf{Abort procedure.}
					\end{enumerate}
					\item Otherwise do the following:
					\begin{enumerate}
						\item Verify that $c-a\equiv 0(\mod(b,d))$.
						\item Let $l=(c-a)\div(b,d)$.
						\item Verify that $l(b,d)=le=c-a=jb-kd=jef-keg$.
						\item Therefore verify that $l=jf-kg$.
						\item Therefore verify that $l\equiv jf(\mod g)$.
						\item Also, using (1) verify that $efh+egi=bh+di=e$.
						\item Therefore verify that $fh+gi=1$.
						\item Therefore verify that $fh\equiv 1(\mod g)$.
						\item Therefore verify that $lh\equiv jfh\equiv j(\mod g)$.
						\item Therefore using \procedurehr{1.05}, verify that $lhb\equiv jb(\mod bg=[b,d])$.
						\item \textbf{Therefore verify that $x\equiv a+jb\equiv a+lhb\equiv\chi_{b,d}(a,c)(\mod[b,d])$.}
					\end{enumerate}
				\end{enumerate}
		\procedure{1.40}
			\objective
				Choose two integers $a,c$ and two positive integers $b,d$ in such a way that $a\equiv c(\mod(b,d))$. The objective of the following instructions is to show that either $0<0$ or $\chi_{b,d}(a,c)=\chi_{d,b}(c,a)$.
			\implementation
				\begin{enumerate}
					\item Execute \procedurehr{1.09} on $\langle b,d\rangle$ and let $\langle f,g,h,i,j\rangle$ receive.
					\item Verify that $ib+jd=f=(b,d)$.
					\item Execute \procedurehr{1.09} on $\langle d,b\rangle$ and let $\langle k,l,m,n,p\rangle$ receive.
					\item Verify that $pb+nd=k=(d,b)=(b,d)$.
					\item Execute \procedurehr{1.12} on $\langle b,p,n,d\rangle$ and let $\langle q\rangle$ receive.
					\item Therefore verify that $n=j-qg$.
					\item Now using \procedurehr{1.33}, verify that $\chi_{b,d}(a,c)$
					\begin{enumerate}
						\item $=(a+((c-a)\div(b,d))ib)\mod[b,d]$
						\item $=(a+((c-a)\div(b,d))(f-jd))\mod[b,d]$
						\item $=(a+((c-a)\div(b,d))f+((a-c)\div(b,d))jd)\mod[b,d]$
						\item $=(a+(c-a)+((a-c)\div(b,d))jd)\mod[b,d]$
						\item $=(c+((a-c)\div(d,b))(n+qg)d)\mod[b,d]$
						\item $=(c+((a-c)\div(d,b))dn+((a-c)\div(d,b))q[b,d])\mod[b,d]$
						\item $=(c+((a-c)\div(d,b))dn)\mod[b,d]$
						\item $=(c+((a-c)\div(d,b))dn)\mod[d,b]$
						\item $=\chi_{d,b}(c,a)$.
					\end{enumerate}
				\end{enumerate}
		\procedure{1.41}
			\objective
				Choose three integers $x,a,c$ and two positive integers $b,d$ such that $a\equiv c(\mod(b,d))$ and $x\equiv\chi_{b,d}(a,c)(\mod[b,d])$. The objective of the following instructions is to show that $x\equiv a(\mod b)$.
			\implementation
				\begin{enumerate}
					\item Execute \procedurehr{1.09} on $\langle b,d\rangle$ and let $\langle e,f,g,h,i\rangle$.
					\item Verify that $x\mod[b,d]=\chi_{b,d}(a,c)\mod[b,d]$.
					\item Therefore verify that $x\mod(bg)=\chi_{b,d}(a,c)\mod(bg)$.
					\item Therefore verify that $(x\mod(bg))\mod b=(\chi_{b,d}(a,c)\mod(bg))\mod b$.
					\item \textbf{Therefore using \procedurehr{1.03}, verify that $x\mod b=\chi_{b,d}(a,c)\mod b=(a+((c-a)\div(b,d))hb)\mod b=a\mod b$.}
				\end{enumerate}
		\procedure{1.42}
			\objective
				Choose three integers $x,a,c$ and two positive integers $b,d$ such that $a\equiv c(\mod(b,d))$ and $x\equiv\chi_{b,d}(a,c)(\mod[b,d])$. The objective of the following instructions is to either show that $0<0$ or to show that $x\equiv a(\mod b)$ and $x\equiv c(\mod d)$.
			\implementation
				\begin{enumerate}
					\item Execute \procedurehr{1.41} on $\langle x,a,c,b,d\rangle$.
					\item \textbf{Therefore verify that $x\equiv a(\mod b)$.}
					\item Now using \procedurehr{1.40}, verify that $x\equiv\chi_{b,d}(a,c)\equiv\chi_{d,b}(c,a)(\mod[d,b])$
					\item Execute \procedurehr{1.41} on $\langle x,c,a,d,b\rangle$.
					\item \textbf{Therefore verify that $x\equiv c(\mod d)$.}
				\end{enumerate}
		\procedure{1.43}
			\objective
				Choose two integers $a,c$ and three positive integers $b,d,e$ such that $a\equiv c(\mod(b,d))$. The objective of the following instructions is to show that $\chi_{b,d}(ea,ec)=e\chi_{b,d}(a,c)$.
			\implementation
				\begin{enumerate}
					\item Verify that $\chi_{b,d}(a,c)\equiv a(\mod b)$.
					\item Therefore using \procedurehr{1.05}, verify that $e\chi_{b,d}(a,c)\equiv ea(\mod b)$.
					\item Verify that $\chi_{b,d}(a,c)\equiv c(\mod d)$.
					\item Therefore using \procedurehr{1.05}, verify that $e\chi_{b,d}(a,c)\equiv ec(\mod d)$.
					\item Also using \procedurehr{1.02} and the precondition, verify that $ea\equiv ec(\mod(b,d))$.
					\item Therefore using \procedurehr{1.39}, verify that $e\chi_{b,d}(a,c)\equiv\chi_{b,d}(ea,ec)(\mod[b,d])$.
					\item \textbf{Therefore verify that $e\chi_{b,d}(a,c)=\chi_{b,d}(ea,ec)$.}
				\end{enumerate}
		\procedure{1.44}
			\objective
				Choose two integers $a,c$ and three positive integers $b,d,e$ such that $a\equiv c(\mod(eb,ed))$. The objective of the following instructions is to show that $\chi_{eb,ed}(a,c)\mod[b,d]=\chi_{b,d}(a,c)$.
			\implementation
				\begin{enumerate}
					\item Verify that $\chi_{eb,ed}(a,c)\equiv a(\mod eb)$.
					\item Therefore using \procedurehr{1.03}, verify that $\chi_{eb,ed}(a,c)\equiv a(\mod b)$.
					\item Verify that $\chi_{eb,ed}(a,c)\equiv c(\mod ed)$.
					\item Therefore using \procedurehr{1.03}, verify that $\chi_{eb,ed}(a,c)\equiv c(\mod d)$.
					\item Now verify that $a\equiv c(\mod e(b,d))$.
					\item Therefore using \procedurehr{1.03}, verify that $a\equiv c(\mod (b,d))$.
					\item Therefore using \procedurehr{1.39}, verify that $\chi_{eb,ed}(a,c)\equiv\chi_{b,d}(a,c)(\mod[b,d])$.
					\item \textbf{Therefore verify that $\chi_{eb,ed}(a,c)\mod[b,d]=\chi_{b,d}(a,c)$.}
				\end{enumerate}
		\procedure{1.45}
			\objective
				Choose three integers $a,c,e$ and three positive integers $b,d,f$ such that $a\equiv e(\mod(b,f))$, and $c\equiv e(\mod(d,f))$. The objective of the following instructions is to show that either $0<0$ or $\chi_{b,d}(a,c)\equiv e(\mod([b,d],f))$.
			\implementation
				\begin{enumerate}
					\item Execute \procedurehr{1.09} on $\langle b,f\rangle$ and let $\langle g_0,h_0,i_0,j_0,k_0\rangle$ receive.
					\item Execute \procedurehr{1.09} on $\langle d,f\rangle$ and let $\langle g_1,h_1,i_1,j_1,k_1\rangle$ receive.
					\item Verify that $e\equiv a(\mod(b,f))$.
					\item Verify that $e\equiv c(\mod(d,f))$.
					\item Therefore using \procedurehr{1.39} and \procedurehr{1.44}, verify that $e$
					\begin{enumerate}
						\item $\equiv\chi_{(b,f),(d,f)}(a,c)$
						\item $\equiv\chi_{(b,f)h_1,(d,f)h_2}(a,c)$
						\item $=\chi_{b,d}(a,c)(\mod[(b,f),(d,f)])$.
					\end{enumerate}
					\item \textbf{Therefore using \procedurehr{1.36}, verify that $e\equiv\chi_{b,d}(a,c)(\mod([b,d],f))$.}
				\end{enumerate}
		\procedure{1.46}
			\objective
				Choose three integers $a,c,e$ and three positive integers $b,d,f$ such that $a\equiv c(\mod(b,d))$, $a\equiv e(\mod(b,f))$, and $c\equiv e(\mod(d,f))$. Execute \procedurehr{1.45} on $\langle a,c,e,b,d,f\rangle$. Execute \procedurehr{1.45} on $\langle c,e,a,d,f,b\rangle$. The objective of the following instructions is to show that $\chi_{[b,d],f}(\chi_{b,d}(a,c),e)=\chi_{b,[d,f]}(a,\chi_{d,f}(c,e))$.
			\implementation
				\begin{enumerate}
					\item Verify that $\chi_{[b,d],f}(\chi_{b,d}(a,c),e)\equiv e(\mod f)$.
					\item Verify that $\chi_{[b,d],f}(\chi_{b,d}(a,c),e)\equiv \chi_{b,d}(a,c)(\mod [b,d]=gb=hd)$.
					\item Therefore using \procedurehr{1.03}, verify that $\chi_{[b,d],f}(\chi_{b,d}(a,c),e)\equiv \chi_{b,d}(a,c)\equiv a(\mod b)$.
					\item Also using \procedurehr{1.03}, verify that $\chi_{[b,d],f}(\chi_{b,d}(a,c),e)\equiv \chi_{b,d}(a,c)\equiv c(\mod d)$.
					\item Therefore using (1), (4), and \procedurehr{1.39}, verify that $\chi_{[b,d],f}(\chi_{b,d}(a,c),e)\equiv\chi_{d,f}(c,e)(\mod[d,f])$.
					\item Therefore using (3), (5), and \procedurehr{1.39}, verify that $\chi_{[b,d],f}(\chi_{b,d}(a,c),e)\equiv\chi_{b,[d,f]}(a,\chi_{d,f}(c,e))(\mod [b,[d,f]]=[[b,d],f])$.
					\item \textbf{Therefore verify that $\chi_{[b,d],f}(\chi_{b,d}(a,c),e)=\chi_{b,[d,f]}(a,\chi_{d,f}(c,e))$.}
				\end{enumerate}
		\declaration{1.15}
			The notation \gls{intchib_0,b_1,...,b_n-1a_0a_1,...,a_n-1} will be used to contextually refer to one of the following integers:
			\begin{enumerate}
				\item $\chi_{b_0,[b_1,b_2,\cdots,b_{n-1}]}(a_0,\allowbreak\chi_{b_1,b_2,\cdots,b_{n-1}}(a_1,a_2,\cdots,a_{n-1}))$
				\item $\chi_{[b_0,b_1],[b_2,b_3,\cdots,b_{n-1}]}(\chi_{b_0,b_1}(a_0,a_1),\allowbreak\chi_{b_2,b_3,\cdots,b_{n-1}}(a_2,a_3,\cdots,a_{n-1})$
				\item $\vdots$
				\item $\chi_{[b_0,b_1,\cdots,b_{n-2}],b_{n-1}}(\chi_{b_0,b_1,\cdots,b_{n-2}}(a_0,a_1,\cdots,a_{n-2}),\allowbreak a_{n-1})$
			\end{enumerate}
		\declaration{1.16}
			The notation \gls{intphi(n)} will be used as a shorthand for the sublist of $[0:n]$ where each entry $x$ is such that $(x,n)=1$.
		\procedure{1.47}
			\objective
				Choose an integer $a$ and a positive integer $b$ such that $(a,b)=1$. The objective of the following instructions is to either show that $0<0$ or to show that each element of $a\phi(b)\mod b$ is in $\phi(b)$.
			\implementation
				\begin{enumerate}
					\item Verify that $(a,b)=1$.
					\item For $i$ in $[0:\lvert\phi(b)\rvert]$, do the following:
					\begin{enumerate}
						\item Using \declarationhr{1.16}, verify that $(\phi(b)_i,b)=1$.
						\item Execute \procedurehr{1.20} on $\langle a,\phi(b)_i,b\rangle$.
						\item Therefore verify that $(a\phi(b)_i,b)=1$.
						\item Execute \procedurehr{1.17} on $\langle a\phi(b)_i\mod b,a\phi(b)_i,b\rangle$.
						\item Therefore verify that $(a\phi(b)_i\mod b,b)=(a\phi(b)_i,b)=1$.
						\item Also verify that $0\le a\phi(b)_i\mod b<b$.
						\item Therefore verify that $a\phi(b)_i\mod b$ is contained in the list $\phi(b)$.
					\end{enumerate}
					\item \textbf{Therefore verify that each element of $a\phi(b)\mod b$ is in $\phi(b)$.}
				\end{enumerate}
		\procedure{1.48}
			\objective
				Choose an integer $a$ and a positive integer $b$ such that $(a,b)=1$. The objective of the following instructions is to either show that $0\ne 0$ or to show that each element of $a\phi(b)\mod b$ is distinct.
			\implementation
				\begin{enumerate}
					\item Execute \procedurehr{1.09} on $\langle a,b\rangle$ and let $\langle r,t,u,v,w\rangle$ receive.
					\item Verify that $va+wb=r=(a,b)=1$.
					\item Therefore verify that $va\equiv 1(\mod b)$.
					\item Now for $i$ in $[0:\lvert\phi(b)\rvert]$, do the following:
					\begin{enumerate}
						\item For $j$ in $[i+1:\lvert\phi(b)\rvert]$, do the following:
						\begin{enumerate}
							\item If $a\phi(b)_i\equiv a\phi(b)_j(\mod b)$, then do the following:
							\begin{enumerate}
								\item Verify that $\phi(b)_i\equiv va\phi(b)_i\equiv va\phi(b)_j\equiv\phi(b)_j(\mod b)$.
								\item Therefore verify that $\phi(b)_i=\phi(b)_j$.
								\item Also verify that $i\ne j$.
								\item Therefore using \declarationhr{1.16}, verify that $\phi(b)_i\ne\phi(b)_j$.
								\item \textbf{Therefore using (B) and (D), verify that $\phi(b)_i\ne\phi(b)_i$.}
								\item \textbf{Abort procedure.}
							\end{enumerate}
							\item Otherwise, do the following:
							\begin{enumerate}
								\item Verify that $a\phi(b)_i\not\equiv a\phi(b)_j(\mod b)$.
							\end{enumerate}
						\end{enumerate}
					\end{enumerate}
					\item \textbf{Therefore verify that $a\phi(b)\mod b$ is composed of distinct elements.}
				\end{enumerate}
		\procedure{1.49}
			\objective
				Choose an integer $a$ and a positive integer $b$ such that $(a,b)=1$. The objective of the following instructions is to either show that $0<0$ or to show that $a\phi(b)\mod b$ is a rearrangement of $\phi(b)$.
			\implementation
				\begin{enumerate}
					\item Execute \procedurehr{1.47} on $\langle a,b\rangle$.
					\item Therefore verify that each element of $a\phi(b)\mod b$ is in $\phi(b)$.
					\item Verify that $\lvert a\phi(b)\mod b\rvert=\lvert \phi(b)\rvert$.
					\item Execute \procedurehr{1.48} on $\langle a,b\rangle$.
					\item Therefore verify that $a\phi(b)\mod b$ is composed of distinct elements.
					\item \textbf{Therefore verify that $a\phi(b)\mod b$ is a rearrangement of $\phi(b)$.}
				\end{enumerate}
		\procedure{1.50}
			\objective
				Choose an integer $a$ and a positive integer $b$ such that $(a,b)=1$. The objective of the following instructions is to show that either $0<0$ or $a^{\lvert\phi(b)\rvert}\equiv 1(\mod b)$.
			\implementation
				\begin{enumerate}
					\item For $i$ in $[0:\lvert\phi(b)\rvert]$, do the following:
					\begin{enumerate}
						\item Execute \procedurehr{1.09} on $\langle\phi(b)_i,b\rangle$ and let $\langle r_i,t_i,u_i,v_i,w_i\rangle$.
						\item Using \declarationhr{1.16}, verify that $v_i\phi(b)_i+w_ib=r_i=(\phi(b)_i,b)=1$.
						\item Therefore verify that $v_i\phi(b)_i\equiv 1(\mod b)$.
					\end{enumerate}
					\item Therefore using \procedurehr{1.49}, verify that $\prod_i^{[0:{\lvert\phi(b)\rvert}]}\phi(b)_i\equiv\prod_i^{[0:{\lvert\phi(b)\rvert}]}a\phi(b)_i\equiv a^{\lvert\phi(b)\rvert}\prod_i^{[0:{\lvert\phi(b)\rvert}]}\phi(b)_i(\mod b)$.
					\item \textbf{Therefore verify that $1\equiv\prod_i^{[0:{\lvert\phi(b)\rvert}]}(v_i\phi(b)_i)=\prod_i^{[0:{\lvert\phi(b)\rvert}]}v_i\prod_i^{[0:{\lvert\phi(b)\rvert}]}\phi(b)_i\equiv a^{\lvert\phi(b)\rvert}\prod_i^{[0:{\lvert\phi(b)\rvert}]}\phi(b)_i\prod_i^{[0:{\lvert\phi(b)\rvert}]}v_i\equiv a^{\lvert\phi(b)\rvert}(\mod b)$.}
				\end{enumerate}
		\declaration{1.17}
			The notation \gls{intatimesb} as a shorthand for the $\lvert a\rvert\times\lvert b\rvert$ matrix such that for $i$ in $[0:\lvert a\rvert]$, for $j$ in $[0:\lvert b\rvert]$, $(a\times b)_{i,j}=\langle a_i,b_j\rangle$.
		\procedure{1.52}
			\objective
				Choose two positive integers $a,b$ such that $(a,b)=1$. The objective of the following instructions is to show that each entry of $\chi_{a,b}([0:a]\times[0:b])$ is in $[0:ab]$.
			\implementation
				\begin{enumerate}
					\item Let $h=\chi_{a,b}([0:a]\times[0:b])$.
					\item Therefore verify that $0\le h_{i,j}<[a,b]=[a,b](a,b)=ab$ for $i$ in $[0:a]$, for $j$ in $[0:b]$.
					\item \textbf{Therefore verify that each entry of $h$ is in $[0:ab]$.}
				\end{enumerate}
		\procedure{1.53}
			\objective
				Choose two positive integers $a,b$ such that $(a,b)=1$. The objective of the following instructions is to either show that $0<0$ or to show that each entry of $\chi_{a,b}([0:a]\times[0:b])$ is distinct.
			\implementation
				\begin{enumerate}
					\item Let $h=\chi_{a,b}([0:a]\times[0:b])$.
					\item For each distinct unordered pair of index pairs $\langle i,j\rangle$ and $\langle k,l\rangle$ of $h$, do the following:
					\begin{enumerate}
						\item If $h_{i,j}=h_{k,l}$, then do the following:
						\begin{enumerate}
							\item Verify that $\chi_{a,b}([0:a]_i,[0:b]_j)=h_{i,j}=h_{k,l}=\chi_{a,b}([0:a]_k,[0:b]_l)$.
							\item Verify that $\chi_{a,b}(i,j)=\chi_{a,b}(k,l)$.
							\item Therefore using \procedurehr{1.42}, verify that $i\equiv\chi_{a,b}(i,j)=\chi_{a,b}(k,l)\equiv k(\mod a)$.
							\item Therefore verify that $i=k$.
							\item Also using \procedurehr{1.42}, verify that $j\equiv\chi_{a,b}(i,j)=\chi_{a,b}(k,l)\equiv l(\mod b)$.
							\item Therefore verify that $j=l$.
							\item Therefore verify that $\langle i,j\rangle=\langle k,l\rangle$.
							\item Using (2), verify that $\langle i,j\rangle\ne\langle k,l\rangle$.
							\item \textbf{Therefore verify that $\langle i,j\rangle\ne\langle i,j\rangle$.}
							\item \textbf{Abort procedure.}
						\end{enumerate}
						\item Otherwise do the following:
						\begin{enumerate}
							\item Verify that $h_{i,j}\ne h_{k,l}$.
						\end{enumerate}
					\end{enumerate}
					\item \textbf{Therefore verify that each entry of $h$ is distinct.}
				\end{enumerate}
		\procedure{1.54}
			\objective
				Choose two positive integers $a,b$ such that $(a,b)=1$. The objective of the following instructions is to show that either $0<0$ or $\chi_{a,b}([0:a]\times[0:b])$ is a rearrangement $[0:ab]$.
			\implementation
				\begin{enumerate}
					\item Let $h=\chi_{a,b}([0:a]\times[0:b])$.
					\item Execute \procedurehr{1.52} on $\langle a,b\rangle$.
					\item Therefore verify that each element of $h$ is in $[0:ab]$.
					\item Also verify that $h$ has the same number of entries as $[0:ab]$.
					\item Execute \procedurehr{1.53} on $\langle a,b\rangle$.
					\item Therefore verify that $h$ is composed of distinct elements.
					\item \textbf{Therefore verify that $h$ is a rearrangement of $[0:ab]$.}
				\end{enumerate}
		\procedure{1.55}
			\objective
				Choose two positive integers $a,b$ such that $(a,b)=1$. The objective of the following instructions is to either show that $0<0$ or to show that each entry of $\chi_{a,b}(\phi(a)\times\phi(b))$ is in $\phi(ab)$.
			\implementation
				\begin{enumerate}
					\item Let $h=\chi_{a,b}(\phi(a)\times\phi(b))$.
					\item Now, for each index pair $\langle i,j\rangle$ of $h$, do the following:
					\begin{enumerate}
						\item Verify that $0\le h_{i,j}<[a,b]=[a,b](a,b)=ab$.
						\item Verify that $h_{i,j}=\chi_{a,b}(\phi(a)_i,\phi(b)_j)\equiv\phi(a)_i(\mod a)$.
						\item Execute \procedurehr{1.17} on $\langle h_{i,j},\phi(a)_i,a\rangle$.
						\item Therefore verify that $(a,h_{i,j})=(h_{i,j},a)=(\phi(a)_i,a)=1$.
						\item Verify that $h_{i,j}=\chi_{a,b}(\phi(a)_i,\phi(b)_j)\equiv\phi(b)_j(\mod b)$.
						\item Execute \procedurehr{1.17} on $\langle h_{i,j},\phi(b)_j,b\rangle$.
						\item Therefore verify that $(b,h_{i,j})=(h_{i,j},b)=(\phi(b)_j,b)=1$.
						\item Therefore verify that $(h_{i,j},ab)=(ab,h_{i,j})=1$.
						\item Therefore verify that $h_{i,j}$ is in $\phi(ab)$.
					\end{enumerate}
					\item \textbf{Therefore verify that each entry of $\chi_{a,b}(\phi(a)\times\phi(b))$ is in $\phi(ab)$.}
				\end{enumerate}
		\procedure{1.56}
			\objective
				Choose two positive integers $a,b$ such that $(a,b)=1$. The objective of the following instructions is to either show that $0<0$ or to show that each entry of $\phi(ab)$ is in $\chi_{a,b}(\phi(a)\times\phi(b))$.
			\implementation
				\begin{enumerate}
					\item For $i$ in $[0:\lvert\phi(ab)\rvert]$, do the following:
					\begin{enumerate}
						\item Verify that $(\phi(ab)_i,ab)=1$.
						\item Verify that $\phi(ab)_i\equiv\phi(ab)_i\mod a(\mod a)$.
						\item Therefore using \procedurehr{1.17}, verify that $(\phi(ab)_i\mod a,a)=(\phi(ab)_i,a)=1$.
						\item Also verify that $0\le \phi(ab)_i\mod a<a$.
						\item Therefore verify that $\phi(ab)_i\mod a$ is amongst $\phi(a)$.
						\item Verify that $\phi(ab)_i\equiv\phi(ab)_i\mod b(\mod b)$.
						\item Also using \procedurehr{1.17}, verify that $(\phi(ab)_i\mod b,b)=(\phi(ab)_i,b)=1$.
						\item Also verify that $0\le \phi(ab)_i\mod b<b$.
						\item Therefore verify that $\phi(ab)_i\mod b$ is amongst $\phi(b)$.
						\item Therefore verify that $\langle \phi(ab)_i\mod a,\phi(ab)_i\mod b\rangle$ is amongst $\phi(a)\times\phi(b)$.
						\item Also using (b) and (f) and \procedurehr{1.39}, verify that $\phi(ab)_i\equiv\chi_{a,b}(\phi(ab)_i\mod a,\phi(ab)_i\mod b)(\mod[a,b]=[a,b](a,b)=ab)$.
						\item Therefore verify that $\phi(ab)_i=\chi_{a,b}(\phi(ab)_i\mod a,\phi(ab)_i\mod b)$.
						\item Therefore using (j) and (l), verify that $\phi(ab)_i$ is amongst $\chi_{a,b}(\phi(a)\times\phi(b))$.
					\end{enumerate}
					\item \textbf{Therefore verify that each entry of $\phi(ab)$ is in $\chi_{a,b}(\phi(a)\times\phi(b))$.}
				\end{enumerate}
		\procedure{1.57}
			\objective
				Choose two positive integers $a,b$ such that $(a,b)=1$. The objective of the following instructions is to either show that $0<0$ or to show that $\phi(ab)$ is a rearrangement of $\chi_{a,b}(\phi(a)\times\phi(b))$ and that $\lvert\phi(ab)\rvert=\lvert\phi(a)\rvert\lvert\phi(b)\rvert$.
			\implementation
				\begin{enumerate}
					\item Execute \procedurehr{1.54} on $\langle a,b\rangle$.
					\item Therefore verify that $\chi_{a,b}([0:a]\times[0:b])$ are a rearrangement of $[0:ab]$.
					\item Verify that $\chi_{a,b}(\phi(a)\times\phi(b))$ is a submatrix of $\chi_{a,b}([0:a]\times[0:b])$.
					\item Therefore verify that the entries of $\chi_{a,b}(\phi(a)\times\phi(b))$ are distinct.
					\item Execute \procedurehr{1.55} on $\langle a,b\rangle$.
					\item Therefore verify that the entries of $\chi_{a,b}(\phi(a)\times\phi(b))$ are in $\phi(ab)$.
					\item Verify that that the entries of $\phi(ab)$ are distinct.
					\item Execute \procedurehr{1.56} on $\langle a,b\rangle$.
					\item Therefore verify that the entries of $\phi(ab)$ are in $\chi_{a,b}(\phi(a)\times\phi(b))$.
					\item \textbf{Therefore verify that $\phi(ab)$ is a rearrangement of $\chi_{a,b}(\phi(a)\times\phi(b))$.}
					\item \textbf{Therefore verify that $\lvert\phi(ab)\rvert=\lvert\chi_{a,b}(\phi(a)\times\phi(b))\rvert=\lvert\phi(a)\times\phi(b)\rvert=\lvert\phi(a)\rvert\lvert\phi(b)\rvert$.}
				\end{enumerate}
		\declaration{1.18}
			The notation \gls{int[P]}, where $P$ is a condition, will be used as a shorthand for $1$ if $P$, otherwise it will stand for $0$.
		\declaration{1.32}
			The notation \gls{inta_+}, where $a$ is a list, will be used as a shorthand for $0$ if $a$ is empty, otherwise it will be a shorthand for the sum of the entries of $a$.
		\declaration{1.19}
			The notation \gls{intsigmaf(r)}, where $R$ is a list and $f[r]$ is a function of $r$, will be used as a shorthand for $f(R)_+$.
		\procedure{1.58}
			\objective
				Choose a positive integer $a$ and a prime $b$. The objective of the following instructions is to show that either $0<0$ or $\lvert\phi(b^a)\rvert=b^a-b^{a-1}$.
			\implementation
				\begin{enumerate}
					\item Using \procedurehr{1.21}, verify that $\sum_r^{[0:{b^a}]}[(r,b^a)=1]\le\sum_r^{[0:{b^a}]}[(r,b)=1]$.
					\item Using \procedurehr{1.20}, verify that $\sum_r^{[0:{b^a}]}[(r,b)=1]\le\sum_r^{[0:{b^a}]}[(r,b^a)=1]$.
					\item Therefore verify that $\sum_r^{[0:{b^a}]}[(r,b^a)=1]=\sum_r^{[0:{b^a}]}[(r,b)=1]$.
					\item Using \procedurehr{1.13}, verify that $\sum_r^{[0:{b^a}]}[(r,b)=1]\le\sum_r^{[0:{b^a}]}[r\mod b\ne 0]$.
					\item Using \procedurehr{1.22}, verify that $\sum_r^{[0:{b^a}]}[r\mod b\ne 0]\le\sum_r^{[0:{b^a}]}[(r,b)=1]$.
					\item Therefore verify that $\sum_r^{[0:{b^a}]}[(r,b)=1]=\sum_r^{[0:{b^a}]}[r\mod b\ne 0]$.
					\item \textbf{Therefore using (3) and (6), verify that $\lvert\phi(b^a)\rvert=\sum_r^{[0:{b^a}]}[(r,b^a)=1]=\sum_r^{[0:{b^a}]}[(r,b)=1]=\sum_r^{[0:{b^a}]}[r\mod b\ne 0]=\sum_r^{[0:{b^a}]}(1-[r\mod b=0])=b^a-b^{a-1}$.}
				\end{enumerate}
		\procedure{1.59}
			\objective
				Choose a list of primes $a$. Let $b$ be the list of distinct primes in $a$. Let $c$ be a list such that $c_i$ is the multiplicity of $b_i$ in $a$ for $i=1$ to $i=\lvert b\rvert$. The objective of the following instructions is to show that either $0<0$ or $\lvert\phi(a_*)\rvert=\prod_i^{[0:{\lvert b\rvert}]}({b_i}^{c_i}-{b_i}^{c_i-1})$.
			\implementation
				\begin{enumerate}
					\item If $a=\langle\rangle$, then do the following:
					\begin{enumerate}
						\item Verify that $\lvert b\rvert=\lvert a\rvert=0$.
						\item \textbf{Therefore verify that $\phi(a_*)=\phi(1)=1=\prod_i^{[0:{\lvert b\rvert}]}({b_i}^{c_i}-{b_i}^{c_i-1})$.}
					\end{enumerate}
					\item Otherwise, do the following:
					\begin{enumerate}
						\item Verify that $a_*=\prod_i^{[0:{\lvert b\rvert}]} {b_i}^{c_i}$.
						\item Verify that $\lvert a\rvert>0$.
						\item Therefore verify that $\lvert c\rvert=\lvert b\rvert>0$.
						\item Therefore using \procedurehr{1.30}, verify that $({b_0}^{c_0},\prod_i^{[1:{\lvert b\rvert}]}{b_i}^{c_i})=1$.
						\item Let $d$ be the list $a$ with all instances of $a_0$ removed.
						\item Verify that $\lvert d\rvert<\lvert a\rvert$.
						\item Now execute \procedurehr{1.59} on $\langle d\rangle$.
						\item Hence verify that $\phi(d_*)=\phi(\prod_i^{[1:{\lvert b\rvert}]}{b_i}^{c_i})=\prod_i^{[1:{\lvert b\rvert}]}({b_i}^{c_i}-{b_i}^{c_i-1})$.
						\item \textbf{Therefore using (d), (h), \procedurehr{1.57} and \procedurehr{1.58}, verify that $\lvert\phi(a_*)\rvert=\lvert\phi(\prod_i^{[0:{\lvert b\rvert}]} {b_i}^{c_i})\rvert=\lvert\phi({b_0}^{c_0}\prod_i^{[1:{\lvert b\rvert}]}{b_i}^{c_i})\rvert=\lvert\phi({b_0}^{c_0})\rvert\lvert\phi(\prod_i^{[1:{\lvert b\rvert}]}{b_i}^{c_i})\rvert=({b_0}^{c_0}-{b_0}^{c_0-1})\lvert\phi(\prod_i^{[1:{\lvert b\rvert}]}{b_i}^{c_i})\rvert=({b_0}^{c_0}-{b_0}^{c_0-1})\prod_i^{[1:{\lvert b\rvert}]}({b_i}^{c_i}-{b_i}^{c_i-1})=\prod_i^{[0:{\lvert b\rvert}]}({b_i}^{c_i}-{b_i}^{c_i-1})$.}
					\end{enumerate}
				\end{enumerate}
		\declaration{1.20}
			The notation \gls{inta^ulb} will be used as a shorthand for $\prod_i^{[0:b]}(a-i)$.
		\declaration{1.33}
			The notation \gls{inta^olb} will be used as a shorthand for $\prod_i^{[0:b]}(a+i)$.
		\procedure{1.60}
			\objective
				Choose a list of distinct elements $a$ and a non-negative integer $b$ such that $b\le\lvert a\rvert$. Let $c$ be a list of length-$b$ permutations of $a$. The objective of the following instructions is to show that $\lvert c\rvert=\lvert a\rvert^{\ul{b}}$.
			\implementation
				\begin{enumerate}
					\item If $\lvert b\rvert>0$, then do the following:
					\begin{enumerate}
						\item For each entry $d$ in $a$, do the following:
						\begin{enumerate}
							\item Let $e$ be the list formed by removing $d$ from $a$.
							\item Verify that the entries of $e$ are distinct.
							\item Verify that $\lvert e\rvert=\lvert a\rvert-1$.
							\item Now execute \procedurehr{1.60} on $\langle e,b-1\rangle$.
							\item Therefore verify that the number of length-$b-1$ permutations of $e$ is $\lvert e\rvert^{\ul{b-1}}$.
							\item Therefore verify that the number of length-$b$ permutations of $a$ beginning with $d$ is $\lvert e\rvert^{\ul{b-1}}=(\lvert a\rvert-1)^{\ul{b-1}}$.
						\end{enumerate}
						\item Therefore verify that the number of length-$b$ permutations of $a$ beginning with any entry of $a$ is $\lvert a\rvert(\lvert a\rvert-1)^{\ul{b-1}}=\lvert a\rvert^{\ul{b}}$.
						\item Therefore verify that the number of length-$b$ permutations of $a$ are $\lvert a\rvert^{\ul{b}}$.
						\item \textbf{Therefore verify that $\lvert c\rvert=\lvert a\rvert^{\ul{b}}$.}
					\end{enumerate}
					\item Otherwise do the following:
					\begin{enumerate}
						\item Verify that $b=0$.
						\item Verify that the number of length-$0$ permutations of $a$ is $1$.
						\item \textbf{Therefore verify that $\lvert c\rvert=1=\lvert a\rvert^{\ul{0}}=\lvert a\rvert^{\ul{b}}$.}
					\end{enumerate}
				\end{enumerate}
		\declaration{1.21}
			The notation \gls{intnchooser} will be used as a shorthand for $n^{\ul{r}}\div(r!)$.
		\procedure{1.61}
			\objective
				Choose a list of distinct elements $n$ and a non-negative integer $r$ such that $r\le\lvert n\rvert$. Let $b$ be the largest list of length-$r$ sublists of $n$ such that no two of them are permutations of each other. The objective of the following instructions is to either show that $b$ contains at least two permutations of the same list, construct a list larger than $b$ that is also a list of length-$r$ sublists of $n$ such that no two of them are permutations of each other, or to show that $\lvert b\rvert=\binom{\lvert n\rvert}{r}$ and that $\lvert n\rvert^{\ul{r}}\mod r!=0$.
			\implementation
				\begin{enumerate}
					\item Let $a$ and $f$ be a list of all the permutations of $n$.
					\item Using \procedurehr{1.60}, verify that $\lvert a\rvert=\lvert n\rvert^{\ul{\lvert n\rvert}}$.
					\item For each list $c$ in $b$, do the following:
					\begin{enumerate}
						\item Using \procedurehr{1.60}, verify that the number of permutations of $c$ is $r!$.
						\item Let $d$ be the list obtained by removing the elements of $c$ from $n$.
						\item Using \procedurehr{1.60}, verify that the number of permutations of $d$ is $(n-r)!$.
						\item Let $e$ be the list of permutations of $n$ beginning with a permutations of $c$.
						\item Verify that $\lvert e\rvert=\lvert c\rvert\lvert d\rvert=r!(\lvert n\rvert-r)!$.
						\item If $e$ is not a sublist of $a$, then do the following:
						\begin{enumerate}
							\item Let $g$ be a list in $e$ that is not in $a$.
							\item Verify that $e$ is a sublist of $f$.
							\item Therefore verify that $g$ was in $a$ but then was removed.
							\item Therefore verify that the variable $c$ was formerly equal to a permutation of the current $c$.
							\item \textbf{Therefore verify that $b$ contains at least two permutations of $c$.}
							\item \textbf{Abort procedure.}
						\end{enumerate}
						\item Otherwise, do the following:
						\begin{enumerate}
							\item Verify that $e$ is a sublist of $a$.
							\item Remove the lists in $e$ from $a$.
						\end{enumerate}
					\end{enumerate}
					\item If $a\ne\langle\rangle$, then do the following:
					\begin{enumerate}
						\item Let $g$ be a list in $a$.
						\item Let $h$ be the sublist of $g$ corresponding to its first $r$ elements.
						\item Therefore verify that the permutations of $n$ beginning with a permutation of $h$ were never removed from $a$.
						\item Therefore verify that the variable $c$ was never equal to a permutation of $h$.
						\item Therefore verify that no permutation of $h$ is in $b$.
						\item \textbf{Therefore verify that $b^{\frown}\langle h\rangle$ is larger than $b$ and is also a list of length-$r$ sublists of $n$ such that no two of them are permutations of each other.}
						\item \textbf{Abort procedure.}
					\end{enumerate}
					\item Otherwise do the following:
					\begin{enumerate}
						\item Verify that $\lvert n\rvert!\mod(r!(\lvert n\rvert-r)!)=0$.
						\item Therefore verify that $\lvert n\rvert!=(\lvert n\rvert!\div(r!(\lvert n\rvert-r)!))r!(\lvert n\rvert-r)!$.
						\item Therefore verify that $\lvert n\rvert!\div(\lvert n\rvert-r)!=(\lvert n\rvert!\div(r!(\lvert n\rvert-r)!))r!$.
						\item \textbf{Therefore verify that $n^{\ul{r}}\mod r!=(\lvert n\rvert!\div(\lvert n\rvert-r)!)\mod r!=0$.}
						\item Also verify that (3) iterated $\lvert n\rvert!\div(r!(\lvert n\rvert-r)!)$ times.
						\item \textbf{Therefore using \procedurehr{1.08}, verify that $\lvert b\rvert=\lvert n\rvert!\div(r!(\lvert n\rvert-r)!)=(\lvert n\rvert!\div(\lvert n\rvert-r)!)\div(r!)=n^{\ul{r}}\div(r!)=\binom{n}{r}$.}
					\end{enumerate}
				\end{enumerate}
		\procedure{1.62}
			\objective
				Choose two positive integers $a,b$. The objective of the following instructions is to show that $\binom{a}{b}=\binom{a-1}{b-1}+\binom{a-1}{b}$.
			\implementation
				\begin{enumerate}
					\item Using \procedurehr{1.05} and \procedurehr{1.06}, verify that $\binom{a-1}{b-1}+\binom{a-1}{b}$
					\begin{enumerate}
						\item $=(a-1)^{\ul{b-1}}\div(b-1)!+(a-1)^{\ul{b}}\div b!$
						\item $=((a-1)^{\ul{b-1}}b)\div b!+(a-1)^{\ul{b}}\div b!$
						\item $=((a-1)^{\ul{b-1}}b+(a-1)^{\ul{b}})\div b!$
						\item $=((a-1)^{\ul{b-1}}b+(a-1)^{\ul{b-1}}(a-b))\div b!$
						\item $=((a-1)^{\ul{b-1}}a)\div b!$
						\item $=a^{\ul{b}}\div b!$
						\item $=\binom{a}{b}$.
					\end{enumerate}
				\end{enumerate}
		\procedure{1.63}
			\objective
				Choose an integer $x$ and a non-negative integer $a$. The objective of the following instructions is to show that the $(1+x)^a=\sum_r^{[0:{a+1}]}\binom{a}{r}x^r$.
			\implementation
				\begin{enumerate}
					\item If $a=0$, then do the following:
					\begin{enumerate}
						\item \textbf{Verify that $(1+x)^a=(1+x)^0=1=\sum_r^{[0:1]}\binom{0}{r}x^r=\sum_r^{[0:{a+1}]}\binom{a}{r}x^r$.}
					\end{enumerate}
					\item Otherwise, do the following:
					\begin{enumerate}
						\item Verify that $a>0$.
						\item Therefore verify that $a-1\ge 0$.
						\item Execute \procedurehr{1.63} on $\langle x,a-1\rangle$.
						\item Therefore verify that $(1+x)^{a-1}=\sum_r^{[0:{a}]}\binom{a-1}{r}x^r$.
						\item Therefore using \procedurehr{1.62}, verify that $(1+x)^a$
						\begin{enumerate}
							\item $=(1+x)(1+x)^{a-1}$
							\item $=(1+x)\sum_r^{[0:{a}]}\binom{a-1}{r}x^r$
							\item $=\sum_r^{[0:{a}]}\binom{a-1}{r}x^r+\sum_r^{[0:{a}]}\binom{a-1}{r}x^{r+1}$
							\item $=\sum_r^{[0:{a+1}]}\binom{a-1}{r}x^r+\sum_r^{[1:{a+1}]}\binom{a-1}{r-1}x^{r}$
							\item $=1+\sum_r^{[1:{a+1}]}(\binom{a-1}{r}+\binom{a-1}{r-1})x^{r}$
							\item $=1+\sum_r^{[1:{a+1}]}\binom{a}{r}x^{r}$
							\item $=\sum_r^{[0:{a+1}]}\binom{a}{r}x^{r}$.
						\end{enumerate}
					\end{enumerate}
				\end{enumerate}
		\procedure{1.64}
			\objective
				Choose an integer $r$ and a prime $n$ such that $0<r<n$. The objective of the following instructions is to show that either $0\ne 0$ or $\binom{n}{r}\mod n=0$.
			\implementation
				\begin{enumerate}
					\item Using \procedurehr{1.61}, verify that $\binom{n}{r}r!=n^{\ul{r}}\equiv 0(\mod n)$.
					\item If $\binom{n}{r}\mod n\ne 0$, then do the following:
					\begin{enumerate}
						\item Verify that $i\mod n\ne 0$ for $i=1$ to $i=r$.
						\item Therefore using \procedurehr{1.23}, verify that $r!\mod n\ne 0$.
						\item Therefore using (2) and (b), verify that $\binom{n}{r}r!\mod n\ne 0$.
						\item \textbf{Therefore using (1) and (c), verify that $0\ne 0$.}
						\item \textbf{Abort procedure.}
					\end{enumerate}
					\item Otherwise, do the following:
					\begin{enumerate}
						\item \textbf{Verify that $\binom{n}{r}\mod n=0$.}
					\end{enumerate}
				\end{enumerate}
	\twocolumn[\part{Rational Arithmetic}]
		\declaration{2.12}
			The phrase "\gls{rational}" will be used as a shorthand for an ordered pair comprising an integer followed by a non-zero natural number.
		\declaration{2.13}
			The phrase "the numerator of $a$" and the notation \gls{ratNu}, where $a$ is a rational number, will be used as a shorthand for the first entry of $a$.
		\declaration{2.14}
			The phrase "the denominator of $a$" and the notation \gls{ratDe}, where $a$ is a rational number, will be used as a shorthand for the second entry of $a$.
		\declaration{2.15}
			The phrase "\gls{rata=b}", where $a,b$ are rational numbers, will be used as a shorthand for "$\Nu(a)\De(b)=\De(a)\Nu(b)$".
		\procedure{2.27}
			\objective
				Choose a rational number $a$. The objective of the following instructions is to show that $a=a$.
			\implementation
				\begin{enumerate}
					\item Verify that $\Nu(a)\De(a)=\De(a)\Nu(a)$.
					\item \textbf{Hence verify that $a=a$.}
				\end{enumerate}
		\procedure{2.28}
			\objective
				Choose two rational numbers $a,b$ such that $a=b$. The objective of the following instructions is to show that $b=a$.
			\implementation
				\begin{enumerate}
					\item Verify that $\Nu(a)\De(b)=\De(a)\Nu(b)$.
					\item Hence verify that $\Nu(b)\De(a)=\De(b)\Nu(a)$.
					\item \textbf{Hence verify that $b=a$.}
				\end{enumerate}
		\procedure{2.29}
			\objective
				Choose three rational numbers $a,b,c$ such that $a=b$ and $b=c$. The objective of the following instructions is to show that $a=c$.
			\implementation
				\begin{enumerate}
					\item Using \declarationhr{2.15}, verify that $\Nu(a)\De(b)=\De(a)\Nu(b)$.
					\item Using \declarationhr{2.15}, verify that $\Nu(b)\De(c)=\De(b)\Nu(c)$.
					\item If $\Nu(b)\ne 0$, then do the following:
					\begin{enumerate}
						\item Hence verify that $\Nu(a)\De(b)\Nu(b)\De(c)=\De(a)\Nu(b)\De(b)\Nu(c)$.
						\item Hence verify that $\Nu(a)\De(c)=\De(a)\Nu(c)$.
					\end{enumerate}
					\item Otherwise do the following:
					\begin{enumerate}
						\item Using \declarationhr{2.12}, verify that $\De(b)\ne 0$.
						\item Verify that $\Nu(a)\De(b)=\De(a)\Nu(b)=0$.
						\item Hence verify that $\Nu(a)=0$.
						\item Verify that $0=\Nu(b)\De(c)=\De(b)\Nu(c)$.
						\item Hence verify that $\Nu(c)=0$.
						\item Hence verify that $\Nu(a)\De(c)=0=\De(a)\Nu(c)$.
					\end{enumerate}
					\item \textbf{Hence verify that $a=c$.}
				\end{enumerate}
		\declaration{2.16}
			The notation \gls{rata+b}, where $a,b$ are rational numbers, will be used as a shorthand for the pair $\langle\Nu(a)\De(b)+\De(a)\Nu(b),\De(a)\De(b)\rangle$.
		\procedure{2.30}
			\objective
				Choose two rational numbers $a,b,c,d$ such that $a=c$ and $b=d$. The objective of the following instructions is to show that $a+b=c+d$.
			\implementation
				\begin{enumerate}
					\item Using \declarationhr{2.15}, verify that $\Nu(a)\De(c)=\De(a)\Nu(c)$.
					\item Using \declarationhr{2.15}, verify that $\Nu(b)\De(d)=\De(b)\Nu(d)$.
					\item Hence verify that $a+b$
					\begin{enumerate}
						\item $=\langle\Nu(a),\De(a)\rangle+\langle\Nu(b),\De(b)\rangle$
						\item $=\langle\Nu(a)\De(b)+\De(a)\Nu(b),\De(a)\De(b)\rangle$
						\item $=\langle\De(c)\De(d)(\Nu(a)\De(b)+\De(a)\Nu(b)),\De(c)\De(d)(\De(a)\De(b))\rangle$
						\item $=\langle\Nu(a)\De(c)\De(b)\De(d)+\De(a)\De(c)\Nu(b)\De(d),\De(c)\De(d)\De(a)\De(b)\rangle$
						\item $=\langle\De(a)\Nu(c)\De(b)\De(d)+\De(a)\De(c)\De(b)\Nu(d),\De(c)\De(d)\De(a)\De(b)\rangle$
						\item $=\langle\De(a)\De(b)(\Nu(c)\De(d)+\De(c)\Nu(d)),\De(a)\De(b)(\De(c)\De(d))\rangle$
						\item $=\langle\Nu(c)\De(d)+\De(c)\Nu(d),\De(c)\De(d)\rangle$
						\item $=\langle\Nu(c),\De(c)\rangle+\langle\Nu(d),\De(d)\rangle$
						\item $=c+d$.
					\end{enumerate}
				\end{enumerate}
		\procedure{2.31}
			\objective
				Choose three rational numbers $a,b,c$. The objective of the following instructions is to show that $(a+b)+c=a+(b+c)$.
			\implementation
				\begin{enumerate}
					\item Verify that $(a+b)+c$
					\begin{enumerate}
						\item $=\langle\Nu(a)\De(b)+\De(a)\Nu(b),\De(a)\De(b)\rangle+\langle\Nu(c),\De(c)\rangle$
						\item $=\langle(\Nu(a)\De(b)+\De(a)\Nu(b))\De(c)+(\De(a)\De(b))\Nu(c),(\De(a)\De(b))\De(c)\rangle$
						\item $=\langle\Nu(a)(\De(b)\De(c))+\De(a)(\Nu(b)\De(c)+\De(b)\Nu(c)),\De(a)(\De(b)\De(c))\rangle$
						\item $=\langle\Nu(a),\De(a)\rangle+\langle\Nu(b)\De(c)+\De(b)\Nu(c),\De(b)\De(c)\rangle$
						\item $=a+(b+c)$.
					\end{enumerate}
				\end{enumerate}
		\procedure{2.32}
			\objective
				Choose two rational numbers $a,b$. The objective of the following instructions is to show that $a+b=b+a$.
			\implementation
				\begin{enumerate}
					\item $a+b$
					\begin{enumerate}
						\item $=\langle\Nu(a)\De(b)+\De(a)\Nu(b),\De(a)\De(b)\rangle$
						\item $=\langle\Nu(b)\De(a)+\De(b)\Nu(a),\De(b)\Nu(a)\rangle$
						\item $=b+a$.
					\end{enumerate}
				\end{enumerate}
		\declaration{2.17}
			The notation \gls{rata}, where $a$ is an integer, will contextually be used as a shorthand for the pair $\langle a,1\rangle$.
		\procedure{2.33}
			\objective
				Choose a rational number $a$. The objective of the following instructions is to show that $0+a=a$.
			\implementation
				\begin{enumerate}
					\item Verify that $0+a$
					\begin{enumerate}
						\item $=\langle 0,1\rangle+\langle\Nu(a),\De(a)\rangle$
						\item $=\langle 0\De(a)+1\Nu(a),1\De(a)\rangle$
						\item $=\langle\Nu(a),\De(a)\rangle$
						\item $=a$.
					\end{enumerate}
				\end{enumerate}
		\declaration{2.18}
			The notation \gls{rat-a}, where $a$ is a rational number, will be used as a shorthand for the pair $\langle-\Nu(a),\De(a)\rangle$.
		\procedure{2.34}
			\objective
				Choose two rational numbers $a,b$ such that $a=b$. The objective of the following instructions is to show that $-a=-b$.
			\implementation
				\begin{enumerate}
					\item Using \declarationhr{2.15}, verify that $\Nu(a)\De(b)=\De(a)\Nu(b)$.
					\item Hence verify that $-a$
					\begin{enumerate}
						\item $=\langle-\Nu(a),\De(a)\rangle$
						\item $=\langle-\Nu(a)\De(b),\De(a)\De(b)\rangle$
						\item $=\langle-\De(a)\Nu(b),\De(a)\De(b)\rangle$
						\item $=\langle-\Nu(b),\De(b)\rangle$
						\item $=-b$.
					\end{enumerate}
				\end{enumerate}
		\procedure{2.35}
			\objective
				Choose a rational number $a$. The objective of the following instructions is to show that $-a+a=0$.
			\implementation
				\begin{enumerate}
					\item Verify that $-a+a$
					\begin{enumerate}
						\item $=(-a)+a$
						\item $=\langle-\Nu(a),\De(a)\rangle+\langle\Nu(a),\De(a)\rangle$
						\item $=\langle-\Nu(a)\De(a)+\De(a)\Nu(a),\De(a)^2\rangle$
						\item $=\langle 0,\De(a)^2\rangle$
						\item $=\langle 0,1\rangle$
						\item $=0$.
					\end{enumerate}
				\end{enumerate}
		\declaration{2.19}
			The notation \gls{ratab}, where $a,b$ are rational numbers, will be used as a shorthand for the pair $\langle\Nu(a)\Nu(b),\De(a)\De(b)\rangle$.
		\procedure{2.36}
			\objective
				Choose two rational numbers $a,b,c,d$ such that $a=c$ and $b=d$. The objective of the following instructions is to show that $ab=cd$.
			\implementation
				\begin{enumerate}
					\item Using \declarationhr{2.15}, verify that $\Nu(a)\De(c)=\De(a)\Nu(c)$.
					\item Using \declarationhr{2.15}, verify that $\Nu(b)\De(d)=\De(b)\Nu(d)$.
					\item Hence verify that $ab$
					\begin{enumerate}
						\item $=\langle\Nu(a),\De(a)\rangle\langle\Nu(b),\De(b)\rangle$
						\item $=\langle\Nu(a)\Nu(b),\De(a)\De(b)\rangle$
						\item $=\langle(\De(c)\De(d))\Nu(a)\Nu(b),(\De(c)\De(d))\De(a)\De(b)\rangle$
						\item $=\langle(\Nu(a)\De(c))(\Nu(b)\De(d)),\De(c)\De(d)\De(a)\De(b)\rangle$
						\item $=\langle(\De(a)\Nu(c))(\De(b)\Nu(d)),\De(c)\De(d)\De(a)\De(b)\rangle$
						\item $=\langle(\De(a)\De(b))\Nu(c)\Nu(d),(\De(a)\De(b))\De(c)\De(d)\rangle$
						\item $=\langle\Nu(c)\Nu(d),\De(c)\De(d)\rangle$
						\item $=\langle\Nu(c),\De(c)\rangle\langle\Nu(d),\De(d)\rangle$
						\item $=cd$.
					\end{enumerate}
				\end{enumerate}
		\procedure{2.37}
			\objective
				Choose three rational numbers $a,b,c$. The objective of the following instructions is to show that $(ab)c=a(bc)$.
			\implementation
				\begin{enumerate}
					\item Verify that $(ab)c$
					\begin{enumerate}
						\item $=\langle\Nu(a)\Nu(b),\De(a)\De(b)\rangle\langle\Nu(c),\De(c)\rangle$
						\item $=\langle\Nu(a)\Nu(b)\Nu(c),\De(a)\De(b)\De(c)\rangle$
						\item $=\langle\Nu(a),\De(a)\rangle\langle\Nu(b)\Nu(c),\De(b)\De(c)\rangle$
						\item $=a(bc)$.
					\end{enumerate}
				\end{enumerate}
		\procedure{2.38}
			\objective
				Choose two rational numbers $a,b$. The objective of the following instructions is to show that $ab=ba$.
			\implementation
				\begin{enumerate}
					\item $ab$
					\begin{enumerate}
						\item $=\langle\Nu(a)\Nu(b),\De(a)\De(b)\rangle$
						\item $=\langle\Nu(b)\Nu(a),\De(b)\De(a)\rangle$
						\item $=ba$.
					\end{enumerate}
				\end{enumerate}
		\procedure{2.39}
			\objective
				Choose a rational number $a$. The objective of the following instructions is to show that $1a=a$.
			\implementation
				\begin{enumerate}
					\item Verify that $1a$
					\begin{enumerate}
						\item $=\langle 1,1\rangle\langle\Nu(a),\De(a)\rangle$
						\item $=\langle 1\Nu(a),1\De(a)\rangle$
						\item $=\langle\Nu(a),\De(a)\rangle$
						\item $=a$.
					\end{enumerate}
				\end{enumerate}
		\declaration{2.20}
			The notation \gls{rat1/a}, where $a$ is a rational number such that $\Nu(a)>0$, will be used as a shorthand for the pair $\langle\De(a),\Nu(a)\rangle$.
		\declaration{2.21}
			The notation $\frac{1}{a}$, where $a$ is a rational number such that $\Nu(a)<0$, will be used as a shorthand for the pair $\langle-\De(a),-\Nu(a)\rangle$.
		\procedure{2.40}
			\objective
				Choose two rational numbers $a,b$ such that $a=b$ and $a\ne 0$. The objective of the following instructions is to show that $\frac{1}{a}=\frac{1}{b}$.
			\implementation
				\begin{enumerate}
					\item Using \declarationhr{2.15}, verify that $\Nu(a)\De(b)=\De(a)\Nu(b)$.
					\item Using \declarationhr{2.15} and \declarationhr{2.17}, verify that $\Nu(a)=\Nu(a)\De(0)\ne\De(a)\Nu(0)=0$.
					\item Hence using \declarationhr{2.12}, verify that $\De(a)\Nu(b)=\Nu(a)\De(b)\ne 0$.
					\item Hence verify that $\Nu(b)\ne 0$.
					\item If $\Nu(a)\Nu(b)>0$, then do the following:
					\begin{enumerate}
						\item Verify that $\frac{1}{a}$
						\begin{enumerate}
							\item $=\langle\De(a)\Nu(b),\Nu(a)\Nu(b)\rangle$
							\item $=\langle\Nu(a)\De(b),\Nu(a)\Nu(b)\rangle$
							\item $=\frac{1}{b}$.
						\end{enumerate}
					\end{enumerate}
					\item Otherwise do the following:
					\begin{enumerate}
						\item Verify that $\Nu(a)\Nu(b)<0$.
						\item Hence verify that $\frac{1}{a}$
						\begin{enumerate}
							\item $=\langle-\De(a)\Nu(b),-\Nu(a)\Nu(b)\rangle$
							\item $=\langle-\Nu(a)\De(b),-\Nu(a)\Nu(b)\rangle$
							\item $=\frac{1}{b}$.
						\end{enumerate}
					\end{enumerate}
				\end{enumerate}
		\procedure{2.41}
			\objective
				Choose a rational number $a$ such that $a\ne 0$. The objective of the following instructions is to show that $\frac{1}{a}a=1$.
			\implementation
				\begin{enumerate}
					\item Using \declarationhr{2.15} and \declarationhr{2.17}, verify that $\Nu(a)=\Nu(a)\De(0)\ne\De(a)\Nu(0)=0$.
					\item If $\Nu(a)>0$, then do the following:
					\begin{enumerate}
						\item Verify that $\frac{1}{a}a$
						\begin{enumerate}
							\item $=\langle\De(a),\Nu(a)\rangle\langle\Nu(a),\De(a)\rangle$
							\item $=\langle\De(a)\Nu(a),\Nu(a)\De(a)\rangle$
							\item $=\langle 1,1\rangle$
							\item $=1$.
						\end{enumerate}
					\end{enumerate}
					\item Otherwise do the following:
					\begin{enumerate}
						\item Verify that $\Nu(a)<0$.
						\item Hence verify that $\frac{1}{a}a$
						\begin{enumerate}
							\item $=\langle-\De(a),-\Nu(a)\rangle\langle\Nu(a),\De(a)\rangle$
							\item $=\langle-\De(a)\Nu(a),-\Nu(a)\De(a)\rangle$
							\item $=\langle 1,1\rangle$
							\item $=1$.
						\end{enumerate}
					\end{enumerate}
				\end{enumerate}
		\procedure{2.42}
			\objective
				Choose three rational numbers $a,b,c$. The objective of the following instructions is to show that $a(b+c)=ab+ac$.
			\implementation
				\begin{enumerate}
					\item $a(b+c)$
					\begin{enumerate}
						\item $=\langle\Nu(a),\De(a)\rangle\langle\Nu(b)\De(c)+\De(b)\Nu(c),\De(b)\De(c)\rangle$
						\item $=\langle\Nu(a)(\Nu(b)\De(c)+\De(b)\Nu(c)),\De(a)(\De(b)\De(c))\rangle$
						\item $=\langle\Nu(a)\Nu(b)\De(c)+\Nu(a)\De(b)\Nu(c),\De(a)\De(b)\De(c)\rangle$
						\item $=\langle\De(a)(\Nu(a)\Nu(b)\De(c)+\Nu(a)\De(b)\Nu(c)),\De(a)(\De(a)\De(b)\De(c))\rangle$
						\item $=\langle(\Nu(a)\Nu(b))(\De(a)\De(c))+(\De(a)\De(b))(\Nu(a)\Nu(c)),(\De(a)\De(b))(\De(a)\De(c))\rangle$
						\item $=\langle\Nu(a)\Nu(b),\De(a)\De(b)\rangle+\langle\Nu(a)\Nu(c),\De(a)\De(c)\rangle$
						\item $=ab+ac$.
					\end{enumerate}
				\end{enumerate}
		\declaration{2.22}
			The phrase "\gls{rata<b}", where $a,b$ are rational numbers, will be used as a shorthand for "$\Nu(a)\De(b)<\De(a)\Nu(b)$".
		\procedure{2.43}
			\objective
				Choose four rational numbers $a,b,c,d$ such that $a<b$, $a=c$ and $b=d$. The objective of the following instructions is to show that $c<d$.
			\implementation
				\begin{enumerate}
					\item Using \declarationhr{2.15}, verify that $\Nu(a)\De(c)=\De(a)\Nu(c)$.
					\item Using \declarationhr{2.15}, verify that $\Nu(b)\De(d)=\De(b)\Nu(d)$.
					\item Using \declarationhr{2.22}, verify that $\Nu(a)\De(b)<\De(a)\Nu(b)$.
					\item Hence verify that $\Nu(c)\De(d)\De(a)\De(b)$
					\begin{enumerate}
						\item $=\Nu(a)\De(c)\De(d)\De(b)$
						\item $<\De(a)\Nu(b)\De(c)\De(d)$
						\item $=\De(b)\Nu(d)\De(a)\De(c)$.
					\end{enumerate}
					\item Hence verify that $\Nu(c)\De(d)<\De(c)\Nu(d)$.
					\item \textbf{Hence verify that $c<d$.}
				\end{enumerate}
		\procedure{2.44}
			\objective
				Choose three rational numbers $a,b,c$ such that $a<b$. The objective of the following instructions is to show that $a+c<b+c$.
			\implementation
				\begin{enumerate}
					\item Using \declarationhr{2.22}, verify that $\Nu(a)\De(b)<\De(a)\Nu(b)$.
					\item Using \declarationhr{2.12}, verify that $0<\De(c)$.
					\item Hence verify that $\Nu(a+c)\De(b+c)$
					\begin{enumerate}
						\item $=(\Nu(a)\De(c)+\De(a)\Nu(c))\De(b)\De(c)$
						\item $=\Nu(a)\De(c)\De(b)\De(c)+\De(a)\Nu(c)\De(b)\De(c)$
						\item $<\De(a)\De(c)\Nu(b)\De(c)+\De(a)\Nu(c)\De(b)\De(c)$
						\item $=(\Nu(b)\De(c)+\Nu(c)\De(b))\De(a)\De(c)$
						\item $=\Nu(b+c)\De(a+c)$.
					\end{enumerate}
					\item \textbf{Hence verify that $a+c<b+c$.}
				\end{enumerate}
		\procedure{2.45}
			\objective
				Choose two rational numbers $a,b$ such that $a<b$. The objective of the following instructions is to show that $a\ne b$ and $b\not<a$.
			\implementation
				\begin{enumerate}
					\item Verify that $\Nu(a)\De(b)<\De(a)\Nu(b)$.
					\item Hence verify that $\Nu(a)\De(b)\ne\De(a)\Nu(b)$.
					\item \textbf{Hence verify that $a\ne b$.}
					\item Also verify that $\Nu(b)\De(a)\not<\De(b)\Nu(a)$.
					\item \textbf{Hence verify that $b\not<a$.}
				\end{enumerate}
		\procedure{2.46}
			\objective
				Choose two rational numbers $a,b$ such that $a=b$. The objective of the following instructions is to show that $a\not<b$ and $b\not<a$.
			\implementation
				Implementation is analogous to that of \procedurehr{2.45}.
		\procedure{2.47}
			\objective
				Choose two rational numbers $a,b$ such that $a\ne b$. The objective of the following instructions is to show that $a<b$ or $b<a$.
			\implementation
				\begin{enumerate}
					\item Verify that $\Nu(a)\De(b)\ne\De(a)\Nu(b)$.
					\item If $\Nu(a)\De(b)<\De(a)\Nu(b)$, then do the following:
					\begin{enumerate}
						\item \textbf{Verify that $a<b$.}
					\end{enumerate}
					\item Otherwise do the following:
					\begin{enumerate}
						\item Verify that $\Nu(b)\De(a)<\De(b)\Nu(a)$.
						\item \textbf{Hence verify that $b<a$.}
					\end{enumerate}
				\end{enumerate}
		\procedure{2.48}
			\objective
				Choose two rational numbers $a,b$ such that $a\not<b$. The objective of the following instructions is to show that $a=b$ or $b<a$.
			\implementation
				Implementation is analogous to that of \procedurehr{2.47}.
		\procedure{2.49}
			\objective
				Choose two rational numbers $a,b$ such that $0<a$ and $0<b$. The objective of the following instructions is to show that $0<a+b$.
			\implementation
				\begin{enumerate}
					\item Using \declarationhr{2.22}, verify that $0=\Nu(0)\De(a)<\De(0)\Nu(a)=\Nu(a)$.
					\item Using \declarationhr{2.12}, verify that $0<\De(a)$.
					\item Using \declarationhr{2.22}, verify that $0=\Nu(0)\De(b)<\De(0)\Nu(b)=\Nu(b)$.
					\item Using \declarationhr{2.12}, verify that $0<\De(b)$.
					\item Hence verify that $\Nu(0)\De(a+b)=0<\Nu(a)\De(b)+\De(a)\Nu(b)=\De(0)\Nu(a+b)$.
					\item \textbf{Hence verify that $0<a+b$.}
				\end{enumerate}
		\procedure{2.50}
			\objective
				Choose two rational numbers $a,b$ such that $0<a$ and $0<b$. The objective of the following instructions is to show that $0<ab$.
			\implementation
				\begin{enumerate}
					\item Using \declarationhr{2.22}, verify that $0=\Nu(0)\De(a)<\De(0)\Nu(a)=\Nu(a)$.
					\item Using \declarationhr{2.22}, verify that $0=\Nu(0)\De(b)<\De(0)\Nu(b)=\Nu(b)$.
					\item Hence verify that $\Nu(0)\De(ab)=0<\Nu(a)\Nu(b)=\De(0)\Nu(ab)$.
					\item \textbf{Hence verify that $0<ab$.}
				\end{enumerate}
		\procedure{2.81}
			\objective
				Choose two rational numbers $a,b$. The objective of the following instructions is to show that $\lVert ab\rVert=\lVert a\rVert\lVert b\rVert$.
			\implementation
				Implementation is analogous to that of \procedurehr{1.87}.
		\procedure{2.82}
			\objective
				Choose two rational numbers $a,b$. The objective of the following instructions is to show that $\lVert a+b\rVert\le\lVert a\rVert+\lVert b\rVert$.
			\implementation
				Implementation is analogous to that of \procedurehr{1.88}.
		\procedure{2.83}
			\objective
				Choose two rational numbers $a,b$. The objective of the following instructions is to show that $\lVert a\rVert-\lVert b\rVert\le\lVert a-b\rVert$.
			\implementation
				Implementation is analogous to that of \procedurehr{1.89}.
		\procedure{2.84}
			\objective
				Choose a rational number $a$. The objective of the following instructions is to show that $a=\sgn(a)\lVert a\rVert$.
			\implementation
				Implementation is analogous to that of \procedurehr{1.90}.
		\declaration{2.02}
			The notation \gls{ratfloor}, where $a$ is a rational number, will be used as a shorthand for $\Nu(a)\div\De(a)$.
		\declaration{2.03}
			The notation \gls{ratceil}, where $a$ is a rational number, will be used as a shorthand for $(\Nu(a)\div\De(a))+1$.
		\procedure{2.04}
			\objective
				Choose a rational number $r\ne 1$ and an integer $n\ge 0$. The objective of the following instructions is to show that $\sum_t^{[0:n]} r^t=\frac{1-r^n}{1-r}$.
			\implementation
				\begin{enumerate}
					\item Verify that $r\sum_t^{[0:n]} r^t=\sum_t^{[0:n]} r^{t+1}=\sum_t^{[1:{n+1}]} r^t$.
					\item Therefore verify that $(1-r)\sum_t^{[0:n]} r^t=\sum_t^{[0:n]} r^t-\sum_t^{[1:{n+1}]} r^t=1-r^n$.
					\item \textbf{Therefore verify that $\sum_t^{[0:n]} r^t=\frac{1-r^n}{1-r}$.}
				\end{enumerate}
		\procedure{2.05}
			\objective
				Choose a rational $0<r<1$ and an integer $n\ge 0$. The objective of the following instructions is to show that $\sum_t^{[0:n]} r^t<\frac{1}{1-r}$.
			\implementation
				\begin{enumerate}
					\item \textbf{Using \procedurehr{2.04}, verify that $\sum_t^{[0:n]} r^t=\frac{1-r^n}{1-r}<\frac{1}{1-r}$.}
				\end{enumerate}
		\procedure{2.06}
			\objective
				Choose a non-negative integer $a$ and a rational number $x$. The objective of the following instructions is to show that $(1+x)^a=\sum_r^{[0:{a+1}]}\binom{a}{r}x^r$.
			\implementation
				Instructions are analogous to those of \procedurehr{1.63}.
		\procedure{2.07}
			\objective
				Choose an integer $r\ge 0$ and a rational number $x\ge -1$. The objective of the following instructions is to show that $(1+x)^r\ge 1+rx$.
			\implementation
				\begin{enumerate}
					\item If $-1\le x<0$, then do the following:
					\begin{enumerate}
						\item Using \procedurehr{2.04}, verify that $(1+x)^r$
						\begin{enumerate}
							\item $=1+(1+x)^r-1$
							\item $=1+x\frac{(1+x)^r-1}{(1+x)-1}$
							\item $=1+x\sum_k^{[0:r]} (1+x)^k$
							\item $\ge 1+x\sum_k^{[0:r]} 1$
							\item $=1+rx$.
						\end{enumerate}
					\end{enumerate}
					\item Otherwise, do the following:
					\begin{enumerate}
						\item Verify that $x\ge 0$.
						\item Using \procedurehr{2.06}, verify that $(1+x)^r$
						\begin{enumerate}
							\item $=\sum_k^{[0:{r+1}]}\binom{r}{k}x^k$
							\item $\ge \binom{r}{0}x^0+\binom{r}{1}x^1$
							\item $=1+rx$
						\end{enumerate}
					\end{enumerate}
				\end{enumerate}
		\declaration{2.08}
			The notation \gls{ratminc}, where $c$ is a list, will be used as a shorthand for $\infty$ if $c$ is empty, otherwise it will stand for the minimum entry of $c$.
		\declaration{2.23}
			The notation \gls{ratmincr}, where $R$ is a list and $c[r]$ is a function of $r$, will be used as a shorthand for $\min(c(R))$.
		\declaration{2.11}
			The notation \gls{ratmaxc}, where $c$ is a list, will be used as a shorthand for $-\infty$ if $c$ is empty, otherwise it will stand for the maximum entry of $c$.
		\declaration{2.24}
			The notation \gls{ratmaxcr}, where $R$ is a list and $c[r]$ is a function of $r$, will be used as a shorthand for $\max(c(R))$.
		\declaration{2.25}
			The phrase "\gls{ratpolynomial}" will be used as a shorthand for a list of rational numbers.
		\declaration{2.26}
			The notation \gls{ratpolya_i}, where $a$ is a polynomial and $i$ is a natural number such that $i\ge\lvert a\rvert$, will be used as a shorthand for $0$.
		\declaration{2.27}
			The phrase "\gls{ratpolya=b}", where $a,b$ are polynomials, will be used as a shorthand for "$a_i=b_i$ for each $i\in[0:\max(\lvert a\rvert,\lvert b\rvert)]$".
		\declaration{2.28}
			The notation \gls{ratpolyLambda} will be used as a shorthand for $\sum_r^{[0:\lvert a\rvert]}a_rb^r$.
		\procedure{2.51}
			\objective
				Choose two polynomials $a,b$ and a rational number $c$ such that $a=b$. The objective of the following instructions is to show that $\Lambda(a,c)=\Lambda(b,c)$.
			\implementation
				\begin{enumerate}
					\item Verify that $\Lambda(a,c)$
					\begin{enumerate}
						\item $=\sum_r^{[0:\lvert a\rvert]}a_rc^r$
						\item $=\sum_r^{[0:\max(\lvert a\rvert,\lvert b\rvert)]}a_rc^r$
						\item $=\sum_r^{[0:\max(\lvert a\rvert,\lvert b\rvert)]}b_rc^r$
						\item $=\sum_r^{[0:\lvert b\rvert]}b_rc^r$
						\item $=\Lambda(b,c)$.
					\end{enumerate}
				\end{enumerate}
		\procedure{2.52}
			\objective
				Choose a natural number $c$ and two polynomials $a,b$ such that $a=b$. The objective of the following instructions is to show that $a_c=b_c$.
			\implementation
				\begin{enumerate}
					\item If $c<\max(\lvert a\rvert,\lvert b\rvert)$, then do the following:
					\begin{enumerate}
						\item \textbf{Verify that $a_c=b_c$.}
					\end{enumerate}
					\item Otherwise do the following:
					\begin{enumerate}
						\item Verify that $c\ge\max(\lvert a\rvert,\lvert b\rvert)$.
						\item \textbf{Hence verify that $a_c=0=b_c$.}
					\end{enumerate}
				\end{enumerate}
		\procedure{2.53}
			\objective
				Choose a polynomial $a$. The objective of the following instructions is to show that $a=a$.
			\implementation
				\begin{enumerate}
					\item Verify that $a_i=a_i$ for each $i\in[0:\max(\lvert a\rvert,\lvert a\rvert)]$.
					\item \textbf{Hence verify that $a=a$.}
				\end{enumerate}
		\procedure{2.54}
			\objective
				Choose two polynomials $a,b$ such that $a=b$. The objective of the following instructions is to show that $b=a$.
			\implementation
				\begin{enumerate}
					\item Verify that $a_i=b_i$ for each $i\in[0:\max(\lvert a\rvert,\lvert b\rvert)]$.
					\item Hence verify that $b_i=a_i$ for each $i\in[0:\max(\lvert b\rvert,\lvert a\rvert)]$.
					\item \textbf{Hence verify that $b=a$.}
				\end{enumerate}
		\procedure{2.55}
			\objective
				Choose three polynomials $a,b,c$ such that $a=b$ and $b=c$. The objective of the following instructions is to show that $a=c$.
			\implementation
				\begin{enumerate}
					\item Using \declarationhr{2.27}, verify that $a_i=b_i$ for each $i\in[0:\max(\lvert a\rvert,\lvert b\rvert,\lvert c\rvert)]$.
					\item Using \declarationhr{2.27}, verify that $b_i=c_i$ for each $i\in[0:\max(\lvert a\rvert,\lvert b\rvert,\lvert c\rvert)]$.
					\item Hence verify that $a_i=c_i$ for each $i\in[0:\max(\lvert a\rvert,\lvert b\rvert,\lvert c\rvert)]$.
					\item \textbf{Hence verify that $a=c$.}
				\end{enumerate}
		\declaration{2.37}
			The notation \gls{ratlistcomp}, where $f[j]$ is a function of $j$ and $R$ is a list, will be used as a shorthand for $\langle f(R)\rangle$.
		\declaration{2.29}
			The notation \gls{ratpolya+b}, where $a,b$ are polynomials, will be used as a shorthand for the list $\langle a_i+b_i\for i\in[0:\max(\lvert a\rvert,\lvert b\rvert)]\rangle$.
		\procedure{2.56}
			\objective
				Choose two polynomials $a,b$ and a rational number $c$. The objective of the following instructions is to show that $\Lambda(a+b,c)=\Lambda(a,c)+\Lambda(b,c)$.
			\implementation
				\begin{enumerate}
					\item Verify that $\Lambda(a+b,c)$
					\begin{enumerate}
						\item $=\Lambda(\langle a_r+b_r\for r\in[0:\max(\lvert a\rvert,\lvert b\rvert)]\rangle,c)$
						\item $=\sum_r^{[0:\max(\lvert a\rvert,\lvert b\rvert)]}(a_r+b_r)c^r$
						\item $=\sum_r^{[0:\max(\lvert a\rvert,\lvert b\rvert)]}a_rc^r+\sum_r^{[0:\max(\lvert a\rvert,\lvert b\rvert)]}b_rc^r$
						\item $=\sum_r^{[0:\lvert a\rvert]}a_rc^r+\sum_r^{[0:\lvert b\rvert]}b_rc^r$
						\item $=\Lambda(a,c)+\Lambda(b,c)$.
					\end{enumerate}
				\end{enumerate}
		\procedure{2.57}
			\objective
				Choose a natural number $c$ and two polynomials $a,b$. The objective of the following instructions is to show that $(a+b)_c=a_c+b_c$.
			\implementation
				\begin{enumerate}
					\item If $c<\max(\lvert a\rvert,\lvert b\rvert)$, then do the following:
					\begin{enumerate}
						\item \textbf{Verify that $(a+b)_c=a_c+b_c$.}
					\end{enumerate}
					\item Otherwise do the following:
					\begin{enumerate}
						\item Verify that $c\ge\max(\lvert a\rvert,\lvert b\rvert)$.
						\item Hence verify that $a_c=0$.
						\item Also verify that $b_c=0$.
						\item Also verify that $(a+b)_c=0$.
						\item \textbf{Hence verify that $(a+b)_c=a_c+b_c$.}
					\end{enumerate}
				\end{enumerate}
		\procedure{2.58}
			\objective
				Choose four polynomials $a,b,c,d$ such that $a=c$ and $b=d$. The objective of the following instructions is to show that $a+b=c+d$.
			\implementation
				\begin{enumerate}
					\item Verify that $a_i=c_i$ for each $i\in[0:\max(\lvert a\rvert,\lvert b\rvert,\lvert c\rvert,\lvert d\rvert)]$.
					\item Verify that $b_i=d_i$ for each $i\in[0:\max(\lvert a\rvert,\lvert b\rvert,\lvert c\rvert,\lvert d\rvert)]$.
					\item Hence verify that $a+b$
					\begin{enumerate}
						\item $=\langle a_i+b_i\for i\in[0:\max(\lvert a\rvert,\lvert b\rvert,\lvert c\rvert,\lvert d\rvert)]\rangle$
						\item $=\langle c_i+d_i\for i\in[0:\max(\lvert a\rvert,\lvert b\rvert,\lvert c\rvert,\lvert d\rvert)]\rangle$
						\item $=c+d$.
					\end{enumerate}
				\end{enumerate}
		\procedure{2.59}
			\objective
				Choose three polynomials $a,b,c$. The objective of the following instructions is to show that $(a+b)+c=a+(b+c)$.
			\implementation
				\begin{enumerate}
					\item Verify that $(a+b)+c$
					\begin{enumerate}
						\item $\langle(a+b)_i+c_i\for i\in[0:\max(\lvert a+b\rvert,\lvert c\rvert)]\rangle$
						\item $\langle(a_i+b_i)+c_i\for i\in[0:\max(\lvert a\rvert,\lvert b\rvert,\lvert c\rvert)]\rangle$
						\item $\langle a_i+(b_i+c_i)\for i\in[0:\max(\lvert a\rvert,\lvert b+c\rvert)]\rangle$
						\item $\langle a_i+(b+c)_i\for i\in[0:\max(\lvert a\rvert,\lvert b+c\rvert)]\rangle$
						\item $=a+(b+c)$.
					\end{enumerate}
				\end{enumerate}
		\procedure{2.60}
			\objective
				Choose two polynomials $a,b$. The objective of the following instructions is to show that $a+b=b+a$.
			\implementation
				\begin{enumerate}
					\item Verify that $a+b$
					\begin{enumerate}
						\item $=\langle a_i+b_i\for i\in[0:\max(\lvert a\rvert,\lvert b\rvert)]\rangle$
						\item $=\langle b_i+a_i\for i\in[0:\max(\lvert b\rvert,\lvert a\rvert)]\rangle$
						\item $=b+a$.
					\end{enumerate}
				\end{enumerate}
		\declaration{2.30}
			The notation \gls{ratpolya}, where $a$ is a rational number, will contextually be used as a shorthand for the list $\langle a\rangle$.
		\procedure{2.61}
			\objective
				Choose a polynomial $a$. The objective of the following instructions is to show that $0+a=a$.
			\implementation
				\begin{enumerate}
					\item Verify that $0+a$
					\begin{enumerate}
						\item $=\langle 0_i+a_i\for i\in[0:\lvert a\rvert]\rangle$
						\item $=\langle 0+a_i\for i\in[0:\lvert a\rvert]\rangle$
						\item $=a$.
					\end{enumerate}
				\end{enumerate}
		\declaration{2.31}
			The notation \gls{ratpoly-a}, where $a$ is a polynomial, will be used as a shorthand for the list $\langle-a_i\for i\in[0:\lvert a\rvert]\rangle$.
		\procedure{2.00}
			\objective
				Choose a polynomial $a$ and a rational number $b$. The objective of the following instructions is to show that $\Lambda(-a,b)=-\Lambda(a,b)$.
			\implementation
				\begin{enumerate}
					\item Verify that $\Lambda(-a,b)$
					\begin{enumerate}
						\item $=\Lambda(\langle-a_i\for i\in[0:\lvert a\rvert]\rangle,b)$
						\item $=\sum_j^{[0:\lvert a\rvert]}(-a_j)b^j$
						\item $=-\sum_j^{[0:\lvert a\rvert]}a_jb^j$
						\item $=-\Lambda(a,b)$.
					\end{enumerate}
				\end{enumerate}
		\procedure{2.62}
			\objective
				Choose two polynomials $a,b$ such that $a=b$. The objective of the following instructions is to show that $-a=-b$.
			\implementation
				\begin{enumerate}
					\item Verify that $a_i=b_i$ for $i\in[0:\max(\lvert a\rvert,\lvert b\rvert)]$.
					\item Hence verify that $-a$
					\begin{enumerate}
						\item $=\langle-a_i\for i\in[0:\max(\lvert a\rvert,\lvert b\rvert)]\rangle$
						\item $=\langle-b_i\for i\in[0:\max(\lvert a\rvert,\lvert b\rvert)]\rangle$
						\item $=-b$.
					\end{enumerate}
				\end{enumerate}
		\procedure{2.63}
			\objective
				Choose a polynomial $a$. The objective of the following instructions is to show that $-a+a=0$.
			\implementation
				\begin{enumerate}
					\item Verify that $-a+a$
					\begin{enumerate}
						\item $=(-a)+a$
						\item $=\langle-a_i\for i\in[0:\lvert a\rvert]\rangle+\langle a_i\for i\in[0:\lvert a\rvert]\rangle$
						\item $=\langle-a_i+a_i\for i\in[0:\lvert a\rvert]\rangle$
						\item $=\langle 0\for i\in[0:\lvert a\rvert]\rangle$
						\item $=0$.
					\end{enumerate}
				\end{enumerate}
		\declaration{2.32}
			The notation \gls{ratpolyab}, where $a,b$ are integers, will be used as a shorthand for the list $\langle\sum_r^{[0:i+1]}a_rb_{i-r}\for i\in[0:\lvert a\rvert+\lvert b\rvert-1]\rangle$.
		\procedure{2.64}
			\objective
				Choose two polynomials $a,b$ and a rational number $c$. The objective of the following instructions is to show that $\Lambda(ab,c)=\Lambda(a,c)\Lambda(b,c)$.
			\implementation
				\begin{enumerate}
					\item Verify that $\Lambda(ab,c)$
					\begin{enumerate}
						\item $=\Lambda(\langle\sum_r^{[0:j+1]}a_rb_{j-r}\for j\in[0:\lvert a\rvert+\lvert b\rvert-1]\rangle,c)$
						\item $=\sum_j^{[0:\lvert a\rvert+\lvert b\rvert-1]}(\sum_r^{[0:j+1]}a_rb_{j-r})c^j$
						\item $=\sum_j^{[0:\lvert a\rvert+\lvert b\rvert-1]}\sum_r^{[0:j+1]}a_rc^rb_{j-r}c^{j-r}$
						\item $=\sum_r^{[0:\lvert a\rvert+\lvert b\rvert-1]}\sum_j^{[r:\lvert a\rvert+\lvert b\rvert-1]}a_rc^rb_{j-r}c^{j-r}$
						\item $=\sum_r^{[0:\lvert a\rvert+\lvert b\rvert-1]}a_rc^r\sum_j^{[r:\lvert a\rvert+\lvert b\rvert-1]}b_{j-r}c^{j-r}$
						\item $=\sum_r^{[0:\lvert a\rvert+\lvert b\rvert-1]}a_rc^r\sum_j^{[0:\lvert a\rvert+\lvert b\rvert-1-r]}b_{j}c^{j}$
						\item $=\sum_r^{[0:\lvert a\rvert]}a_rc^r\sum_j^{[0:\lvert a\rvert+\lvert b\rvert-1-r]}b_{j}c^{j}$
						\item $=\sum_r^{[0:\lvert a\rvert]}a_rc^r\sum_j^{[0:\lvert b\rvert]}b_{j}c^{j}$
						
						\item $=(\sum_j^{[0:\lvert a\rvert]}a_jc^j)(\sum_j^{[0:\lvert b\rvert]}b_jc^j)$
						\item $=\Lambda(a,c)\Lambda(b,c)$.
					\end{enumerate}
				\end{enumerate}
		\procedure{2.65}
			\objective
				Choose a natural number $c$ and two polynomials $a,b$. The objective of the following instructions is to show that $(ab)_c=\sum_r^{[0:c+1]}a_rb_{c-r}$.
			\implementation
				\begin{enumerate}
					\item If $c<\lvert a\rvert+\lvert b\rvert-1$, then do the following:
					\begin{enumerate}
						\item \textbf{Verify that $(ab)_c=\sum_r^{[0:c+1]}a_rb_{c-r}$.}
					\end{enumerate}
					\item Otherwise do the following:
					\begin{enumerate}
						\item Verify that $c\ge\lvert a\rvert+\lvert b\rvert-1$.
						\item Verify that $(ab)_c$
						\begin{enumerate}
							\item $=0$
							\item $=\sum_r^{[0:\lvert a\rvert]}0a_r+\sum_r^{[\lvert a\rvert:c+1]}0b_{c-r}$
							\item $=\sum_r^{[0:\lvert a\rvert]}a_rb_{c-r}+\sum_r^{[\lvert a\rvert:c+1]}a_rb_{c-r}$
							\item $=\sum_r^{[0:c+1]}a_rb_{c-r}$.
						\end{enumerate}
					\end{enumerate}
				\end{enumerate}
		\procedure{2.66}
			\objective
				Choose four polynomials $a,b,c,d$ such that $a=c$ and $b=d$. The objective of the following instructions is to show that $ab=cd$.
			\implementation
				\begin{enumerate}
					\item Using \declarationhr{2.27}, verify that $a_i=c_i$ for $i\in[0:\max(\lvert a\rvert,\lvert c\rvert)+\max(\lvert b\rvert,\lvert d\rvert)-1]$.
					\item Using \declarationhr{2.27}, verify that $b_i=d_i$ for $i\in[0:\max(\lvert a\rvert,\lvert c\rvert)+\max(\lvert b\rvert,\lvert d\rvert)-1]$.
					\item Hence verify that $ab$
					\begin{enumerate}
						\item $=\langle\sum_r^{[0:i+1]}a_rb_{i-r}\for i\in[0:\max(\lvert a\rvert,\lvert c\rvert)+\max(\lvert b\rvert,\lvert d\rvert)-1]\rangle$
						\item $=\langle\sum_r^{[0:i+1]}c_rd_{i-r}\for i\in[0:\max(\lvert a\rvert,\lvert c\rvert)+\max(\lvert b\rvert,\lvert d\rvert)-1]\rangle$
						\item $=cd$.
					\end{enumerate}
				\end{enumerate}
		\procedure{2.67}
			\objective
				Choose three polynomials $a,b,c$. The objective of the following instructions is to show that $(ab)c=a(bc)$.
			\implementation
				\begin{enumerate}
					\item Verify that $(ab)c$
					\begin{enumerate}
						\item $=\langle\sum_t^{[0:j+1]}(ab)_tc_{j-t}\for j\in[0:\lvert ab\rvert+\lvert c\rvert-1]\rangle$
						\item $=\langle\sum_t^{[0:j+1]}\langle\sum_r^{[0:i+1]}a_rb_{i-r}\for i\in[0:\lvert a\rvert+\lvert b\rvert-1]\rangle_tc_{j-t}\for j\in[0:\lvert a\rvert+\lvert b\rvert+\lvert c\rvert-2]\rangle$
						\item $=\langle\sum_t^{[0:j+1]}\sum_r^{[0:t+1]}a_rb_{t-r}c_{j-t}\for j\in[0:\lvert a\rvert+\lvert b\rvert+\lvert c\rvert-2]\rangle$
						\item $=\langle\sum_r^{[0:j+1]}\sum_t^{[r:j+1]}a_rb_{t-r}c_{j-t}\for j\in[0:\lvert a\rvert+\lvert b\rvert+\lvert c\rvert-2]\rangle$
						\item $=\langle\sum_r^{[0:j+1]}a_r\sum_t^{[r:j+1]}b_{t-r}c_{j-t}\for j\in[0:\lvert a\rvert+\lvert b\rvert+\lvert c\rvert-2]\rangle$
						\item $=\langle\sum_r^{[0:j+1]}a_r\sum_t^{[0:j-r+1]}b_{t}c_{j-r-t}\for j\in[0:\lvert a\rvert+\lvert b\rvert+\lvert c\rvert-2]\rangle$
						\item $=\langle\sum_r^{[0:j+1]}a_r\langle\sum_t^{[0:i+1]}b_{t}c_{i-t}\for i\in[0:\lvert b\rvert+\lvert c\rvert-1]\rangle_{j-r}\for j\in[0:\lvert a\rvert+\lvert b\rvert+\lvert c\rvert-2]\rangle$
						\item $=\langle\sum_r^{[0:j+1]}a_r(bc)_{j-r}\for j\in[0:\lvert a\rvert+\lvert bc\rvert-1]\rangle$
						\item $=a(bc)$.
					\end{enumerate}
				\end{enumerate}
		\procedure{2.68}
			\objective
				Choose two polynomials $a,b$. The objective of the following instructions is to show that $ab=ba$.
			\implementation
				\begin{enumerate}
					\item $ab$
					\begin{enumerate}
						\item $=\langle\sum_r^{[0:i+1]}a_rb_{i-r}\for i\in[0:\lvert a\rvert+\lvert b\rvert-1]\rangle$
						\item $=\langle\sum_r^{[0:i+1]}b_ra_{i-r}\for i\in[0:\lvert a\rvert+\lvert b\rvert-1]\rangle$
						\item $=ba$.
					\end{enumerate}
				\end{enumerate}
		\procedure{2.69}
			\objective
				Choose a polynomial $a$. The objective of the following instructions is to show that $1a=a$.
			\implementation
				\begin{enumerate}
					\item Verify that $1a$
					\begin{enumerate}
						\item $=\langle\sum_r^{[0:i+1]}1_ra_{i-r}\for i\in[0:\lvert 1\rvert+\lvert a\rvert-1]\rangle$
						\item $=\langle 1_0a_{i-0}\for i\in[0:\lvert a\rvert]\rangle$
						\item $=\langle a_{i}\for i\in[0:\lvert a\rvert]\rangle$
						\item $=a$.
					\end{enumerate}
				\end{enumerate}
		\procedure{2.70}
			\objective
				Choose three polynomials $a,b,c$. The objective of the following instructions is to show that $a(b+c)=ab+ac$.
			\implementation
				\begin{enumerate}
					\item $a(b+c)$
					\begin{enumerate}
						\item $=\langle\sum_r^{[0:i+1]}a_r(b+c)_{i-r}\for i\in[0:\lvert a\rvert+\lvert b+c\rvert-1]\rangle$
						\item $=\langle\sum_r^{[0:i+1]}a_r(b_{i-r}+c_{i-r})\for i\in[0:\lvert a\rvert+\lvert b+c\rvert-1]\rangle$
						\item $=\langle\sum_r^{[0:i+1]}(a_rb_{i-r}+a_rc_{i-r})\for i\in[0:\lvert a\rvert+\lvert b+c\rvert-1]\rangle$
						\item $=\langle\sum_r^{[0:i+1]}a_rb_{i-r}+\sum_r^{[0:i+1]}a_rc_{i-r}\for i\in[0:\lvert a\rvert+\lvert b+c\rvert-1]\rangle$
						\item $=\langle\sum_r^{[0:i+1]}a_rb_{i-r}\for i\in[0:\lvert a\rvert+\lvert b\rvert-1]\rangle+\langle\sum_r^{[0:i+1]}a_rc_{i-r}\for i\in[0:\lvert a\rvert+\lvert c\rvert-1]\rangle$
						\item $=ab+ac$.
					\end{enumerate}
				\end{enumerate}
		\declaration{2.33}
			The notation \gls{ratpolylambda} will be used as a shorthand for the list $\langle 0,1\rangle$.
		\procedure{2.71}
			\objective
				Choose a polynomial $a$. The objective of the following instructions is to show that $\lambda a=\langle 0\rangle^{\frown}a$.
			\implementation
				\begin{enumerate}
					\item Verify that $\lvert\lambda a\rvert=\lvert\lambda\rvert+\lvert a\rvert-1=\lvert a\rvert+1$.
					\item For $j\in[1:\lvert a\rvert+1]$, do the following:
					\begin{enumerate}
						\item Verify that $(\lambda a)_j$
						\begin{enumerate}
							\item $=\sum_r^{[0:j+1]}\lambda_ra_{j-r}$
							\item $=\sum_r^{[0:j+1]}[r=1]a_{j-r}$
							\item $=a_{j-1}$
						\end{enumerate}
					\end{enumerate}
					\item Verify that $(\lambda a)_0=\sum_r^{[0:1]}\lambda_ra_{0-r}=\lambda_0a_{0}=0$.
					\item \textbf{Hence verify that $\lambda a=\langle 0\rangle^{\frown}a$.}
				\end{enumerate}
		\procedure{2.72}
			\objective
				Choose a natural number $n$. The objective of the following instructions is to show that $\lambda^n=\langle[j=n]\for j\in[0:n+1]\rangle$.
			\implementation
				\begin{enumerate}
					\item If $n=0$, then do the following:
					\begin{enumerate}
						\item Verify that $\lambda^n$
						\begin{enumerate}
							\item $=\lambda^0$
							\item $=\langle 1\rangle$
							\item $=\langle[j=0]\for j\in[0:1]\rangle$
							\item $=\langle[j=n]\for j\in[0:n+1]\rangle$.
						\end{enumerate}
					\end{enumerate}
					\item Otherwise do the following:
					\begin{enumerate}
						\item Execute \procedurehr{2.25} on $\langle n-1\rangle$.
						\item Hence verify that $\lambda^{n-1}=\langle[j=n-1]\for j\in[0:n]\rangle$.
						\item Hence verify that $\lambda^n$
						\begin{enumerate}
							\item $=\lambda\lambda^{n-1}$
							\item $=\lambda\langle[j=n-1]\for j\in[0:n]\rangle$
							\item $=\langle 0\rangle^{\frown}\langle[j=n-1]\for j\in[0:n]\rangle$
							\item $=\langle[j=n]\for j\in[0:n+1]\rangle$.
						\end{enumerate}
					\end{enumerate}
				\end{enumerate}
		\declaration{2.34}
			The notation \gls{ratpolydeg}, where $a$ is a polynomial such that $a\ne 0$, will be used as a shorthand for the largest natural number $j<\lvert a\rvert$ such that $a_j\ne 0$.
		\procedure{2.73}
			\objective
				Choose two polynomials $a,b$ such that $a=b$ and $a\ne 0$. The objective of the following instructions is to show that $\deg(a)=\deg(b)$.
			\implementation
				\begin{enumerate}
					\item For $j\in[\max(\lvert a\rvert,\lvert b\rvert):0]$, do the following:
					\begin{enumerate}
						\item If $a_j=0$, then do the following:
						\begin{enumerate}
							\item Verify that $0=a_j=b_j$.
						\end{enumerate}
						\item Otherwise do the following:
						\begin{enumerate}
							\item Verify that $0\ne a_j=b_j$.
							\item Hence verify that $j<\min(\lvert a\rvert,\lvert b\rvert)$.
							\item \textbf{Hence verify that $\deg(a)=j=\deg(b)$.}
						\end{enumerate}
					\end{enumerate}
				\end{enumerate}
		\procedure{2.74}
			\objective
				Let $\deg(0)=-1$. Choose two polynomials $a,b$ such that $\deg(a)<\deg(b)$. The objective of the following instructions is to show that $\deg(a+b)=\deg(b)$.
			\implementation
				\begin{enumerate}
					\item For $j\in[\max(\lvert a\rvert,\lvert b\rvert):\deg(b)+1]$, do the following:
					\begin{enumerate}
						\item Verify that $j>\deg(b)>\deg(a)$.
						\item Hence verify that $a_j=b_j=0$.
						\item Hence verify that $(a+b)_j=a_j+b_j=0$.
					\end{enumerate}
					\item Verify that $\deg(b)>\deg(a)$.
					\item Hence verify that $(a+b)_{\deg(b)}=a_{\deg(b)}+b_{\deg(b)}=0+b_{\deg(b)}=b_{\deg(b)}\ne 0$.
					\item \textbf{Hence verify that $\deg(a+b)=\deg(b)$.}
				\end{enumerate}
		\procedure{2.75}
			\objective
				Let $\deg(0)=-1$. Choose two polynomials $a,b$. The objective of the following instructions is to show that $\deg(a+b)\le\max(\deg(a),\deg(b))$.
			\implementation
				\begin{enumerate}
					\item For $j\in[\max(\lvert a\rvert,\lvert b\rvert):\max(\deg(a),\deg(b))+1]$, do the following:
					\begin{enumerate}
						\item Verify that $j>\deg(a)$.
						\item Verify that $j>\deg(b)$.
						\item Hence verify that $a_j=b_j=0$.
						\item Hence verify that $(a+b)_j=a_j+b_j=0$.
					\end{enumerate}
					\item \textbf{Hence verify that $\deg(a+b)\le\max(\deg(a),\deg(b))$.}
				\end{enumerate}
		\procedure{2.76}
			\objective
				Let $\deg(0)=-1$. Choose a polynomial $a$. The objective of the following instructions is to show that $\deg(-a)=\deg(a)$.
			\implementation
				\begin{enumerate}
					\item For $j\in[\lvert a\rvert:\deg(a)+1]$, do the following:
					\begin{enumerate}
						\item Verify that $j>\deg(a)$.
						\item Hence verify that $a_j=0$.
						\item Hence verify that $(-a)_j=-(a_j)=-0=0$.
					\end{enumerate}
					\item Verify that $a_{\deg(a)}\ne 0$.
					\item Hence verify that $(-a)_{\deg(a)}=-(a_{\deg(a)})\ne 0$.
					\item \textbf{Hence verify that $\deg(-a)=\deg(a)$.}
				\end{enumerate}
		\procedure{2.77}
			\objective
				Choose two polynomials $a,b$ such that $a\ne 0$ and $b\ne 0$. The objective of the following instructions is to show that $(ab)_{\deg(a)+\deg(b)}=a_{\deg(a)}b_{\deg(b)}\ne 0$.
			\implementation
				\begin{enumerate}
					\item Verify that $a_{\deg(a)}\ne 0$.
					\item Verify that $b_{\deg(b)}\ne 0$.
					\item Hence verify that $(ab)_{\deg(a)+\deg(b)}$
					\begin{enumerate}
						\item $=\sum_r^{[0:\deg(a)+\deg(b)+1]}a_rb_{\deg(a)+\deg(b)-r}$
						\item $=\sum_r^{[0:\deg(a)]}a_rb_{\deg(a)+\deg(b)-r}+a_{\deg(a)}b_{\deg(a)+\deg(b)-\deg(a)}+\sum_r^{[\deg(a)+1:\deg(a)+\deg(b)+1]}a_rb_{\deg(a)+\deg(b)-r}$
						\item $=\sum_r^{[0:\deg(a)]}0a_r+a_{\deg(a)}b_{\deg(b)}+\sum_r^{[\deg(a)+1:\deg(a)+\deg(b)+1]}0b_{\deg(a)+\deg(b)-r}$
						\item $=a_{\deg(a)}b_{\deg(b)}$
						\item $\ne 0$.
					\end{enumerate}
				\end{enumerate}
		\procedure{2.78}
			\objective
				Choose two polynomials $a,b$ such that $a\ne 0$ and $b\ne 0$. The objective of the following instructions is to show that $\deg(ab)=\deg(a)+\deg(b)$.
			\implementation
				\begin{enumerate}
					\item For $j\in[\deg(a)+\deg(b)+1:\lvert a\rvert+\lvert b\rvert-1]$, do the following:
					\begin{enumerate}
						\item Verify that $(ab)_j$
						\begin{enumerate}
							\item $=\sum_r^{[0:j+1]}a_rb_{j-r}$
							\item $=\sum_r^{[0:\deg(a)+1]}a_rb_{j-r}+\sum_r^{[\deg(a)+1:j+1]}a_rb_{j-r}$
							\item $=\sum_r^{[0:\deg(a)+1]}0a_r+\sum_r^{[\deg(a)+1:j+1]}0b_{j-r}$
							\item $=0$.
						\end{enumerate}
					\end{enumerate}
					\item Now using \procedurehr{2.77}, verify that $(ab)_{\deg(a)+\deg(b)}=a_{\deg(a)}b_{\deg(b)}\ne 0$.
					\item \textbf{Hence verify that $\deg(ab)=\deg(a)+\deg(b)$.}
				\end{enumerate}
		\declaration{2.00}
			The phrase "\gls{ratpolymonic}" will be used to refer to polynomials $p$ such that $p\ne 0$ and $p_{\deg(p)}=1$.
		\declaration{2.01}
			The notation \gls{ratpolymon}, where $p$ is a polynomial such that $p\ne 0$, will be used as a shorthand for $\frac{p}{p_{\deg(p)}}$.
		\procedure{2.25}
			\objective
				Choose two polynomials, $a,b$ such that $b\ne 0$. The objective of the following instructions is to construct two polynomials $u,w$ such that $a=ub+w$ and $\deg(w)<\deg(b)$.
			\implementation
				\begin{enumerate}
					\item If $\deg(a)\ge\deg(b)$, then do the following:
					\begin{enumerate}
						\item Let $y=\frac{a_{\deg(a)}}{b_{\deg(b)}}\lambda^{\deg(a)-\deg(b)}$
						\item Let $e=a-yb$.
						\item Verify that $\deg(e)<\deg(a)$.
						\item Execute \procedurehr{2.25} on the tuple $\langle e,b\rangle$ and let $\langle c,d\rangle$ receive.
						\item Verify that $cb+d=e$.
						\item \textbf{Verify that $\deg(d)<\deg(b)$.}
						\item Therefore verify that $cb+d=a-yb$
						\item \textbf{Therefore verify that $(y+c)b+d=a$.}
						\item \textbf{Now yield the tuple $\langle y+c, d\rangle$.}
					\end{enumerate}
					\item Otherwise do the following:
					\begin{enumerate}
						\item \textbf{Verify that $0b+a=a$.}
						\item \textbf{Verify that $\deg(a)<\deg(b)$.}
						\item \textbf{Yield the tuple $\langle 0,a\rangle$.}
					\end{enumerate}
				\end{enumerate}
		\declaration{2.35}
			The notation \gls{ratpolydiv}, where $a,b$ are polynomials, will be used to refer to the first part of the pair yielded by executing \procedurehr{2.25} on $\langle a,b\rangle$.
		\declaration{2.36}
			The notation \gls{ratpolymod}, where $a,b$ are polynomials, will be used to refer to the second part of the pair yielded by executing \procedurehr{2.25} on $\langle a,b\rangle$.
		\procedure{2.79}
			\objective
				Choose a polynomial $a$ and a rational number $b$. The objective of the following instructions is to show that $a\mod(\lambda-b)=\Lambda(a,b)$.
			\implementation
				\begin{enumerate}
					\item Let $d=\lambda-b$.
					\item Verify that $d\ne 0$.
					\item Let $c=a\div d$.
					\item Verify that $a=cd+(a\mod d)$.
					\item Also verify that $\deg(a\mod d)<\deg(d)=1$.
					\item Hence verify that $\deg(a\mod d)=0$.
					\item Now verify that $\Lambda(a,b)$
					\begin{enumerate}
						\item $=\Lambda(cd+(a\mod d),b)$
						\item $=\Lambda(cd,b)+\Lambda(a\mod d,b)$
						\item $=\Lambda(c,b)\Lambda(d,b)+\Lambda(a\mod d,b)$
						\item $=\Lambda(c,b)(-b+b)+\Lambda(a\mod d,b)$
						\item $=0\Lambda(c,b)+\Lambda(a\mod d,b)$
						\item $=\Lambda(a\mod d,b)$
						\item $=a\mod d$.
					\end{enumerate}
				\end{enumerate}
		\procedure{2.80}
			\objective
				Choose a polynomial $p\ne 0$ and rational numbers $a_0<a_1<\cdots<a_{\deg(p)-2}<a_{\deg(p)-1}$ in such a way that $\Lambda(p,a_i)=0$ for $i\in[0:\deg(p)]$. The objective of the following instructions is to either show that $p=q_0\prod_j^{[0:{n}]}(\lambda-a_j)$ or $0\ne 0$.
			\implementation
				\begin{enumerate}
					\item Let $n=\deg(p)$.
					\item Let $q=p$.
					\item For $i$ in $[0:n]$, do the following:
					\begin{enumerate}
						\item Verify that $p=q\prod_k^{[0:i]}(\lambda-a_k)$.
						\item If $\Lambda(q,a_i)\ne 0$, do the following:
						\begin{enumerate}
							\item Verify that $\Lambda(p,a_i)=\Lambda(q\prod_k^{[0:i]}(\lambda-a_k),a_i)=\Lambda(q,a_i)\prod_k^{[0:i]}\Lambda(\lambda-a_k,a_i)=\Lambda(q,a_i)\prod_k^{[0:i]}(a_i-a_k)\ne 0$.
							\item Therefore using the precondition and (i), verify that $0\ne 0$.
							\item \textbf{Abort procedure.}
						\end{enumerate}
						\item Otherwise do the following:
						\begin{enumerate}
							\item Let $b=q$.
							\item Let $q=b\div(\lambda-a_i)$.
							\item Verify that $\Lambda(b,a_i)=0$.
							\item Execute \procedurehr{2.79} on $\langle b,a_i\rangle$.
							\item Hence verify that $b=(\lambda-a_i)q+b\mod(\lambda-a_i)=(\lambda-a_i)q$.
							\item Hence verify that $p=q\prod_j^{[0:{i+1}]}(\lambda-a_j)$.
						\end{enumerate}
					\end{enumerate}
					\item Now verify that $0\ne p=q\prod_j^{[0:{n}]}(\lambda-a_j)$.
					\item Hence verify that $q\ne 0$.
					\item Hence verify that $n=\deg(p)=\deg(q)+\sum_j^{[0:n]}\deg(\lambda-a_j)=\deg(q)+n$.
					\item Hence verify that $\deg(q)=0$.
					\item Hence verify that $q=q_0\ne 0$.
					\item \textbf{Hence verify that $p=q_0\prod_j^{[0:{n}]}(\lambda-a_j)$.}
				\end{enumerate}
		\procedure{2.16}
			\objective
				Choose a polynomial $p\ne 0$ and rational numbers $a_0<a_1<\cdots<a_{\deg(p)-1}<a_{\deg(p)}$ in such a way that $\Lambda(p,a_i)=0$ for $i\in[0:\deg(p)+1]$. The objective of the following instructions is to show that $0\ne 0$.
			\implementation
				\begin{enumerate}
					\item Let $n=\deg(p)$.
					\item Execute \procedurehr{2.80} on $\langle p,a\rangle$.
					\item Hence verify that $p=q_0\prod_j^{[0:{n}]}(\lambda-a_j)$.
					\item Hence verify that $\Lambda(p,a_{n})=\Lambda(q_0\prod_j^{[0:{n}]}(\lambda-a_j),a_{n})=\Lambda(q_0,a_{n})\prod_j^{[0:{n}]}\Lambda(\lambda-a_j,a_{n})=q_0\prod_j^{[0:{n}]}(a_{n}-a_j)\ne 0$.
					\item Therefore using the precondition and (4), verify that $0=\Lambda(p,a_{n})\ne 0$.
					\item \textbf{Abort procedure.}
				\end{enumerate}
		\procedure{2.15}
			\objective
				Choose a polynomial $p$ and a rational number $X$. The objective of the following instructions is to construct a rational number $a$ and a procedure $q(y,z)$ to show that $\lvert\Lambda(p,z)-\Lambda(p,y)\rvert\le a\lvert z-y\rvert$ when two rational numbers $y,z$ such that $\lvert y\rvert\le X$ and $\lvert z\rvert\le X$ are chosen.
			\implementation
				\begin{enumerate}
					\item Let $a=\sum_{r}^{[1:\lvert p\rvert]}r\lvert p_r\rvert X^{r-1}$.
					\item Let $q(y,z)$ be the following procedure:
					\begin{enumerate}
						\item Verify that $\lvert\Lambda(p,z)-\Lambda(p,y)\rvert$
						\begin{enumerate}
							\item $=\lvert(\sum_{r}^{[0:\lvert p\rvert]} p_rz^r)-(\sum_{r}^{[0:\lvert p\rvert]} p_ry^r)\rvert$
							\item $=\lvert\sum_{r}^{[1:\lvert p\rvert]} p_r(z^r-y^r)\rvert$
							\item $=\lvert\sum_{r}^{[1:\lvert p\rvert]} p_r(z-y)\sum_t^{[0:r]}z^ty^{r-1-t}\rvert$
							\item $=\lvert(z-y)\sum_{r}^{[1:\lvert p\rvert]} p_r\sum_t^{[0:r]}z^ty^{r-1-t}\rvert$
							\item $=\lvert z-y\rvert\lvert\sum_{r}^{[1:\lvert p\rvert]} p_r\sum_t^{[0:r]}z^ty^{r-1-t}\rvert$
							\item $\le\lvert z-y\rvert\sum_{r}^{[1:\lvert p\rvert]}\lvert p_r\sum_t^{[0:r]}z^ty^{r-1-t}\rvert$
							\item $=\lvert z-y\rvert\sum_{r}^{[1:\lvert p\rvert]}\lvert p_r\rvert\lvert\sum_t^{[0:r]}z^ty^{r-1-t}\rvert$
							\item $\le\lvert z-y\rvert\sum_{r}^{[1:\lvert p\rvert]}\lvert p_r\rvert\sum_t^{[0:r]}\lvert z^ty^{r-1-t}\rvert$
							\item $=\lvert z-y\rvert\sum_{r}^{[1:\lvert p\rvert]}\lvert p_r\rvert\sum_t^{[0:r]}\lvert z\rvert^t\lvert y\rvert^{r-1-t}$
							\item $\le\lvert z-y\rvert\sum_{r}^{[1:\lvert p\rvert]}\lvert p_r\rvert\sum_t^{[0:r]}X^tX^{r-1-t}$
							\item $=\lvert z-y\rvert\sum_{r}^{[1:\lvert p\rvert]}\lvert p_r\rvert\sum_t^{[0:r]}X^{r-1}$
							\item $=\lvert z-y\rvert\sum_{r}^{[1:\lvert p\rvert]}r\lvert p_r\rvert X^{r-1}$
							\item $=a\lvert z-y\rvert$
						\end{enumerate}
					\end{enumerate}
					\item \textbf{Yield the tuple $\langle a,q\rangle$.}
				\end{enumerate}
		\procedure{2.17}
			\objective
				Choose a polynomial $f$. Choose rational numbers $a<b$ such that $\sgn(\Lambda(f,a))=-\sgn(\Lambda(f,b))$. Choose a rational number target $B>0$. The objective of the following instructions is to construct a rational number $d$ such that $a\le d\le b$ and $\lvert f(d)\rvert<B$.
			\implementation
				\begin{enumerate}
					\item Execute \procedurehr{2.15} on $\langle f,\max(\lvert a\rvert,\lvert b\rvert)\rangle$ and let $\langle G,q\rangle$ receive the result.
					\item Let $c=a$ and $d=b$.
					\item Until $G\lvert d-c\rvert<B$
					\begin{enumerate}
						\item Let $e=\frac{c+d}{2}$.
						\item \textbf{Verify that $a\le c<e<d\le b$.}
						\item If $\sgn(\Lambda(f,c))=-\sgn(\Lambda(f,e))$, then do the following:
						\begin{enumerate}
							\item Let $d=e$.
						\end{enumerate}
						\item Otherwise if $\sgn(\Lambda(f,e))=-\sgn(\Lambda(f,d))$, then do the following:
						\begin{enumerate}
							\item Let $c=e$.
						\end{enumerate}
						\item Otherwise if $\Lambda(f,e)=0$, then do the following:
						\begin{enumerate}
							\item \textbf{Verify that $\lvert\Lambda(f,e)\rvert=0<B$.}
							\item \textbf{Yield the tuple $\langle e\rangle$.}
						\end{enumerate}
					\end{enumerate}
					\item Execute procedure $q$ on $\langle c,d\rangle$.
					\item \textbf{Hence verify that $\lvert\Lambda(f,c)\rvert<\lvert\Lambda(f,d)-\Lambda(f,c)\rvert\le G\lvert d-c\rvert<B$.}
					\item \textbf{Yield the tuple $\langle c\rangle$.}
				\end{enumerate}
		\procedure{2.18}
			\objective
				Choose a polynomial $f\ne 0$ and pairs of rational numbers $(a_{\deg(f)},b_{\deg(f)}),(a_{\deg(f)-1},b_{\deg(f)-1}),\cdots,(a_0,b_0)$ in such a way that:
				\begin{enumerate}
					\item $a_{\deg(f)}<b_{\deg(f)}\le a_{\deg(f)-1}<b_{\deg(f)-1}\le\cdots\le a_1<b_1\le a_0<b_0$.
					\item $\sgn(\Lambda(f,a_i))=-\sgn(\Lambda(f,b_i))$ for $i\in[0:\deg(f)+1]$.
				\end{enumerate}
				The objective of the following instructions is to show that $1=-1$.
			\implementation
				\begin{enumerate}
					\item If $\deg(f)>0$:
					\begin{enumerate}
						\item Let $B=\min_k^{[0:\deg(f)-1]}\min(\lvert\Lambda(f,a_k)\rvert,\lvert\Lambda(f,b_k)\rvert)$.
						\item For $k\in[0:\deg(f)]$, verify that $\lvert\Lambda(f,a_k)\rvert\ge B$.
						\item Execute \procedurehr{2.17} on the formal polynomial $f$, interval $(a_{\deg(f)}, b_{\deg(f)})$, and target of $B$. Let the tuple $\langle d\rangle$ receive the result.
						\item Verify that $\lvert\Lambda(f,d)\rvert<B$.
						\item Let $h=f\div(\lambda-d)$.
						\item Execute \procedurehr{2.79} on $\langle f,d\rangle$.
						\item Hence verify that $f=(\lambda-d)h+f\mod(\lambda-d)=(\lambda-d)h+\Lambda(f,d)$.
						\item Hence verify that $0\ne f-\Lambda(f,d)=(\lambda-d)h$.
						\item Hence verify that $h\ne 0$.
						\item Hence verify that $\deg(f)=\deg(f-\Lambda(f,d))=\deg((\lambda-d)h)=\deg(\lambda-d)+\deg(h)=1+\deg(h)$.
						\item Hence verify that $\deg(h)=\deg(f)-1$.
						\item For $k\in[0:\deg(h)+1]$, do the following:
						\begin{enumerate}
							\item If $\Lambda(f,a_k)\ge B$, in-order verify that:
							\begin{enumerate}
								\item $\Lambda(f,a_k)\ge B>\lvert\Lambda(f,d)\rvert\ge\Lambda(f,d)$.
								\item $\Lambda(f,a_k)-\Lambda(f,d)>0$.
								\item $(a_k-d)\Lambda(h,a_k)>0$.
								\item \textbf{$\Lambda(h,a_k)>0$.}
								\item $\Lambda(f,b_k)\le-B<-\lvert\Lambda(f,d)\rvert\le\Lambda(f,d)$.
								\item $\Lambda(f,b_k)-\Lambda(f,d)<0$.
								\item $(b_k-d)\Lambda(h,b_k)<0$.
								\item \textbf{$\Lambda(h,b_k)<0$.}
							\end{enumerate}
							\item Otherwise, if $\Lambda(f,a_k)\le -B$, do the following:
							\begin{enumerate}
								\item \textbf{Using steps analogous to (ji), verify that $\Lambda(h,a_k)<0$.}
								\item \textbf{Using steps analogous to (ji), verify that $\Lambda(h,b_k)>0$.}
							\end{enumerate}
						\end{enumerate}
						\item Execute \procedurehr{2.18} on $h$ and $a_{\deg(h)}<b_{\deg(h)}\le a_{\deg(h)-1}<b_{\deg(h)-1}\le\cdots\le a_1<b_1\le a_0<b_0$.
					\end{enumerate}
					\item Otherwise, do the following:
					\begin{enumerate}
						\item Verify that $\deg(f)=0$.
						\item Therefore verify that $f=f_0\ne 0$.
						\item Therefore verify that $\sgn(f_0)=\sgn(\Lambda(f,a_0))=-\sgn(\Lambda(f,b_0))=-\sgn(f_0)$.
						\item \textbf{Therefore verify that $1=-1$.}
						\item \textbf{Abort procedure.}
					\end{enumerate}
				\end{enumerate}
		\procedure{2.19}
			\objective
				Choose two lists of polynomials $s,q$ in such a way that:
				\begin{enumerate}
					\item $\lvert s\rvert>1$.
					\item For $i$ in $[0:\lvert s\rvert]$, $\deg(s_i)=i$.
					\item For $i$ in $[0:\lvert s\rvert]$, $\sgn((s_i)_i)=\sgn((s_m)_m)$.
					\item For $i$ in $[1:\lvert s\rvert-1]$, $s_{i-1}+s_{i+1}=q_is_i$.
				\end{enumerate}
				The objective of the following instructions is to construct lists of polynomials $g,h$ such that $g_is_{i+1}+h_is_i=1$ for $i$ in $[0:\lvert s\rvert-1]$.
			\implementation
				\begin{enumerate}
					\item Let $m=\lvert s\rvert-1$
					\item Let $g=h=\langle\rangle$.
					\item If $m>1$, do the following:
					\begin{enumerate}
						\item Verify that $q_{m-1}s_{m-1}-s_{m}=s_{m-2}$.
						\item Execute \procedurehr{2.19} on $s_{[0:m]}$ and $q_{[1:m-1]}$ and let the tuple $\langle,,g,h\rangle$ receive.
						\item Verify that $g_{m-2}s_{m-1}+h_{m-2}s_{m-2}=1$.
						\item Let $g_{m-1}=-h_{m-2}$.
						\item Let $h_{m-1}=g_{m-2}+h_{m-2}q_{m-1}$.
						\item Therefore verify that $g_{m-1}s_{m}+h_{m-1}s_{m-1}$
						\begin{enumerate}
							\item $=g_{m-2}s_{m-1}+h_{m-2}(q_{m-1}s_{m-1}-s_{m})$
							\item $=g_{m-2}s_{m-1}+h_{m-2}s_{m-2}$
							\item $=1$.
						\end{enumerate}
					\end{enumerate}
					\item Otherwise, if $m=1$ do the following:
					\begin{enumerate}
						\item Let $g_0=0$.
						\item Let $h_0=\frac{1}{s_0}$.
						\item \textbf{Therefore verify that $g_0s_1+h_0s_0=1$.}
					\end{enumerate}
					\item \textbf{Yield the tuple $\langle s,q,g,h\rangle$.}
				\end{enumerate}
		\declaration{2.10}
			The notation \gls{ratpolyJ}, where $s$ is a list of polynomials and $x$ is a rational number, will be used as a shorthand for the number of changes observed when the list $\sgn(\Lambda(s,x))$ is iterated through linearly.
		\procedure{2.20}
			\objective
				Execute \procedurehr{2.19} and let $\langle s,q,g,h\rangle$ receive. Choose a rational number $X$. The objective of the following instructions is to construct a rational number $l$ and a procedure $u(c,d)$ to show that either $0<0$ or $\lvert J_{s}(d)-J_{s}(c)\rvert=[\sgn(\Lambda(s_{\lvert s\rvert-1},c))\ne\sgn(\Lambda(s_{\lvert s\rvert-1},d))]$, when rational numbers $c,d$ such that $\lvert c\rvert\le X$, $\lvert d\rvert\le X$, $\lvert d-c\rvert\le l$, $0\not\in\Lambda(s,c)$, and $0\not\in\Lambda(s,d)$ are chosen.
			\implementation
				\begin{enumerate}
					\item Execute \procedurehr{2.15} on $\langle s,X\rangle$ and let $\langle G,t\rangle$ receive the result.
					\item Let $B=\max_i^{[0:\lvert s\rvert]}G_i$.
					\item Let $C=\max_i^{[0:\lvert s\rvert-1]}\max(\lvert\Lambda(\lVert g_i\rVert,X)\rvert,\lvert\Lambda(\lVert h_i\rVert,X)\rvert)$.
					\item Let $D=\max_i^{[1:\lvert s\rvert-1]}\max(\lvert\Lambda(\lVert q_i\rVert,X)\rvert,2)$.
					\item Let $l=\frac{1}{BCD}$.
					\item Let $u(c,d)$ be the following procedure:
					\begin{enumerate}
						\item Let $i=0$.
						\item If $i+1<\lvert s\rvert$, do the following:
						\begin{enumerate}
							\item Verify that $\sgn(\Lambda(s_i,c))=\sgn(\Lambda(s_i,d))$.
							\item Verify that $J_{s_{[0:i+1]}}(c)=J_{s_{[0:i+1]}}(d)$.
							\item If $\sgn(\Lambda(s_{i+1},c))=\sgn(\Lambda(s_{i+1},d))$, do the following:
							\begin{enumerate}
								\item Verify that $J_{s_{[0:i+2]}}(c)=J_{s_{[0:i+2]}}(d)$.
								\item Set $i$ to $i+1$ and go to (b).
							\end{enumerate}
							\item Otherwise if $\sgn(\Lambda(s_{i+1},c))=\sgn(\Lambda(s_{i+1},d))$ or $i+2\ge\lvert s\rvert$, do the following:
							\begin{enumerate}
								\item Verify that $\sgn(\Lambda(s_{i+1},c))\ne\sgn(\Lambda(s_{i+1},d))$.
								\item Verify that $\lvert J_{s_{[0:i+2]}}(c)-J_{s_{[0:i+2]}}(d)\rvert=1$.
								\item Verify that $i+2=\lvert s\rvert$.
								\item Go to (c).
							\end{enumerate}
							\item Execute \hyperref[sec:procedure 2.5 auxilliary procedure]{procedure 2.5 auxilliary procedure} on $i$.
							\item If $\sgn(\Lambda(s_{i+2},c))\ne\sgn(\Lambda(s_{i+2},d))$, do the following:
							\begin{enumerate}
								\item Execute procedure $t_{i+2}$ on $\langle c,d\rangle$.
								\item Hence verify that $\lvert\Lambda(s_{i+2},c)\rvert<\lvert\Lambda(s_{i+2},d)-\Lambda(s_{i+2},c)\rvert=\lvert (d-c)G_{i+2}\rvert\le\frac{1}{BCD}\cdot B=\frac{1}{CD}=\frac{1}{C}\cdot\frac{1}{D}\le\frac{1}{C}(1-\frac{1}{D})$.
								\item Using (A) and (i), verify that $\frac{1}{C}(1-\frac{1}{D})<\lvert s_{i+2}(c)\rvert<\frac{1}{C}(1-\frac{1}{D})$.
								\item \textbf{Abort procedure.}
							\end{enumerate}
							\item Otherwise if $\sgn(\Lambda(s_i,c))=\sgn(\Lambda(s_{i+2},c))$, do the following:
							\begin{enumerate}
								\item Verify that $2\frac{1}{C}(1-\frac{1}{D})<\lvert\Lambda(s_i,c)\rvert+\lvert\Lambda(s_{i+2},c)\rvert=\lvert\Lambda(s_i,c)+\Lambda(s_{i+2},c)\rvert=\lvert q_{i+1}(c)\Lambda(s_{i+1},c)\rvert<D\frac{1}{CD}$.
								\item Verify that $2(1-\frac{1}{D})<1$.
								\item Using (B) and the construction of $D$, verify that $2\le D<2$.
								\item \textbf{Abort procedure.}
							\end{enumerate}
							\item Otherwise, do the following:
							\begin{enumerate}
								\item Verify that $\sgn(\Lambda(s_i,d))=\sgn(\Lambda(s_i,c))\ne\sgn(\Lambda(s_{i+2},c))=\sgn(\Lambda(s_{i+2},d))$.
								\item Therefore verify that $1=J_{s_{[0:i+3]}}(c)-J_{s_{[0:i+1]}}(c)=J_{s_{[0:i+3]}}(d)-J_{s_{[0:i+1]}}(d)$.
								\item Therefore verify that $J_{s_{[0:i+1]}}(c)+1=J_{s_{[0:i+3]}}(c)=J_{s_{[0:i+3]}}(d)=J_{s_{[0:i+1]}}(d)+1$.
								\item Set $i$ to $i+2$ and go to (b).
							\end{enumerate}
						\end{enumerate}
						\item If $\sgn(\Lambda(s_{\lvert s\rvert-1},c))=\sgn(\Lambda(s_{\lvert s\rvert-1},d))$, then do the following:
						\begin{enumerate}
							\item \textbf{Verify that $J_s(c)=J_s(d)$.}
						\end{enumerate}
						\item Otherwise do the following:
						\begin{enumerate}
							\item \textbf{Verify that $\lvert J_s(d)-J_s(c)\rvert=1$.}
						\end{enumerate}
					\end{enumerate}
					\item \textbf{Yield the tuple $\langle l,u\rangle$.}
				\end{enumerate}
			\subsubsection*{Auxilliary Procedure}\label{sec:procedure 2.5 auxilliary procedure}
				\paragraph{Objective}
					Choose a non-negative integer $i<m$ and natural numbers $c,d$ such that $\sgn(\Lambda(s_{i+1},c))\ne\sgn(\Lambda(s_{i+1},d))$ and $i+2\le m$. The objective of the following instructions is to show that $\lvert\Lambda(s_{i+1},c)\rvert<\frac{1}{CD}$, $\lvert\Lambda(s_{i+1},d)\rvert<\frac{1}{CD}$, $\frac{1}{C}(1-\frac{1}{D})<\lvert\Lambda(s_{i},c)\rvert$, $\frac{1}{C}(1-\frac{1}{D})<\lvert\Lambda(s_{i},d)\rvert$, $\frac{1}{C}(1-\frac{1}{D})<\lvert\Lambda(s_{i+2},c)\rvert$, and $\frac{1}{C}(1-\frac{1}{D})<\lvert\Lambda(s_{i+2},d)\rvert$.
				\paragraph{Implementation}
					\begin{enumerate}
						\item Verify the following in order:
						\begin{enumerate}
							\item Execute procedure $t_{i+1}$ on $\langle c,d\rangle$.
							\item $\lvert\Lambda(s_{i+1},c)\rvert<\lvert\Lambda(s_{i+1},c)-\Lambda(s_{i+1},d)\rvert=\lvert c-d\rvert\lvert G_{i+1}\rvert\le\lvert c-d\rvert B\le\frac{1}{BCD}\cdot B=\frac{1}{CD}$
							\item $\lvert\Lambda(s_{i+1},d)\rvert<\lvert\Lambda(s_{i+1},c)-\Lambda(s_{i+1},d)\rvert\le\frac{1}{CD}$
							\item $1=\Lambda(g_{i},c)\Lambda(s_{i+1},c)+\Lambda(h_{i},c)\Lambda(s_{i},c)=\lvert\Lambda(g_{i},c)\Lambda(s_{i+1},c)+\Lambda(h_{i},c)\Lambda(s_{i},c)\rvert\le\lvert\Lambda(g_{i},c)\rvert\lvert\Lambda(s_{i+1},c)\rvert+\lvert\Lambda(h_{i},c)\rvert\lvert\Lambda(s_{i},c)\rvert<C(\frac{1}{CD}+\lvert\Lambda(s_{i},c)\rvert)$
							\item $\frac{1}{C}(1-\frac{1}{D})<\lvert\Lambda(s_{i},c)\rvert$
							\item $1<C(\frac{1}{CD}+\lvert\Lambda(s_{i},d)\rvert)$
							\item $\frac{1}{C}(1-\frac{1}{D})<\lvert\Lambda(s_{i},d)\rvert$
							\item $1=\Lambda(g_{i+1},c)\Lambda(s_{i+2},c)+\Lambda(h_{i+1},c)\Lambda(s_{i+1},c)=\lvert\Lambda(g_{i+1},c)\Lambda(s_{i+2},c)+\Lambda(h_{i+1},c)\Lambda(s_{i+1},c)\rvert\le\lvert\Lambda(g_{i+1},c)\rvert\lvert\Lambda(s_{i+2},c)\rvert+\lvert\Lambda(h_{i+1},c)\rvert\lvert\Lambda(s_{i+1},c)\rvert<C(\lvert\Lambda(s_{i+2},c)\rvert+\frac{1}{CD})$
							\item $\frac{1}{C}(1-\frac{1}{D})<\lvert\Lambda(s_{i+2},c)\rvert$
							\item $1<C(\lvert\Lambda(s_{i+2},d)\rvert+\frac{1}{CD})$
							\item $\frac{1}{C}(1-\frac{1}{D})<\lvert\Lambda(s_{i+2},d)\rvert$
						\end{enumerate}
					\end{enumerate}
		\procedure{2.21}
			\objective
				Choose a polynomial $p\ne 0$. Choose a rational number $k>1+\max_{i}^{[0:\deg(p)]}\lvert\frac{p_i}{p_{\deg(p)}}\rvert$. The objective of the following instructions is to show that $\sgn(\Lambda(p,k))=\sgn(p_{\deg(p)})$.
			\implementation
				\begin{enumerate}
					\item Let $n=\deg(p)$.
					\item In reverse order verify the following:
					\begin{enumerate}
						\item \textbf{$\sgn(\Lambda(p,k))=\sgn(p_{\deg(p)})$}
						\item $\sgn(p_nk^n+p_{n-1}k^{n-1}+\cdots+p_0k^0)=\sgn(p_{n})$
						\item $\sgn(k^n+\frac{p_{n-1}}{p_n}k^{n-1}+\cdots+\frac{p_0}{p_n}k^0)=1$
						\item $k^n+\frac{p_{n-1}}{p_n}k^{n-1}+\cdots+\frac{p_0}{p_n}k^0>0$
						\item $k^n>-(\frac{p_{n-1}}{p_n}k^{n-1}+\cdots+\frac{p_0}{p_n}k^0)$
						\item $k^n>\lvert \frac{p_{n-1}}{p_n}k^{n-1}+\cdots+\frac{p_0}{p_n}k^0\rvert$
						\item $k^n>\lvert\max_i^{[0:n]}\lvert\frac{p_i}{p_n}\rvert(k^{n-1}+\cdots+k^0)\rvert$
						\item $k^n>\max_i^{[0:n]}\lvert \frac{p_i}{p_n}\rvert\frac{k^n-1}{k-1}$
						\item $k^{n+1}-k^n>\max_i^{[0:n]}\lvert\frac{p_i}{p_n}\rvert(k^n-1)$
						\item $k^{n+1}-(1+\max_i^{[0:n]}\lvert\frac{p_i}{p_n}\rvert)k^n+\max_i^{[0:n]}\lvert\frac{p_i}{p_n}\rvert>0$
						\item $k>1+\max_i^{[0:n]}\lvert\frac{p_i}{p_n}\rvert$
					\end{enumerate}
				\end{enumerate}
		\procedure{2.22}
			\objective
				Choose a polynomial $p\ne 0$. Choose a rational number $k<-(1+\max_i^{[0:\deg(p)]}\lvert\frac{p_i}{p_{\deg(p)}}\rvert)$. The objective of the following instructions is to show that $\sgn(\Lambda(p,k))=(-1)^{\deg(p)}\sgn(p_{\deg(p)})$.
			\implementation
				\begin{enumerate}
					\item Let $t=\deg(p)$.
					\item Let $q=\langle(-1)^{t-i}p_i\for i\in[0:t+1]\rangle$.
					\item Verify that $k<-(1+\max_i^{[1:t+1]}\lvert\frac{q_i}{q_{\deg(q)}}\rvert)$.
					\item Therefore verify that $-k>1+\max_i^{[0:t]}\lvert\frac{q_i}{q_{\deg(q)}}\rvert$.
					\item Execute \procedurehr{2.21} on $\langle q,-k\rangle$.
					\item Hence verify that $(-1)^t\sgn(\Lambda(p,k))$
					\begin{enumerate}
						\item $=\sgn((-1)^t\Lambda(p,k))$
						\item $=\sgn((-1)^t\sum_i^{[0:{t+1}]} p_ik^i)$
						\item $=\sgn(\sum_i^{[0:{t+1}]} (-1)^i(-1)^{t-i}p_ik^i)$
						\item $=\sgn(\sum_i^{[0:{t+1}]} q_i(-k)^i)$
						\item $=\sgn(\Lambda(q,-k))$
						\item $=\sgn(q_t)$
						\item $=\sgn(p_t)$.
					\end{enumerate}
					\item \textbf{Therefore verify that $\sgn(\Lambda(p,k))=(-1)^t(-1)^t\sgn(\Lambda(p,k))=(-1)^t\sgn(p_t)$.}
				\end{enumerate}
		\procedure{2.23}
			\objective
				Choose a list of polynomials, $s$, and rational numbers $a,l,c$ such that $a<c$ and $l>0$. The objective of the following instructions is to either show that $0<0$ or to construct a list of rational numbers, $b$, such that $a=b_0<b_1<\cdots<b_{\lvert b\rvert-1}=c$, $b_{i}-b_{i-1}\le l$ for $i$ in $[1:\lvert b\rvert]$, and $0\not\in\Lambda(s,b_i)$ for $i$ in $[1:\lvert b\rvert-1]$.
			\implementation
				\begin{enumerate}
					\item Let $e=\langle\langle\rangle,\langle\rangle,\cdots,\langle\rangle\rangle$.
					\item Let $f=\sum_r^{[0:{\lvert s\rvert}]}\deg(s_r)$.
					\item Let $b=\langle a\rangle$.
					\item Let $d=b_1$.
					\item While $d+l<c$, do the following:
					\begin{enumerate}
						\item Let $m=l$.
						\item While $0\in\Lambda(s,d+m)$ and $\sum\lvert e\rvert\le f$, do the following:
						\begin{enumerate}
							\item Let $0\le i<\lvert s\rvert$ be an integer such that $\Lambda(s_i,d+m)=0$.
							\item Append $d+m$ onto $e_i$.
							\item Set $m=\frac{m}{2}$
						\end{enumerate}
						\item If $\sum\lvert e\rvert>f$, then do the following:
						\begin{enumerate}
							\item If $\lvert e_i\rvert\le\deg(s_i)$ for $0\le i<\lvert s\rvert$, then do the following:
							\begin{enumerate}
								\item Verify that $\sum\lvert e\rvert\le f$.
								\item Therefore using (c), verify that $\sum\lvert e\rvert\le f<\sum\lvert e\rvert$.
								\item \textbf{Abort procedure.}
							\end{enumerate}
							\item Otherwise, do the following:
							\begin{enumerate}
								\item Let $0\le i<\lvert s\rvert$ be an integer such that $\lvert e_i\rvert>\deg(s_i)$.
								\item Execute \procedurehr{2.16} on $s_i$ and a sorted $e_i$.
								\item \textbf{Abort procedure.}
							\end{enumerate}
						\end{enumerate}
						\item Otherwise, do the following:
						\begin{enumerate}
							\item \textbf{Verify that $0\not\in\Lambda(s,d+m)$.}
							\item Append $d+m$ onto $b$.
							\item \textbf{Verify that $0<b_{\lvert b\rvert-1}-b_{\lvert b\rvert-2}=m\le l$.}
							\item Set $d$ to $d+m$.
							\item Using (5), verify that $d<c$.
						\end{enumerate}
					\end{enumerate}
					\item Verify that $d<c$.
					\item Verify that $d+l\ge c$.
					\item \textbf{Therefore verify that $0<c-d\le l$.}
					\item Append $c$ onto $b$.
					\item \textbf{Yield $\langle b\rangle$.}
				\end{enumerate}
		\procedure{2.24}
			\objective
				Execute \procedurehr{2.19} and let $\langle s,q,g,h\rangle$ receive. Let $m=\lvert s\rvert-1$. The objective of the following instructions is to either show that $0<0$ or to construct two lists of rational numbers $c,d$ such that $c_0<d_0\le c_1<d_1\le\cdots\le c_{m-1}<d_{m-1}$ and $0\ne\sgn(\Lambda(s_m,c_i))=-\sgn(\Lambda(s_m,d_i))$ for $i$ in $[0:m]$.
			\implementation
				\begin{enumerate}
					\item Let $U=1+\max_i^{[0:\lvert s\rvert]}\left(1+\max_j^{[1:i+1]}\lvert\frac{(s_i)_{i-j}}{(s_i)_i}\rvert\right)$
					\item Using \procedurehr{2.21}, verify that $J(U)=0$.
					\item Using \procedurehr{2.22}, verify that $J(-U)=m$.
					\item Execute \procedurehr{2.20} on the tuple $\langle s,q,U\rangle$ and let $\langle l,u\rangle$ receive.
					\item Execute \procedurehr{2.23} on $s$ with endpoints $-U,U$ and a step size of $l$ and let $\langle e\rangle$ receive the result.
					\item Let $c=d=\langle\rangle$.
					\item For $i=1$ to $i=\lvert e\rvert-1$:
					\begin{enumerate}
						\item Execute procedure $u$ on the tuple $\langle e_{i-1},e_{i}\rangle$.
						\item If $J_m(e_{i-1})\ne J_m(e_{i})$, then do the following:
						\begin{enumerate}
							\item Append $e_{i-1}$ to $c$.
							\item Append $e_{i}$ to $d$.
							\item Verify that $0\ne\lvert J_{s}(d_{\lvert d\rvert-1})-J_{s}(c_{\lvert c\rvert-1})\rvert=[\sgn(\Lambda(s_{\lvert s\rvert-1},c_{\lvert c\rvert-1}))\ne\sgn(\Lambda(s_{\lvert s\rvert-1},d_{\lvert d\rvert-1}))]$.
							\item Therefore verify that $\sgn(s_m(c_{\lvert c\rvert-1}))\ne\sgn(s_m(d_{\lvert d\rvert-1}))$.
							\item Therefore verify that $\lvert J_m(d_{\lvert d\rvert-1})-J_m(c_{\lvert c\rvert-1})\rvert=1$.
							\item Also verify that $0\not\in\Lambda(s,c_{\lvert c\rvert-1})$.
							\item Hence verify that $\Lambda(s_m,c_{\lvert c\rvert-1})\ne 0$.
							\item Also verify that $0\not\in\Lambda(s,d_{\lvert d\rvert-1})$.
							\item Hence verify that $\Lambda(s_m,d_{\lvert d\rvert-1})\ne 0$.
							\item \textbf{Therefore verify that $0\ne\sgn(s_m(c_{\lvert c\rvert-1}))=-\sgn(s_m(d_{\lvert d\rvert-1}))$.}
							\item \textbf{Also verify that $d_{\lvert d\rvert-2}\le c_{\lvert c\rvert-1}<d_{\lvert d\rvert-1}$.}
						\end{enumerate}
					\end{enumerate}
					\item If $\lvert c\rvert=\lvert d\rvert<m$, then do the following:
					\begin{enumerate}
						\item Verify that each change of $J_m(x)$ over the course of (7) was by $1$.
						\item Verify that $J_m(x)$ changed less than $m$ times over the course of (12).
						\item Therefore verify that $\lvert J_m(U)-J_m(-U)\rvert<m$.
						\item Therefore using (2) and (3), verify that $m=\lvert J_m(U)-J_m(-U)\rvert<m$.
						\item \textbf{Abort procedure.}
					\end{enumerate}
					\item Otherwise, do the following:
					\begin{enumerate}
						\item \textbf{Verify that $m\le\lvert c\rvert=\lvert d\rvert$.}
						\item \textbf{Yield the tuple $\langle c,d\rangle$.}
					\end{enumerate}
				\end{enumerate}
		\procedure{2.26}
			\objective
				Choose two lists of polynomials $s,q$ and a non-negative integer $k$ in such a way that, letting $m=\lvert s\rvert-1$,
				\begin{enumerate}
					\item $k<m$.
					\item For $k\le i\le m$, $\deg(s_i)=i$.
					\item For $k<i<m$, $s_{i-1}+s_{i+1}=q_is_i$.
				\end{enumerate}
				Let $\deg(0)=-1$. The objective of the following instructions is to construct polynomials $g,h$ such that $s_k=gs_{m-1}+hs_m$, $\deg(g)=m-1-k$, and $\deg(h)=m-2-k$.
			\implementation
				\begin{enumerate}
					\item If $k<m-2$, do the following:
					\begin{enumerate}
						\item Verify that $s_k+s_{k+2}=q_{k+1}s_{k+1}$.
						\item Therefore verify that $s_k=q_{k+1}s_{k+1}-s_{k+2}$.
						\item Execute \procedurehr{2.26} on $s,q,k+1$ and let the tuple $\langle g_1,h_1\rangle$ receive.
						\item Verify that $s_{k+1}=g_1s_{m-1}+h_1s_m$.
						\item Verify that $\deg(g_1)=m-1-(k+1)=m-k-2$.
						\item Verify that $\deg(h_1)=m-2-(k+1)=m-k-3$.
						\item Execute \procedurehr{2.26} on $s,q,k+2$ and let the tuple $\langle g_2,h_2\rangle$ receive.
						\item Verify that $s_{k+2}=g_2s_{m-1}+h_2s_m$.
						\item Verify that $\deg(g_2)=m-1-(k+2)=m-k-3$.
						\item Verify that $\deg(h_2)=m-2-(k+2)=m-k-4$.
						\item Let $g=q_{k+1}g_1-g_2$.
						\item \textbf{Verify that $\deg(g)=\max(1+(m-k-2),m-k-3)=m-1-k$.}
						\item Let $h=q_{k+1}h_1-h_2$.
						\item \textbf{Verify that $\deg(h)=\max(1+(m-k-3),m-k-4)=m-2-k$.}
						\item \textbf{Verify that $s_k=q_{k+1}(g_1s_{m-1}+h_1s_m)-(g_2s_{m-1}+h_2s_m)=(q_{k+1}g_1-g_2)s_{m-1}+(q_{k+1}h_1-h_2)s_m=gs_{m-1}+hs_m$.}
					\end{enumerate}
					\item Otherwise, if $k=m-2$ do the following:
					\begin{enumerate}
						\item Verify that $s_{m-2}+s_m=q_{m-1}s_{m-1}$.
						\item Let $g=q_{m-1}$.
						\item \textbf{Verify that $\deg(g)=1=m-1-k$.}
						\item Let $h=-1$.
						\item \textbf{Verify that $\deg(h)=0=m-2-k$.}
						\item \textbf{Therefore verify that $s_k=s_{m-2}=q_{m-1}s_{m-1}-s_m=gs_{m-1}+hs_m$.}
					\end{enumerate}
					\item Otherwise, if $k=m-1$ do the following:
					\begin{enumerate}
						\item Let $g=1$.
						\item \textbf{Verify that $\deg(g)=0=m-1-k$.}
						\item Let $h=0$.
						\item \textbf{Verify that $\deg(h)=-1=m-2-k$.}
						\item \textbf{Verify that $s_k=s_{m-1}=gs_{m-1}+hs_m$.}
					\end{enumerate}
					\item \textbf{Yield the tuple $\langle g,h\rangle$.}
				\end{enumerate}
	\twocolumn[\part{Complex Arithmetic}]
		\declaration{3.19}
			The phrase "\gls{complex}" will be used as a shorthand for an ordered pair of rational numbers.
		\declaration{3.20}
			The phrase "the real part of $a$" and the notation \gls{complexRe}, where $a$ is a complex number, will be used as a shorthand for the first entry of $a$.
		\declaration{3.21}
			The phrase "the imaginary part of $a$" and the notation \gls{complexIm}, where $a$ is a complex number, will be used as a shorthand for the second entry of $a$.
		\declaration{3.22}
			The phrase "\gls{complexa=b}", where $a,b$ are complex numbers, will be used as a shorthand for "$\Re(a)=\Re(b)$ and $\Im(a)=\Im(b)$".
		\procedure{3.68}
			\objective
				Choose a complex number $a$. The objective of the following instructions is to show that $a=a$.
			\implementation
				\begin{enumerate}
					\item Verify that $\Re(a)=\Re(a)$.
					\item Verify that $\Im(a)=\Im(a)$.
					\item \textbf{Hence verify that $a=a$.}
				\end{enumerate}
		\procedure{3.69}
			\objective
				Choose two complex numbers $a,b$ such that $a=b$. The objective of the following instructions is to show that $b=a$.
			\implementation
				\begin{enumerate}
					\item Verify that $\Re(a)=\Re(b)$.
					\item Hence verify that $\Re(b)=\Re(a)$.
					\item Verify that $\Im(a)=\Im(b)$.
					\item Hence verify that $\Im(b)=\Im(a)$.
					\item \textbf{Hence verify that $b=a$.}
				\end{enumerate}
		\procedure{3.70}
			\objective
				Choose three complex numbers $a,b,c$ such that $a=b$ and $b=c$. The objective of the following instructions is to show that $a=c$.
			\implementation
				\begin{enumerate}
					\item Verify that $\Re(a)=\Re(b)$.
					\item Verify that $\Re(b)=\Re(c)$.
					\item Hence verify that $\Re(a)=\Re(c)$.
					\item Verify that $\Im(a)=\Im(b)$.
					\item Verify that $\Im(b)=\Im(c)$.
					\item Hence verify that $\Im(a)=\Im(c)$.
					\item \textbf{Hence verify that $a=c$.}
				\end{enumerate}
		\declaration{3.23}
			The notation \gls{complexa+b}, where $a,b$ are complex numbers, will be used as a shorthand for the pair $\langle\Re(a)+\Re(b),\Im(a)+\Im(b)\rangle$.
		\procedure{3.71}
			\objective
				Choose two complex numbers $a,b,c,d$ such that $a=c$ and $b=d$. The objective of the following instructions is to show that $a+b=c+d$.
			\implementation
				\begin{enumerate}
					\item Using \declarationhr{3.22}, verify that $\Re(a)=\Re(c)$.
					\item Using \declarationhr{3.22}, verify that $\Im(a)=\Im(c)$.
					\item Using \declarationhr{3.22}, verify that $\Re(b)=\Re(d)$.
					\item Using \declarationhr{3.22}, verify that $\Im(b)=\Im(d)$.
					\item Hence verify that $a+b$
					\begin{enumerate}
						\item $=\langle\Re(a),\Im(a)\rangle+\langle\Re(b),\Im(b)\rangle$
						\item $=\langle\Re(a)+\Re(b),\Im(a)+\Im(b)\rangle$
						\item $=\langle\Re(c)+\Re(d),\Im(c)+\Im(d)\rangle$
						\item $=\langle\Re(c),\Im(c)\rangle+\langle\Re(d),\Im(d)\rangle$
						\item $=c+d$.
					\end{enumerate}
				\end{enumerate}
		\procedure{3.72}
			\objective
				Choose three complex numbers $a,b,c$. The objective of the following instructions is to show that $(a+b)+c=a+(b+c)$.
			\implementation
				\begin{enumerate}
					\item Verify that $(a+b)+c$
					\begin{enumerate}
						\item $=\langle\Re(a)+\Re(b),\Im(a)+\Im(b)\rangle+\langle\Re(c),\Im(c)\rangle$
						\item $=\langle(\Re(a)+\Re(b))+\Re(c),(\Im(a)+\Im(b))+\Im(c)\rangle$
						\item $=\langle\Re(a)+(\Re(b)+\Re(c)),\Im(a)+(\Im(b)+\Im(c))\rangle$
						\item $=\langle\Re(a),\Im(a)\rangle+\langle\Re(b)+\Re(c),\Im(b)+\Im(c)\rangle$
						\item $=a+(b+c)$.
					\end{enumerate}
				\end{enumerate}
		\procedure{3.73}
			\objective
				Choose two complex numbers $a,b$. The objective of the following instructions is to show that $a+b=b+a$.
			\implementation
				\begin{enumerate}
					\item $a+b$
					\begin{enumerate}
						\item $=\langle\Re(a)+\Re(b),\Im(a)+\Im(b)\rangle$
						\item $=\langle\Re(b)+\Re(a),\Im(b)+\Im(a)\rangle$
						\item $=b+a$.
					\end{enumerate}
				\end{enumerate}
		\declaration{3.24}
			The notation \gls{complexa}, where $a$ is a rational number, will contextually be used as a shorthand for the pair $\langle a,0\rangle$.
		\procedure{3.74}
			\objective
				Choose a complex number $a$. The objective of the following instructions is to show that $0+a=a$.
			\implementation
				\begin{enumerate}
					\item Verify that $0+a$
					\begin{enumerate}
						\item $=\langle 0,0\rangle+\langle\Re(a),\Im(a)\rangle$
						\item $=\langle 0+\Re(a),0+\Im(a)\rangle$
						\item $=\langle\Re(a),\Im(a)\rangle$
						\item $=a$.
					\end{enumerate}
				\end{enumerate}
		\declaration{3.25}
			The notation \gls{complex-a}, where $a$ is a complex number, will be used as a shorthand for the pair $\langle-\Re(a),-\Im(a)\rangle$.
		\procedure{3.75}
			\objective
				Choose a complex number $a$. The objective of the following instructions is to show that $-a+a=0$.
			\implementation
				\begin{enumerate}
					\item Verify that $-a+a$
					\begin{enumerate}
						\item $=(-a)+a$
						\item $=\langle-\Re(a),-\Im(a)\rangle+\langle\Re(a),\Im(a)\rangle$
						\item $=\langle-\Re(a)+\Re(a),-\Im(a)+\Im(a)\rangle$
						\item $=\langle 0,0\rangle$
						\item $=0$.
					\end{enumerate}
				\end{enumerate}
		\declaration{3.26}
			The notation \gls{complexab}, where $a,b$ are complex numbers, will be used as a shorthand for the pair $\langle\Re(a)\Re(b)-\Im(a)\Im(b),\Re(a)\Im(b)+\Im(a)\Re(b)\rangle$.
		\procedure{3.76}
			\objective
				Choose four complex numbers $a,b,c,d$ such that $a=c$ and $b=d$. The objective of the following instructions is to show that $ab=cd$.
			\implementation
				\begin{enumerate}
					\item Using \declarationhr{3.22}, verify that $\Re(a)=\Re(c)$.
					\item Using \declarationhr{3.22}, verify that $\Im(a)=\Im(c)$.
					\item Using \declarationhr{3.22}, verify that $\Re(b)=\Re(d)$.
					\item Using \declarationhr{3.22}, verify that $\Im(b)=\Im(d)$.
					\item Hence verify that $ab$
					\begin{enumerate}
						\item $=\langle\Re(a),\Im(a)\rangle\langle\Re(b),\Im(b)\rangle$
						\item $=\langle\Re(a)\Re(b)-\Im(a)\Im(b),\Re(a)\Im(b)+\Im(a)\Re(b)\rangle$
						\item $=\langle\Re(c)\Re(d)-\Im(c)\Im(d),\Re(c)\Im(d)+\Im(c)\Re(d)\rangle$
						\item $=\langle\Re(c),\Im(c)\rangle\langle\Re(d),\Im(d)\rangle$
						\item $=cd$.
					\end{enumerate}
				\end{enumerate}
		\procedure{3.77}
			\objective
				Choose three complex numbers $a,b,c$. The objective of the following instructions is to show that $(ab)c=a(bc)$.
			\implementation
				\begin{enumerate}
					\item Verify that $(ab)c$
					\begin{enumerate}
						\item $=\langle\Re(a)\Re(b)-\Im(a)\Im(b),\Re(a)\Im(b)+\Im(a)\Re(b)\rangle\langle\Re(c),\Im(c)\rangle$
						\item $=\langle(\Re(a)\Re(b)-\Im(a)\Im(b))\Re(c)-(\Re(a)\Im(b)+\Im(a)\Re(b))\Im(c),(\Re(a)\Re(b)-\Im(a)\Im(b))\Im(c)+(\Re(a)\Im(b)+\Im(a)\Re(b))\Re(c)\rangle$
						\item $=\langle\Re(a)(\Re(b)\Re(c)-\Im(b)\Im(c))-\Im(a)(\Re(b)\Im(c)+\Im(b)\Re(c)),\Re(a)(\Re(b)\Im(c)+\Im(b)\Re(c))+\Im(a)(\Re(b)\Re(c)-\Im(b)\Im(c))\rangle$
						\item $=\langle\Re(a),\Im(a)\rangle\langle\Re(b)\Re(c)-\Im(b)\Im(c),\Re(b)\Im(c)+\Im(b)\Re(c)\rangle$
						\item $=a(bc)$.
					\end{enumerate}
				\end{enumerate}
		\procedure{3.78}
			\objective
				Choose two complex numbers $a,b$. The objective of the following instructions is to show that $ab=ba$.
			\implementation
				\begin{enumerate}
					\item $ab$
					\begin{enumerate}
						\item $=\langle\Re(a)\Re(b)-\Im(a)\Im(b),\Re(a)\Im(b)+\Im(a)\Re(b)\rangle$
						\item $=\langle\Re(b)\Re(a)-\Im(b)\Im(a),\Re(b)\Im(a)+\Im(b)\Re(a)\rangle$
						\item $=ba$.
					\end{enumerate}
				\end{enumerate}
		\procedure{3.79}
			\objective
				Choose a complex number $a$. The objective of the following instructions is to show that $1a=a$.
			\implementation
				\begin{enumerate}
					\item Verify that $1a$
					\begin{enumerate}
						\item $=\langle 1,0\rangle\langle\Re(a),\Im(a)\rangle$
						\item $=\langle 1\Re(a)-0\Im(a),1\Im(a)+0\Re(a)\rangle$
						\item $=\langle\Re(a),\Im(a)\rangle$
						\item $=a$.
					\end{enumerate}
				\end{enumerate}
		\declaration{3.02}
			The notation \gls{complexconj}, where $a$ is a complex number, will be used as a shorthand for $\langle\Re(a),-\Im(a)\rangle$.
		\procedure{3.00}
			\objective
				Choose two complex numbers $a,b$. The objective of the following instructions is to show that $\conj{a+b}=\conj{a}+\conj{b}$.
			\implementation
				\begin{enumerate}
					\item Verify that $\conj{a+b}$
					\begin{enumerate}
						\item $=\langle\Re(a+b),-\Im(a+b)\rangle$
						\item $=\langle\Re(a)+\Re(b),-\Im(a)-\Im(b)\rangle$
						\item $=\conj{a}+\conj{b}$.
					\end{enumerate}
				\end{enumerate}
		\procedure{3.01}
			\objective
				Choose two complex numbers $a,b$. The objective of the following instructions is to show that $\conj{ab}=\conj{a}\conj{b}$.
			\implementation
				\begin{enumerate}
					\item Verify that $\conj{ab}$
					\begin{enumerate}
						\item $=\langle\Re(ab),-\Im(ab)\rangle$
						\item $=\langle\Re(a)\Re(b)-\Im(a)\Im(b)),-\Re(a)\Im(b)-\Im(a)\Re(b)\rangle$
						\item $=\langle\Re(a),-\Im(a)\rangle\langle\Re(b),-\Im(b)\rangle$
						\item $=\conj{a}\conj{b}$.
					\end{enumerate}
				\end{enumerate}
		\procedure{3.04}
			\objective
				Choose a list of complex numbers $a$. The objective of the following instructions is to show that $\lVert\sum_r^{[0:{\lvert a\rvert}]}a_r\rVert^2\le\lvert a\rvert\sum_r^{[0:{\lvert a\rvert}]}\lVert a_r\rVert^2$.
			\implementation
				\begin{enumerate}
					\item Verify that $\lVert\sum_r^{[0:{\lvert a\rvert}]}a_r\rVert^2$
					\begin{enumerate}
						\item $=\sum_r^{[0:{\lvert a\rvert}]}\sum_k^{[0:{\lvert a\rvert}]}a_r\conj{a_k}$
						\item $=\sum_r^{[0:{\lvert a\rvert}]}\lVert a_r\rVert^2+2\sum_r^{[0:\lvert a\rvert]}\sum_k^{[r+1:\lvert a\rvert]}(\Re(a_r)\Re(a_k)+\Im(a_r)\Im(a_k))$
						\item $=\sum_r^{[0:{\lvert a\rvert}]}\lVert a_r\rVert^2+2\sum_r^{[0:\lvert a\rvert]}\sum_k^{[r+1:\lvert a\rvert]}(\Re(a_r)^2-(\Re(a_r)-\Re(a_k))^2+\Re(a_k)^2+\Im(a_r)^2-(\Im(a_r)-\Im(a_k))^2+\Im(a_k)^2)$
						\item $\le\sum_r^{[0:{\lvert a\rvert}]}\lVert a_r\rVert^2+\sum_r^{[0:\lvert a\rvert]}\sum_k^{[r+1:\lvert a\rvert]}(\Re(a_r)^2+\Re(a_k)^2+\Im(a_r)^2+\Im(a_k)^2)$
						\item $\le\sum_r^{[0:{\lvert a\rvert}]}\lVert a_r\rVert^2+\sum_r^{[0:\lvert a\rvert]}\sum_k^{[r+1:\lvert a\rvert]}(\lVert a_r\rVert^2+\lVert a_k\rVert^2)$
						\item $\le\sum_r^{[0:{\lvert a\rvert}]}\lVert a_r\rVert^2+\frac{1}{2}\sum_r^{[0:\lvert a\rvert]}\sum_k^{[0:r]^{\frown}[r+1:\lvert a\rvert]}(\lVert a_r\rVert^2+\lVert a_k\rVert^2)$
						\item $\le\sum_r^{[0:{\lvert a\rvert}]}\lVert a_r\rVert^2+\frac{1}{2}(\sum_r^{[0:{\lvert a\rvert}]}(\lvert a\rvert-1)\lVert a_r\rVert^2+\sum_k^{[0:{\lvert a\rvert}]}(\lvert a\rvert-1)\lVert a_k\rVert^2)$
						\item $\le\sum_r^{[0:{\lvert a\rvert}]}\lVert a_r\rVert^2+\sum_r^{[0:{\lvert a\rvert}]}(\lvert a\rvert-1)\lVert a_r\rVert^2$
						\item $\le\lvert a\rvert\sum_r^{[0:{\lvert a\rvert}]}\lVert a_r\rVert^2$
					\end{enumerate}
				\end{enumerate}
		\procedure{3.05}
			\objective
				Choose a list of complex numbers $a$. The objective of the following instructions is to show that $\frac{\lVert a_0\rVert^2}{\lvert a\rvert}-\sum_r^{[1:{\lvert a\rvert}]}\lVert a_r\rVert^2\le\lVert a_0-\sum_r^{[1:{\lvert a\rvert}]}a_r\rVert^2$.
			\implementation
				\begin{enumerate}
					\item Using \procedurehr{3.04}, verify that $\lVert a_0\rVert^2$
					\begin{enumerate}
						\item $=\lVert\sum_r^{[1:{\lvert a\rvert}]}a_r+(a_0-\sum_r^{[1:{\lvert a\rvert}]}a_r)\rVert^2$
						\item $\le \lvert a\rvert\sum_r^{[1:{\lvert a\rvert}]}\lVert a_r\rVert^2+\lvert a\rvert\lVert a_0-\sum_r^{[1:{\lvert a\rvert}]}a_r\rVert^2$
					\end{enumerate}
					\item \textbf{Therefore verify that $\frac{\lVert a_0\rVert^2}{\lvert a\rvert}-\sum_r^{[1:{\lvert a\rvert}]}\lVert a_r\rVert^2\le\lVert a_0-\sum_r^{[1:{\lvert a\rvert}]}a_r\rVert^2$.}
				\end{enumerate}
		\declaration{3.03}
			The notation \gls{complexabs^2}, where $a$ is a complex number, will be used as a shorthand for $\Re(a)^2+\Im(a)^2$.
		\procedure{3.02}
			\objective
				Choose a complex number $a$. The objective of the following instructions is to show that $a\conj{a}=\lVert a\rVert^2$.
			\implementation
				\begin{enumerate}
					\item \textbf{Verify that $a\conj{a}=\lVert a\rVert^2$.}
				\end{enumerate}
		\declaration{3.27}
			The notation \gls{complex1/a}, where $a$ is a complex number, will be used as a shorthand for the pair $\frac{1}{\lVert a\rVert^2}\conj{a}$.
		\procedure{3.81}
			\objective
				Choose a complex number $a$ such that $a\ne 0$. The objective of the following instructions is to show that $\frac{1}{a}a=1$.
			\implementation
				\begin{enumerate}
					\item Using \declarationhr{3.22}, verify that $\Re(a)\ne\Re(0)=0$ or $\Im(a)\ne\Im(0)=0$.
					\item Hence verify that $\lVert a\rVert^2=\Re(a)^2+\Im(a)^2>0$.
					\item Hence verify that $\frac{1}{a}a$
					\begin{enumerate}
						\item $=(\frac{1}{\Vert a\rVert^2}\conj{a})a$
						\item $=\frac{1}{\Vert a\rVert^2}(\conj{a}a)$
						\item $=\frac{1}{\Vert a\rVert^2}\Vert a\rVert^2$
						\item $=1$.
					\end{enumerate}
				\end{enumerate}
		\procedure{3.82}
			\objective
				Choose three complex numbers $a,b,c$. The objective of the following instructions is to show that $a(b+c)=ab+ac$.
			\implementation
				\begin{enumerate}
					\item $a(b+c)$
					\begin{enumerate}
						\item $=\langle\Re(a),\Im(a)\rangle\langle\Re(b)+\Re(c),\Im(b)+\Im(c)\rangle$
						\item $=\langle\Re(a)(\Re(b)+\Re(c))-\Im(a)(\Im(b)+\Im(c)),\Re(a)(\Im(b)+\Im(c))+\Im(a)(\Re(b)+\Re(c))\rangle$
						\item $=\langle(\Re(a)\Re(b)-\Im(a)\Im(b))+(\Re(a)\Re(c)-\Im(a)\Im(c)),(\Re(a)\Im(b)+\Im(a)\Re(b))+(\Re(a)\Im(c)+\Im(a)\Re(c))\rangle$
						\item $=\langle\Re(a)\Re(b)-\Im(a)\Im(b),\Re(a)\Im(b)+\Im(a)\Re(b)\rangle+\langle\Re(a)\Re(c)-\Im(a)\Im(c),\Re(a)\Im(c)+\Im(a)\Re(c)\rangle$
						\item $=ab+ac$.
					\end{enumerate}
				\end{enumerate}
		\declaration{3.28}
			The notation \gls{complexi} will be used as a shorthand for $\langle 0,1\rangle$.
		\declaration{3.05}
			The notation \gls{complexexp}, where $a$ is a complex number, will be used as a shorthand for $(1+\frac{a}{n})^n$.
		\procedure{3.07}
			\objective
				Choose a rational number $a$ and a positive integer $n$ such that $-n<a$. The objective of the following instructions is to show that $\exp_n(a)\ge 1+a$.
			\implementation
				\begin{enumerate}
					\item Using \procedurehr{2.07}, verify that $\exp_n(a)$
					\begin{enumerate}
						\item $=(1+\frac{a}{n})^n$
						\item $\ge 1+n\frac{a}{n}$
						\item $=1+a$.
					\end{enumerate}
				\end{enumerate}
		\procedure{3.08}
			\objective
				Choose a rational number $a$ and a positive integer $n$ such that $-n<a<1$. The objective of the following instructions is to show that $\exp_n(a)\le\frac{1}{1-a}$.
			\implementation
				\begin{enumerate}
					\item Using \procedurehr{2.07}, verify that $\exp_n(a)$
					\begin{enumerate}
						\item $=(\frac{n+a}{n})^n$
						\item $=(\frac{n}{n+a})^{-n}$
						\item $=\frac{1}{(1+\frac{-a}{n+a})^n}$
						\item $\le\frac{1}{1+\frac{-an}{n+a}}$
						\item $\le\frac{1}{1-a}$.
					\end{enumerate}
				\end{enumerate}
		\procedure{3.09}
			\objective
				Choose a rational number $a$ and a positive integer $n$ such that $a>-n$. The objective of the following instructions is to show that $\frac{\exp_{n+1}(a)}{\exp_n(a)}\ge 1$.
			\implementation
				\begin{enumerate}
					\item Using \procedurehr{2.07}, verify that $\frac{\exp_{n+1}(a)}{\exp_n(a)}$
					\begin{enumerate}
						\item $=\frac{(\frac{n+1+a}{n+1})^{n}}{(\frac{n+a}{n})^{n}}(1+\frac{a}{n+1})$
						\item $=(\frac{(n+1+a)n}{(n+1)(n+a)})^n(1+\frac{a}{n+1})$
						\item $=(\frac{n^2+n+na}{n^2+an+n+a})^n(1+\frac{a}{n+1})$
						\item $=(1-\frac{a}{(n+1)(n+a)})^n(1+\frac{a}{n+1})$
						\item $\ge(1-\frac{an}{(n+1)(n+a)})(1+\frac{a}{n+1})$
						\item $=1+\frac{a(n+a)}{(n+1)(n+a)}-\frac{an}{(n+1)(n+a)}-\frac{a^2n}{(n+1)^2(n+a)}$
						\item $=1+\frac{a^2}{(n+1)(n+a)}-\frac{a^2n}{(n+1)^2(n+a)}$
						\item $=1+\frac{a^2}{(n+1)^2(n+a)}$
						\item $\ge 1$
					\end{enumerate}
				\end{enumerate}
		\procedure{3.10}
			\objective
				Choose a rational number $X\ge 0$. The objective of the following instructions is to construct two rational numbers $a,b$, and a procedure, $p$, to show that $\exp_n(x)\le a$ when given a rational number $x$ and a positive integer $n\ge b$ such that $\lvert x\rvert\le X$.
			\implementation
				\begin{enumerate}
					\item Let $a=2^{\lceil 2X\rceil}$.
					\item Let $b=X$.
					\item Let $p(x,n)$ be the following procedure:
					\begin{enumerate}
						\item Verify that $\lvert x\rvert\le X$.
						\item Therefore verify that $-X\le x\le X$
						\item Therefore verify that $-1\le\frac{x}{n}\le 1$.
						\item Therefore verify that $0\le 1+\frac{x}{n}\le 2$.
						\item Hence using \procedurehr{3.08} and \procedurehr{3.09}, verify that $\exp_n(x)$
						\begin{enumerate}
							\item $\le\exp_n(X)$
							\item $\le(1+\frac{X}{\lceil 2X\rceil n})^{\lceil 2X\rceil n}$
							\item $=((1+\frac{\frac{X}{\lceil 2X\rceil}}{n})^n)^{\lceil 2X\rceil}$
							\item $=\exp_n(\frac{X}{\lceil 2X\rceil})^{\lceil 2X\rceil}$
							\item $\le(\frac{1}{1-\frac{X}{\lceil 2X\rceil}})^{\lceil 2X\rceil}$
							\item $\le 2^{\lceil 2X\rceil}$
							\item $=a$.
						\end{enumerate}
					\end{enumerate}
					\item \textbf{Yield the tuple $\langle a,b,p\rangle$.}
				\end{enumerate}
		\procedure{3.11}
			\objective
				Choose a rational number $X\le 0$. The objective of the following instructions is to construct two rational numbers $a>0,b$, and a procedure $p(x,n)$ to show that $\exp_n(x)\ge a$ when given a rational number $x$ and a positive integer $n>b$ such that $X\le x\le 0$.
			\implementation
				\begin{enumerate}
					\item Execute \procedurehr{3.10} on $\langle -2X\rangle$ and let $\langle c,d,q\rangle$ receive.
					\item Let $a=c^{-1}$.
					\item Let $b=\max(-2X,d)$.
					\item Let $p(x,n)$ be the following procedure:
					\begin{enumerate}
						\item Verify that $X\le x\le 0$.
						\item Therefore verify that $2X\le 2x\le 0$.
						\item Therefore verify that $0\le -2x\le -2X$.
						\item Hence execute procedure $q$ on $\langle -2x,n\rangle$.
						\item Therefore verify that $\exp_n(-2x)\le c$.
						\item Also verify that $n>b\ge -2X\ge -2x\ge 0$.
						\item Therefore verify that $-\frac{n}{2}\le x\le 0$.
						\item Therefore verify that $\frac{n}{2}\le n+x<n$.
						\item Hence verify that $\exp_n(x)$
						\begin{enumerate}
							\item $=(\frac{n+x}{n})^n$
							\item $=(\frac{n}{n+x})^{-n}$
							\item $=(1-\frac{x}{n+x})^{-n}$
							\item $\ge (1-\frac{x}{\frac{1}{2}n})^{-n}$
							\item $=(1-\frac{2x}{n})^{-n}$
							\item $=(\exp_n(-2x))^{-1}$
							\item $\ge c^{-1}$
							\item $=a$.
						\end{enumerate}
					\end{enumerate}
					\item \textbf{Yield the tuple $\langle a,b,p\rangle$.}
				\end{enumerate}
		\procedure{3.12}
			\objective
				Choose a rational number $X\ge 0$. The objective of the following instructions is to construct two rational numbers $a>0,b$, and a procedure $p(x,n)$ to show that $\exp_n(x)\ge a$ when given a rational number $x$ and a positive integer $n>b$ such that $\lvert x\rvert\le X$.
			\implementation
				\begin{enumerate}
					\item Execute \procedurehr{3.11} on $\langle -X\rangle$ and let $\langle c,b,q\rangle$ receive.
					\item Let $a=\min(1,c)$.
					\item Let $p(x,n)$ be the following procedure:
					\begin{enumerate}
						\item If $x<0$, then do the following:
						\begin{enumerate}
							\item Verify that $\lvert x\rvert\le X$.
							\item Therefore verify that $-X\le x\le 0$.
							\item Hence execute procedure $q$ on $x$.
							\item \textbf{Hence verify that $\exp_n(x)\ge c\ge a$.}
						\end{enumerate}
						\item Otherwise do the following:
						\begin{enumerate}
							\item Verify that $x\ge 0$.
							\item \textbf{Using \procedurehr{3.07}, verify that $\exp_n(x)\ge 1+x\ge 1\ge a$.}
						\end{enumerate}
					\end{enumerate}
					\item \textbf{Yield the tuple $\langle a,b,p\rangle$.}
				\end{enumerate}
		\procedure{3.13}
			\objective
				Choose a rational number $X\ge 0$. The objective of the following instructions is to construct rational numbers $a,b$ and a procedure, $p(x,n)$, to show that $\lVert\exp_n(x)\rVert^2\le a$ when a complex number $x$ and a positive integer $n>b$ such that $\lVert x\rVert^2\le X$ are chosen.
			\implementation
				\begin{enumerate}
					\item Let $c=2\max(1,X)+X$.
					\item Execute \procedurehr{3.10} on $\langle c\rangle$ and let $\langle a,b,q\rangle$.
					\item Let $p(x,n)$ be the following procedure:
					\begin{enumerate}
						\item Verify that $n>b$.
						\item Let $y=2\lvert\Re(x)\rvert+\lVert x\rVert^2$.
						\item Verify that $\lvert\Re(x)\rvert^2\le\lVert x\rVert^2\le X$.
						\item Therefore verify that $\lvert\Re(x)\rvert\le\max(1,X)$.
						\item Therefore verify that $\lvert y\rvert=y\le 2\max(1,X)+X=c$.
						\item Hence execute procedure $q$ on $\langle y,n\rangle$.
						\item Hence verify that $\exp_n(y)\le a$.
						\item Now using \procedurehr{3.02} verify that $\lVert\exp_n(x)\rVert^2$
						\begin{enumerate}
							\item $=\exp_n(x)\conj{\exp_n(x)}$
							\item $=(1+\frac{x}{n})^n(1+\frac{\conj{x}}{n})^n$
							\item $=(1+\frac{2\Re(x)}{n}+\frac{\lVert x\rVert^2}{n^2})^n$
							\item $\le(1+\frac{2\lvert\Re(x)\rvert}{n}+\frac{\lVert x\rVert^2}{n^2})^n$
							\item $\le(1+\frac{2\lvert\Re(x)\rvert+\lVert x\rVert^2}{n})^n$
							\item $=\exp_n(y)$
							\item $\le a$.
						\end{enumerate}
					\end{enumerate}
					\item \textbf{Yield the tuple $\langle a,b,p\rangle$.}
				\end{enumerate}
		\procedure{3.14}
			\objective
				Choose a rational number $X\ge 0$. The objective of the following instructions is to construct two rational numbers $a,b$ and a procedure, $p(x,n)$, to show that $\lVert\exp_n(x)\rVert^2\ge a$ when a rational number $x$ and a positive integer $n>b$ such that $\lVert x\rVert^2\le X$ are chosen.
			\implementation
				\begin{enumerate}
					\item Let $c=2\max(1,X)+X$.
					\item Execute \procedurehr{3.12} on $\langle c\rangle$ and let $\langle a,d,q\rangle$ receive.
					\item Let $b=\max(c,d)$.
					\item Let $p(x,n)$ be the following procedure:
					\begin{enumerate}
						\item Verify that $n>b\ge d$.
						\item Let $y=2\lvert\Re(x)\rvert+\lVert x\rVert^2$.
						\item Verify that $\lvert -y\rvert=y\le 2\max(1,X)+X=c$.
						\item Hence execute procedure $q$ on $\langle -y,n\rangle$.
						\item Hence verify that $\exp_n(-y)\ge a$.
						\item Also, verify that $n>b\ge c\ge y$.
						\item Hence verify that $\lVert\exp_n(x)\rVert^2$
						\begin{enumerate}
							\item $=\exp_n(x)\conj{\exp_n(x)}$
							\item $=(1+\frac{x}{n})^n(1+\frac{\conj{x}}{n})^n$
							\item $=(1+\frac{2\Re(x)}{n}+\frac{\lVert x\rVert^2}{n^2})^n$
							\item $\ge(1-\frac{2\lvert\Re(x)\rvert}{n}-\frac{\lVert x\rVert^2}{n^2})^n$
							\item $\ge(1-\frac{2\lvert\Re(x)\rvert+\lVert x\rVert^2}{n})^n$
							\item $=\exp_n(-y)$
							\item $\ge a$.
						\end{enumerate}
					\end{enumerate}
					\item \textbf{Yield the tuple $\langle a,b,p\rangle$.}
				\end{enumerate}
		\procedure{3.15}
			\objective
				Choose a rational number $X\ge 0$. The objective of the following instructions is to construct rational numbers $a,b$ such that $a>0$, and a procedure, $p$, to show that $\lVert\exp_n(x)\exp_n(y)-\exp_n(x+y)\rVert^2\le\frac{a\lVert xy\rVert^2}{n^2}$ when two complex numbers $x,y$ and a positive integer $n>b$ such that $2(\lVert x\rVert^2+\lVert y\rVert^2)\le X$ are chosen.
			\implementation
				\begin{enumerate}
					\item Execute \procedurehr{3.13} on $\langle X\rangle$ and let $\langle c,b,q\rangle$ receive.
					\item Let $a=\max(1,c)^3$.
					\item Let $p(x,y,n)$ be the following procedure:
					\begin{enumerate}
						\item Verify that $\lVert x\rVert^2\le X$.
						\item Hence execute $q$ on $\langle x,n\rangle$.
						\item Hence verify that $\lVert\exp_n(x)\rVert^2\le c$.
						\item Verify that $\lVert y\rVert^2\le X$.
						\item Hence execute $q$ on $\langle y,n\rangle$.
						\item Hence verify that $\lVert\exp_n(y)\rVert^2\le c$.
						\item Verify that $\lVert x+y\rVert^2\le 2(\lVert x\rVert^2+\lVert y\rVert^2)\le X$.
						\item Hence execute $q$ on $\langle x+y,n\rangle$.
						\item Hence verify that $\lVert\exp_n(x+y)\rVert^2\le c$.
						\item Hence using \procedurehr{3.04}, verify that $\lVert\exp_n(x)\exp_n(y)-\exp_n(x+y)\rVert^2$
						\begin{enumerate}
							\item $=\lVert(1+\frac{x}{n})^n(1+\frac{y}{n})^n-(1+\frac{x+y}{n})^n\rVert^2$
							\item $=\lVert(1+\frac{x+y}{n}+\frac{xy}{n^2})^n-(1+\frac{x+y}{n})^n\rVert^2$
							\item $=\lVert\frac{xy}{n^2}\sum_r^{[0:n]}(1+\frac{x+y}{n}+\frac{xy}{n^2})^r(1+\frac{x+y}{n})^{n-1-r}\rVert^2$
							\item $=\frac{\lVert xy\rVert^2}{n^4}\lVert\sum_r^{[0:n]}(1+\frac{x}{n})^r(1+\frac{y}{n})^r(1+\frac{x+y}{n})^{n-1-r}\rVert^2$
							\item $=\frac{\lVert xy\rVert^2}{n^3}\sum_r^{[0:n]}\lVert 1+\frac{x}{n}\rVert^{2r}\lVert 1+\frac{y}{n}\rVert^{2r}\lVert 1+\frac{x+y}{n}\rVert^{2(n-1-r)}$
							\item $\le\frac{\lVert xy\rVert^2}{n^3}\sum_r^{[0:n]}\max(1,\lVert\exp_n(x)\rVert^{2})\allowbreak\max(1,\lVert\exp_n(y)\rVert^{2})\allowbreak\max(1,\lVert\exp_n(x+y)\rVert^{2})$
							\item $\le\frac{\lVert xy\rVert^2}{n^3}\sum_r^{[0:n]}\max(1,c)^3$
							\item $=\frac{\lVert xy\rVert^2\max(1,c)^3}{n^2}$
							\item $=\frac{a\lVert xy\rVert^2}{n^2}$.
						\end{enumerate}
					\end{enumerate}
					\item \textbf{Yield the tuple $\langle a,b,p\rangle$.}
				\end{enumerate}
		\procedure{3.16}
			\objective
				Choose a rational number $X\ge 0$ and an integer $K\ge 0$. The objective of the following instructions is to construct rational numbers $a,b$ and a procedure, $p(x,k,n)$, to show that $\lVert\exp_n(kx)-\exp_n(x)^k\rVert^2\le\frac{a\lVert x\rVert^4}{n^2}$ when a complex number $x$, a non-negative integer $k$, and a positive integer $n$ such that $n>b$, $k\le K$, $\lVert x\rVert^2\le X$ are chosen.
			\implementation
				\begin{enumerate}
					\item Execute \procedurehr{3.13} on $\langle X\rangle$ and let $\langle c,d,q\rangle$ receive.
					\item Execute \procedurehr{3.15} on $\langle 2K^2X\rangle$ and let $\langle e,f,t\rangle$ receive.
					\item Let $a=e\max(1,c)^KK^4$
					\item Let $b=\max(d,f)$.
					\item Let $p(x,k,n)$ be the following procedure:
					\begin{enumerate}
						\item Execute procedure $q$ on $\langle x,n\rangle$.
						\item Hence verify that $\lVert\exp_n(x)\rVert^2\le c$.
						\item For $r$ in $[0:k]$, do the following:
						\begin{enumerate}
							\item Verify that $2(\lVert x\rVert^2+\lVert(k-r-1)x\rVert^2)$
							\begin{enumerate}
								\item $=2(\lVert x\rVert^2+(k-r-1)^2\lVert x\rVert^2)$
								\item $\le 2(\lVert x\rVert^2+(k-1)^2\lVert x\rVert^2)$
								\item $=2(k^2-2k+2)\lVert x\rVert^2$
								\item $\le 2k^2\lVert x\rVert^2$
								\item $\le 2K^2X$.
							\end{enumerate}
							\item Now execute procedure $t$ on $\langle x, (k-r-1)x,n\rangle$.
							\item Hence verify that $\lVert\exp_n(x)\exp_n((k-r-1)x)-\exp_n((k-r)x)\rVert^2\le\frac{e\lVert x\rVert^2\lVert(k-r-1)x\rVert^2}{n^2}\le\frac{ek^2\lVert x\rVert^4}{n^2}$.
						\end{enumerate}
						\item Hence using (ciii), verify that $\lVert\exp_n(kx)-\exp_n(x)^k\rVert^2$
						\begin{enumerate}
							\item $=\lVert\sum_r^{[0:k]}(\exp_n(x)^r\exp_n((k-r)x)-\exp_n(x)^{r+1}\exp_n((k-r-1)x))\rVert^2$
							\item $\le k\sum_r^{[0:k]}\lVert\exp_n(x)^r\exp_n((k-r)x)-\exp_n(x)^{r+1}\exp_n((k-r-1)x)\rVert^2$
							\item $=k\sum_r^{[0:k]}\lVert\exp_n(x)\rVert^{2r}\lVert\exp_n((k-r)x)-\exp_n(x)\exp_n((k-r-1)x)\rVert^2$
							\item $\le k\sum_r^{[0:k]}\max(1,c)^k\cdot\frac{ek^2\lVert x\rVert^4}{n^2}$
							\item $=\max(1,c)^k\cdot\frac{ek^4\lVert x\rVert^4}{n^2}$
							\item \textbf{$\le\frac{a\lVert x\rVert^4}{n^2}$.}
						\end{enumerate}
					\end{enumerate}
					\item \textbf{Yield the tuple $\langle a,b,p\rangle$.}
				\end{enumerate}
		\declaration{3.29}
			The notation \gls{complexdiff}, where $x,z$ are complex numbers such that $z\ne 0$ and $f[x]$ is a function of $x$, will be used as a shorthand for $\frac{f(y+z)-f(y)}{z}$.
		\procedure{3.83}
			\objective
				Choose two functions $f[x],g[x]$ and two complex numbers $y,z$ such that $z\ne 0$. The objective of the following instructions is to show that $\diff_{x=y}^z(f(x)+g(x))=\diff_{x=y}^zf(x)+\diff_{x=y}^zg(x)$.
			\implementation
				\begin{enumerate}
					\item Verify that $\diff_{x=y}^z(f(x)+g(x))$
					\begin{enumerate}
						\item $=\frac{(f(y+z)+g(y+z))-(f(y)+g(y))}{z}$
						\item $=\frac{f(y+z)-f(y)}{z}+\frac{g(y+z)-g(y)}{z}$
						\item $=\diff_{x=y}^zf(x)+\diff_{x=y}^zg(x)$.
					\end{enumerate}
				\end{enumerate}
		\procedure{3.84}
			\objective
				Choose a functions $f[x]$ and complex numbers $a,y,z$ such that $z\ne 0$. The objective of the following instructions is to show that $\diff_{x=y}^z(af(x))=a\diff_{x=y}^zf(x)$.
			\implementation
				\begin{enumerate}
					\item Verify that $\diff_{x=y}^z(af(x))$
					\begin{enumerate}
						\item $=\frac{af(y+z)-af(y)}{z}$
						\item $=a\frac{f(y+z)-f(y)}{z}$
						\item $=a\diff_{x=y}^zf(x)$.
					\end{enumerate}
				\end{enumerate}
		\procedure{3.17}
			\objective
				Choose a complex number $x$. The objective of the following instructions is to show that $\lvert\Re(x)\rvert\le\max(1,\lVert x\rVert^2)$.
			\implementation
				\begin{enumerate}
					\item If $\lvert\Re(x)\rvert<1$, then do the following:
					\begin{enumerate}
						\item \textbf{Verify that $\lvert\Re(x)\rvert<1\le\max(1,\lVert x\rVert^2)$.}
					\end{enumerate}
					\item Otherwise do the following:
					\begin{enumerate}
						\item \textbf{Verify that $\lvert\Re(x)\rvert\le\Re(x)^2\le\lVert x\rVert^2\le\max(1,\lVert x\rVert^2)$.}
					\end{enumerate}
				\end{enumerate}
		\procedure{3.18}
			\objective
				Choose a rational number $D\ge 0$. The objective of the following instructions is to construct two rational numbers $a,c$ and a procedure, $p(dx,n)$, to show that $\lVert\diff_{x=0}^{+dx}\exp_n(x)-1\rVert^2\le a\lVert dx\rVert^2$ when a complex number $dx$ and a positive integer $n>c$ such that $\lVert dx\rVert^2\le D$ is chosen.
			\implementation
				\begin{enumerate}
					\item Let $e=2\max(1,D)+D$.
					\item Execute \procedurehr{3.10} on $\langle e\rangle$ and let $\langle d,c,q\rangle$ receive.
					\item Let $a=\max(1,d)$.
					\item Let $p(dx,n)$ be the following procedure:
					\begin{enumerate}
						\item Verify that $n>c$.
						\item Hence execute procedure $q$ on $\langle D,n\rangle$.
						\item Hence verify that $\exp_n(D)\le d$.
						\item Now using \procedurehr{2.04}, \procedurehr{3.04} and \procedurehr{3.17}, verify that $\lVert\exp_n(dx)-1-dx\rVert^2$
						\begin{enumerate}
							\item $=\lVert(1+\frac{dx}{n})^n-1-dx\rVert^2$
							\item $=\lVert\frac{dx}{n}\sum_r^{[0:n]}(1+\frac{dx}{n})^r-n\frac{dx}{n}\rVert^2$
							\item $=\lVert\frac{dx}{n}\sum_r^{[0:n]}((1+\frac{dx}{n})^r-1)\rVert^2$
							\item $=\lVert\frac{dx}{n}\sum_r^{[0:n]}\frac{dx}{n}\sum_k^{[0:r]}(1+\frac{dx}{n})^k\rVert^2$
							\item $=\frac{\lVert dx\rVert^4}{n^4}\lVert\sum_r^{[0:n]}\sum_k^{[0:r]}(1+\frac{dx}{n})^k\rVert^2$
							\item $\le\frac{\lVert dx\rVert^4}{n^2}\sum_r^{[0:n]}\sum_k^{[0:r]}\lVert 1+\frac{dx}{n}\rVert^{2k}$
							\item $\le\frac{\lVert dx\rVert^4}{n^2}\sum_r^{[0:n]}\sum_k^{[0:r]}\max(1,\lVert 1+\frac{dx}{n}\rVert^{2n})$
							\item $=\frac{\lVert dx\rVert^4}{n^2}\sum_r^{[0:n]}\sum_k^{[0:r]}\max(1,(1+\frac{2\Re(dx)}{n}+\frac{\lVert dx\rVert^2}{n^2})^{n})$
							\item $\le\frac{\lVert dx\rVert^4}{n^2}\sum_r^{[0:n]}\sum_k^{[0:r]}\max(1,(1+\frac{2\lvert\Re(dx)\rvert}{n}+\frac{\lVert dx\rVert^2}{n^2})^{n})$
							\item $\le\frac{\lVert dx\rVert^4}{n^2}\sum_r^{[0:n]}\sum_k^{[0:r]}\max(1,(1+\frac{e}{n})^{n})$
							\item $\le\frac{\lVert dx\rVert^4}{n^2}\sum_r^{[0:n]}\sum_k^{[0:r]}\max(1,d)$
							\item $\le a\lVert dx\rVert^4$.
						\end{enumerate}
						\item \textbf{Therefore verify that $\lVert\diff_{x=0}^{+dx}\exp_n(x)-1\rVert^2\le a\lVert dx\rVert^2$.}
					\end{enumerate}
					\item \textbf{Yield the tuple $\langle a,c,p\rangle$.}
				\end{enumerate}
		\procedure{3.19}
			\objective
				Choose a rational number $X\ge 0$. The objective of the following instructions is to construct three rational numbers $a,b,d$, and a procedure, $p(x,dx,n)$, to show that $\lVert\diff_{y=x}^{+dx}\exp_n(y)-\exp_n(x)\rVert^2\le\frac{a}{n^2}+b\lVert dx\rVert^2$ when two complex numbers $x,dx$ and a positive integer $n>d$ such that $2(\lVert x\rVert^2+\lVert dx\rVert^2)\le X$ are chosen. 
			\implementation
				\begin{enumerate}
					\item Execute \procedurehr{3.15} on $\langle X\rangle$ and let $\langle e,f,q\rangle$ receive.
					\item Execute \procedurehr{3.18} on $\langle X\rangle$ and let $\langle h,j,r\rangle$ receive.
					\item Execute \procedurehr{3.13} on $\langle X\rangle$ and let $\langle l,m,t\rangle$ receive.
					\item Let $a=2eX$.
					\item Let $b=2lh$.
					\item Let $d=\max(f,j,m)$.
					\item Let $p(x,dx,n)$ be the following procedure:
					\begin{enumerate}
						\item Verify that $n>d\ge f$.
						\item Hence execute procedure $q$ on $\langle x,dx,n\rangle$.
						\item Therefore verify that $\lVert\exp_n(x)\exp_n(dx)-\exp_n(x+dx)\rVert^2\le\frac{e\lVert xdx\rVert^2}{n^2}$.
						\item Verify that $n>d\ge j$.
						\item Execute procedure $r$ on $\langle dx,n\rangle$.
						\item Therefore verify that $\lVert\frac{\exp_n(dx)-1}{dx}-1\rVert^2\le h\lVert dx\rVert^2$.
						\item Verify that $n>d\ge m$.
						\item Execute procedure $t$ on $\langle x,n\rangle$.
						\item Therefore verify that $\lVert\exp_n(x)\rVert^2\le l$.
						\item Using \procedurehr{3.04}, verify that $\lVert\exp_n(x+dx)-\exp_n(x)-dx\exp_n(x)\rVert^2$
						\begin{enumerate}
							\item $=\lVert\exp_n(x+dx)-\exp_n(x)\exp_n(dx)+\exp_n(x)\exp_n(dx)-\exp_n(x)-dx\exp_n(x)\rVert^2$
							\item $\le 2\lVert\exp_n(x+dx)-\exp_n(x)\exp_n(dx)\rVert^2+2\lVert\exp_n(x)\rVert^2\lVert\exp_n(dx)-1-dx\rVert^2$
							\item $\le\frac{2e\lVert x\rVert^2\lVert dx\rVert^2}{n^2}+2lh\lVert dx\rVert^4$
							\item $\le\frac{2eX\lVert dx\rVert^2}{n^2}+2lh\lVert dx\rVert^4$
						\end{enumerate}
						\item \textbf{Therefore verify that $\lVert\diff_{y=x}^{+dx}\exp_n(y)-\exp_n(x)\rVert^2\le\frac{a}{n^2}+b\lVert dx\rVert^2$.}
					\end{enumerate}
					\item \textbf{Yield the tuple $\langle a,b,d,p\rangle$.}
				\end{enumerate}
		\procedure{3.20}
			\objective
				Choose two rational numbers $X\ge 0$. The objective of the following instructions is to construct two rational numbers $a,c$ and a procedure, $p(x,dx,n)$, to show that $\lVert\diff_{y=x}^{+dx}\exp_n(y)\rVert^2\le a$ when two rational numbers $x,dx$ and a positive integer $n>c$ such that $2(\lVert x\rVert^2+\lVert dx\rVert^2)\le X$ are chosen.
			\implementation
				\begin{enumerate}
					\item Execute \procedurehr{3.19} on $\langle X\rangle$ and let $\langle d,e,f,q\rangle$ receive.
					\item Execute \procedurehr{3.13} on $\langle X\rangle$ and let $\langle h,j,r\rangle$ receive.
					\item Let $a=(2d+eX+2h)$.
					\item Let $c=\max(j,f)$.
					\item Let $p(x,dx,n)$ be the following procedure:
					\begin{enumerate}
						\item Verify that $n>c\ge f$.
						\item Execute procedure $q$ on $\langle x,dx,n\rangle$.
						\item Hence verify that $\lVert\frac{\exp_n(x+dx)-\exp_n(x)}{dx}-\exp_n(x)\rVert^2\le\frac{d}{n^2}+e\lVert dx\rVert^2$.
						\item Verify that $n>c\ge j$.
						\item Execute procedure $r$ on $\langle x,n\rangle$.
						\item Hence verify that $\lVert\exp_n(x)\rVert^2\le h$.
						\item Now using \procedurehr{3.04}, verify that $\lVert\exp_n(x+dx)-\exp_n(x)\rVert^2$
						\begin{enumerate}
							\item $=\lVert\exp_n(x+dx)-\exp_n(x)-dx\exp_n(x)+dx\exp_n(x)\rVert^2$
							\item $\le 2\lVert\exp_n(x+dx)-\exp_n(x)-dx\exp_n(x)\rVert^2+2\lVert dx\rVert^2\lVert\exp_n(x)\rVert^2$
							\item $\le 2d\frac{\lVert dx\rVert^2}{n^2}+2e\lVert dx\rVert^4+2h\lVert dx\rVert^2$
							\item $\le (2d+eX+2h)\lVert dx\rVert^2$.
						\end{enumerate}
						\item Therefore verify that $\lVert\diff_{y=x}^{+dx}\exp_n(y)\rVert^2\le a$.
					\end{enumerate}
					\item \textbf{Yield the tuple $\langle a,c,p\rangle$.}
				\end{enumerate}
		\declaration{3.15}
			The notation \gls{complexisqrt}, where $x$ is a non-negative rational number, will be used as a shorthand for the result of executing the following instructions:
			\begin{enumerate}
				\item Verify that $\lceil x\rceil^2>x$.
				\item \textbf{Yield the smallest integer $r$ in $[0:\lceil x\rceil+1]$ such that $r^2>x$.}
			\end{enumerate}
		\procedure{3.21}
			\objective
				Choose a rational number $X\ge 0$. The objective of the following instructions is to construct two rational numbers $a,N$, and a procedure, $p(x,n)$, to show that $\lVert\sum_r^{[0:{n+1}]}\frac{x^r}{r!}-\exp_n(x)\rVert^2\le\frac{a\lVert x\rVert^4}{n}$ when a complex number $x$ and an integer $n>N$ such that $\lVert x\rVert^2\le X$ are chosen.
			\implementation
				\begin{enumerate}
					\item Let $N=\lceil\sqrt{2X}\rceil$.
					\item Let $a=\sum_r^{[0:{N}]}\frac{X^{r}}{r!^2}+2\frac{X^{N}}{N!^2}$.
					\item Let $p(x,n)$ be the following procedure:
					\begin{enumerate}
						\item Using \procedurehr{2.06}, \procedurehr{3.04}, \procedurehr{2.05}, and \procedurehr{2.07}, verify that $\lVert\sum_r^{[0:{n+1}]}\frac{x^r}{r!}-\exp_n(x)\rVert^2$
						\begin{enumerate}
							\item $=\lVert\sum_r^{[0:{n+1}]}\frac{x^r}{r!}-\sum_r^{[0:{n+1}]}\frac{n^{\ul{r}}}{r!}\cdot\frac{x^r}{n^r}\rVert^2$
							\item $=\lVert\sum_r^{[1:{n+1}]}(1-\frac{n^{\ul{r}}}{n^r})\frac{x^r}{r!}\rVert^2$
							\item $\le n\sum_r^{[1:{n+1}]}\lVert 1-\frac{n^{\ul{r}}}{n^r}\rVert^2\lVert\frac{x^r}{r!}\rVert^2$
							\item $\le n\sum_r^{[2:{n+1}]}\lVert 1-\frac{(n-r+1)^{r}}{n^r}\rVert^2\lVert\frac{x^r}{r!}\rVert^2$
							\item $\le n\sum_r^{[2:{n+1}]}\lVert 1-(1-\frac{r-1}{n})^{r}\rVert^2\lVert\frac{x^r}{r!}\rVert^2$
							\item $\le n\sum_r^{[2:{n+1}]}\lVert 1-(1-\frac{(r-1)r}{n})\rVert^2\lVert\frac{x^r}{r!}\rVert^2$
							\item $=n\sum_r^{[2:{n+1}]}\lVert \frac{(r-1)r}{n}\rVert^2\lVert\frac{x^r}{r!}\rVert^2$
							\item $=\frac{1}{n}\sum_r^{[2:{n+1}]}\lVert\frac{x^r}{(r-2)!}\rVert^2$
							\item $=\frac{\lVert x\rVert^4}{n}\sum_r^{[0:{n-1}]}\frac{\lVert x\rVert^{2r}}{r!^2}$
							\item $\le\frac{\lVert x\rVert^4}{n}\sum_r^{[0:{n-1}]}\frac{X^{r}}{r!^2}$
							\item $=\frac{\lVert x\rVert^4}{n}(\sum_r^{[0:{N}]}\frac{X^{r}}{r!^2}+\sum_r^{[N:{n-1}]}\frac{X^{r}}{r!^2})$
							\item $\le\frac{\lVert x\rVert^4}{n}(\sum_r^{[0:{N}]}\frac{X^{r}}{r!^2}+\frac{X^{N}}{N!^2}\sum_r^{[0:{n-N-1}]}\frac{X^{r}}{N^{2r}})$
							\item $\le\frac{\lVert x\rVert^4}{n}(\sum_r^{[0:{N}]}\frac{X^{r}}{r!^2}+\frac{X^{N}}{N!^2}\sum_r^{[0:{n-N-1}]}(\frac{1}{2})^r)$
							\item $\le\frac{\lVert x\rVert^4}{n}(\sum_r^{[0:{N}]}\frac{X^{r}}{r!^2}+2\frac{X^{N}}{N!^2})$
							\item $\le\frac{a\lVert x\rVert^4}{n}$
						\end{enumerate}
					\end{enumerate}
					\item \textbf{Yield the tuple $\langle a,N,p\rangle$.}
				\end{enumerate}
		\declaration{3.17}
			The notation \gls{complexcos}, where $z$ is a complex number and $n$ is a positive integer, will be used as a shorthand for $\frac{\exp_n(iz)+\exp_n(-iz)}{2}$.
		\procedure{3.22}
			\objective
				Choose a rational number $x$ and a positive integer $n$. The objective of the following instructions is to show that $\Re(\exp_n(ix))=\cos_n(x)$.
			\implementation
				\begin{enumerate}
					\item Verify that $\Re(\exp_n(ix))$
					\begin{enumerate}
						\item $=\frac{\exp_n(ix)+\conj{\exp_n(ix)}}{2}$
						\item $=\frac{\exp_n(ix)+\exp_n(\conj{ix})}{2}$
						\item $=\frac{\exp_n(ix)+\exp_n(-ix)}{2}$
						\item $=\cos_n(x)$.
					\end{enumerate}
				\end{enumerate}
		\declaration{3.18}
			The notation \gls{complexsin}, where $z$ is a complex number and $n$ is a positive integer, will be used as a shorthand for $\frac{\exp_n(iz)-\exp_n(-iz)}{2i}$.
		\procedure{3.23}
			\objective
				Choose a rational number $x$ and a positive integer $n$. The objective of the following instructions is to show that $\Im(\exp_n(ix))=\sin_n(x)$.
			\implementation
				\begin{enumerate}
					\item Verify that $\Im(\exp_n(ix))$
					\begin{enumerate}
						\item $=\frac{\exp_n(ix)-\conj{\exp_n(ix)}}{2i}$
						\item $=\frac{\exp_n(ix)-\exp_n(\conj{ix})}{2i}$
						\item $=\frac{\exp_n(ix)-\exp_n(-ix)}{2i}$
						\item $=\sin_n(x)$.
					\end{enumerate}
				\end{enumerate}
		\procedure{3.24}
			\objective
				Choose a rational number $X\ge 0$. The objective of the following instructions is to construct two rational numbers $a,b$, and a procedure, $p(x,y,n)$, to show that $\lVert\cos_n(x)\cos_n(y)-\sin_n(x)\sin_n(y)-\cos_n(x+y)\rVert^2\le\frac{a\lVert xy\rVert^2}{n^2}$ when two complex numbers $x,y$ and a positive integer $n>b$ such that $2(\lVert x\rVert^2+\lVert y\rVert^2)\le X$ are chosen.
			\implementation
				\begin{enumerate}
					\item Execute \procedurehr{3.15} on $\langle X\rangle$ and let $\langle a,b,q\rangle$ receive.
					\item Let $p(x,y,n)$ be the following procedure:
					\begin{enumerate}
						\item Verify that $2(\lVert ix\rVert^2+\lVert iy\rVert^2)=2(\lVert x\rVert^2+\lVert y\rVert^2)\le X$.
						\item Execute procedure $q$ on $\langle ix,iy,n\rangle$.
						\item Hence verify that $\lVert\exp_n(ix)\exp_n(iy)-\exp_n(ix+iy)\rVert^2\le\frac{a\lVert i^2xy\rVert^2}{n^2}=\frac{a\lVert xy\rVert^2}{n^2}$.
						\item Verify that $2(\lVert -ix\rVert^2+\lVert -iy\rVert^2)=2(\lVert x\rVert^2+\lVert y\rVert^2)\le X$.
						\item Execute procedure $q$ on $\langle -ix,-iy,n\rangle$.
						\item Hence verify that $\lVert\exp_n(-ix)\exp_n(-iy)-\exp_n(-ix-iy)\rVert^2\le\frac{a\lVert (-i)^2xy\rVert^2}{n^2}=\frac{a\lVert xy\rVert^2}{n^2}$.
						\item Using \procedurehr{3.04}, verify that $\lVert\cos_n(x)\cos_n(y)-\sin_n(x)\sin_n(y)-\cos_n(x+y)\rVert^2$
						\begin{enumerate}
							\item $=\lVert\frac{(\exp_n(ix)+\exp_n(-ix))(\exp_n(iy)+\exp_n(-iy))}{4}-\frac{\exp_n(ix)-\exp_n(-ix))(\exp_n(iy)-\exp_n(-iy))}{4i^2}-\cos_n(x+y)\rVert^2$
							\item $=\lVert\frac{\exp_n(ix)\exp_n(iy)}{2}+\frac{\exp_n(-ix)\exp_n(-iy)}{2}-\frac{\exp_n(i(x+y))+\exp_n(-i(x+y))}{2}\rVert^2$
							\item $\le\frac{\lVert\exp_n(ix)\exp_n(iy)-\exp_n(i(x+y))\rVert^2}{2}+\frac{\lVert\exp_n(-ix)\exp_n(-iy)-\exp_n(-i(x+y))\rVert^2}{2}$
							\item $\le\frac{a\lVert xy\rVert^2}{n^2}$.
						\end{enumerate}
					\end{enumerate}
					\item \textbf{Yield the tuple $\langle a,b,p\rangle$.}
				\end{enumerate}
		\procedure{3.25}
			\objective
				Choose a rational number $X\ge 0$. The objective of the following instructions is to construct two rational numbers $a,b$, and a procedure, $p(x,y,n)$, to show that $\lVert\sin_n(x)\cos_n(y)-\cos_n(x)\sin_n(y)-\sin_n(x+y)\rVert^2\le\frac{a\lVert xy\rVert^2}{n^2}$ when two complex numbers $x,y$ and a positive integer $n>b$ such that $2(\lVert x\rVert^2+\lVert y\rVert^2)\le X$ are chosen.
			\implementation
				Implementation is analogous to that of \procedurehr{3.24}.
		\procedure{3.26}
			\objective
				 Choose a rational number $X\ge 0$. The objective of the following instructions is to construct two rational numbers $a,b$, and a procedure, $p$, to show that $\lVert\cos_n(x)^2+\sin_n(x)^2-1\rVert^2\le\frac{a\lVert x\rVert^4}{n^2}$ when a complex number $x$ and a positive integer $n$ such that $4\lVert x\rVert^2\le X$ are chosen.
			\implementation
				\begin{enumerate}
					\item Execute \procedurehr{3.15} on $\langle X\rangle$ and let $\langle a,b,q\rangle$ receive.
					\item Let $p(x,n)$ be the following procedure:
					\begin{enumerate}
						\item Verify that $2(\lVert ix\rVert^2+\lVert -ix\rVert^2)=4\lVert X\rVert^2\le X$.
						\item Execute procedure $q$ on $\langle ix,-ix,n\rangle$.
						\item Hence verify that $\lVert\exp_n(ix)\exp_n(-ix)-\exp_n(ix-ix)\rVert^2\le\frac{a\lVert-i^2x^2\rVert^2}{n^2}=\frac{a\lVert x\rVert^4}{n^2}$.
						\item Hence verify that $\lVert\cos_n(x)^2+\sin_n(x)^2-1\rVert^2$
						\begin{enumerate}
							\item $=\lVert\frac{(\exp_n(ix)+\exp_n(-ix))^2}{4}+\frac{(\exp_n(ix)-\exp_n(-ix))^2}{4i^2}-1\rVert^2$
							\item $=\lVert\exp_n(ix)\exp_n(-ix)-1\rVert^2$
							\item $=\lVert\exp_n(ix)\exp_n(-ix)-\exp_n(ix-ix)\rVert^2$
							\item $\le\frac{a\lVert x\rVert^4}{n^2}$.
						\end{enumerate}
					\end{enumerate}
					\item \textbf{Yield the tuple $\langle a,b,p\rangle$.}
				\end{enumerate}
		\procedure{3.27}
			\objective
				Choose a rational number $X\ge 0$. The objective of the following instructions is to construct three rational numbers $a,b,d$ and a procedure, $p(x,dx,n)$, to show that $\lVert\diff_{y=x}^{+dx}\cos_n(y)+\sin_n(x)\rVert^2\le\frac{a}{n^2}+b\lVert dx\rVert^2$ when two complex numbers $x,dx$ and a positive integer $n>d$ such that $2(\lVert x\rVert^2+\lVert dx\rVert^2)\le X$ are chosen.
			\implementation
				\begin{enumerate}
					\item Execute \procedurehr{3.19} on $\langle X\rangle$ and let $\langle a,b,d,q\rangle$.
					\item Let $p(x,dx,n)$ be the following procedure:
					\begin{enumerate}
						\item Verify that $2(\lVert ix\rVert^2+\lVert idx\rVert^2)=2(\lVert x\rVert^2+\lVert dx\rVert^2)\le X$.
						\item Verify that $\lVert idx\rVert^2=\lVert dx\rVert^2\le X$.
						\item Hence execute procedure $q$ on $\langle ix,idx,n\rangle$.
						\item Hence verify that $\lVert\frac{\exp_n(ix+idx)-\exp_n(ix)}{idx}-\exp_n(ix)\rVert^2\le\frac{a}{n^2}+b\lVert idx\rVert^2$.
						\item Verify that $2(\lVert -ix\rVert^2+\lVert -idx\rVert^2)=2(\lVert x\rVert^2+\lVert dx\rVert^2)\le X$.
						\item Verify that $\lVert -idx\rVert^2=\lVert dx\rVert^2\le X$.
						\item Hence execute procedure $q$ on $\langle -ix,-idx,n\rangle$.
						\item Hence verify that $\lVert\frac{\exp_n(-ix-idx)-\exp_n(-ix)}{-idx}-\exp_n(-ix)\rVert^2\le\frac{a}{n^2}+b\lVert -idx\rVert^2$.
						\item Using \procedurehr{3.04}, verify that $\lVert\cos_n(x+dx)-\cos_n(x)+\sin_n(x)dx\rVert^2$
						\begin{enumerate}
							\item $=\lVert\frac{\exp_n(i(x+dx))+\exp_n(-i(x+dx))}{2}-\frac{\exp_n(ix)+\exp_n(-ix)}{2}+\frac{dx\exp_n(ix)-dx\exp_n(-ix)}{2i}\rVert^2$
							\item $\le\frac{\lVert\exp_n(ix+idx)-\exp_n(ix)-idx\exp_n(ix)\rVert^2}{2}+\frac{\lVert\exp_n(-ix-idx)-\exp_n(-ix)-(-idx)\exp_n(-ix)\rVert^2}{2}$
							\item $\le\frac{\lVert idx\rVert^2(\frac{a}{n^2}+b\lVert idx\rVert^2)}{2}+\frac{\lVert -idx\rVert^2(\frac{a}{n^2}+b\lVert -idx\rVert^2)}{2}$
							\item $=\lVert dx\rVert^2(\frac{a}{n^2}+b\lVert dx\rVert^2)$.
						\end{enumerate}
						\item \textbf{Therefore verify that $\lVert\diff_{y=x}^{+dx}\cos_n(y)+\sin_n(x)\rVert^2\le\frac{a}{n^2}+b\lVert dx\rVert^2$.}
					\end{enumerate}
					\item \textbf{Yield the tuple $\langle a,b,d,p\rangle$.}
				\end{enumerate}
		\procedure{3.28}
			\objective
				Choose a rational number $X\ge 0$. The objective of the following instructions is to construct three rational numbers $a,b,d$ and a procedure, $p(x,dx,n)$, to show that $\lVert\diff_{y=x}^{+dx}\sin_n(y)-\cos_n(x)\rVert^2\le\frac{a}{n^2}+b\lVert dx\rVert^2$ when two complex numbers $x,dx$ and a positive integer $n>d$ such that $2(\lVert x\rVert^2+\lVert dx\rVert^2)\le X$ are chosen.
			\implementation
				Implementation is analogous to that of \procedurehr{3.27}.
		\procedure{3.29}
			\objective
				Choose a rational number $X\ge 0$. The objective of the following instructions is to construct two rational numbers $a,N$, and a procedure, $p$, to show that $\lVert\sum_r^{[0:{\lceil\frac{n}]}{2}\rceil}\frac{(-1)^rx^{2r}}{(2r)!}-\cos_n(x)\rVert^2\le\frac{a\lVert x\rVert^4}{n}$ when a complex number $x$ and an integer $n>N$ such that $\lVert x\rVert^2\le X$ is chosen.
			\implementation
				\begin{enumerate}
					\item Execute \procedurehr{3.21} on $\langle X\rangle$ and let $\langle a,N,q\rangle$ receive.
					\item Let $p(x,n)$ be the following procedure:
					\begin{enumerate}
						\item Verify that $\lVert ix\rVert^2=\lVert x\rVert^2\le X$.
						\item Execute procedure $q$ on $\langle ix,n\rangle$.
						\item Hence verify that $\lVert\sum_r^{[0:{n+1}]}\frac{(ix)^r}{r!}-\exp_n(ix)\rVert^2\le\frac{a\lVert ix\rVert^4}{n}$.
						\item Verify that $\lVert -ix\rVert^2=\lVert x\rVert^2\le X$.
						\item Execute procedure $q$ on $\langle -ix,n\rangle$.
						\item Hence verify that $\lVert\sum_r^{[0:{n+1}]}\frac{(-ix)^r}{r!}-\exp_n(-ix)\rVert^2\le\frac{a\lVert -ix\rVert^4}{n}$.
						\item Hence using \procedurehr{3.04}, verify that $\lVert\sum_r^{[0:{\lceil\frac{n}]}{2}\rceil}\frac{(-1)^rx^{2r}}{(2r)!}-\cos_n(x)\rVert^2$
						\begin{enumerate}
							\item $=\lVert\sum_r^{[0:{n+1}]}\frac{[r\mod 2=0](-1)^{\frac{r}{2}}x^{r}}{r!}-\cos_n(x)\rVert^2$
							\item $=\lVert\sum_r^{[0:{n+1}]}\frac{(i^r+(-i)^r)x^{r}}{2(r!)}-\frac{\exp_n(ix)+\exp_n(-ix)}{2}\rVert^2$
							\item $=\frac{1}{4}\lVert\sum_r^{[0:{n+1}]}\frac{(ix)^{r}}{r!}-\exp_n(ix)+\sum_r^{[0:{n+1}]}\frac{(-ix)^{r}}{r!}-\exp_n(-ix)\rVert^2$
							\item $\le\frac{1}{2}(\lVert\sum_r^{[0:{n+1}]}\frac{(ix)^{r}}{r!}-\exp_n(ix)\rVert^2+\lVert\sum_r^{[0:{n+1}]}\frac{(-ix)^{r}}{r!}-\exp_n(-ix)\rVert^2)$
							\item $\le\frac{1}{2}(\frac{a\lVert ix\rVert^4}{n}+\frac{a\lVert -ix\rVert^4}{n})$
							\item $\le\frac{a\lVert x\rVert^4}{n}$.
						\end{enumerate}
					\end{enumerate}
				\end{enumerate}
		\procedure{3.30}
			\objective
				Choose a rational number $X\ge 0$. The objective of the following instructions is to construct two rational numbers $a,N$, and a procedure, $p$, to show that $\lVert\sum_r^{[0:{\lfloor\frac{n+1}]}{2}\rfloor}\frac{(-1)^rx^{2r+1}}{(2r+1)!}-\sin_n(x)\rVert^2\le\frac{a\lVert x\rVert^4}{n}$ when a complex number $x$ and an integer $n>N$ such that $\lVert x\rVert^2\le X$ is chosen.
			\implementation
				Implementation is analogous to that of \procedurehr{3.29}.
		\declaration{3.30}
			The notation \gls{complexintegral}, where $R$ is a non-empty list of complex numbers and $f[\#r,r,dr]$ is a function of $\#r,r,dr$, will be used as a shorthand for $\sum_t^{[0:\lvert R\rvert-1]}f(t,R_t,R_{t+1}-R_{t})$.
		\procedure{3.86}
			\objective
				Choose two functions $f[\#r,r,dr],g[\#r,r,dr]$, and a non-empty list of complex numbers $R$. The objective of the following instructions is to show that $\int_r^R(f(\#r,r,dr)+g(\#r,r,dr))=\int_r^Rf(\#r,r,dr)+\int_r^Rg(\#r,r,dr)$.
			\implementation
				\begin{enumerate}
					\item Verify that $\int_r^R(f(\#r,r,dr)+g(\#r,r,dr))$
					\begin{enumerate}
						\item $=\sum_t^{[0:\lvert R\rvert-1]}(f(t,R_t,R_{t+1}-R_t)+g(t,R_t,R_{t+1}-R_t))$
						\item $=\sum_t^{[0:\lvert R\rvert-1]}f(t,R_t,R_{t+1}-R_t)+\sum_t^{[0:\lvert R\rvert-1]}g(t,R_t,R_{t+1}-R_t)$
						\item $=\int_r^Rf(\#r,r,dr)+\int_r^Rg(\#r,r,dr)$
					\end{enumerate}
				\end{enumerate}
		\procedure{3.87}
			\objective
				Choose a complex number $a$, a function $f[\#r,r,dr]$, and a non-empty list of complex numbers $R$. The objective of the following instructions is to show that $\int_r^Raf(\#r,r,dr)=a\int_r^Rf(\#r,r,dr)$.
			\implementation
				\begin{enumerate}
					\item Verify that $\int_r^Raf(\#r,r,dr)$
					\begin{enumerate}
						\item $=\sum_t^{[0:\lvert R\rvert-1]}af(t,R_t,R_{t+1}-R_t)$
						\item $=a\sum_t^{[0:\lvert R\rvert-1]}f(t,R_t,R_{t+1}-R_t)$
						\item $=a\int_r^Rf(\#r,r,dr)$
					\end{enumerate}
				\end{enumerate}
		\procedure{3.88}
			\objective
				Choose a function $f[r]$ and two non-empty lists of complex numbers $R,S$ such that $R_{\lvert R\rvert-1}=S_0$. The objective of the following instructions is to show that $\int_r^{R^\frown S}f(r)dr=\int_r^Rf(r)dr+\int_r^Sf(r)dr$.
			\implementation
				\begin{enumerate}
					\item Let $T=R^\frown S$.
					\item Verify that $\int_r^{T}f(r)dr$
					\begin{enumerate}
						\item $=\sum_t^{[0:\lvert T\rvert-1]}f(T_t)(T_{t+1}-T_t)$
						\item $=\sum_t^{[0:\lvert R\rvert-1]}f(T_t)(T_{t+1}-T_t)+\sum_t^{[\lvert R\rvert-1:\lvert R\rvert]}f(T_t)(T_{t+1}-T_t)+\sum_t^{[\lvert R\rvert:\lvert T\rvert-1]}f(T_t)(T_{t+1}-T_t)$
						\item $=\sum_t^{[0:\lvert R\rvert-1]}f(R_t)(R_{t+1}-R_t)+f(T_{\lvert R\rvert-1})(T_{\lvert R\rvert}-T_{\lvert R\rvert-1})+\sum_t^{[\lvert R\rvert:\lvert T\rvert-1]}f(S_{t-\lvert R\rvert})(S_{t+1-\lvert R\rvert}-S_{t-\lvert R\rvert})$
						\item $=\sum_t^{[0:\lvert R\rvert-1]}f(R_t)(R_{t+1}-R_t)+f(T_{\lvert R\rvert-1})(S_0-R_{\lvert R\rvert-1})+\sum_t^{[0:\lvert S\rvert-1]}f(S_{t})(S_{t+1}-S_{t})$
						\item $=\int_r^Rf(r)dr+\int_r^Sf(r)dr$.
					\end{enumerate}
				\end{enumerate}
		\procedure{3.33}
			\objective
				Choose a function $f[r]$ and a list of complex numbers $R$. The objective of the following instructions is to show that $\lVert\int_r^Rf(r)dr\rVert^2\le(\lvert R\rvert-1)^2\max(\lVert\Delta R\rVert^2)\max(\lVert f(R_{[0:\lvert R\rvert-1]})\rVert^2)$
			\implementation
				\begin{enumerate}
					\item Verify that $\lVert\int_r^Rf(r)dr\rVert^2$
					\begin{enumerate}
						\item $\le(\lvert R\rvert-1)\int_r^R\lVert f(r)\rVert^2\lVert dr\rVert^2$
						\item $\le(\lvert R\rvert-1)\int_r^R\max(\lVert f(R_{[0:\lvert R\rvert-1]})\rVert^2)\max(\lVert\Delta R\rVert^2)$
						\item $\le(\lvert R\rvert-1)^2\max(\lVert f(R_{[0:\lvert R\rvert-1]})\rVert^2)\max(\lVert\Delta R\rVert^2)$.
					\end{enumerate}
				\end{enumerate}
		\procedure{3.34}
			\objective
				Choose a list of functions $F$ and a list of complex numbers $R$ such that $F_0(R_1)=F_1(R_1),F_1(R_2)=F_2(R_2),\cdots,F_{\lvert R\rvert-3}(R_{\lvert R\rvert-2})=F_{\lvert R\rvert-2}(R_{\lvert R\rvert-2})$. The objective of the following instructions is to show that $\int_r^Rdr\diff_{z=r}^{dr}F_{\#r}(z)=F_{\lvert R\rvert-2}(R_{\lvert R\rvert-1})-F_0(R_0)$.
			\implementation
				\begin{enumerate}
					\item Verify that $\int_r^Rdr\diff_{z=r}^{dr}F_{\#r}(z)$
					\begin{enumerate}
						\item $=\int_r^Rdr(\frac{F_{\#r}(r+dr)-F_{\#r}(r)}{dr})$
						\item $=\int_r^R(F_{\#r}(r+dr)-F_{\#r}(r))$
						\item $=\sum_k^{[0:\lvert R\rvert-1]}(F_{k}(R_{k+1})-F_{k}(R_k))$
						\item $=-F_0(R_0)+\sum_k^{[0:\lvert R\rvert-2]}(F_{k}(R_{k+1})-F_{k+1}(R_{k+1}))+F_{\lvert R\rvert-2}(R_{\lvert R\rvert-1})$
						\item $=-F_0(R_0)+\sum_k^{[0:\lvert R\rvert-2]}0+F_{\lvert R\rvert-2}(R_{\lvert R\rvert-1})$
						\item $=F_{\lvert R\rvert-2}(R_{\lvert R\rvert-1})-F_0(R_0)$.
					\end{enumerate}
				\end{enumerate}
		\procedure{3.41}
			\objective
				Choose three rational numbers $X>0$, $Y\ge 0$, and $Z\ge 0$. The objective of the following instructions is to construct rational numbers $a,b,d$, and a procedure, $p(x,y[x],dx,n)$, to show that $\lVert\diff_{z=x}^{+dx}(z\exp_n(-y(z)))\rVert^2\le\frac{a}{n^2}+b\lVert dx\rVert^2$ when a function $y[x]$, two complex numbers $x,dx$ and two positive integers $n>d,m$ such that $\diff_{z=x}^{+dx}y(z)=\frac{1}{x}$, $\lVert x\rVert^2\ge X$, $\lVert dx\rVert^2\le Y$, and $\lVert y(x)\rVert^2\le Z$ are chosen.
			\implementation
				\begin{enumerate}
					\item Execute \procedurehr{3.13} on $\langle Z\rangle$ and let $\langle a_3,b_3,p_3\rangle$ receive.
					\item Execute \procedurehr{3.19} on $\langle 2(Z+\frac{Y}{X})\rangle$ and let $\langle a_4,b_4,c_4,p_4\rangle$ receive.
					\item Let $a=4a_4(1+\frac{Y}{X})$.
					\item Let $b=\frac{2}{X}(2b_4+\frac{2b_4Y}{X}+a_3)$.
					\item Let $d=\max(b_3,c_4)$.
					\item Let $p(x,y[x],dx,n)$ be the following procedure:
					\begin{enumerate}
						\item Execute procedure $p_3$ on $\langle-y(x),n\rangle$.
						\item Hence verify that $\lVert\exp_n(-y(x))\rVert^2\le a_3$.
						\item Execute procedure $p_4$ on $\langle -y(x),-\frac{dx}{x},n\rangle$.
						\item Hence verify that $\lVert dx\rVert^2\lVert\diff_{z=x}^{+dx}\exp_n(-y(z))+\frac{1}{x}\exp_n(-y(x))\rVert^2=\lVert\frac{dx}{x}\rVert^2\lVert\diff_{z=-y(x)}^{-\frac{dx}{x}}\exp_n(z)-\exp_n(-y(x))\rVert^2\le\frac{a_4}{n^2}\lVert\frac{dx}{x}\rVert^2+b_4\lVert\frac{dx}{x}\rVert^4$.
						\item Hence verify that $\lVert dx\diff_{z=x}^{dx}(z\exp_n(-y(z)))\rVert^2$
						\begin{enumerate}
							\item $=\lVert(x+dx)(\exp_n(-y(x)-\frac{dx}{x})-\exp_n(-y(x))+\frac{dx}{x}\exp_n(-y(x)))-\frac{dx^2}{x}\exp_n(-y(x))\rVert^2$
							\item $\le 2\lVert x+dx\rVert^2\lVert \exp_n(-y(x)-\frac{dx}{x})-\exp_n(-y(x))+\frac{dx}{x}\exp_n(-y(x))\rVert^2+2\frac{\lVert dx\rVert^4}{\lVert x\rVert^2}\lVert\exp_n(-y(x))\rVert^2$
							\item $\le 4(\lVert x\rVert^2+\lVert dx\rVert^2)(\frac{a_4}{n^2}\frac{\lVert dx\rVert^2}{\lVert x\rVert^2}+b_4\frac{\lVert dx\rVert^4}{\lVert x\rVert^4})+2\frac{\lVert dx\rVert^4}{X}a_3$
							\item $\le 4(\frac{a_4}{n^2}\lVert dx\rVert^2+b_4\frac{\lVert dx\rVert^4}{\lVert x\rVert^2})+4(\frac{a_4}{n^2}\frac{\lVert dx\rVert^4}{\lVert x\rVert^2}+b_4\frac{\lVert dx\rVert^6}{\lVert x\rVert^4})+2\frac{\lVert dx\rVert^4}{X}a_3$.
							\item $\le 4(\frac{a_4}{n^2}\lVert dx\rVert^2+b_4\frac{\lVert dx\rVert^4}{X})+4(\frac{a_4}{n^2}\frac{\lVert dx\rVert^4}{X}+b_4\frac{\lVert dx\rVert^6}{X^2})+2\frac{\lVert dx\rVert^4}{X}a_3$.
						\end{enumerate}
						\item \textbf{Therefore verify that $\lVert\diff_{z=x}^{dx}(z\exp_n(-y(z)))\rVert^2\le\frac{a}{n^2}+b\lVert dx\rVert^2$.}
					\end{enumerate}
					\item \textbf{Yield the tuple $\langle a,b,d,p\rangle$.}
				\end{enumerate}
		\declaration{3.31}
			The notation \gls{complexDeltaX}, where $X$ is a list, will be used as a shorthand for $\langle X_1-X_0,X_2-X_1,\cdots,X_{\lvert X\rvert-1}-X_{\lvert X\rvert-2}\rangle$.
		\procedure{3.42}
			\objective
				Choose two rational numbers $X>0,Z\ge 0$. The objective of the following instructions is to construct rational numbers $a,h,e$ and a procedure $p(R,n)$ to show that $\lVert R_{\lvert R\rvert-1}\exp_n(-\int_r^R\frac{dr}{r})-R_0\rVert^2\le\frac{a}{n^2}+h\max(\lVert\Delta R\rVert^2)$ when a non-empty list of complex numbers $R$ and a positive integer $n$ such that $n>e$, $\min(\lVert R_{[0:\lvert R\rvert-1]}\rVert^2)\ge X$, and $(\lvert R\rvert-1)^2\max(\lVert\Delta R\rVert^2)\le Z$ are chosen.
			\implementation
				\begin{enumerate}
					\item Execute \procedurehr{3.41} on $\langle X,Z,\frac{Z}{X}\rangle$ and let $\langle c,d,e,q\rangle$ receive.
					\item Let $a=cZ$.
					\item Let $h=dZ$.
					\item Let $p(R,n)$ be the following procedure:
					\begin{enumerate}
						\item For $k$ in $[0:\lvert R\rvert-1]$, do the following:
						\begin{enumerate}
							\item Let $y_k(R_{k})=\int_r^{R_{[0:k+1]}}\frac{dr}{r}$.
							\item Let $y_k(R_{k+1})=\int_r^{R_{[0:k+2]}}\frac{dr}{r}$.
							\item Verify that $\diff_{z=R_{k}}^{R_{k+1}-R_{k}}y_k(z)=\frac{1}{R_{k}}$.
							\item Using \procedurehr{3.33}, verify that $\lVert y_k(R_{k})\rVert^2\le\frac{Z}{X}$.
							\item Verify that $\lVert R_{k}\rVert^2\ge X$.
							\item Verify that $\lVert R_{k+1}-R_{k}\rVert^2\le Z$.
							\item Execute procedure $q$ on $\langle R_{k},y_{k+1},R_{k+1}-R_{k},n\rangle$.
							\item Hence verify that $\lVert\diff_{z=R_{k}}^{R_{k+1}-R_{k}}(z\exp_n(-y_k(z)))\rVert^2\le\lVert R_{k+1}-R_{k}\rVert^2(\frac{c}{n^2}+d\lVert R_{k+1}-R_{k}\rVert^2)$
						\end{enumerate}
						\item Hence using \procedurehr{3.34}, verify that $\lVert R_{\lvert R\rvert-1}\exp_n(-\int_r^R\frac{dr}{r})-R_0\rVert^2$
						\begin{enumerate}
							\item $=\lVert R_{\lvert R\rvert-1}\exp_n(-y_{\lvert R\rvert-2}(R_{\lvert R\rvert-1}))-R_0\exp_n(-y_0(R_0))\rVert^2$
							\item $=\lVert\int_{k}^{R}dk\diff_{z=k}^{dk}(z\exp_n(-y_{\#k}(z))\rVert^2$
							\item $\le(\lvert R\rvert-1)\int_{k}^{R}\lVert dk\diff_{z=k}^{dk}(z\exp_n(-y_{\#k}(z))\rVert^2$
							\item $\le(\lvert R\rvert-1)\int_{k}^{R}\lVert dk\rVert^2(\frac{c}{n^2}+d\lVert dk\rVert^2)$
							\item $\le(\lvert R\rvert-1)^2\max(\lVert\Delta R\rVert^2)(\frac{c}{n^2}+d\lVert\max(\lVert\Delta R\rVert^2)\rVert^2)$
							\item $\le Z(\frac{c}{n^2}+d\lVert\max(\lVert\Delta R\rVert^2)\rVert^2)$
							\item $\le\frac{a}{n^2}+h\max(\lVert\Delta R\rVert^2)$
						\end{enumerate}
					\end{enumerate}
					\item \textbf{Yield the tuple $\langle a,h,e,p\rangle$.}
				\end{enumerate}
		\procedure{3.43}
			\objective
				Choose two rational numbers $X>0,Y\ge 0,Z\ge 0$. The objective of the following instructions is to construct rational numbers $a,z,b$ and a procedure $p(R,n)$ to show that $\lVert R_{\lvert R\rvert-1}-R_0\exp_n(\int_r^R\frac{dr}{r})\rVert^2\le\frac{a}{n^2}+z\max(\lVert\Delta R\rVert^2)$ when a list of complex numbers $R$ and an integer $n$ such that $n>b$, $\min(\lVert R_{[0:\lvert R\rvert-1]}\rVert^2)\ge X$, $\lVert R_{\lvert R\rvert-1}\rVert^2\le Y$, and $(\lvert R\rvert-1)^2\max(\lVert\Delta R\rVert^2)\le Z$ are chosen.
			\implementation
				\begin{enumerate}
					\item Execute \procedurehr{3.13} on $\langle\frac{Z}{X}\rangle$ and let $\langle g,h,t\rangle$ receive.
					\item Execute \procedurehr{3.15} on $\langle\frac{4Z}{X}\rangle$ and let $\langle c,d,q\rangle$ receive.
					\item Execute \procedurehr{3.42} on $\langle X,Z\rangle$ and let $\langle e,k,f,r\rangle$ receive.
					\item Let $a=2(\frac{cYZ^2}{X^2}+2ge)$.
					\item Let $z=2gk$.
					\item Let $b=\max(d,f,h)$.
					\item Let $p(R,n)$ be the following procedure:
					\begin{enumerate}
						\item Now execute procedure $r$ on $\langle R,n\rangle$.
						\item Hence verify that $\lVert R_{\lvert R\rvert-1}\exp_n(-\int_r^R\frac{dr}{r})-R_0\rVert^2\le\frac{e}{n^2}+k\max(\lVert\Delta R\rVert^2)$.
						\item Using \procedurehr{3.33}, verify that $\lVert\int_r^R\frac{dr}{r}\rVert^2\le\frac{Z}{X}$.
						\item Execute procedure $t$ on $\langle\int_r^R\frac{dr}{r},n\rangle$.
						\item Hence verify that $\lVert\exp_n(\int_r^R\frac{dr}{r})\rVert^2\le g$.
						\item Now verify that $2(\lVert\int_r^R\frac{dr}{r}\rVert^2+\lVert-\int_r^R\frac{dr}{r}\rVert^2)\le\frac{4Z}{X}$.
						\item Execute procedure $q$ on $\langle\int_r^R\frac{dr}{r},-\int_r^R\frac{dr}{r},n\rangle$.
						\item Hence verify that $\lVert\exp_n(\int_r^R\frac{dr}{r})\exp_n(-\int_r^R\frac{dr}{r})-\exp_n(0)\rVert^2\le\frac{c\lVert\int_r^R\frac{dr}{r}\rVert^4}{n^2}\le\frac{cZ^2}{n^2X^2}$.
						\item Hence verify that $\lVert R_{\lvert R\rvert-1}-R_0\exp_n(\int_r^R\frac{dr}{r})\rVert^2$
						\begin{enumerate}
							\item $=\lVert R_{\lvert R\rvert-1}\exp_n(0)-R_0\exp_n(\int_r^R\frac{dr}{r})\rVert^2$
							\item $=\lVert R_{\lvert R\rvert-1}\exp_n(0)-R_{\lvert R\rvert-1}\exp_n(\int_r^R\frac{dr}{r})\exp_n(-\int_r^R\frac{dr}{r})+R_{\lvert R\rvert-1}\exp_n(\int_r^R\frac{dr}{r})\exp_n(-\int_r^R\frac{dr}{r})-R_0\exp_n(\int_r^R\frac{dr}{r})\rVert^2$
							\item $\le 2\lVert R_{\lvert R\rvert-1}\rVert^2\lVert\exp_n(0)-\exp_n(\int_r^R\frac{dr}{r})\exp_n(-\int_r^R\frac{dr}{r})\rVert^2+2\lVert\exp_n(\int_r^R\frac{dr}{r})\rVert^2\lVert R_{\lvert R\rvert-1}\exp_n(-\int_r^R\frac{dr}{r})-R_0\rVert^2$
							\item $\le 2\frac{cZ^2\lVert R_{\lvert R\rvert-1}\rVert^2}{n^2X^2}+2g(\frac{e}{n^2}+k\max(\lVert\Delta R\rVert^2))$
							\item $=\frac{a}{n^2}+z\max(\lVert\Delta R\rVert^2)$.
						\end{enumerate}
					\end{enumerate}
					\item \textbf{Yield the tuple $\langle a,z,b,p\rangle$.}
				\end{enumerate}
		\declaration{3.32}
			The notation \gls{complexmu}, where $a,b$ are complex numbers and $n$ is a positive integer, will be used as a shorthand for $\langle a+0b,a+1b,\cdots,a+(n-1)b\rangle$.
		\declaration{3.33}
			The notation \gls{complextau}, where $n$ is a positive integer, will be used as a shorthand for $\frac{4}{i}\int_r^{\mu(1,\frac{i-1}{n},n+2)}\frac{dr}{r}$.
		\procedure{3.47}
			\objective
				Choose a positive integer $n$. The objective of the following instructions is to show that $\tau_n=4\int_r^{\mu(0,\frac{1}{n},n+2)}\frac{dr}{2r^2-2r+1}$.
			\implementation
				\begin{enumerate}
					\item Verify that $i\tau_n$
					\begin{enumerate}
						\item $=2(\int_r^{\mu(1,\frac{i-1}{n},n+2)}\frac{dr}{r}+\int_r^{\mu(1,\frac{i-1}{n},n+2)}\frac{dr}{r})$
						\item $=2(\int_r^{\mu(1,\frac{i-1}{n},n+2)}\frac{dr}{r}+\int_r^{\mu(1,\frac{i-1}{n},n+2)}\frac{dr}{1+i-r})$
						\item $=2\int_r^{\mu(1,\frac{i-1}{n},n+2)}\frac{dr(1+i)}{r(1+i-r)}$
						\item $=2\int_r^{\mu(0,\frac{i-1}{n},n+2)}\frac{dr(1+i)}{(r+1)(i-r)}$
						\item $=2\int_r^{\mu(0,\frac{1}{n},n+2)}\frac{dr(i-1)(1+i)}{(r(i-1)+1)(i-r(i-1))}$
						\item $=2\int_r^{\mu(0,\frac{1}{n},n+2)}\frac{-2dr}{2ir^2-2ir+i}$
					\end{enumerate}
					\item \textbf{Therefore verify that $\tau_n=4\int_r^{\mu(0,\frac{1}{n},n+2)}\frac{dr}{2r^2-2r+1}$.}
				\end{enumerate}
		\procedure{3.48}
			\objective
				Choose a positive integer $n$. The objective of the following instructions is to show that $\Im(\tau_n)=0$.
			\implementation
				\begin{enumerate}
					\item \textbf{Using \procedurehr{3.47}, verify that $\Im(\tau_n)=\Im(4\int_r^{\mu(0,\frac{1}{n},n+2)}\frac{dr}{2r^2-2r+1})=0$.}
				\end{enumerate}
		\procedure{3.49}
			\objective
				The objective of the following instructions is to construct a rational number $a$ such that $a\ge 4$, and a procedure, $p(n)$, to show that $\tau_n\ge a$ when a positive integer $n$ is chosen.
			\implementation
				\begin{enumerate}
					\item Let $a=4$.
					\item \textbf{Verify that $a\ge 4$.}
					\item Let $p(n)$ be the following procedure:
					\begin{enumerate}
						\item Verify that $2r^2-2r+1=2r(r-1)+1\le 1$ for $r\in\mu(0,\frac{1}{n},n+1)$.
						\item \textbf{Therefore, using \procedurehr{3.47}, verify that $\tau_n=4\int_r^{\mu(0,\frac{1}{n},n+2)}\frac{dr}{2r^2-2r+1}\ge 4\int_r^{\mu(0,\frac{1}{n},n+2)}\frac{dr}{1}=4(1+\frac{1}{n})\ge a$.}
					\end{enumerate}
					\item \textbf{Yield the tuple $\langle a,p\rangle$.}
				\end{enumerate}
		\procedure{3.50}
			\objective
				The objective of the following instructions is to construct rational numbers $a,b$ such that $a\ge 4$ and $a^2<48$, and a procedure, $p(n)$, to show that $\tau_n\le a$ when a positive integer $n$ such that $n>b$ is chosen.
			\implementation
				\begin{enumerate}
					\item Let $a=\frac{66060329}{9535825}$.
					\item \textbf{Verify that $a\ge 4$.}
					\item \textbf{Verify that $a^2=\frac{4363967067588241}{90931958430625}<48$.}
					\item Let $b=260$.
					\item Let $p(n)$ be the following procedure:
					\begin{enumerate}
						\item Let $i_{12}=\langle 1\rangle$.
						\item For $k\in[0:12]$, do the following:
						\begin{enumerate}
							\item Let $i_k$ be the sublist of $\mu(0,\frac{1}{n},n+1)$ where each entry $x$ is such that $\frac{k}{12}\le x<\frac{k+1}{12}$.
							\item Append $(i_{k+1})_{0}$ onto $i_k$.
						\end{enumerate}
						\item Therefore, using \procedurehr{3.47}, verify that $\tau_n$
						\begin{enumerate}
							\item $=4\int_r^{\mu(0,\frac{1}{n},n+2)}\frac{dr}{2r^2-2r+1}$
							\item $=4(\sum_k^{[0:{12}]}\int_r^{i_k}\frac{dr}{2r^2-2r+1}+\frac{1}{n})$
							\item $=4(\sum_k^{[0:{12}]}\int_r^{i_k}\frac{dr}{2(r-\frac{1}{2})^2+\frac{1}{2}}+\frac{1}{n})$
							\item $=4(\sum_k^{[0:{6}]}\int_r^{i_k}\frac{dr}{2(r-\frac{1}{2})^2+\frac{1}{2}}+\sum_k^{[6:{12}]}\int_r^{i_k}\frac{dr}{2(r-\frac{1}{2})^2+\frac{1}{2}}+\frac{1}{n})$
							\item $\le 4(\sum_k^{[0:{6}]}\int_r^{i_k}\frac{dr}{2(\frac{k+1}{12})^2-2(\frac{k+1}{12})+1}+\sum_k^{[6:{12}]}\int_r^{i_k}\frac{dr}{2(\frac{k}{12})^2-2(\frac{k}{12})+1}+\frac{1}{n})$
							\item $=4(2\sum_k^{[0:{6}]}\int_r^{i_k}\frac{dr}{2(\frac{k+1}{12})^2-2(\frac{k+1}{12})+1}+\frac{1}{n})$
							\item $\le 4(2(\frac{1}{12}+\frac{1}{n})\sum_k^{[0:{6}]}\frac{1}{2(\frac{k+1}{12})^2-2(\frac{k+1}{12})+1}+\frac{1}{n})$
							\item $=\frac{969318}{146705}+\frac{12218636}{146705n}$
							\item $\le\frac{969318}{146705}+\frac{12218636}{146705b}$
							\item $=a$.
						\end{enumerate}
					\end{enumerate}
					\textbf{Yield the tuple $\langle a,b,p\rangle$.}
				\end{enumerate}
		\procedure{3.51}
			\objective
				Choose a positive integer $m$. The objective of the following instructions is to show that $\min(\lVert\mu(1,\frac{i-1}{m},m+1)\rVert^2)\ge\frac{1}{2}$.
			\implementation
				\begin{enumerate}
					\item For $r$ in $\mu(0,\frac{1}{m},m+1)$, do the following:
					\begin{enumerate}
						\item Verify that $\lVert 1+r(i-1)\rVert^2$
						\begin{enumerate}
							\item $=\lVert 1-r+ri\rVert^2$
							\item $=(1-r)^2+r^2$
							\item $=1-2r+2r^2$
							\item $=2(r-\frac{1}{2})^2+\frac{1}{2}$
							\item $\ge\frac{1}{2}$.
						\end{enumerate}
					\end{enumerate}
					\item \textbf{Therefore verify that $\min(\lVert\mu(1,\frac{i-1}{m},m+1)\rVert^2)\ge\frac{1}{2}$.}
				\end{enumerate}
		\procedure{3.53}
			\objective
				The objective of the following instructions is to construct rational numbers $a,b,c$, and a procedure, $p(n,m)$, to show that $\lVert\exp_n(\frac{1}{4}\tau_m i)-i\rVert^2\le\frac{a}{n^2}+\frac{b}{m^2}$ when two positive integers $n,m$ such that $n>c$ are chosen.
			\implementation
				\begin{enumerate}
					\item Execute \procedurehr{3.43} on $\langle\frac{1}{2},5,8\rangle$ and let $\langle e,d,c,q\rangle$ receive.
					\item Let $a=2e$.
					\item Let $b=4(1+d)$.
					\item Let $p(n,m)$ be the following procedure:
					\begin{enumerate}
						\item Let $R=\mu(1,\frac{i-1}{m},m+2)$.
						\item Using \procedurehr{3.51}, verify that $\min(\lVert R_{[0:\lvert R\rvert-1]}\rVert^2)\ge\frac{1}{2}$.
						\item Also verify that $\lVert R_{\lvert R\rvert-1}\rVert^2=\lVert 1+\frac{i-1}{m}(m+1)\rVert^2=(1+\frac{1}{m})^2+(\frac{1}{m})^2\le 5$.
						\item Verify that $(\lvert R\rvert-1)^2\max(\lVert\Delta R\rVert^2)=(m+1)^2\lVert\frac{i-1}{m}\rVert^2=2(1+\frac{1}{m})^2\le 8$.
						\item Now execute procedure $q$ on $\langle R,n\rangle$.
						\item Hence verify that $\lVert R_{\lvert R\rvert-1}-R_0\exp_n(\int_r^R\frac{dr}{r})\rVert^2\le\frac{e}{n^2}+\frac{2}{m^2}d$.
						\item Now verify that $R_0=1$.
						\item Also verify that $R_{\lvert R\rvert-1}=i+\frac{i-1}{m}$.
						\item Hence verify that $\lVert i-\exp_n(\frac{1}{4}\tau_m i)\rVert^2$
						\begin{enumerate}
							\item $=\lVert\frac{1-i}{m}+R_{\lvert R\rvert-1}-R_0\exp_n(\int_r^R\frac{dr}{r})\rVert^2$
							\item $\le 2\lVert\frac{1-i}{m}\rVert^2+2\lVert R_{\lvert R\rvert-1}-R_0\exp_n(\int_r^R\frac{dr}{r})\rVert^2$
							\item $\le 2\frac{2}{m^2}+2(\frac{e}{n^2}+\frac{2}{m^2}d)$
							\item $\le\frac{a}{n^2}+\frac{b}{m^2}$.
						\end{enumerate}
					\end{enumerate}
					\item \textbf{Yield the tuple $\langle a,b,c,p\rangle$.}
				\end{enumerate}
		\procedure{3.54}
			\objective
				The objective of the following instructions is to construct rational numbers $a,b,c$, and a procedure, $p(n,m)$, to show that $\lVert\exp_n(-\frac{1}{4}\tau_m i)+i\rVert^2\le\frac{a}{n^2}+\frac{b}{m^2}$ when two positive integers $n,m$ such that $n>c$ are chosen.
			\implementation
				Implementation is analogous to that of \procedurehr{3.53}.
		\procedure{3.55}
			\objective
				Choose an integer $K\ge 0$. The objective of the following instructions is to construct rational numbers $a,b,c,d$, and a procedure, $p(n,k,m)$, to show that $\lVert\exp_n(\frac{k}{4}\tau_m i)-i^k\rVert^2\le\frac{a}{n^2}+\frac{b}{m^2}$ when a non-negative integer $k$ and two positive integers $n,m$ such that $k\le K$, $n>c$, and $m>d$ are chosen.
			\implementation
				\begin{enumerate}
					\item Execute \procedurehr{3.50} and let $\langle a_1,d,p_1\rangle$ receive.
					\item Execute \procedurehr{3.16} on $\langle(\frac{a_1}{4})^2,K\rangle$ and let $\langle a_2,b_2,p_2\rangle$ receive.
					\item Execute \procedurehr{3.53} and let $\langle a_3,b_3,c_3,p_3\rangle$ receive.
					\item Execute \procedurehr{3.13} on $\langle(\frac{a_1}{4})^2\rangle$ and let $\langle a_4,b_4,p_4\rangle$ receive.
					\item Let $a=2a_2{a_1}^4+2K^2a_3\max(1,a_4)^K$.
					\item Let $b=2K^2b_3\max(1,a_4)^K$.
					\item Let $c=\max(b_2,b_4,c_3)$.
					\item Let $p(n,k,m)$ be the following procedure:
					\begin{enumerate}
						\item Execute procedure $p_1$ on $\langle m\rangle$.
						\item Hence verify that $\tau_m\le z$.
						\item Hence verify that $\lVert\frac{1}{4}\tau_m i\rVert^2=\lVert\frac{1}{4}\tau_m\rVert^2\le(\frac{a_1}{4})^2$.
						\item Execute procedure $p_2$ on $\langle\frac{1}{4}\tau_m i,k,n\rangle$.
						\item Hence verify that $\lVert\exp_n(\frac{k}{4}\tau_m i)-\exp_n(\frac{1}{4}\tau_m i)^k\rVert^2\le\frac{a_2\lVert\frac{1}{4}\tau_m i\rVert^4}{n^2}\le\frac{a_2{a_1}^4}{n^2}$.
						\item Execute procedure $p_3$ on $\langle n,m\rangle$.
						\item Hence verify that $\lVert\exp_n(\frac{1}{4}\tau_m i)-i\rVert^2\le\frac{a_3}{n^2}+\frac{b_3}{m^2}$.
						\item Execute procedure $p_4$ on $\langle\frac{1}{4}\tau_m i,n\rangle$.
						\item Hence verify that $\lVert\exp_n(\frac{1}{4}\tau_m i)\rVert^2\le a_4$.
						\item Verify that $\lVert\exp_n(\frac{k}{4}\tau_m i)-i^k\rVert^2$
						\begin{enumerate}
							\item $\le\lVert\exp_n(\frac{k}{4}\tau_m i)-\exp_n(\frac{1}{4}\tau_m i)^k+\exp_n(\frac{1}{4}\tau_m i)^k-i^k\rVert^2$
							\item $\le 2\lVert\exp_n(\frac{k}{4}\tau_m i)-\exp_n(\frac{1}{4}\tau_m i)^k\rVert^2+2\lVert\exp_n(\frac{1}{4}\tau_m i)^k-i^k\rVert^2$
							\item $\le\frac{2a_2{a_1}^4}{n^2}+2\lVert\exp_n(\frac{1}{4}\tau_m i)-i\rVert^2\lVert\sum_r^{[0:k]}\exp_n(\frac{1}{4}\tau_m i)^ri^{k-1-r}\rVert^2$
							\item $\le\frac{2a_2{a_1}^4}{n^2}+2k(\frac{a_3}{n^2}+\frac{b_3}{m^2})\sum_r^{[0:k]}\lVert\exp_n(\frac{1}{4}\tau_m i)^ri^{k-1-r}\rVert^2$
							\item $=\frac{2a_2{a_1}^4}{n^2}+2k(\frac{a_3}{n^2}+\frac{b_3}{m^2})\sum_r^{[0:k]}\lVert\exp_n(\frac{1}{4}\tau_m i)^r\rVert^2$
							\item $\le\frac{2a_2{a_1}^4}{n^2}+2k(\frac{a_3}{n^2}+\frac{b_3}{m^2})\sum_r^{[0:k]}\max(1,a_4)^k$
							\item $=\frac{2a_2{a_1}^4}{n^2}+2k^2(\frac{a_3}{n^2}+\frac{b_3}{m^2})\max(1,a_4)^k$
							\item $\le\frac{a}{n^2}+\frac{b}{m^2}$.
						\end{enumerate}
					\end{enumerate}
					\item \textbf{Yield the tuple $\langle a,b,c,d,p\rangle$.}
				\end{enumerate}
		\procedure{3.56}
			\objective
				Choose an integer $K\ge 0$. The objective of the following instructions is to construct rational numbers $a,b,c,d$, and a procedure, $p(n,k,m)$, to show that $\lVert\exp_n(\frac{k}{4}\tau_m i)-i^k\rVert^2\le\frac{a}{n^2}+\frac{b}{m^2}$ when an integer $k$ and two positive integers $n,m$ such that $\lvert k\rvert\le K$, $n>c$, and $m>d$ are chosen.
			\implementation
				Implementation is an extension of that of \procedurehr{3.55} using \procedurehr{3.54}.
		\procedure{3.57}
			\objective
				Choose an integer $K\ge 0$. The objective of the following instructions is to construct rational numbers $a,b,c,d$, and a procedure, $p(n,k,m)$, to show that $\lVert\cos_n(\frac{k}{4}\tau_m)-\frac{i^k+(-i)^k}{2}\rVert^2\le\frac{a}{n^2}+\frac{b}{m^2}$ when an integer $k$ and two positive integers $n,m$ such that $\lvert k\rvert\le K$, $n>c$, and $m>d$ are chosen.
			\implementation
				\begin{enumerate}
					\item Execute \procedurehr{3.56} on $\langle K\rangle$ and let $\langle a,b,c,d,q\rangle$ receive.
					\item Let $p(n,k,m)$ be the following procedure:
					\begin{enumerate}
						\item Execute procedure $q$ on $\langle n,k,m\rangle$.
						\item Hence verify that $\lVert\exp_n(\frac{k}{4}\tau_m i)-i^k\rVert^2\le\frac{a}{n^2}+\frac{b}{m^2}$.
						\item Execute procedure $q$ on $\langle n,-k,m\rangle$.
						\item Hence verify that $\lVert\exp_n(-\frac{k}{4}\tau_m i)-i^{-k}\rVert^2\le\frac{a}{n^2}+\frac{b}{m^2}$.
						\item Hence verify that $\lVert\cos_n(\frac{k}{4}\tau_m)-\frac{i^k+(-i)^k}{2}\rVert^2$
						\begin{enumerate}
							\item $=\lVert\frac{\exp_n(\frac{k}{4}\tau_m i)+\exp_n(-\frac{k}{4}\tau_m i)}{2}-\frac{i^k+(-i)^k}{2}\rVert^2$
							\item $=\lVert\frac{\exp_n(\frac{k}{4}\tau_m i)-i^k}{2}+\frac{\exp_n(-\frac{k}{4}\tau_m i)-(-i)^k}{2}\rVert^2$
							\item $\le\frac{1}{2}\lVert\exp_n(\frac{k}{4}\tau_m i)-i^k\rVert^2+\frac{1}{2}\lVert\exp_n(-\frac{k}{4}\tau_m i)-(-i)^k\rVert^2$
							\item $\le\frac{a}{n^2}+\frac{b}{m^2}$.
						\end{enumerate}
					\end{enumerate}
					\item \textbf{Yield the tuple $\langle a,b,c,d,p\rangle$.}
				\end{enumerate}
		\procedure{3.58}
			\objective
				Choose an integer $K\ge 0$. The objective of the following instructions is to construct rational numbers $a,b,c,d$, and a procedure, $p(n,k,m)$, to show that $\lVert\sin_n(\frac{k}{4}\tau_m)-\frac{i^k-(-i)^k}{2i}\rVert^2\le\frac{a}{n^2}+\frac{b}{m^2}$ when an integer $k$ and two positive integers $n,m$ such that $\lvert k\rvert\le K$, $n>c$, and $m>d$ are chosen.
			\implementation
				Implementation is analogous to that of \procedurehr{3.57}.
		\procedure{3.59}
			\objective
				Choose two integers $X\ge 0,K\ge 0$. The objective of the following instructions is to construct rational numbers $a,b,c,d$, and a procedure, $p(x,n,k,m)$, to show that $\lVert\exp_n(x+\frac{k}{4}\tau_m i)-i^k\exp_n(x)\rVert^2\le\frac{a}{n^2}+\frac{b}{m^2}$ when an integer $k$ and two positive integers $n,m$ such that $\lVert x\rVert^2\le X$, $\lvert k\rvert\le K$, $n>c$, and $m>d$ are chosen.
			\implementation
				\begin{enumerate}
					\item Execute \procedurehr{3.50} and let $\langle a_1,b_1,p_1\rangle$ receive.
					\item Execute \procedurehr{3.15} on $\langle 2X+\frac{K{a_1}^2}{8}\rangle$ and let $\langle a_2,b_2,p_2\rangle$ receive.
					\item Execute \procedurehr{3.13} on $\langle X\rangle$ and let $\langle a_3,b_3,p_3\rangle$ receive.
					\item Execute \procedurehr{3.56} on $\langle K\rangle$ and let $\langle a_4,b_4,c_4,d_4,p_4\rangle$ receive.
					\item Let $a=\frac{a_2XK^2{a_1}^2}{8}+2a_3a_4$.
					\item Let $b=2a_3b_4$.
					\item Let $c=\max(b_2,b_3,c_4)$.
					\item Let $d=\max(b_1,d_4)$.
					\item Let $p(x,n,k,m)$ be the following procedure:
					\begin{enumerate}
						\item Verify that $\lVert x\rVert^2\le X$.
						\item Execute procedure $p_1$ on $\langle m\rangle$.
						\item Hence verify that $\tau_m\le a_1$.
						\item Hence verify that $\lVert\frac{k}{4}\tau_m i\rVert^2=\frac{k^2{\tau_m}^2}{16}\le\frac{K^2{a_1}^2}{16}$.
						\item Hence verify that $2(\lVert x\rVert^2+\lVert\frac{k}{4}\tau_m i\rVert^2)\le 2X+\frac{K^2{a_1}^2}{8}$.
						\item Now execute procedure $p_2$ on $\langle x,\frac{k}{4}\tau_m i,n\rangle$.
						\item Hence verify that $\lVert\exp_n(x)\exp_n(\frac{k}{4}\tau_m i)-\exp_n(x+\frac{k}{4}\tau_m i)\rVert^2\le\frac{a_2\lVert x\rVert^2\lVert\frac{k}{4}\tau_m i\rVert^2}{n^2}\le\frac{a_2XK^2{a_1}^2}{16n^2}$.
						\item Execute procedure $p_3$ on $\langle x,n\rangle$.
						\item Hence verify that $\lVert\exp_n(x)\rVert^2\le a_3$.
						\item Execute procedure $p_4$ on $\langle n,k,m\rangle$.
						\item Hence verify that $\lVert\exp_n(\frac{k}{4}\tau_mi)-i^k\rVert^2\le\frac{a_4}{n^2}+\frac{b_4}{m^2}$.
						\item Verify that $\lVert\exp_n(x+\frac{k}{4}\tau_m i)-i^k\exp_n(x)\rVert^2$
						\begin{enumerate}
							\item $=\lVert\exp_n(x+\frac{k}{4}\tau_m i)-\exp_n(x)\exp_n(\frac{k}{4}\tau_m i)+\exp_n(x)\exp_n(\frac{k}{4}\tau_m i)-i^k\exp_n(x)\rVert^2$
							\item $\le 2\lVert\exp_n(x+\frac{k}{4}\tau_m i)-\exp_n(x)\exp_n(\frac{k}{4}\tau_m i)\rVert^2+2\lVert\exp_n(x)\rVert^2\lVert\exp_n(\frac{k}{4}\tau_m i)-i^k\rVert^2$
							\item $\le\frac{a_2XK^2{a_1}^2}{8n^2}+2a_3(\frac{a_4}{n^2}+\frac{b_4}{m^2})$
							\item $=\frac{a}{n^2}+\frac{b}{m^2}$.
						\end{enumerate}
					\end{enumerate}
					\item \textbf{Yield the tuple $\langle a,b,c,d,p\rangle$.}
				\end{enumerate}
		\procedure{3.89}
			\objective
				Choose a positive integer $K$. The objective of the following instructions is to construct rational numbers $a,b,c,d$, and a procedure, $p(n,k,m)$, to show that $\lVert\exp_n(\frac{k}{K}\tau_m i)^K-1\rVert^2\le\frac{a}{n^2}+\frac{b}{m^2}$ when an integer $k$ and positive integers $n,m$ such that $0\le k<K$, $n\ge c$, and $m>d$ are chosen.
			\implementation
				\begin{enumerate}
					\item Execute \procedurehr{3.50} and let $\langle a_1,b_1,p_1\rangle$ receive.
					\item Execute \procedurehr{3.16} on $\langle {a_1}^2,K\rangle$ and let $\langle a_2,b_2,p_2\rangle$ receive.
					\item Execute \procedurehr{3.56} on $\langle 4K\rangle$ and let $\langle a_3,b_3,c_3,d_3,p_3\rangle$ receive.
					\item Let $a=2a_2{a_1}^2+2a_3$.
					\item Let $b=2b_3$.
					\item Let $c=\max(b_2,c_3)$.
					\item Let $d=\max(b_1,d_3)$.
					\item Let $p(n,k,m)$ be the following procedure:
					\begin{enumerate}
						\item Execute procedure $p_1$ on $\langle m\rangle$.
						\item Hence verify that $\tau_m\le a_1$.
						\item Execute procedure $p_2$ on $\langle\frac{k}{K}\tau_mi,K,n\rangle$.
						\item Hence verify that $\lVert\exp_n(K\frac{k}{K}\tau_mi)-\exp_n(\frac{k}{K}\tau_mi)^K\rVert^2\le\frac{a_2\lVert\frac{k}{K}\tau_mi\rVert^2}{n^2}\le\frac{a_2\lVert\tau_m\rVert^2}{n^2}\le\frac{a_2{a_1}^2}{n^2}$.
						\item Execute procedure $p_3$ on $\langle n,4k,m\rangle$.
						\item Hence verify that $\lVert\exp_n(\frac{4k}{4}\tau_mi)-i^{4k}\rVert^2\le\frac{a_3}{n^2}+\frac{b_3}{m^2}$.
						\item Verify that $\lVert\exp_n(\frac{k}{K}\tau_mi)^K-1\rVert^2$
						\begin{enumerate}
							\item $=\lVert\exp_n(\frac{k}{K}\tau_mi)^K-\exp_n(k\tau_mi)+\exp_n(k\tau_mi)-1\rVert^2$
							\item $\le 2\lVert\exp_n(\frac{k}{K}\tau_mi)^K-\exp_n(k\tau_mi)\rVert^2+2\lVert\exp_n(k\tau_mi)-1\rVert^2$
							\item $\le\frac{2a_2{a_1}^2}{n^2}+2(\frac{a_3}{n^2}+\frac{b_3}{m^2})$
							\item $=\frac{a}{n^2}+\frac{b}{m^2}$.
						\end{enumerate}
					\end{enumerate}
					\item \textbf{Yield the tuple $\langle a,b,c,d,p\rangle$.}
				\end{enumerate}
		\procedure{3.90}
			\objective
				Choose a two rationals $M,N$ such that $0<M$ and $N^2<12$. The objective of the following instructions is to construct rational numbers $a,b$ such that $a>0$, and a procedure, $p(x,n)$, to show that $\lVert\cos_n(x)-1\rVert^2\ge a$ when a rational number $x$ and a positive integer $n$ such that $M\le\lvert x\rvert\le N$ and $n>b$ are chosen.
			\implementation
				\begin{enumerate}
					\item Let $a=\frac{M^4}{16}(-1+\frac{N^2}{12})^2$.
					\item \textbf{Verify that $a>0$.}
					\item Let $b=4$.
					\item Let $p(x,n)$ be the following procedure:
					\begin{enumerate}
						\item Using \procedurehr{3.22}, verify that $(\cos_n(x)-1)^2$
						\begin{enumerate}
							\item $=(\frac{1}{2}((1+\frac{xi}{n})^n+(1-\frac{xi}{n})^n)-1)^2$
							\item $=(\frac{1}{2}(\sum_r^{[0:{n+1}]}\frac{n^{\ul{r}}}{r!}(\frac{x}{n})^ri^r+\sum_r^{[0:{n+1}]}\frac{n^{\ul{r}}}{r!}(\frac{x}{n})^r(-i)^r)-1)^2$
							\item $=(\sum_r^{[0:{\lfloor\frac{n}]}{2}\rfloor+1}\frac{n^{\ul{2r}}}{(2r)!}(\frac{x}{n})^{2r}(-1)^r-1)^2$
							\item $=(\sum_r^{[1:{\lfloor\frac{n}]}{2}\rfloor+1}\frac{n^{\ul{2r}}}{(2r)!}(\frac{x}{n})^{2r}(-1)^r)^2$
							\item $=(\sum_r^{[1:{\lfloor\frac{\lfloor\frac{n}]}{2}\rfloor}{2}\rfloor+1}(-\frac{n^{\ul{4r-2}}}{(4r-2)!}(\frac{x}{n})^{4r-2}+\frac{n^{\ul{4r}}}{(4r)!}(\frac{x}{n})^{4r})-\frac{n^{\ul{2\lfloor\frac{n}{2}\rfloor}}}{(2\lfloor\frac{n}{2}\rfloor)!}(\frac{x}{n})^{2\lfloor\frac{n}{2}\rfloor}[\lfloor\frac{n}{2}\rfloor\mod 2=1])^2$
							\item $\ge(\sum_r^{[1:{\lfloor\frac{\lfloor\frac{n}]}{2}\rfloor}{2}\rfloor+1}\frac{n^{\ul{4r-2}}}{(4r-2)!}(\frac{x}{n})^{4r-2}(-1+\frac{(n-4r+2)^{\ul{2}}}{(4r)^{\ul{2}}}(\frac{x}{n})^{2}))^2$
							\item $\ge(\sum_r^{[1:{\lfloor\frac{\lfloor\frac{n}]}{2}\rfloor}{2}\rfloor+1}\frac{n^{\ul{4r-2}}}{(4r-2)!}(\frac{x}{n})^{4r-2}(-1+\frac{1}{(4r)^{\ul{2}}}(x)^{2}))^2$
							\item $\ge(\sum_r^{[1:{\lfloor\frac{\lfloor\frac{n}]}{2}\rfloor}{2}\rfloor+1}\frac{n^{\ul{4r-2}}}{(4r-2)!}(\frac{x}{n})^{4r-2}(-1+\frac{1}{12}x^{2}))^2$
							\item $\ge(\sum_r^{[1:{\lfloor\frac{\lfloor\frac{n}]}{2}\rfloor}{2}\rfloor+1}\frac{n^{\ul{4r-2}}}{(4r-2)!}(\frac{x}{n})^{4r-2}(-1+\frac{N^2}{12}))^2$
							\item $\ge(\frac{n^{\ul{2}}}{2}(\frac{x}{n})^{2}(-1+\frac{N^2}{12}))^2$
							\item $\ge(\frac{1}{4}x^{2}(-1+\frac{N^2}{12}))^2$
							\item $\ge(\frac{M^2}{4}(-1+\frac{N^2}{12}))^2$
							\item $=a$
						\end{enumerate}
					\end{enumerate}
					\item \textbf{Yield the tuple $\langle a,b,p\rangle$.}
				\end{enumerate}
		\procedure{3.60}
			\objective
				Choose a positive integer $K$. The objective of the following instructions is to construct rational numbers $a,b,c$ such that $a>0$, and a procedure, $p(n,k,m)$, to show that $\lVert\exp_n(\frac{k}{K}\tau_mi)-1\rVert^2\ge a$ when an integer $k$ and positive integers $n,m$ such that $0<\lvert k\rvert\le\frac{K}{2}$, $n>b$, and $m>c$ are chosen.
			\implementation
				\begin{enumerate}
					\item Execute \procedurehr{3.50} and let $\langle a_1,c,p_1\rangle$ receive.
					\item Verify that $(\frac{a_1}{2})^2<12$.
					\item Execute \procedurehr{3.49} and let $\langle a_2,p_2\rangle$ receive.
					\item Verify that $a_2>0$.
					\item Execute \procedurehr{3.90} on $\langle\frac{a_2}{K},\frac{a_1}{2}\rangle$ and let $\langle a,b,p_3\rangle$ receive.
					\item \textbf{Verify that $a>0$.}
					\item Let $p(n,k,m)$ be the following procedure:
					\begin{enumerate}
						\item Verify that $1\le\lvert k\rvert\le\frac{K}{2}$.
						\item Hence verify that $\frac{1}{K}\le\frac{\lvert k\rvert}{K}\le\frac{1}{2}$.
						\item Execute procedure $p_1$ on $\langle m\rangle$.
						\item Execute procedure $p_2$ on $\langle m\rangle$.
						\item Hence verify that $0<\frac{a_2}{K}\le\frac{1}{K}\tau_m\le\frac{\lvert k\rvert}{K}\tau_m\le\frac{1}{2}\tau_m\le\frac{a_1}{2}$.
						\item Hence execute procedure $p_3$ on $\langle\frac{k}{K}\tau_m,n\rangle$.
						\item Hence verify that $(\cos_n(\frac{k}{K}\tau_m)-1)^2\ge a$.
						\item Using \procedurehr{3.22}, verify that $\lVert\exp_n(\frac{k}{K}\tau_mi)-1\rVert^2$
						\begin{enumerate}
							\item $\ge\Re(\exp_n(\frac{k}{K}\tau_mi)-1)^2$
							\item $=(\cos_n(\frac{k}{K}\tau_m)-1)^2$
							\item $\ge a$
						\end{enumerate}
					\end{enumerate}
					\item \textbf{Yield the tuple $\langle a,b,c,p\rangle$.}
				\end{enumerate}
		\procedure{3.61}
			\objective
				Choose a positive integer $K$. The objective of the following instructions is to construct rational numbers $a,b,c$ such that $a>0$, and a procedure, $p(n,j,k,m)$, to show that $\lVert\exp_n(\frac{k}{K}\tau_mi)-\exp_n(\frac{j}{K}\tau_mi)\rVert^2\ge a$ when positive integers $n,j,k,m$ such that $-K<j\le k<K$, $0<k-j\le\frac{K}{2}$, $n\ge b$, and $m\ge c$ are chosen.
			\implementation
				\begin{enumerate}
					\item Execute \procedurehr{3.50} and let $\langle a_1,b_1,p_1\rangle$ receive.
					\item Execute \procedurehr{3.14} on $\langle {a_1}^2\rangle$ and let $\langle a_2,b_2,p_2\rangle$ receive.
					\item Execute \procedurehr{3.60} on $\langle K\rangle$ and let $\langle a_3,b_3,c_3,p_3\rangle$ receive.
					\item Execute \procedurehr{3.15} on $\langle 4{a_1}^2\rangle$ and let $\langle a_4,b_4,p_4\rangle$ receive.
					\item Let $a=\frac{1}{4}a_2a_3$.
					\item Let $b=\max(\lceil\sqrt{\frac{4a_4{a_1}^4}{a_2a_3}}\rceil,b_3,b_4,b_2)$.
					\item Let $c=\max(b_1,c_3)$.
					\item Let $p(n,j,k,m)$ be the following procedure:
					\begin{enumerate}
						\item Verify that $-K<j<K$.
						\item Hence verify that $-1<\frac{j}{K}<1$.
						\item Hence verify that $\lVert\frac{j}{K}\rVert^2<1$.
						\item Execute procedure $p_1$ on $\langle m\rangle$.
						\item Hence verify that $\lVert\frac{j}{K}\tau_mi\rVert^2=\lVert\frac{j}{K}\rVert^2\lVert\tau_m\rVert^2\le\lVert\tau_m\rVert^2\le{a_1}^2$.
						\item Execute procedure $p_2$ on $\langle\frac{j}{K}\tau_mi,n\rangle$.
						\item Hence verify that $\lVert\exp_n(\frac{j}{K}\tau_mi)\rVert^2\ge a_2>0$.
						\item Execute procedure $p_3$ on $\langle n,k-j,m\rangle$.
						\item Hence verify that $\lVert\exp_n(\frac{k-j}{K}\tau_mi)-1\rVert^2\ge a_3>0$.
						\item Verify that $0<\frac{k-j}{K}\le\frac{1}{2}$.
						\item Hence verify that $\lVert\frac{k-j}{K}\tau_mi\rVert^2\le\lVert\frac{k-j}{K}\rVert^2\lVert\tau_m\rVert^2\le\lVert\tau_m\rVert^2\le{a_1}^2$.
						\item Hence verify that $2(\lVert\frac{j}{K}\tau_mi\rVert^2+\lVert\frac{k-j}{K}\tau_mi\rVert^2)\le 4{a_1}^2$.
						\item Execute procedure $p_4$ on $\langle\frac{k-j}{K}\tau_mi,\frac{j}{K}\tau_mi,n\rangle$.
						\item Hence verify that $\lVert\exp_n(\frac{k-j}{K}\tau_mi)\exp_n(\frac{j}{K}\tau_mi)-\exp_n(\frac{k-j}{K}\tau_mi+\frac{j}{K}\tau_mi)\rVert^2\le\frac{a_4\lVert\frac{k-j}{K}\tau_mi\rVert^2\lVert\frac{j}{K}\tau_mi\rVert^2}{n^2}\le\frac{a_4{a_1}^4}{n^2}$.
						\item Hence using \procedurehr{3.05}, verify that $\lVert\exp_n(\frac{k}{K}\tau_mi)-\exp_n(\frac{j}{K}\tau_mi)\rVert^2$
						\begin{enumerate}
							\item $=\lVert\exp_n(\frac{k-j}{K}\tau_mi+\frac{j}{K}\tau_mi)-\exp_n(\frac{k-j}{K}\tau_mi)\exp_n(\frac{j}{K}\tau_mi)+\exp_n(\frac{k-j}{K}\tau_mi)\exp_n(\frac{j}{K}\tau_mi)-\exp_n(\frac{j}{K}\tau_mi)\rVert^2$
							\item $\ge\frac{1}{2}\lVert\exp_n(\frac{k-j}{K}\tau_mi)\exp_n(\frac{j}{K}\tau_mi)-\exp_n(\frac{j}{K}\tau_mi)\rVert^2-\lVert\exp_n(\frac{k-j}{K}\tau_mi+\frac{j}{K}\tau_mi)-\exp_n(\frac{k-j}{K}\tau_mi)\exp_n(\frac{j}{K}\tau_mi)\rVert^2$
							\item $=\frac{1}{2}\lVert\exp_n(\frac{j}{K}\tau_mi)\rVert^2\lVert\exp_n(\frac{k-j}{K}\tau_mi)-1\rVert^2-\lVert\exp_n(\frac{k-j}{K}\tau_mi+\frac{j}{K}\tau_mi)-\exp_n(\frac{k-j}{K}\tau_mi)\exp_n(\frac{j}{K}\tau_mi)\rVert^2$
							\item $\ge\frac{1}{2}a_2a_3-\frac{a_4{a_1}^4}{n^2}$
							\item $\ge a$.
						\end{enumerate}
					\end{enumerate}
					\item \textbf{Yield the tuple $\langle a,b,c,p\rangle$.}
				\end{enumerate}
		\procedure{3.62}
			\objective
				Choose a positive integer $K$. The objective of the following instructions is to construct rational numbers $a,b,c$ such that $a>0$, and a procedure, $p(n,j,k,m)$, to show that $\lVert\exp_n(\frac{k}{K}\tau_mi)-\exp_n(\frac{j}{K}\tau_mi)\rVert^2\ge a$ when positive integers $n,j,k,m$ such that $0\le j\le k<K$, $\frac{K}{2}\le k-j<K$, $n\ge b$, and $m\ge c$ are chosen.
			\implementation
				\begin{enumerate}
					\item Execute \procedurehr{3.61} on $\langle K\rangle$ and let $\langle a_1,b_1,c_1,p_1\rangle$ receive.
					\item Execute \procedurehr{3.50} and let $\langle a_2,b_2,p_2\rangle$ receive.
					\item Execute \procedurehr{3.59} on $\langle {a_2}^2,4\rangle$ and let $\langle a_3,b_3,c_3,d_3,p_3\rangle$ receive.
					\item Let $a=\frac{1}{4}a_1$.
					\item Let $b=\max(\lceil\sqrt{\frac{8a_3}{a_1}}\rceil,b_1,c_3)$.
					\item Let $c=\max(\lceil\sqrt{\frac{8b_3}{a_1}}\rceil,b_2,c_1,d_3)$.
					\item Let $p(n,j,k,m)$ be the following procedure:
					\begin{enumerate}
						\item Verify that $-\frac{K}{2}\le k-K<j<\frac{K}{2}$.
						\item Also verify that $0<j-(k-K)\le\frac{K}{2}$.
						\item Execute procedure $p_1$ on $\langle n,k-K,j,m\rangle$.
						\item Hence verify that $\lVert\exp_n(\frac{j}{K}\tau_mi)-\exp_n(\frac{k-K}{K}\tau_mi)\rVert^2\ge a_1$.
						\item Execute procedure $p_2$ on $\langle m\rangle$.
						\item Hence verify that $\tau_m\le a_2$.
						\item Hence verify that $\lVert\frac{k}{K}\tau_mi\rVert^2=\lVert\frac{k}{K}\rVert^2\lVert\tau_m\rVert^2\le\lVert\tau_m\rVert^2\le{a_2}^2$.
						\item Now execute procedure $p_3$ on $\langle\frac{k}{K}\tau_mi,n,-4,m\rangle$.
						\item Hence verify that $\lVert i^{-4}\exp_n(\frac{k}{K}\tau_mi)-\exp_n(\frac{k}{K}\tau_mi-\frac{4}{4}\tau_mi)\rVert^2\le\frac{a_3}{n^2}+\frac{b_3}{m^2}$.
						\item Now verify that $\lVert\exp_n(\frac{k}{K}\tau_mi)-\exp_n(\frac{j}{K}\tau_mi)\rVert^2$
						\begin{enumerate}
							\item $=\lVert\exp_n(\frac{k}{K}\tau_mi)-\exp_n(\frac{k}{K}\tau_mi-\tau_mi)+\exp_n(\frac{k-K}{K}\tau_mi)-\exp_n(\frac{j}{K}\tau_mi)\rVert^2$
							\item $\ge\frac{1}{2}\lVert\exp_n(\frac{k-K}{K}\tau_mi)-\exp_n(\frac{j}{K}\tau_mi)\rVert^2-\lVert\exp_n(\frac{k}{K}\tau_mi)-\exp_n(\frac{k}{K}\tau_mi-\tau_mi)\rVert^2$
							\item $\ge\frac{1}{2}a_1-\frac{a_3}{n^2}-\frac{b_3}{m^2}$
							\item $\ge a$.
						\end{enumerate}
					\end{enumerate}
					\item \textbf{Yield the tuple $\langle a,b,c,p\rangle$.}
				\end{enumerate}
		\procedure{3.63}
			\objective
				Choose a positive integer $K$. The objective of the following instructions is to construct rational numbers $a,b,c$ such that $a>0$, and a procedure, $p(n,j,k,m)$, to show that $\lVert\exp_n(\frac{k}{K}\tau_mi)-\exp_n(\frac{j}{K}\tau_mi)\rVert^2\ge a$ when positive integers $n,j,k,m$ such that $0\le j\le k<K$, $0<k-j<K$, $n\ge b$, and $m\ge c$ are chosen.
			\implementation
				\begin{enumerate}
					\item Execute \procedurehr{3.61} on $\langle K\rangle$ and let $\langle a_1,b_1,c_1,p_1\rangle$ receive.
					\item Execute \procedurehr{3.62} on $\langle K\rangle$ and let $\langle a_2,b_2,c_2,p_2\rangle$ receive.
					\item Let $a=\min(a_1,a_2)$.
					\item \textbf{Verify that $a>0$.}
					\item Let $b=\max(b_1,b_2)$.
					\item Let $c=\max(c_1,c_2)$.
					\item Let $p(n,j,k,m)$ be the following procedure:
					\begin{enumerate}
						\item If $k-j\le\frac{K}{2}$, then do the following:
						\begin{enumerate}
							\item Execute procedure $p_1$ on $\langle n,j,k,m\rangle$.
							\item \textbf{Hence verify that $\lVert\exp_n(\frac{k}{K}\tau_mi)-\exp_n(\frac{j}{K}\tau_mi)\rVert^2\ge a_1\ge a$.}
						\end{enumerate}
						\item Otherwise if $k-j>\frac{K}{2}$, then do the following:
						\begin{enumerate}
							\item Execute procedure $p_2$ on $\langle n,j,k,m\rangle$.
							\item \textbf{Hence verify that $\lVert\exp_n(\frac{k}{K}\tau_mi)-\exp_n(\frac{j}{K}\tau_mi)\rVert^2\ge a_2\ge a$.}
						\end{enumerate}
					\end{enumerate}
					\item \textbf{Yield the tuple $\langle a,b,c,p\rangle$.}
				\end{enumerate}
		\declaration{3.34}
			The phrase "\gls{complexpolynomial}" will be used to indicate that the declarations and procedures pertaining to polynomials are being used but with the provison that all uses of rational numbers therein are substituted with uses of complex numbers.
		\procedure{3.64}
			\objective
				Choose a positive integer $K$. The objective of the following instructions is to construct rational numbers $a,b,c,d$, and a procedure, $p(n,m)$, to construct a list of complex numbers $z$ and a list of complex polynomials $q$ such that,
				\begin{enumerate}
					\item $z_k=\exp_n(\frac{k}{K}\tau_mi)$ for $k\in[0:K]$
					\item $q_K=\lambda^K-1$
					\item $q_{K-1}=\sum_r^{[0:K]}\lambda^r$
					\item $q_{k+1}=(\lambda-z'_k)q_k+\Lambda(q_{k+1},z'_k)$ for $k\in[0:K]$
					\item $(q_k)_{\deg(q_k)}=1$ for $k\in[0:K+1]$
					\item $\lVert\Lambda(q_k,z_j)\rVert^2\le\frac{a}{n^2}+\frac{b}{m^2}$ for $j\in[0:k]$, for $k\in[0:K+1]$
				\end{enumerate}
				when two positive integers $n,m$ such that $n>c$ and $m>d$ are chosen.
			\implementation
				\begin{enumerate}
					\item Execute \procedurehr{3.59} on $\langle K\rangle$ and let $\langle a_1,b_1,c_1,d_1,p_1\rangle$ receive.
					\item Execute \procedurehr{3.63} on $\langle K\rangle$ and let $\langle a_2,b_2,c_2,p_2\rangle$ receive.
					\item Let $a=\max(1,\frac{4}{a_2})^Ka_1$.
					\item Let $b=\max(1,\frac{4}{a_2})^Kb_1$.
					\item Let $c=\max(c_1,b_2)$.
					\item Let $d=\max(c_2,d_1)$.
					\item Let $p(n,m)$ be the following procedure:
					\begin{enumerate}
						\item \textbf{Let $q_K=\lambda^K-1$.}
						\item For $k\in[K:0]$, do the following:
						\begin{enumerate}
							\item \textbf{Let $z_k=\exp_n(\frac{k}{K}\tau_mi)$.}
							\item Execute procedure $p_1$ on $\langle n,k,m\rangle$.
							\item \textbf{Hence verify that $\lVert\Lambda(q_K,z_k)\rVert^2\le\frac{a_1}{n^2}+\frac{b_1}{m^2}$.}
						\end{enumerate}
						\item For $k\in[K:0]$, do the following:
						\begin{enumerate}
							\item Let $q_k=q_{k+1}\div(\lambda-z'_k)$.
							\item Let $r_k=q_{k+1}\mod(\lambda-z'_k)$.
							\item Verify that $\deg(r_k)<\deg(\lambda-z'_k)=1$.
							\item Hence verify that $\deg(r_k)=0$.
							\item Verify that $q_{k+1}=(\lambda-z'_k)q_k+r_k$.
							\item \textbf{Hence verify that $1=(q_{k+1})_{\deg(q_{k+1})}=((\lambda-z'_k)q_k+r_k)_{\deg(q_{k+1})}=(q_k)_{\deg(q_k)}$.}
							\item Also verify that $\Lambda(q_{k+1},z'_k)=\Lambda(\lambda-z'_k,z'_k)\Lambda(q_k,z'_k)+\Lambda(r_k,z'_k)=(z'_k-z'_k)\Lambda(q_k,z'_k)+r_k=r_k$.
							\item \textbf{Hence verify that $q_{k+1}=(\lambda-z'_k)q_k+\Lambda(q_{k+1},z'_k)$.}
							\item For $j\in[k:0]$, do the following:
							\begin{enumerate}
								\item Verify that $\Lambda(q_{k+1},z_j)=\Lambda(\lambda-z'_k,z_j)\Lambda(q_k,z_j)+\Lambda(q_{k+1},z'_k)$.
								\item Hence verify that $\Lambda(q_{k+1},z_j)-\Lambda(q_{k+1},z'_k)=(z_j-z'_k)\Lambda(q_k,z_j)$.
								\item Execute procedure $p_2$ on $\langle n,\min(j,k),\max(j,k),m\rangle$.
								\item Hence verify that $\lVert z_j-z'_k\rVert^2\ge a_2$.
								\item Hence verify that $2\lVert\Lambda(q_{k+1},z_j)\rVert^2+2\lVert\Lambda(q_{k+1},z'_k)\rVert^2\ge\lVert z_j-z'_k\rVert^2\lVert\Lambda(q_k,z_j)\rVert^2\ge a_2\lVert\Lambda(q_k,z_j)\rVert^2$.
								\item Hence verify that $\lVert\Lambda(q_k,z_j)\rVert^2$
								\item $\le\frac{1}{a_2}(2\lVert\Lambda(q_{k+1},z_j)\rVert^2+2\lVert\Lambda(q_{k+1},z'_k)\rVert^2)$
								\item $=\frac{2}{a_2}(\lVert\Lambda(q_{k+1},z_j)\rVert^2+\lVert\Lambda(q_{k+1},z'_k)\rVert^2)$
								\item $\le\frac{2}{a_2}((\frac{4}{a_2})^{K-k-1}(\frac{a_1}{n^2}+\frac{b_1}{m^2})+(\frac{4}{a_2})^{K-k-1}(\frac{a_1}{n^2}+\frac{b_1}{m^2}))$
								\item $=\frac{4}{a_2}(\frac{4}{a_2})^{K-k-1}(\frac{a_1}{n^2}+\frac{b_1}{m^2})$
								\item $=(\frac{4}{a_2})^{K-k}(\frac{a_1}{n^2}+\frac{b_1}{m^2})$
							\end{enumerate}
							\item Now using (cvii), verify that $(\lambda-1)\sum_r^{[0:K]}\lambda^r$
							\begin{enumerate}
								\item $=\lambda^K-1$
								\item $=q_K$
								\item $=(\lambda-z'_{K-1})q_{K-1}+\Lambda(q_K,z'_{K-1})$
								\item $=(\lambda-1)q_{K-1}+\Lambda(\lambda^K-1,1)$
								\item $=(\lambda-1)q_{K-1}$.
							\end{enumerate}
							\item \textbf{Hence verify that $\sum_r^{[0:K]}\lambda^r=q_{K-1}$.}
						\end{enumerate}
						\item \textbf{Yield the tuple $\langle z,q\rangle$.}
					\end{enumerate}
					\item \textbf{Yield the tuple $\langle a,b,c,d,p\rangle$.}
				\end{enumerate}
		\procedure{3.65}
			\objective
				Choose a rational number $X$ and a positive integer $K$. The objective of the following instructions is to construct rational numbers $a,b,c,d$, and a procedure, $p(x,n,m)$, to show that $\lVert \sum_r^{[0:K]}x^r-\prod_r^{[1:K]}(x-\exp_n(\frac{r}{K}\tau_mi))\rVert^2\le\frac{a}{n^2}+\frac{b}{m^2}$ when a complex number $x$ and positive integers $n,m$ such that $n>c$, $m>d$, and $\lVert x\rVert^2\le X$ are chosen.
			\implementation
				\begin{enumerate}
					\item Execute \procedurehr{3.64} on $\langle K\rangle$ and let $\langle a_1,b_1,c_1,d_1,p_1\rangle$ receive.
					\item Execute \procedurehr{3.50} and let $\langle a_2,b_2,p_2\rangle$ receive.
					\item Execute \procedurehr{3.13} on $\langle {a_2}^2\rangle$ and let $\langle a_3,b_3,p_3\rangle$ receive.
					\item Let $l=(K-1)\sum_k^{[0:K-1]}2^{K-k-2}(X+a_3)^{K-k-2}$.
					\item Let $a=a_1l$.
					\item Let $b=b_1l$.
					\item Let $c=\max(c_1,b_3)$.
					\item Let $d=\max(d_1,b_2)$.
					\item Let $p(x,n,m)$ be the following procedure:
					\begin{enumerate}
						\item Execute procedure $p_2$ on $\langle m\rangle$.
						\item Hence verify that $\tau_m\le a_2$.
						\item Execute procedure $p_1$ on $\langle n,m\rangle$ and let $\langle z,t\rangle$ receive.
						\item For $j\in[1:K]$, do the following:
						\begin{enumerate}
							\item Verify that $\lVert\frac{j}{K}\tau_mi\rVert^2=\lVert\frac{j}{K}\rVert^2\lVert\tau_m\rVert^2\le\lVert\tau_m\rVert^2\le a_2$.
							\item Execute procedure $p_3$ on $\langle\frac{j}{K}\tau_mi,n\rangle$.
							\item Hence verify that $\lVert z_j\rVert^2=\lVert\exp_n(\frac{j}{K}\tau_mi)\rVert^2\le a_3$.
						\end{enumerate}
						\item Hence verify that $\lVert\sum_r^{[0:K]}x^r-\prod_r^{[1:K]}(x-z_r)\rVert^2$
						\begin{enumerate}
							\item $=\lVert\Lambda(\sum_r^{[0:K]}\lambda^r,x)-\prod_r^{[1:K]}(x-z_r)\rVert^2$
							\item $=\lVert\Lambda(t_{K-1},x)-\prod_r^{[1:K]}(x-z_r)\rVert^2$
							\item $=\lVert\Lambda(\prod_j^{[0:K-1]}(\lambda-z'_j)+\sum_k^{[0:K-1]}\Lambda(t_{k+1},z'_{k})\prod_j^{[k+1:K-1]}(\lambda-z'_j),x)-\prod_r^{[1:K]}(x-z_r)\rVert^2$
							\item $=\lVert\prod_j^{[0:K-1]}(x-z'_j)+\sum_k^{[0:K-1]}\Lambda(t_{k+1},z'_{k})\prod_j^{[k+1:K-1]}(x-z'_j)-\prod_r^{[1:K]}(x-z_r)\rVert^2$
							\item $=\lVert\sum_k^{[0:K-1]}\Lambda(t_{k+1},z'_{k})\prod_j^{[k+1:K-1]}(x-z'_j)\rVert^2$
							\item $\le(K-1)\sum_k^{[0:K-1]}\lVert\Lambda(t_{k+1},z'_{k})\rVert^2\lVert\prod_j^{[k+1:K-1]}(x-z'_j)\rVert^2$
							\item $\le(K-1)\sum_k^{[0:K-1]}(\frac{a_1}{n^2}+\frac{b_1}{m^2})\prod_j^{[k+1:K-1]}\lVert x-z'_j\rVert^2$
							\item $\le(\frac{a_1}{n^2}+\frac{b_1}{m^2})(K-1)\sum_k^{[0:K-1]}\prod_j^{[k+1:K-1]}2(\lVert x\rVert^2+\lVert z'_j\rVert^2)$
							\item $\le(\frac{a_1}{n^2}+\frac{b_1}{m^2})(K-1)\sum_k^{[0:K-1]}2^{K-k-2}(X+a_3)^{K-k-2}$
							\item $=\frac{a}{n^2}+\frac{b}{m^2}$.
						\end{enumerate}
					\end{enumerate}
					\item \textbf{Yield the tuple $\langle a,b,c,d,p\rangle$.}
				\end{enumerate}
		\procedure{3.66}
			\objective
				Choose a rational number $X$ and a positive integer $K$. The objective of the following instructions is to construct rational numbers $a,b,c,d$, and a procedure, $p(x,n,m)$, to show that $\lVert x^K-1-\prod_r^{[0:K]}(x-\exp_n(\frac{r}{K}\tau_mi))\rVert^2\le\frac{a}{n^2}+\frac{b}{m^2}$ when a complex number $x$ and positive integers $n,m$ such that $n>c$, $m>d$, and $\lVert x\rVert^2\le X$ are chosen.
			\implementation
				\begin{enumerate}
					\item Execute \procedurehr{3.65} on $\langle X,K\rangle$ and let $\langle a_1,b_1,c,d,p_1\rangle$ receive.
					\item Let $a=2(X+1)a_1$.
					\item Let $b=2(X+1)b_1$.
					\item Let $p(x,n,m)$ be the following procedure:
					\begin{enumerate}
						\item Execute procedure $p_1$ on $\langle x,n,m\rangle$.
						\item Hence verify that $\lVert\sum_r^{[0:K]}x^r-\prod_r^{[1:K]}(x-\exp_n(\frac{r}{K}\tau_mi))\rVert^2\le\frac{a_1}{n^2}+\frac{b_1}{m^2}$.
						\item Hence verify that $\lVert x^K-1-\prod_r^{[0:K]}(x-\exp_n(\frac{r}{K}\tau_mi))\rVert^2$
						\begin{enumerate}
							\item $=\lVert(x-1)\sum_r^{[0:K]}x^r-(x-1)\prod_r^{[1:K]}(x-\exp_n(\frac{r}{K}\tau_mi))\rVert^2$
							\item $=\lVert x-1\rVert^2\lVert\sum_r^{[0:K]}x^r-\prod_r^{[1:K]}(x-\exp_n(\frac{r}{K}\tau_mi))\rVert^2$
							\item $\le\lVert x-1\rVert^2(\frac{a_1}{n^2}+\frac{b_1}{m^2})$
							\item $\le 2(\lVert x\rVert^2+\lVert 1\rVert^2)(\frac{a_1}{n^2}+\frac{b_1}{m^2})$
							\item $\le 2(X+1)(\frac{a_1}{n^2}+\frac{b_1}{m^2})$
							\item $=\frac{a}{n^2}+\frac{b}{m^2}$.
						\end{enumerate}
					\end{enumerate}
					\item \textbf{Yield the tuple $\langle a,b,c,d,p\rangle$.}
				\end{enumerate}
		\procedure{3.67}
			\objective
				Choose a rational number $X$ and a positive integer $K$. The objective of the following instructions is to construct rational numbers $a,b,c,d$, and a procedure, $p(x,n,m)$, to show that $\lVert\exp_K(x)-1-x\prod_r^{[1:K]}(1-\frac{x}{K(\exp_n(\frac{r}{K}\tau_mi)-1)})\rVert^2\le\frac{a}{n^2}+\frac{b}{m^2}$ when a complex number $x$ and positive integers $n,m$ such that $n>c$, $m>d$, and $\lVert x\rVert^2\le X$ are chosen.
			\implementation
				\begin{enumerate}
					\item Execute \procedurehr{3.66} on $\langle 2(1+\frac{X}{K^2}),K\rangle$ and let $\langle a_1,b_1,c_1,d_1,p_1\rangle$ receive.
					\item Execute \procedurehr{3.65} on $\langle 1,K\rangle$ and let $\langle a_2,b_2,c_2,d_2,p_2\rangle$ receive.
					\item Execute \procedurehr{3.63} on $\langle K\rangle$ and let $\langle a_3,b_3,c_3,p_3\rangle$ receive.
					\item Let $l=2\frac{X}{K^2}2^{K-1}(1+\frac{X}{K^2a_3})^{K-1}$.
					\item Let $a=2a_1+la_2$.
					\item Let $b=2b_1+lb_2$.
					\item Let $c=\max(c_1,c_2,b_3)$.
					\item Let $d=\max(d_1,d_2,c_3)$.
					\item Let $p(x,n,m)$ be the following procedure:
					\begin{enumerate}
						\item Verify that $\lVert 1+\frac{x}{K}\rVert^2\le 2(\lVert 1\rVert^2+\lVert\frac{x}{K}\rVert^2)\le 2(1+\frac{X}{K^2})$.
						\item Hence execute procedure $p_1$ on $\langle 1+\frac{x}{K},n,m\rangle$.
						\item Hence verify that $\lVert(1+\frac{x}{K})^K-1-\prod_r^{[0:K]}(1+\frac{x}{K}-\exp_n(\frac{r}{K}\tau_mi))\rVert^2\le\frac{a_1}{n^2}+\frac{b_1}{m^2}$.
						\item Execute procedure $p_2$ on $\langle 1,n,m\rangle$.
						\item Hence verify that $\lVert K-\prod_r^{[1:K]}(1-\exp_n(\frac{r}{K}\tau_mi))\rVert^2=\sum_r^{[0:K]}1^r-\prod_r^{[1:K]}(1-\exp_n(\frac{r}{K}\tau_mi))\rVert^2\le\frac{a_2}{n^2}+\frac{b_2}{m^2}$.
						\item For $j\in[1:K]$, do the following:
						\begin{enumerate}
							\item Execute procedure $p_3$ on $\langle n,0,j,m\rangle$.
							\item Hence verify that $\lVert\exp_n(\frac{j}{K}\tau_mi)-1\rVert^2\ge a_3$.
							\item Let $z_j=K(\exp_n(\frac{r}{K}\tau_mi)-1)$.
						\end{enumerate}
						\item Hence verify that $\lVert\exp_K(x)-1-x\prod_r^{[1:K]}(1-\frac{x}{z_r})\rVert^2$
						\begin{enumerate}
							\item $=\lVert\exp_K(x)-1-\prod_r^{[0:K]}(1+\frac{x}{K}-\exp_n(\frac{r}{K}\tau_mi))+\prod_r^{[0:K]}(1+\frac{x}{K}-\exp_n(\frac{r}{K}\tau_mi))-x\prod_r^{[1:K]}(1-\frac{x}{z_r})\rVert^2$
							\item $\le 2\lVert\exp_K(x)-1-\prod_r^{[0:K]}(1+\frac{x}{K}-\exp_n(\frac{r}{K}\tau_mi))\rVert^2+2\lVert\prod_r^{[0:K]}(1+\frac{x}{K}-\exp_n(\frac{r}{K}\tau_mi))-x\prod_r^{[1:K]}(1-\frac{x}{z_r})\rVert^2$
							\item $\le 2(\frac{a_1}{n^2}+\frac{b_1}{m^2})+2\lVert\frac{x}{K}\prod_r^{[1:K]}(1+\frac{x}{K}-\exp_n(\frac{r}{K}\tau_mi))-x\prod_r^{[1:K]}(1-\frac{x}{z_r})\rVert^2$
							\item $=2(\frac{a_1}{n^2}+\frac{b_1}{m^2})+2\lVert\frac{x}{K}\prod_r^{[1:K]}(1-\exp_n(\frac{r}{K}\tau_mi))\prod_r^{[1:K]}(1-\frac{x}{z_r})-x\prod_r^{[1:K]}(1-\frac{x}{z_r})\rVert^2$
							\item $=2(\frac{a_1}{n^2}+\frac{b_1}{m^2})+2\lVert\frac{x}{K}\prod_r^{[1:K]}(1-\frac{x}{z_r})\rVert^2\lVert\prod_r^{[1:K]}(1-\exp_n(\frac{r}{K}\tau_mi))-K\rVert^2$
							\item $\le 2(\frac{a_1}{n^2}+\frac{b_1}{m^2})+2\frac{X}{K^2}(\prod_r^{[1:K]}\lVert 1-\frac{x}{z_r}\rVert^2)(\frac{a_2}{n^2}+\frac{b_2}{m^2})$
							\item $\le 2(\frac{a_1}{n^2}+\frac{b_1}{m^2})+2\frac{X}{K^2}(\prod_r^{[1:K]} 2(\lVert 1\rVert^2+\lVert\frac{x}{z_r}\rVert^2))(\frac{a_2}{n^2}+\frac{b_2}{m^2})$
							\item $\le 2(\frac{a_1}{n^2}+\frac{b_1}{m^2})+2\frac{X}{K^2}(\prod_r^{[1:K]} 2(1+\frac{X}{K^2a_3}))(\frac{a_2}{n^2}+\frac{b_2}{m^2})$
							\item $=\frac{a}{n^2}+\frac{b}{m^2}$.
						\end{enumerate}
					\end{enumerate}
					\item \textbf{Yield the tuple $\langle a,b,c,d,p\rangle$.}
				\end{enumerate}
	\twocolumn[\part{Matrix Arithmetic}]
		\declaration{4.28}
			The phrase "\gls{matrix}" will be used as a shorthand for a list of equally lengthed lists of polynomials. In particular, the phrase "$m\times n$ matrix" will be used as a shorthand for a length-$m$ list of length-$n$ lists of polynomials.
		\declaration{4.29}
			The notation \gls{matrixentry}, where $A$ is a matrix and $I,J$ are lists of indicies, will be used as a shorthand for $\langle(A_j)_J\for j\in I\rangle$.
		\declaration{4.30}
			The phrase "\gls{matrixA=B}", where $A,B$ are $m\times n$ matrices, will be used as a shorthand for "$A_{i,j}=B_{i,j}$ for $j\in[0:n]$, for $i\in[0:m]$".
		\procedure{4.73}
			\objective
				Choose an $m\times n$ matrix $A$. The objective of the following instructions is to show that $A=A$.
			\implementation
				\begin{enumerate}
					\item Verify that $A_{i,j}=A_{i,j}$ for $j\in[0:n]$, for $i\in[0:m]$.
					\item \textbf{Hence verify that $A=A$.}
				\end{enumerate}
		\procedure{4.74}
			\objective
				Choose two $m\times n$ matrices $A,B$ such that $A=B$. The objective of the following instructions is to show that $B=A$.
			\implementation
				\begin{enumerate}
					\item Verify that $A_{i,j}=B_{i,j}$ for $j\in[0:n]$, for $i\in[0:m]$.
					\item Hence verify that $B_{i,j}=A_{i,j}$ for $j\in[0:n]$, for $i\in[0:m]$.
					\item \textbf{Hence verify that $B=A$.}
				\end{enumerate}
		\procedure{4.75}
			\objective
				Choose three $m\times n$ matrices $A,B,C$ such that $A=B$ and $B=C$. The objective of the following instructions is to show that $A=C$.
			\implementation
				\begin{enumerate}
					\item Verify that $A_{i,j}=B_{i,j}$ for $j\in[0:n]$, for $i\in[0:m]$.
					\item Verify that $B_{i,j}=C_{i,j}$ for $j\in[0:n]$, for $i\in[0:m]$.
					\item Hence verify that $A_{i,j}=C_{i,j}$ for $j\in[0:n]$, for $i\in[0:m]$.
					\item \textbf{Hence verify that $A=C$.}
				\end{enumerate}
		\declaration{4.31}
			The notation \gls{matrixA+B}, where $A,B$ are $m\times n$ matrices, will be used as a shorthand for the list $\langle\langle A_{i,j}+B_{i,j}\for j\in[0:n]\rangle\for i\in[0:m]\rangle$.
		\procedure{4.76}
			\objective
				Choose four $m\times n$ matrices $A,B,C,D$ such that $A=C$ and $B=D$. The objective of the following instructions is to show that $A+B=C+D$.
			\implementation
				\begin{enumerate}
					\item Verify that $A_{i,j}=C_{i,j}$ for $j\in[0:n]$, for $i\in[0:m]$.
					\item Verify that $B_{i,j}=D_{i,j}$ for $j\in[0:n]$, for $i\in[0:m]$.
					\item Hence verify that $A+B$
					\begin{enumerate}
						\item $=\langle\langle A_{i,j}+B_{i,j}\for j\in[0:n]\rangle\for i\in[0:m]\rangle$
						\item $=\langle\langle C_{i,j}+D_{i,j}\for j\in[0:n]\rangle\for i\in[0:m]\rangle$
						\item $=C+D$.
					\end{enumerate}
				\end{enumerate}
		\procedure{4.77}
			\objective
				Choose three $m\times n$ matrices $A,B,C$. The objective of the following instructions is to show that $(A+B)+C=A+(B+C)$.
			\implementation
				\begin{enumerate}
					\item Verify that $(A+B)+C$
					\begin{enumerate}
						\item $=\langle\langle (A+B)_{i,j}+C_{i,j}\for j\in[0:n]\rangle\for i\in[0:m]\rangle$
						\item $=\langle\langle (A_{i,j}+B_{i,j})+C_{i,j}\for j\in[0:n]\rangle\for i\in[0:m]\rangle$
						\item $=\langle\langle A_{i,j}+(B_{i,j}+C_{i,j})\for j\in[0:n]\rangle\for i\in[0:m]\rangle$
						\item $=\langle\langle A_{i,j}+(B+C)_{i,j}\for j\in[0:n]\rangle\for i\in[0:m]\rangle$
						\item $=A+(B+C)$.
					\end{enumerate}
				\end{enumerate}
		\procedure{4.78}
			\objective
				Choose two $m\times n$ matrices $A,B$. The objective of the following instructions is to show that $A+B=B+A$.
			\implementation
				\begin{enumerate}
					\item $A+B$
					\begin{enumerate}
						\item $=\langle\langle A_{i,j}+B_{i,j}\for j\in[0:n]\rangle\for i\in[0:m]\rangle$
						\item $=\langle\langle B_{i,j}+A_{i,j}\for j\in[0:n]\rangle\for i\in[0:m]\rangle$
						\item $=B+A$.
					\end{enumerate}
				\end{enumerate}
		\declaration{4.32}
			The notation \gls{matrix0mn} will contextually be used as a shorthand for the list $\langle\langle 0\for j\in[0:n]\rangle\for i\in[0:m]\rangle$ where the natural numbers $m,n$ are determined by the context.
		\procedure{4.79}
			\objective
				Choose an $m\times n$ matrix $A$. The objective of the following instructions is to show that $0+A=A$.
			\implementation
				\begin{enumerate}
					\item Verify that $0+A$
					\begin{enumerate}
						\item $=0_{m\times n}+A$
						\item $=\langle\langle 0_{i,j}+A_{i,j}\for j\in[0:n]\rangle\for i\in[0:m]\rangle$
						\item $=\langle\langle 0+A_{i,j}\for j\in[0:n]\rangle\for i\in[0:m]\rangle$
						\item $=\langle\langle A_{i,j}\for j\in[0:n]\rangle\for i\in[0:m]\rangle$
						\item $=A$.
					\end{enumerate}
				\end{enumerate}
		\declaration{4.33}
			The notation \gls{matrix-A}, where $A$ is an $m\times n$ matrix, will be used as a shorthand for the list $\langle\langle -A_{i,j}\for j\in[0:n]\rangle\for i\in[0:m]\rangle$.
		\procedure{4.80}
			\objective
				Choose two $m\times n$ matrices $A,B$ such that $A=B$. The objective of the following instructions is to show that $-A=-B$.
			\implementation
				\begin{enumerate}
					\item Verify that $A_{i,j}=B_{i,j}$ for $j\in[0:n]$, for $i\in[0:m]$.
					\item Hence verify that $-A$
					\begin{enumerate}
						\item $=\langle\langle -A_{i,j}\for j\in[0:n]\rangle\for i\in[0:m]\rangle$
						\item $=\langle\langle -B_{i,j}\for j\in[0:n]\rangle\for i\in[0:m]\rangle$
						\item $=-B$.
					\end{enumerate}
				\end{enumerate}
		\procedure{4.81}
			\objective
				Choose a $m\times n$ matrix $A$. The objective of the following instructions is to show that $-A+A=0$.
			\implementation
				\begin{enumerate}
					\item Verify that $-A+A$
					\begin{enumerate}
						\item $\langle\langle (-A)_{i,j}+A_{i,j}\for j\in[0:n]\rangle\for i\in[0:m]\rangle$
						\item $\langle\langle -(A_{i,j})+A_{i,j}\for j\in[0:n]\rangle\for i\in[0:m]\rangle$
						\item $\langle\langle 0\for j\in[0:n]\rangle\for i\in[0:m]\rangle$
						\item $=0$.
					\end{enumerate}
				\end{enumerate}
		\declaration{4.34}
			The notation \gls{matrixAB}, where $A$ is an $m\times n$ matrix and $B$ is an $n\times k$ matrix, will be used as a shorthand for the list $\langle\langle\sum_r^{[0:n]}A_{i,r}B_{r,j}\for j\in[0:k]\rangle\for i\in[0:m]\rangle$.
		\procedure{4.82}
			\objective
				Choose two $m\times n$ matrices $A,C$ and two $n\times k$ matrices $B,D$ such that $A=C$ and $B=D$. The objective of the following instructions is to show that $AB=CD$.
			\implementation
				\begin{enumerate}
					\item Verify that $A_{i,j}=C_{i,j}$ for $j\in[0:n]$, for $i\in[0:m]$.
					\item Verify that $B_{i,j}=D_{i,j}$ for $j\in[0:k]$, for $i\in[0:n]$.
					\item Hence verify that $AB$
					\begin{enumerate}
						\item $=\langle\langle\sum_r^{[0:n]}A_{i,r}B_{r,j}\for j\in[0:k]\rangle\for i\in[0:m]\rangle$
						\item $=\langle\langle\sum_r^{[0:n]}C_{i,r}D_{r,j}\for j\in[0:k]\rangle\for i\in[0:m]\rangle$
						\item $=CD$.
					\end{enumerate}
				\end{enumerate}
		\procedure{4.02}
			\objective
				Choose an $m\times n$ matrix, A, an $n\times p$ matrix, B, and a $p\times q$ matrix, C. The objective of the following instructions is to show that $(AB)C=A(BC)$.
			\implementation
				\begin{enumerate}
					\item Verify that $(AB)C$
					\begin{enumerate}
						\item $=\langle\langle\sum_r^{[0:p]}(AB)_{i,r}C_{r,j}\for j\in[0:q]\rangle\for i\in[0:m]\rangle$
						\item $=\langle\langle\sum_r^{[0:p]}(\sum_l^{[0:n]}A_{i,l}B_{l,r})C_{r,j}\for j\in[0:q]\rangle\for i\in[0:m]\rangle$
						\item $=\langle\langle\sum_r^{[0:p]}\sum_l^{[0:n]}A_{i,l}B_{l,r}C_{r,j}\for j\in[0:q]\rangle\for i\in[0:m]\rangle$
						\item $=\langle\langle\sum_l^{[0:n]}\sum_r^{[0:p]}A_{i,l}B_{l,r}C_{r,j}\for j\in[0:q]\rangle\for i\in[0:m]\rangle$
						\item $=\langle\langle\sum_l^{[0:n]}A_{i,l}\sum_r^{[0:p]}B_{l,r}C_{r,j}\for j\in[0:q]\rangle\for i\in[0:m]\rangle$
						\item $=\langle\langle\sum_l^{[0:n]}A_{i,l}(BC)_{l,j}\for j\in[0:q]\rangle\for i\in[0:m]\rangle$
						\item $=A(BC)$.
					\end{enumerate}
				\end{enumerate}
		\declaration{4.35}
			The notation \gls{matrixamm}, where $a\ne 0$ is a polynomial, will contextually be used as a shorthand for the list $\langle\langle a[i=j]\for j\in[0:m]\rangle\for i\in[0:m]\rangle$.
		\procedure{4.84}
			\objective
				Choose an $m\times n$ matrix, $A$. The objective of the following instructions is to show that $1A=A$.
			\implementation
				\begin{enumerate}
					\item Verify that $1A$
					\begin{enumerate}
						\item $=1_mA$
						\item $=\langle\langle\sum_r^{[0:m]}1_{i,r}A_{r,j}\for j\in[0:n]\rangle\for i\in[0:m]\rangle$
						\item $=\langle\langle\sum_r^{[0:m]}[i=r]A_{r,j}\for j\in[0:n]\rangle\for i\in[0:m]\rangle$
						\item $=\langle\langle A_{i,j}\for j\in[0:n]\rangle\for i\in[0:m]\rangle$
						\item $=A$.
					\end{enumerate}
				\end{enumerate}
		\procedure{4.85}
			\objective
				Choose an $m\times n$ matrix $A$, and two $n\times k$ matrices $B,C$. The objective of the following instructions is to show that $A(B+C)=AB+AC$.
			\implementation
				\begin{enumerate}
					\item $A(B+C)$
					\begin{enumerate}
						\item $=\langle\langle\sum_r^{[0:n]}A_{i,r}(B+C)_{r,j}\for j\in[0:k]\rangle\for i\in[0:m]\rangle$
						\item $=\langle\langle\sum_r^{[0:n]}A_{i,r}(B_{r,j}+C_{r,j})\for j\in[0:k]\rangle\for i\in[0:m]\rangle$
						\item $=\langle\langle\sum_r^{[0:n]}(A_{i,r}B_{r,j}+A_{i,r}C_{r,j})\for j\in[0:k]\rangle\for i\in[0:m]\rangle$
						\item $=\langle\langle\sum_r^{[0:n]}A_{i,r}B_{r,j}+\sum_r^{[0:n]}A_{i,r}C_{r,j}\for j\in[0:k]\rangle\for i\in[0:m]\rangle$
						\item $=\langle\langle\sum_r^{[0:n]}A_{i,r}B_{r,j}\for j\in[0:k]\rangle\for i\in[0:m]\rangle+\langle\langle\sum_r^{[0:n]}\sum_r^{[0:n]}A_{i,r}C_{r,j}\for j\in[0:k]\rangle\for i\in[0:m]\rangle$
						\item $=AB+AC$.
					\end{enumerate}
				\end{enumerate}
		\declaration{4.36}
			The phrase "row $i$ of $A$" and the notation \gls{matrixrow}, where $A$ is an $m\times n$ matrix and $0\le i<m$, will be used as a shorthand for $A_{i,[0:n]}$.
		\declaration{4.37}
			The phrase "column $i$ of $A$" and the notation \gls{matrixcol}, where $A$ is an $m\times n$ matrix and $0\le i<n$, will be used as a shorthand for $A_{[0:m],i}$.
		\procedure{4.00}
			\objective
				Choose an $m\times 2$ matrix, $A$. Let $\deg(0)=\infty$. Let $k=\min(\deg(A_{0,0}),\deg(A_{0,1}))$ and $q=\deg(A_{0,0})$. The objective of the following instructions is to make $A_{0,1}=0$, $\deg(A_{0,0})\le k$, and either leave $A_{*,0}$ unchanged or make $\deg(A_{0,0})<q$ by a sequence of operations whereby, in each step a polynomial times either of the columns is added to the other.
			\implementation
				\begin{enumerate}
					\item Let $A$ be our working matrix.
					\item While $A_{0,1}\ne 0$, do the following:
					\begin{enumerate}
						\item If $\deg(A_{0,0})\le\deg(A_{0,1})$, then:
						\begin{enumerate}
							\item Subtract $\frac{(A_{0,1})_{\deg(A_{0,1})}}{(A_{0,0})_{\deg(A_{0,0})}}\lambda^{\deg(A_{0,1})-\deg(A_{0,0})}$ times $A_{0,0}$ from $A_{0,1}$.
							\item Now verify that either $A_{0,1}$'s degree has decreased or $A_{0,1}=0$.
						\end{enumerate}
						\item Otherwise, do the following:
						\begin{enumerate}
							\item Let $p=\frac{(A_{0,0})_{\deg(A_{0,0})}}{(A_{0,1})_{\deg(A_{0,1})}}\lambda^{\deg(A_{0,0})-\deg(A_{0,1})}$.
							\item If $A_{0,0}=pA_{0,1}$, then do the following:
							\begin{enumerate}
								\item Add $1-p$ times $A_{0,1}$ to $A_{0,0}$.
								\item Verify that now $A_{0,0}=A_{0,1}$.
							\end{enumerate}
							\item Otherwise, do the following:
							\begin{enumerate}
								\item Verify that $A_{0,0}\ne pA_{0,1}$.
								\item Add $-p$ times $A_{0,1}$ to $A_{0,0}$.
							\end{enumerate}
							\item Therefore verify that $A_{0,0}\ne 0$.
							\item Also verify that $A_{0,0}$'s degree has decreased.
						\end{enumerate}
					\end{enumerate}
					\item \textbf{Verify that $A_{0,1}=0$.}
					\item Verify that the changes to $A_{0,0}$, if any, have decreased its degree.
					\item If both operations are well-defined, then do the following:
					\begin{enumerate}
						\item Verify that all changes to $A_{0,1}$ but the last have decreased its degree.
						\item Verify that $\deg(A_{0,0})\le$ the degree of the penultimate value of $A_{0,1}$.
					\end{enumerate}
					\item \textbf{Therefore verify that $\deg(A_{0,0})\le k$.}
					\item If $A_{*,0}$ was changed, then do the following:
					\begin{enumerate}
						\item Verify that $A_{0,0}$ was also changed.
						\item \textbf{Therefore verify that $\deg(A_{0,0})<q$.}
					\end{enumerate}
					\item \textbf{Yield the tuple $\langle A\rangle$.}
				\end{enumerate}
		\declaration{4.01}
			The phrase "\gls{matrixdiag}" will be used as a shorthand for matrix positions such that the row index equals the column index.
		\declaration{4.02}
			The phrase "\gls{matrixdiagmatrix}" will be used to refer to matrices with $0$s in all off-diagonal positions.
		\procedure{4.01}
			\objective
				Choose a $m\times n$ matrix, $A$. The objective of the following instructions is to transform $A$ into an $m\times n$ diagonal matrix by a sequence of operations whereby either a polynomial times any of the columns is added to a different column, or a polynomial times any of the rows is added to a different row.
			\implementation
				\begin{enumerate}
					\item If $m=0$ or $n=0$, then do the following:
					\begin{enumerate}
						\item \textbf{Verify that $A$ is an $m\times n$ diagonal matrix.}
						\item \textbf{Yield the tuple $\langle A\rangle$.}
					\end{enumerate}
					\item Otherwise do the following:
					\item Verify that $m>0$ and $n>0$.
					\item Let $A$ be our working matrix.
					\item Now do the following:
						\begin{enumerate}
						\item While $A_{0,[1:n]}\ne 0$, do the following:
						\begin{enumerate}
							\item Select the $m\times 2$ matrix whose top-right entry coincides with the last non-zero entry of the first row
							\item Apply \procedurehr{4.00} on this submatrix.
							\item Verify that the top-left and top-right entries of the submatrix are now non-zero and zero respectively.
							\item If $A_{*,0}$ was modified by (5aii), then do the following:
							\begin{enumerate}
								\item Verify that $\deg(A_{0,0})$ decreased.
								\item Go back to (5).
							\end{enumerate}
						\end{enumerate}
						\item Now do the same operations as in (a), but this time with the operations themselves reflected across the matrix's diagonal.
					\end{enumerate}
					\item Verify that $A_{0,[1:n]}=0$.
					\item Also verify that $A_{[1:m],0}=0$.
					\item Apply \procedurehr{4.01} on the submatrix $A_{[1:m],[1:n]}$.
					\item Verify that (8)'s execution leaves the first row and column unchanged.
					\item Also verify that $A_{[1:m],[1:n]}$ is now a $(m-1)\times(n-1)$ diagonal matrix.
					\item \textbf{Therefore verify that $A$ is now an $m\times n$ diagonal matrix.}
					\item \textbf{Yield the tuple $\langle A\rangle$.}
				\end{enumerate}
		\declaration{4.04}
			The phrase "\gls{matrixtilt}" will be used to refer to square matrices with only $1$s on the diagonal, a single polynomial off the diagonal, and $0$s everywhere else.
		\procedure{4.03}
			\objective
				Choose a procedure, $A$, and two non-negative integers $m,n$. The objective of the following instructions is, once $A$ has been executed, to construct a list of $m\times m$ tilts, $M$, and a list of $n\times n$ tilts, $N$ such that $M_{\lvert M\rvert-1-i}$ equals $1_m$ after applying the $i^{th}$ row operation carried out by $A$ also on it, and $N_i$ equals $1_n$ after applying the $i^{th}$ row operation carried out by $A$ also on it.
			\implementation
				\begin{enumerate}
					\item Make an empty list, $N$.
					\item Augment procedure $A$ so that each time a polynomial $x$ times a column $i$ is added onto column $j$, an $n\times n$ matrix that only has $1$s on its diagonal, and such that the only non-zero entry off its diagonal is $x$ at position $(i,j)$, is appended onto $N$.
					\item Make an empty list, $M$.
					\item Also augment procedure $A$ so that each time a polynomial $x$ times a row $i$ is added onto row $j$, an $n\times n$ matrix that only has $1$s on its diagonal, and such that the only non-zero entry off its diagonal is $x$ at position $(j,i)$, is prepended onto $M$.
					\item Now run procedure $A$.
					\item \textbf{Yield the tuple $\langle M,N\rangle$.}
				\end{enumerate}
		\procedure{4.04}
			\objective
				Choose a $m\times n$ matrix, $A$. The objective of the following instructions is to show that $1_mA=A=A1_n$.
			\implementation
				\begin{enumerate}
					\item For $0\le r<m$, do the following:
					\begin{enumerate}
						\item For $0\le t<n$, do the following:
						\begin{enumerate}
							\item Verify that $(1_mA)_{r,t}=\sum_u^{[0:m]} (1_m)_{r,u}A_{u,t}=(1_m)_{r,r}A_{r,t}=1*A_{r,t}=A_{r,t}$.
						\end{enumerate}
					\end{enumerate}
					\item \textbf{Therefore verify that $1_mA=A$.}
					\item For $0\le r<m$, do the following:
					\begin{enumerate}
						\item For $0\le t<n$, do the following:
						\begin{enumerate}
							\item Verify that $(A1_n)_{r,t}=\sum_u^{[0:m]} A_{r,u}(1_n)_{u,t}=A_{r,t}(1_n)_{t,t}=A_{r,t}*1=A_{r,t}$.
						\end{enumerate}
					\end{enumerate}
					\item \textbf{Therefore verify that $A1_n=A$.}
				\end{enumerate}
		\declaration{4.05}
			The notation \gls{matrixA^-1}, where $A$ is a list of $m\times m$ tilts, will be used to refer to the result yielded by executing the following instructions:
			\begin{enumerate}
				\item Let $A^{-1}$ be $\langle\rangle$.
				\item For $i$ in $[0:\lvert A\rvert]$, do the following:
				\begin{enumerate}
					\item Let $(j,k)$ be the position of the off diagonal entry of $A_i$.
					\item Let $B$ equal $A_i$ but with entry $(j,k)$ negated.
					\item Now prepend $B$ onto $A^{-1}$.
				\end{enumerate}
				\item \textbf{Yield $\langle A^{-1}\rangle$.}
			\end{enumerate}
		\procedure{4.05}
			\objective
				Choose a list of $m\times m$ tilts, $A$. The objective of the following instructions is to show that $A_*{A^{-1}}_*=1_m$.
			\implementation
				\begin{enumerate}
					\item Verify that $\lvert A\rvert=\lvert A^{-1}\rvert$.
					\item For $i$ in $[0:\lvert A\rvert]$, do the following:
					\begin{enumerate}
						\item Let $(j,k)$ be the position of the off diagonal entry of $A_i$.
						\item Let $B={A^{-1}}_{\lvert A\rvert-1-i}$.
						\item For $r$ in $[0:m]$ and $r\ne j$, do the following:
						\begin{enumerate}
							\item For $t$ in $[0:m]$, do the following:
							\begin{enumerate}
								\item Verify that $(A_iB)_{r,t}=\sum_u^{[0:m]} (A_i)_{r,u}B_{u,t}=(A_i)_{r,r}B_{r,t}=1*B_{r,t}=[r=t]$.
							\end{enumerate}
						\end{enumerate}
						\item For $t$ in $[0:m]$ and $t\ne k$, do the following:
						\begin{enumerate}
							\item Verify that $(A_iB)_{j,t}=\sum_u^{[0:m]} (A_i)_{j,u}B_{u,t}=(A_i)_{j,t}B_{t,t}=(A_i)_{j,t}*1=[j=t]$.
						\end{enumerate}
						\item Verify that $(A_iB)_{j,k}=\sum_u^{[0:m]} (A_i)_{j,u}B_{u,k}=(A_i)_{j,j}B_{j,k}+(A_i)_{j,k}B_{k,k}=1*B_{j,k}+(A_i)_{j,k}*1=B_{j,k}+(A_i)_{j,k}=0$.
						\item Therefore verify that $A_iB=1_m$.
					\end{enumerate}
					\item Therefore using \procedurehr{4.02} and \procedurehr{4.04}, verify that $A_*{A^{-1}}_*$
					\begin{enumerate}
						\item $=A_0\cdots A_{\lvert A\rvert-2}A_{\lvert A\rvert-1}{A^{-1}}_0{A^{-1}}_1\cdots {A^{-1}}_{\lvert A\rvert-1}$
						\item $=A_0\cdots A_{\lvert A\rvert-3}A_{\lvert A\rvert-2}1_m{A^{-1}}_1{A^{-1}}_2\cdots {A^{-1}}_{\lvert A\rvert-1}$
						\item $=A_0\cdots A_{\lvert A\rvert-3}A_{\lvert A\rvert-2}{A^{-1}}_1{A^{-1}}_2\cdots {A^{-1}}_{\lvert A\rvert-1}$
						\item $\vdots$
						\item $=A_01_m{A^{-1}}_{\lvert A\rvert-1}$
						\item $=A_0{A^{-1}}_{\lvert A\rvert-1}$
						\item $=1_m$.
					\end{enumerate}
				\end{enumerate}
		\procedure{4.06}
			\objective
				Choose a list of $m\times m$ tilts, $A$. The objective of the following instructions is to show that $(A^{-1})^{-1}=A$ and ${A^{-1}}_*A_*=1_m$.
			\implementation
				\begin{enumerate}
					\item \textbf{Verify that $(A^{-1})^{-1}=A$.}
					\item \textbf{Therefore using \procedurehr{4.05}, verify that ${A^{-1}}_*A_*={A^{-1}}_*{(A^{-1})^{-1}}_*=1_m$.}
				\end{enumerate}
		\procedure{4.07}
			\objective
				Choose a $2\times 2$ diagonal matrix, $A$. The objective of the following instructions is to construct polynomials $u,v$ and transform $A$ into a $2\times 2$ diagonal matrix, $A'$, such that $A'_{1,1}=uA'_{0,0}$ and $A_{0,0}=vA'_{0,0}$ by a sequence of operations whereby either a polynomial times any of the columns is added to a different column, or a polynomial times any of the rows is added to a different row.
			\implementation
				\begin{enumerate}
					\item Add row $1$ to row $0$.
					\item Now verify that $A_{0,1}=A_{1,1}$.
					\item Set $A'=A$ and let $A'$ be our working matrix.
					\item Let $\langle M,N\rangle$ receive the results of executing \procedurehr{4.03} on the pair $\langle 2,2\rangle$ and the following procedure:
					\begin{enumerate}
						\item Execute \procedurehr{4.00} on $A'$.
					\end{enumerate}
					
					\item Using (4), verify that $M$ is empty.
					\item Using (4) and (5), verify that $AN_*=M_*AN_*=A'$.
					\item Using (6), verify that $A=A1_n=AN_*{N^{-1}}_*=A'{N^{-1}}_*$.
					\item Using (4), verify that $A'_{0,1}=0$.
					\item \textbf{Using (4) and (7), verify that $A_{0,0}=A'_{0,0}{{N^{-1}}_*}_{0,0}+A'_{0,1}{{N^{-1}}_*}_{1,0}=A'_{0,0}{{N^{-1}}_*}_{0,0}$.}
					\item Using (4) and (7), verify that $A_{1,1}=A_{0,1}=A'_{0,0}{{N^{-1}}_*}_{0,1}+A'_{0,1}{{N^{-1}}_*}_{1,1}=A'_{0,0}{{N^{-1}}_*}_{0,1}$.
					\item Using (2), verify that $A_{1,0}=0$.
					\item Using (6) and (11), verify that $A'_{1,0}=A_{1,0}{N_*}_{0,0}+A_{1,1}{N_*}_{1,0}=A_{1,1}{N_*}_{1,0}=A'_{0,0}{{N^{-1}}_*}_{0,1}{N_*}_{1,0}$.
					\item \textbf{Using (6) and (11), verify that $A'_{1,1}=A_{1,0}{N_*}_{0,1}+A_{1,1}{N_*}_{1,1}=A_{1,1}{N_*}_{1,1}=A'_{0,0}{{N^{-1}}_*}_{0,1}{N_*}_{1,1}$.}
					\item Subtract ${{N^{-1}}_*}_{0,1}{N_*}_{1,0}$ times row $0$ from row $1$.
					\item Now using (14) and (12), verify that $A'_{1,0}=0$.
					\item \textbf{Therefore verify that $A'$ is a $2\times 2$ diagonal matrix.}
					\item \textbf{Let $A=A'$.}
					\item \textbf{Yield $\langle{{N^{-1}}_*}_{0,1}{N_*}_{1,1},{{N^{-1}}_*}_{0,0}\rangle$.}
				\end{enumerate}
		\procedure{4.08}
			\objective
				Choose a $m\times n$ matrix, $A$ such that $\min(m,n)>0$. The objective of the following instructions is to define a list of polynomials $u$ and transform $A$ into an $m\times n$ diagonal matrix such that $A_{k,k}=u_kA_{0,0}$ for $k$ in $[0:\min(m,n)]$ by a sequence of operations whereby either a polynomial times any of the columns is added to a different column, or a polynomial times any of the rows is added to a different row.
			\implementation
				\begin{enumerate}
					\item Let $u=\langle 1\rangle$.
					\item Execute \procedurehr{4.01} on $A$.
					\item Verify that $A$ is an $m\times n$ diagonal matrix.
					\item For $j$ in $[1:\min(m,n)]$, do the following:
					\begin{enumerate}
						\item Using (h), verify that $A_{k,k}=u_kA_{0,0}$ for $k$ in $[0:j]$.
						\item Set $A'=A$.
						\item Execute \procedurehr{4.07} on $A'_{\langle 0,j\rangle,\langle 0,j\rangle}$ and let $\langle u_j,v\rangle$ receive.
						\item Using (c), verify that $A$ and $A'$ are the same modulo positions $\langle 0,0\rangle$ and $\langle j,j\rangle$.
						\item Therefore verify that $A'$ is an $m\times n$ diagonal matrix.
						\item Also, using (c), verify that $A'_{j,j}=u_jA'_{0,0}$.
						\item Also, for $k$ in $[1:j]$, do the following:
						\begin{enumerate}
							\item Using (a), (c), and (d), verify that $A'_{k,k}=A_{k,k}=u_kA_{0,0}=u_kA'_{0,0}v$.
							\item Set $u_k=u_kv$.
							\item Hence verify that $A'_{k,k}=u_kA'_{0,0}$.
						\end{enumerate}
						\item Therefore verify that $A_{k,k}=u_kA_{0,0}$ for $k$ in $[0:j+1]$.
						\item Now let $A=A'$.
					\end{enumerate}
					\item \textbf{Hence using (4h), verify that $A_{k,k}=u_kA_{0,0}$ for $k$ in $[0:\min(m,n)]$.}
					\item \textbf{Also, using (4e), verify that $A$ is an $m\times n$ diagonal matrix.}
					\item \textbf{Yield $\langle u\rangle$.}
				\end{enumerate}
		\procedure{4.09}
			\objective
				Choose a $m\times n$ matrix, $A$, and a $n\times k$ matrix, $B$. Choose integers $0\le a<m$, $0\le b<n$, and $0\le c<k$. The objective of the following instructions is to show that
				\begin{enumerate}
					\item $(AB)_{[0:a],[0:c]}=A_{[0:a],[0:b]}B_{[0:b],[0:c]}+A_{[0:a],[b:n]}B_{[b:n],[0:c]}$
					\item $(AB)_{[0:a],[c:k]}=A_{[0:a],[0:b]}B_{[0:b],[c:k]}+A_{[0:a],[b:n]}B_{[b:n],[c:k]}$
					\item $(AB)_{[a:m],[0:c]}=A_{[a:m],[0:b]}B_{[0:b],[0:c]}+A_{[a:m],[b:n]}B_{[b:n],[0:c]}$
					\item $(AB)_{[a:m],[c:k]}=A_{[a:m],[0:b]}B_{[0:b],[c:k]}+A_{[a:m],[b:n]}B_{[b:n],[c:k]}$.
				\end{enumerate}
			\implementation
				\begin{enumerate}
					\item For each $0\le i<a$, do the following:
					\begin{enumerate}
						\item For each $0\le j<c$, do the following:
							\begin{enumerate}
								\item Verify that $(AB)_{i,j}=\sum_{p}^{[0:n]} A_{i,p}B_{p,j}=\sum_{p}^{[0:b]} A_{i,p}B_{p,j}+\sum_p^{[b:n]} A_{i,p}B_{p,j}=\sum_p^{[0:{b}]} (A_{[0:a],[0:b]})_{i,p}(B_{[0:b],[0:c]})_{p,j}+\sum_p^{[0:{n-b}]} (A_{[0:a],[b:n]})_{i,p}(B_{[b:n],[0:c]})_{p,j}=(A_{[0:a],[0:b]}B_{[0:b],[0:c]})_{i,j}+(A_{[0:a],[b:n]}B_{[b:n],[0:c]})_{i,j}$.
							\end{enumerate}
					\end{enumerate}
					\item \textbf{Therefore verify that $(AB)_{[0:a],[0:c]}=A_{[0:a],[0:b]}B_{[0:b],[0:c]}+A_{[0:a],[b:n]}B_{[b:n],[0:c]}$.}
					\item \textbf{Using computations analogous to (1) and (2), show items (2), (3), and (4) of the objective.}
				\end{enumerate}
		\declaration{4.06}
			The phrase "number of rows of $A$" and the notation \gls{matrixrows}, where $A$ is an $m\times n$ matrix, will be used as a shorthand for $m$.
		\declaration{4.07}
			The phrase "number of columns of $A$" and the notation \gls{matrixcols}, where $A$ is an $m\times n$ matrix, will be used as a shorthand for $n$.
		\declaration{4.08}
			The notation \gls{matrixbdiag}, where $C$ is a list of rational square matrices, will be used to refer to the result yielded by executing the following instructions:
			\begin{enumerate}
				\item Let $E$ be a $0\times 0$ matrices.
				\item Now for $i$ in $[0:\lvert C\rvert]$:
				\begin{enumerate}
					\item Add $\cols(C_i)$ columns filled with zeros to the right end of $E$.
					\item Add $\rows(C_i)$ rows filled with zeros to the bottom end of $E$.
					\item Set the bottom-right corner of $E$ equal to $C_i$.
				\end{enumerate}
				\item \textbf{Yield the tuple $\langle E\rangle$.}
			\end{enumerate}
		\procedure{4.10}
			\objective
				Choose a $m\times n$ matrix, $A$. Let $A_{-1,-1}=1$. The objective of the following instructions is to construct the list of polynomials $v$ and transform $A$ into an $m\times n$ diagonal matrix such that $A_{k,k}=v_kA_{k-1,k-1}$ for $k$ in $[0:\min(m,n)]$ by a sequence of operations whereby either a polynomial times any of the columns is added to a different column, or a polynomial times any of the rows is added to a different row.
			\implementation
				\begin{enumerate}
					\item If $\min(m,n)=0$, then do the following:
					\begin{enumerate}
						\item \textbf{Verify that $A$ is an $m\times n$ diagonal matrix.}
						\item \textbf{Yield $\langle\rangle$.}
					\end{enumerate}
					\item Otherwise do the following:
					\begin{enumerate}
						\item Apply \procedurehr{4.08} on $A$, and let $\langle u\rangle$ receive.
						\item Verify that $A$ is an $m\times n$ diagonal matrix.
						\item Verify that $A_{k,k}=u_{k}A_{0,0}$ for $k$ in $[0:\min(m,n)]$.
						\item Let $B,C$ be an $(m-1)\times(n-1)$ diagonal matrix with $u_{1:\lvert u\rvert}$ on the diagonal.
						\item Let $\langle M,N\rangle$ receive the results of executing \procedurehr{4.03} on the pair $\langle m-1,n-1\rangle$ and the following procedure:
						\begin{enumerate}
							\item Execute \procedurehr{4.10} on $C$ and let $\langle w\rangle$ receive.
						\end{enumerate}
						\item Therefore verify that $C$ is an $(m-1)\times(n-1)$ diagonal matrix.
						\item Also verify that $C=M_*BN_*$.
						\item Let $C_{-1,-1}=1$.
						\item Now using (ei), verify that $C_{k,k}=w_kC_{k-1,k-1}$ for $k$ in $[0:\min(m,n)-1]$.
						\item Therefore using (c), verify that $A_{0,0}C=M_*(A_{0,0}B)N_*=M_*A_{[1:m],[1:n]}N_*$.
						\item Premultiply $A$ by $\diag(1,M_k)$ for $k$ in $[\lvert M\rvert:0]$.
						\item Postmultiply $A$ by $\diag(1,N_k)$ for $k$ in $[0:\lvert N\rvert]$.
						\item Now verify that $A_{[1:m],[1:n]}=A_{0,0}C$.
						\item Now let $u=\langle A_{0,0}\rangle^{\frown}w$.
						\item \textbf{Therefore verify that $A_{k,k}=u_kA_{k-1,k-1}$ for $k$ in $[0:\min(m,n)]$.}
						\item \textbf{Yield the tuple $\langle u\rangle$.}
					\end{enumerate}
				\end{enumerate}
		\declaration{4.09}
			The notation \gls{matrixdet}, where $A$ is a $m\times m$ matrix, will be used to refer to the result yielded by executing the following instructions:
			\begin{enumerate}
				\item If $m=0$, then do the following:
				\begin{enumerate}
					\item \textbf{Yield the tuple $\langle 1\rangle$.}
				\end{enumerate}
				\item Otherwise, do the following:
				\begin{enumerate}
					\item Let $h_r=A_{[0:r]^\frown[r+1,m],[1:m]}$ for $r$ in $[0:m]$.
					\item \textbf{Yield the tuple $\langle\sum_r^{[0:m]} (-1)^{r}A_{r,0}\det(h_r)\rangle$.}
				\end{enumerate}
			\end{enumerate}
		\procedure{4.11}
			\objective
				Choose a polynomial $p$. Choose two $1\times m$ matrices, $B$ and $C$. Choose an integer $0\le i<m$. Choose a $m\times m$ matrix, $A$, such that its $i^{th}$ row is $B+pC$. Let $A'$ be $A$ but with the $i^{th}$ row replaced by $B$ and let $A''$ be $A$ but with the $i^{th}$ row replaced by $C$. The objective of the following instructions is to show that $\det(A)=\det(A')+p\det(A'')$.
			\implementation
				\begin{enumerate}
					\item If $m=1$, then do the following:
					\begin{enumerate}
						\item Verify that $i=0$.
						\item \textbf{Therefore verify that $\det(A)=A_{0,0}=B_{0,0}+pC_{0,0}=\det(A')+p\det(A'')$.}
					\end{enumerate}
					\item Otherwise, do the following:
					\begin{enumerate}
						\item For $r$ in $[0:i]$, do the following:
						\begin{enumerate}
							\item Verify that $(A_{[0:r]^\frown[r+1:m],[1:m]})_{i-1,*}=B+pC$.
							\item Verify that $A'_{[0:r]^\frown[r+1:m],[1:m]}$ is $A_{[0:r]^\frown[r+1:m],[1:m]}$ with row $i-1$ replaced by $B$.
							\item Verify that $A''_{[0:r]^\frown[r+1:m],[1:m]}$ is $A_{[0:r]^\frown[r+1:m],[1:m]}$ with row $i-1$ replaced by $C$.
							\item Execute \procedurehr{4.11} on $\langle p,B,C,i-1,A_{[0:r]^\frown[r+1:m],[1:m]}\rangle$.
							\item Therefore verify that $\det(A_{[0:r]^\frown[r+1:m],[1:m]})=\det(A'_{[0:r]^\frown[r+1:m],[1:m]})+p\det(A''_{[0:r]^\frown[r+1:m],[1:m]})$.
						\end{enumerate}
						\item For $r$ in $[i+1:m]$, do the following:
						\begin{enumerate}
							\item Verify that $(A_{[0:r]^\frown[r+1:m],[1:m]})_{i,*}=B+pC$.
							\item Verify that $A'_{[0:r]^\frown[r+1:m],[1:m]}$ is $A_{[0:r]^\frown[r+1:m],[1:m]}$ with row $i$ replaced by $B$.
							\item Verify that $A''_{[0:r]^\frown[r+1:m],[1:m]}$ is $A_{[0:r]^\frown[r+1:m],[1:m]}$ with row $i$ replaced by $C$.
							\item Execute \procedurehr{4.11} on $\langle p,B,C,i,A_{[0:r]^\frown[r+1:m],[1:m]}\rangle$.
							\item Therefore verify that $\det(A_{[0:r]^\frown[r+1:m],[1:m]})=\det(A'_{[0:r]^\frown[r+1:m],[1:m]})+p\det(A''_{[0:r]^\frown[r+1:m],[1:m]})$.
						\end{enumerate}
						\item Therefore using (av) and (bv), verify that $\det(A)$
						\begin{enumerate}
							\item $=\sum_r^{[0:m]} (-1)^rA_{r,0}\det(A_{[0:r]^\frown[r+1:m],[1:m]})$
							\item $=\sum_r^{[0:i]} (-1)^rA_{r,0}\det(A_{[0:r]^\frown[r+1:m],[1:m]})+(-1)^iA_{i,0}\det(A_{[0:i]^\frown[i+1:m],[1:m]})+\sum_r^{[i+1:m]} (-1)^rA_{r,0}\det(A_{[0:r]^\frown[r+1:m],[1:m]})$
							\item $=\sum_r^{[0:i]} (-1)^rA_{r,0}(\det(A'_{[0:r]^\frown[r+1:m],[1:m]})+p\det(A''_{[0:r]^\frown[r+1:m],[1:m]}))+(-1)^i(A'_{i,0}+pA''_{i,0})\det(A_{[0:i]^\frown[i+1:m],[1:m]})+\sum_r^{[i+1:m]} (-1)^rA_{r,0}(\det(A'_{[0:r]^\frown[r+1:m],[1:m]})+p\det(A''_{[0:r]^\frown[r+1:m],[1:m]}))$
							\item $=\sum_r^{[0:i]} (-1)^rA_{r,0}\det(A'_{[0:r]^\frown[r+1:m],[1:m]})+(-1)^iA'_{i,0}\det(A_{[0:i]^\frown[i+1:m],[1:m]})+\sum_r^{[i+1:m]} (-1)^rA_{r,0}\det(A'_{[0:r]^\frown[r+1:m],[1:m]})
								+\sum_r^{[0:i]} (-1)^rA_{r,0}p\det(A''_{[0:r]^\frown[r+1:m],[1:m]})+(-1)^ipA''_{i,0}\det(A_{[0:i]^\frown[i+1:m],[1:m]})+\sum_r^{[i+1:m]} (-1)^rA_{r,0}p\det(A''_{[0:r]^\frown[r+1:m],[1:m]})$
							\item $=\sum_r^{[0:m]} (-1)^rA'_{r,0}\det(A'_{[0:r]^\frown[r+1:m],[1:m]})+p\sum_r^{[0:m]} (-1)^rA''_{r,0}\det(A''_{[0:r]^\frown[r+1:m],[1:m]})$
							\item $=\det(A')+p\det(A'')$.
						\end{enumerate}
					\end{enumerate}
				\end{enumerate}
		\procedure{4.12}
			\objective
				Choose a polynomial $p$. Choose two $m\times 1$ matrices, $B$ and $C$. Choose an integer $0\le i<m$. Choose a $m\times m$ matrix, $A$, such that its $i^{th}$ column is $B+pC$. Let $A'$ be $A$ but with the $i^{th}$ column replaced by $B$ and let $A''$ be $A$ but with the $i^{th}$ column replaced by $C$. The objective of the following instructions is to show that $\det(A)=\det(A')+p\det(A'')$.
			\implementation
				\begin{enumerate}
					\item If $i=0$, then verify that $\det(A)$
					\begin{enumerate}
						\item $=\sum_r^{[0:m]} (-1)^{r}A_{r,0}\det(A_{[0:r]^\frown[r+1:m],[1:m]})$
						\item $=\sum_r^{[0:m]} (-1)^{r}(B+pC)_{r,0}\det(A_{[0:r]^\frown[r+1:m],[1:m]})$
						\item $=\sum_r^{[0:m]} (-1)^{r}(B)_{r,0}\det(A_{[0:r]^\frown[r+1:m],[1:m]})+\sum_r^{[0:m]} (-1)^{r}(pC)_{r,0}\det(A_{[0:r]^\frown[r+1:m],[1:m]})$
						\item $=\sum_r^{[0:m]} (-1)^{r}(B)_{r,0}\det(A_{[0:r]^\frown[r+1:m],[1:m]})+p\sum_r^{[0:m]} (-1)^{r}(C)_{r,0}\det(A_{[0:r]^\frown[r+1:m],[1:m]})$
						\item $=\sum_r^{[0:m]} (-1)^{r}(A')_{r,0}\det(A'_{[0:r]^\frown[r+1:m],[1:m]})+p\sum_r^{[0:m]} (-1)^{r}(A'')_{r,0}\det(A''_{[0:r]^\frown[r+1:m],[1:m]})$
						\item $=\det(A')+p\det(A'')$
					\end{enumerate}
					\item Otherwise, do the following:
					\begin{enumerate}
						\item For $r$ in $[0:m]$, do the following:
						\begin{enumerate}
							\item Execute \procedurehr{4.12} on $\langle p,B_{[0:r]^\frown[r+1:m],0},C_{[0:r]^\frown[r+1:m],0},i-1,A_{[0:r]^\frown[r+1:m],[1:m]}\rangle$.
							\item Therefore verify that $\det(A_{[0:r]^\frown[r+1:m],[1:m]})=\det(A'_{[0:r]^\frown[r+1:m],[1:m]})+p\det(A''_{[0:r]^\frown[r+1:m],[1:m]})$.
						\end{enumerate}
						\item Therefore using (a), verify that $\det(A)$
						\begin{enumerate}
							\item $=\sum_r^{[0:m]} (-1)^{r}A_{r,0}\cdot\det(A_{[0:r]^\frown[r+1:m],[1:m]})$
							\item $=\sum_r^{[0:m]} (-1)^{r}A_{r,0}(\det(A'_{[0:r]^\frown[r+1:m],[1:m]})+p\det(A''_{[0:r]^\frown[r+1:m],[1:m]}))$
							\item $=\sum_r^{[0:m]} (-1)^{r}A'_{r,0}\det(A'_{[0:r]^\frown[r+1:m],[1:m]})+\sum_r^{[0:m]} (-1)^{r}A''_{r,0}p\det(A''_{[0:r]^\frown[r+1:m],[1:m]})$
							\item $=\det(A')+p\det(A'')$.
						\end{enumerate}
					\end{enumerate}
				\end{enumerate}
		\procedure{4.13}
			\objective
				Choose a $m\times m$ matrix, $A$. Choose an integer $0<i<m$. Let $A'$ be $A$ with rows $i-1$ and $i$ swapped. The objective of the following instructions is to show that $\det(A')=-\det(A)$.
			\implementation
				\begin{enumerate}
					\item If $m=2$, then do the following:
					\begin{enumerate}
						\item Verify that $i=1$.
						\item Therefore verify that $\det(A')=A'_{0,0}A'_{1,1}-A'_{1,0}A'_{0,1}=A_{1,0}A_{0,1}-A_{0,0}A_{1,1}=-\det(A)$.
					\end{enumerate}
					\item Otherwise do the following:
					\begin{enumerate}
						\item For $r$ in $[0:i-1]$, do the following:
						\begin{enumerate}
							\item Verify that $A_{[0:r]^\frown[r+1:m],[1:m]}$ is the same as $A'_{[0:r]^\frown[r+1:m],[1:m]}$ but with rows $i-2$ and $i-1$ swapped.
							\item Execute \procedurehr{4.13} on $\langle A_{[0:r]^\frown[r+1:m],[1:m]},i-1\rangle$.
							\item Hence verify that $\det(A'_{[0:r]^\frown[r+1:m],[1:m]})=-\det(A_{[0:r]^\frown[r+1:m],[1:m]})$.
						\end{enumerate}
						\item For $r$ in $[i+1:m]$, do the following:
						\begin{enumerate}
							\item Verify that $A_{[0:r]^\frown[r+1:m],[1:m]}$ is the same as $A'_{[0:r]^\frown[r+1:m],[1:m]}$ but with rows $i-1$ and $i$ swapped.
							\item Execute \procedurehr{4.13} on $\langle A_{[0:r]^\frown[r+1:m],[1:m]},i\rangle$.
							\item Hence verify that $\det(A'_{[0:r]^\frown[r+1:m],[1:m]})=-\det(A_{[0:r]^\frown[r+1:m],[1:m]})$.
						\end{enumerate}
						\item Verify that $\det(A)$
						\begin{enumerate}
							\item $=\sum_r^{[0:m]} (-1)^rA_{r,0}\det(A_{[0:r]^\frown[r+1:m],[1:m]})$
							\item $=\sum_r^{[0:{i-1}]} (-1)^rA_{r,0}\det(A_{[0:r]^\frown[r+1:m],[1:m]})+(-1)^{i-1}A_{i-1,0}\det(A_{[0:i-1]^\frown[i:m],[1:m]})+(-1)^iA_{i,0}\det(A_{[0:i]^\frown[i+1:m],[1:m]})+\sum_r^{[i+1:m]} (-1)^rA_{r,0}\det(A_{[0:r]^\frown[r+1:m],[1:m]})$
							\item $=-\sum_r^{[0:{i-1}]} (-1)^rA'_{r,0}\det(A'_{[0:r]^\frown[r+1:m],[1:m]})-(-1)^{i}A'_{i,0}\det(A'_{[0:i]^\frown[i+1:m],[1:m]})-(-1)^{i-1}A'_{i-1,0}\det(A'_{[0:i-1]^\frown[i:m],[1:m]})-\sum_r^{[i+1:m]} (-1)^rA'_{r,0}\det(A'_{[0:r]^\frown[r+1:m],[1:m]})$
							\item $=-\sum_r^{[0:m]} (-1)^rA'_{r,0}\det(A'_{[0:r]^\frown[r+1:m],[1:m]})$
							\item $=-\det(A')$.
						\end{enumerate}
					\end{enumerate}
				\end{enumerate}
		\procedure{4.14}
			\objective
				Choose a $m\times m$ matrix, $A$. Choose an integer $0<i<m$. Let $A'$ be $A$ with columns $i-1$ and $i$ swapped. The objective of the following instructions is to show that $\det(A')=-\det(A)$.
			\implementation
				\begin{enumerate}
					\item If $i=1$, then verify that $\det(A)$
					\begin{enumerate}
						\item $=\sum_r^{[0:m]} (-1)^{r}A_{r,0}\det(A_{[0:r]^\frown[r+1:m],[1:m]})$
						\item $=\sum_r^{[0:m]} (-1)^{r}A_{r,0}\sum_t^{[r+1:m]} (-1)^{t-1}A_{t,1}\cdot\det(A_{[0:r]^\frown[r+1:t]^\frown[t+1:m],[2:m]})+\sum_t^{[0:m]} (-1)^{t}A_{t,0}\sum_r^{[0:{t}]} (-1)^{r}A_{r,1}\cdot\det(A_{[0:r]^\frown[r+1:t]^\frown[t+1:m],[2:m+1]})$
						\item $=\sum_t^{[0:m]} (-1)^{t-1}A_{t,1}\sum_r^{[0:{t}]} (-1)^{r}A_{r,0}\cdot\det(A_{[0:r]^\frown[r+1:t]^\frown[t+1:m],[2:m+1]})+\sum_r^{[0:m]} (-1)^{r}A_{r,1}\sum_t^{[r+1:m]} (-1)^{t}A_{t,0}\cdot\det(A_{[0:r]^\frown[r+1:t]^\frown[t+1:m],[2:m+1]})$
						\item $=\sum_t^{[0:m]} (-1)^{t-1}A'_{t,0}\sum_r^{[0:{t}]} (-1)^{r}A'_{r,1}\cdot\det(A'_{[0:r]^\frown[r+1:t]^\frown[t+1:m],[2:m+1]})+\sum_r^{[0:m]} (-1)^{r}A'_{r,0}\sum_t^{[r+1:m]} (-1)^{t}A'_{t,1}\cdot\det(A'_{[0:r]^\frown[r+1:t]^\frown[t+1:m],[2:m]})$
						\item $=-(\sum_r^{[0:m]} (-1)^{r}A'_{r,0}\sum_t^{[r+1:m]} (-1)^{t-1}A'_{t,1}\cdot\det(A'_{[0:r]^\frown[r+1:t]^\frown[t+1:m],[2:m]})+\sum_t^{[0:m]} (-1)^{t}A'_{t,0}\sum_r^{[0:{t}]} (-1)^{r}A'_{r,1}\cdot\det(A'_{[0:r]^\frown[r+1:t]^\frown[t+1:m],[2:m]}))$
						\item $=-\det(A')$.
					\end{enumerate}
					\item Otherwise do the following:
					\begin{enumerate}
						\item Verify that $i>1$.
						\item For $r$ in $[0:m]$, do the following:
						\begin{enumerate}
							\item Execute \procedurehr{4.14} on $\langle i-1,A_{[0:r]^\frown[r+1:m],[1:m]}\rangle$.
							\item Therefore verify that $\det(A_{[0:r]^\frown[r+1:m],[1:m]})=-\det(A'_{[0:r]^\frown[r+1:m],[1:m]})$.
						\end{enumerate}
						\item Therefore using (bii), verify that $\det(A)=\sum_r^{[0:m]} (-1)^{r}A_{r,0}\cdot\det(A_{[0:r]^\frown[r+1:m],[1:m]})=\sum_r^{[0:m]} (-1)^{r}A'_{r,0}\cdot(-\det(A'_{[0:r]^\frown[r+1:m],[1:m]}))=-\det(A')$.
					\end{enumerate}
				\end{enumerate}
		\procedure{4.15}
			\objective
				Choose integers $0<i<m$. Choose a $m\times m$ matrix, $A$, such that columns $i-1$ and $i$ are the same. The objective of the following instructions is to show that $\det(A)=0$.
			\implementation
				\begin{enumerate}
					\item Let $A'$ be $A$ with columns $i-1$ and $i$ swapped.
					\item Execute \procedurehr{4.14} on $\langle A, i\rangle$.
					\item Also, verify that $A'=A$.
					\item Therefore verify that $\det(A)=\det(A')=-\det(A)$.
					\item \textbf{Therefore verify that $\det(A)=0$.}
				\end{enumerate}
		\procedure{4.16}
			\objective
				Choose integers $0<i<m$. Choose a $m\times m$ matrix, $A$, such that rows $i-1$ and $i$ are the same. The objective of the following instructions is to show that $\det(A)=0$.
			\implementation
				Instructions are analogous to those of \procedurehr{4.15}.
		\procedure{4.17}
			\objective
				Choose integers $0\le i<m$. Choose an integer $-i\le j<m-i$. Choose a $m\times m$ matrix, $A$. Let $A'$ be $A$ but with column $i$ moved $j$ places. The objective of the following instructions is to show that $\det(A')=(-1)^j\det(A)$.
			\implementation
				\begin{enumerate}
					\item Let $B=\langle A\rangle$.
					\item For $k$ in $[i:i+j]$, do the following:
					\begin{enumerate}
						\item Let $B_{\lvert B\rvert}$ be the result of swapping columns $k$ and $k+1$ of $B_{\lvert B\rvert-1}$.
						\item Using \procedurehr{4.14}, verify that $\det(B_{\lvert B\rvert-1})=-\det(B_{\lvert B\rvert-2})$.
					\end{enumerate}
					\item Verify that $A'=B_{\lvert B\rvert-1}$.
					\item \textbf{Therefore verify that $\det(A')=\det(B_{\lvert B\rvert-1})=(-1)^1\det(B_{\lvert B\rvert-2})=\cdots=(-1)^j\det(B_{0})=(-1)^j\det(A)$.}
				\end{enumerate}
		\procedure{4.18}
			\objective
				Choose integers $0\le i<m$. Choose an integer $-i\le j<m-i$. Choose a $m\times m$ matrix, $A$. Let $A'$ be $A$ but with row $i$ moved $j$ places. The objective of the following instructions is to show that $\det(A')=(-1)^j\det(A)$.
			\implementation
				Instructions are analogous to those of \procedurehr{4.17}.
		\declaration{4.10}
			The notation \gls{matrixC_k(A)}, where $A$ is a $m\times n$ matrix and $k$ is an integer such that $0\le k\le\min(m,n)$, will be used to refer to the $\binom{m}{k}\times\binom{n}{k}$ matrix with the following specification:
			\begin{enumerate}
				\item The rows are labeled by the colexicographically sorted list of increasing length-$k$ sequences whose elements are picked from $[0:m]$.
				\item The columns are labeled by the colexicographically sorted list of increasing length-$k$ sequences whose elements are picked from $[0:n]$.
				\item For each row label $I$: For each column label $J$: The entry at position $(I,J)$ is $\det(A_{I,J})$.
			\end{enumerate}
		\declaration{4.11}
			The notation \gls{matrixlabelledentry} will be used to refer to the entry of $A$ with row label $I$ and column label $J$.
		\procedure{4.19}
			\objective
				Choose two integers $0\le k\le m$. The objective of the following instructions is to show that $C_k(1_m)=1_{\binom{m}{k}}$.
			\implementation
				\begin{enumerate}
					\item For each row label $I$ of $C_k(1_m)$, for each column label $J$ of $C_k(1_m)$, do the following:
					\begin{enumerate}
						\item If $I=J$, then do the following:
						\begin{enumerate}
							\item Verify that $((1_m)_{I,J})_{i,j}=((1_m)_{J,J})_{i,j}=(1_m)_{J_i,J_j}=[J_i=J_j]=[i=j]$ for $0\le i<k$, for $0\le j<k$.
							\item Therefore verify that $(C_k(1_m))_{\ul{I},\ul{J}}=1_k$.
							\item \textbf{Therefore verify that $(C_k(1_m))_{\ul{I},\ul{J}}=\det((1_m)_{I,J})=\det(1_k)=1$.}
						\end{enumerate}
						\item Otherwise, do the following:
						\begin{enumerate}
							\item Verify that $I\ne J$.
							\item Let $i$ be the index of an element of $I$ that is not an element of $J$.
							\item Now verify that $(1_m)_{I_i,j}=[I_i=j]=0$, for each $j$ in $J$.
							\item Therefore verify that $((1_m)_{I,J})_{i,*}=0_{1\times k}$.
							\item \textbf{Therefore verify that $(C_k(1_m))_{\ul{I},\ul{J}}=\det((1_m)_{I,J})=0$.}
						\end{enumerate}
					\end{enumerate}
					\item \textbf{Therefore verify that $C_k(1_m)=1_{\binom{m}{k}}$.}	
				\end{enumerate}
		\procedure{4.20}
			\objective
				Choose an integer $0\le k\le\min(m,n)$. Choose a $m\times m$ tilt, $A$, such that the off diagonal entry is the polynomial $p$ at $(i,j)$. Also choose a $m\times n$ matrix, $B$. The objective of the following instructions is to construct a $\binom{m}{k}\times\binom{m}{k}$ matrix $D$ such that $C_k(AB)=DC_k(B)$.
			\implementation
				\begin{enumerate}
					\item Let $D=C_k(1_m)=1_{\binom{m}{k}}$.
					\item Verify that $AB$ equals $B$, but with its row $i$ having $p$ times $B$'s row $j$ added to it.
					\item Go through the row labels, $I$, of $C_k(AB)$ and do the following:
					\begin{enumerate}
						\item If $i\notin I$, then do the following:
						\begin{enumerate}
							\item Verify that $(AB)_{I,*}=B_{I,*}$.
							\item Therefore for each column label $J$, verify that ${C_k(AB)}_{\ul{I},\ul{J}}=\det((AB)_{I,J})=\det(B_{I,J})={C_k(B)}_{\ul{I},\ul{J}}$.
							\item \textbf{Therefore verify that $(C_k(AB))_{\ul{I},*}=(C_k(B))_{\ul{I},*}$.}
						\end{enumerate}
						\item Otherwise, if $i\in I$, then:
						\begin{enumerate}
							\item Let $I'$ be $I$ but with an in-place replacement of $i$ by $j$.
							\item For each column label $J$: Using \procedurehr{4.12}, verify that ${C_k(AB)}_{\ul{I},\ul{J}}=\det((AB)_{I,J})=\det(B_{I,J})+p*\det(B_{I',J})$.
							\item If $j\in I$, then do the following:
							\begin{enumerate}
								\item Verify that the sequence $I'$ contains two $j$s.
								\item For each column label $J$: Using \procedurehr{4.16} verify that $\det(B_{I',J})=0$.
								\item Therefore for each column label $J$: verify that ${C_k(AB)}_{\ul{I},\ul{J}}=\det(B_{I,J})={C_k(B)}_{\ul{I},\ul{J}}$.
								\item \textbf{Therefore verify that ${C_k(AB)}_{\ul{I},*}={C_k(B)}_{\ul{I},*}$.}
							\end{enumerate}
							\item Otherwise if $j\notin I$, do the following:
							\begin{enumerate}
								\item Let $l$ be the signed number of places that the $j$ introduced above needs to be moved in order to make $I'$ an increasing sequence.
								\item Let $I''$ be obtained from $I'$ by moving the integer $j$ in $I'$ by $l$ places.
								\item For each column label $J$: Using \procedurehr{4.18}, verify that $\det(B_{I',J})=(-1)^l\det(B_{I'',J})$.
								\item Therefore for each column label $J$: Verify that ${C_k(AB)}_{\ul{I},\ul{J}}=\det(B_{I,J})+p*\det(B_{I',J})=\det(B_{I,J})+(-1)^lp*\det(B_{I'',J})$.
								\item Verify that $I''$ is a row label of $C_k(B)$.
								\item Therefore for each column label $J$: Verify that ${C_k(AB)}_{\ul{I},\ul{J}}=\det(B_{I,J})+(-1)^lp*\det(B_{I'',J})={C_k(B)}_{\ul{I},\ul{J}}+(-1)^lp*{C_k(B)}_{\ul{I''},\ul{J}}$.
								\item \textbf{Therefore verify that $(C_k(AB))_{\ul{I},*}=(C_k(B))_{\ul{I},*}+(-1)^lp(C_k(B))_{\ul{I''},*}$.}
								\item \textbf{Set $D_{\ul{I},\ul{I''}}$ to $(-1)^lp$.}
							\end{enumerate}
						\end{enumerate}
						\item \textbf{Therefore verify that ${C_k(AB)}_{\ul{I},*}=D_{\ul{I},*}C_k(B)$.}
					\end{enumerate}
					\item \textbf{Therefore verify that $C_k(AB)=DC_k(B)$.}
					\item \textbf{Yield $\langle D\rangle$.}
				\end{enumerate}
		\procedure{4.21}
			\objective
				Choose an $m\times n$ diagonal matrix, $A$. Also choose an $n\times n$ matrix, $B$. Also choose an integer $0\le k\le\min(m,n)$. The objective of the following instructions is to construct an $\binom{m}{k}\times\binom{n}{k}$ diagonal matrix $D$ such that $C_k(AB)=DC_k(B)$.
			\implementation
				\begin{enumerate}
					\item Let $D=C_k(0_{m\times n})=0_{\binom{m}{k}\times\binom{n}{k}}$.
					\item Verify that $AB$ equals $B_{[0:\min(m,n)],*}$ with each row $i$ multiplied by $A_{i,i}$.
					\item Go through the row labels, $I$, of $C_k(AB)$ and do the following:
					\begin{enumerate}
						\item If $I_k<\min(m,n)$, then do the following:
						\begin{enumerate}
							\item Verify that every element of $I$ is less than $\min(m,n)$.
							\item Let $A_0=A$.
							\item For $i$ in $[0:k]$: Let $A_{i+1}$ equal $A_{i}$ but with position $(I_i,I_i)$ set to $1$.
							\item For each column label $J$: Repeatedly using \procedurehr{4.12}, verify that ${C_k(AB)}_{I,J}$
							\begin{enumerate}
								\item $=\det((AB)_{I,J})$
								\item $=\det((A_0B)_{I,J})$
								\item $=A_{I_0,I_0}\det((A_1B)_{I,J})$
								\item $=A_{I_0,I_0}A_{I_1,I_1}\det((A_2B)_{I,J})$
								\item $\vdots$
								\item $=A_{I_0,I_0}A_{I_1,I_1}\cdots A_{I_{k-1},I_{k-1}}\det((A_kB)_{I,J})$
								\item $=A_{I_0,I_0}A_{I_1,I_1}\cdots A_{I_{k-1},I_{k-1}}\det(B_{I,J})$
								\item $=A_{I_0,I_0}A_{I_1,I_1}\cdots A_{I_{k-1},I_{k-1}}{C_k(B)}_{\ul{I},\ul{J}}$.
							\end{enumerate}
							\item \textbf{Therefore verify that $(C_k(AB))_{\ul{I},*}=A_{I_1,I_1}A_{I_1,I_1}\cdots A_{I_k,I_k}*(C_k(B))_{\ul{I},*}$.}
							\item \textbf{Set $D_{\ul{I},\ul{I}}$ to $A_{I_0,I_0}A_{I_1,I_1}\cdots A_{I_{k-1},I_{k-1}}$.}
						\end{enumerate}
						\item Otherwise if $I_k\ge\min(m,n)$, then do the following:
						\begin{enumerate}
							\item Using the precondition, verify that $A_{I_k,*}=0_{1\times n}$.
							\item Therefore verify that $(AB)_{I_k,*}=0_{1\times n}$.
							\item Therefore verify that $((AB)_{I,*})_{k,*}=0_{1\times n}$.
							\item Therefore for each column label $J$: verify that ${C_k(AB)}_{\ul{I},\ul{J}}=\det((AB)_{I,J})=0$.
							\item \textbf{Therefore verify that $(C_k(AB))_{\ul{I},*}$ is zero.}
						\end{enumerate}
						\item \textbf{Therefore verify that ${C_k(AB)}_{\ul{I},*}=D_{\ul{I},*}C_k(B)$.}
					\end{enumerate}
					\item \textbf{Verify that $D$ is diagonal.}
					\item \textbf{Verify that $C_k(AB)=DC_k(B)$.}
					\item \textbf{Yield $\langle D\rangle$.}
				\end{enumerate}
		\procedure{4.22}
			\objective
				Choose an integer $0\le k\le\min(m,n)$. Choose a $m\times m$ tilt, $A$. Also choose a $m\times n$ matrix, $B$. The objective of the following instructions is to show that $C_k(AB)=C_k(A)C_k(B)$.
			\implementation
				\begin{enumerate}
					\item Execute \procedurehr{4.20} on matrices $A$ and $1_m$ and let $\langle D\rangle$ receive.
					\item Using \procedurehr{4.19}, verify that $C_k(A)=C_k(A1_m)=DC_k(1_m)=D1_{\binom{m}{k}}=D$.
					\item Execute \procedurehr{4.20} on $\langle A,B\rangle$ and let $\langle D'\rangle$ receive.
					\item Verify that $D'=D=C_k(A)$.
					\item \textbf{Therefore verify that $C_k(AB)=D'C_k(B)=C_k(A)C_k(B)$.}
				\end{enumerate}
		\procedure{4.23}
			\objective
				Choose an integer $0\le k\le\min(m,n)$. Choose an $n\times n$ tilt, $A$. Also choose a $m\times n$ matrix, $B$. The objective of the following instructions is to show that $C_k(BA)=C_k(B)C_k(A)$.
			\implementation
				Instructions are analogous to those of \procedurehr{4.22}.
		\procedure{4.24}
			\objective
				Choose an integer $0\le k\le\min(m,n)$. Choose an $m\times n$ diagonal matrix, $A$. Also choose a $n\times n$ matrix, $B$. The objective of the following instructions is to show that $C_k(AB)=C_k(A)C_k(B)$.
			\implementation
				Instructions are analogous to those of \procedurehr{4.22}.
		\procedure{4.25}
			\objective
				Choose a $m\times n$ matrix, $A$. Let $D_{-1,-1}=1$. The objective of the following instructions is to construct a list of $m\times m$ tilts, $M$, an $m\times n$ diagonal matrix, $D$, a list of polynomials, $v$, and a list of $n\times n$ tiltss, $N$, such that $M_*AN_*=D$, $A={M^{-1}}_*D{N^{-1}}_*$, and $D_{i,i}=v_iD_{i-1,i-1}$ for $i$ in $[0:\min(m,n)]$.
			\implementation
				\begin{enumerate}
					\item Let $D$ be a copy of $A$.
					\item Let $\langle M,N\rangle$ receive the results of executing \procedurehr{4.03} on the pair $\langle m,n\rangle$ and the following procedure:
						\begin{enumerate}
							\item Execute \procedurehr{4.10} on the matrix $D$ and let $\langle v\rangle$ receive.
						\end{enumerate}
					\item \textbf{Verify that $D_{i,i}=v_iD_{i-1,i-1}$ for $i$ in $[0:\min(m,n)]$.}
					\item \textbf{Verify that $M_*AN_*=D$.}
					\item Hence verify that $A=1_mA1_n={M^{-1}}_*M_*AN_*{N^{-1}}_*={M^{-1}}_*D{N^{-1}}_*$.
					\item \textbf{Yield the tuple $\langle M,D,v,N\rangle$.}
				\end{enumerate}
		\procedure{4.26}
			\objective
				Choose integers $0\le k\le\min(m,n,p)$. Choose a $m\times n$ matrix, $A$. Also choose a $n\times p$ matrix, $B$. The objective of the following instructions is to show that $C_k(AB)=C_k(A)C_k(B)$.
			\implementation
				\begin{enumerate}
					\item Execute \procedurehr{4.25} on $A$ and let $\langle M,D,,N\rangle$ receive.
					\item Using repeated applications of \procedurehr{4.24}, verify that $C_k(AB)$
					\begin{enumerate}
						\item $=C_k({M^{-1}}_0\cdots {M^{-1}}_{\lvert M\rvert-1}D{N^{-1}}_0\cdots {N^{-1}}_{\lvert N\rvert-1}B)$
						\item $=C_k({M^{-1}}_0)\cdots C_k({M^{-1}}_{\lvert M\rvert-1})*C_k(D)*C_k({N^{-1}}_0)\cdots C_k({N^{-1}}_{\lvert N\rvert-1})C_k(B)$
						\item $=C_k({M^{-1}}_0\cdots {M^{-1}}_{\lvert M\rvert-1}D{N^{-1}}_0\cdots {N^{-1}}_{\lvert N\rvert-1})C_k(B)$
						\item $=C_k(A)C_k(B)$.
					\end{enumerate}
				\end{enumerate}
		\procedure{4.27}
			\objective
				Choose a $m\times m$ matrix, $A$. Let $D$ be a copy of $A$. Execute \procedurehr{4.10} on $D$. The objective of the following instructions is to show that $\det(A)$ is the product of the diagonal entries of $D$.
			\implementation
				\begin{enumerate}
					\item Execute \procedurehr{4.25} on $A$ and let $\langle M,D,,N\rangle$ receive.
					\item Using \procedurehr{4.26}, verify that $\det(A)$
					\begin{enumerate}
						\item $=C_m(A)$
						\item $=C_m({M^{-1}}_0\cdots {M^{-1}}_{\lvert M\rvert-1}D{N^{-1}}_0\cdots {N^{-1}}_{\lvert N\rvert-1})$
						\item $=C_m({M^{-1}}_0)\cdots C_m({M^{-1}}_{\lvert M\rvert-1})C_m(D)C_m({N^{-1}}_0)\allowbreak\cdots C_m({N^{-1}}_{\lvert N\rvert-1})$
						\item $=1\cdots 1C_m(D)1\cdots 1=C_m(D)$
						\item $=\det(D)$
						\item $=\prod_r^{[0:m]} D_{r,r}$.
					\end{enumerate}
				\end{enumerate}
		\declaration{4.12}
			The notation \gls{matrixtranspose}, where $A$ is a $m\times n$ matrix, will be used to refer to the $n\times m$ matrix such that ${A^T}_{i,j}=A_{j,i}$ for $i$ in $[0:n]$, for $j$ in $[0:m]$.
		\procedure{4.28}
			\objective
				Choose a $m\times n$ matrix, $A$, and a $n\times k$ matrix, $B$. The objective of the following instructions is to show that $B^TA^T=(AB)^T$.
			\implementation
				\begin{enumerate}
					\item Verify that $B^TA^T$ and $(AB)^T$ have dimensions $k\times m$.
					\item For $i$ in $[0:k]$: For $j$ in $[0:m]$:
					\begin{enumerate}
						\item Verify that $(B^TA^T)_{i,j}=\sum_l^{[0:n]} B_{l,i}A_{j,l}=\sum_l^{[0:n]} A_{j,l}B_{l,i}=(AB)_{j,i}=((AB)^T)_{i,j}$.
					\end{enumerate}
					\item \textbf{Therefore verify that $B^TA^T=(AB)^T$.}
				\end{enumerate}
		\procedure{4.29}
			\objective
				Choose a $m\times m$ matrix, $A$. The objective of the following instructions is to show that $\det(A^T)=\det(A)$.
			\implementation
				\begin{enumerate}
					\item Execute \procedurehr{4.25} on $A$ and let $\langle M,D,,N\rangle$ receive.
					\item Therefore using procedures \procedurehr{4.27} and \procedurehr{4.28}, verify that $\det(A^T)$
					\begin{enumerate}
						\item $=\det(({M^{-1}}_0\cdots {M^{-1}}_{\lvert M\rvert-1}D{N^{-1}}_0\cdots {N^{-1}}_{\lvert N\rvert-1})^T)$
						\item $=\det(({N^{-1}}_{\lvert N\rvert-1})^T\cdots({N^{-1}}_0)^TD^T({M^{-1}}_{\lvert M\rvert-1})^T\allowbreak\cdots({M^{-1}}_0)^T)$
						\item $=\det(D^T)$
						\item $=\det(D)$
						\item $=\det({M^{-1}}_0\cdots {M^{-1}}_{\lvert M\rvert-1}D{N^{-1}}_0\cdots {N^{-1}}_{\lvert N\rvert-1})$
						\item $=\det(A)$.
					\end{enumerate}
				\end{enumerate}
		\procedure{4.30}
			\objective
				Choose a $m\times n$ matrix, $A$, and an integer $0\le k\le\min(m,n)$. The objective of the following instructions is to show that $C_k(A)^T=C_k(A^T)$.
			\implementation
				\begin{enumerate}
					\item For each row label $I$ of $C_k(A^T)$, do the following:
					\begin{enumerate}
						\item For each column label $J$ of $C_k(A^T)$, do the following:
						\begin{enumerate}
							\item Using \procedurehr{4.29}, verify that $(C_k(A^T))_{\ul{I},\ul{J}}=\det((A^T)_{I,J})=\det(A_{J,I})=(C_k(A))_{\ul{J},\ul{I}}$.
						\end{enumerate}
					\end{enumerate}
					\item \textbf{Therefore verify that $(C_k(A))^T=(C_k(A^T))$.}
				\end{enumerate}
		\procedure{4.31}
			\objective
				Choose a $m\times n$ rational matrix, $A$, and a $m\times p$ rational matrix, $B$. Execute \procedurehr{4.25} on $A$ and let $\langle M,D,,N\rangle$ receive the result. If the indices of the rows of $D$ that are entirely zero are also the indices of the rows of $M_*B$ that are entirely zero, then the objective of the following instructions is to construct a $n\times p$ rational matrix $E$ such that $AE=B$.
			\implementation
				\begin{enumerate}
					\item Verify that $A={M^{-1}}_*D{N^{-1}}_*$.
					\item Verify that ${M^{-1}}_*$, $D$, and ${N^{-1}}_*$ are rational matrices.
					\item Let $C$ be an $n\times p$ matrix with its $i^{th}$ row given as follows:
					\begin{enumerate}
						\item If $D_{i,i}\ne 0$, then do the following:
						\begin{enumerate}
							\item Let row $i$ be row $i$ of $M_*B$ divided by $D_{i,i}$.
						\end{enumerate}
						\item Otherwise, do the following:
						\begin{enumerate}
							\item \textbf{Choose $p$ rational numbers to fill up the row.}
						\end{enumerate}
					\end{enumerate}
					\item Verify that $DC=M_*B$.
					\item Let $E$ be $N_*C$.
					\item \textbf{Therefore using \procedurehr{4.05}, verify that $AE={M^{-1}}_*D{N^{-1}}_*E={M^{-1}}_*D{N^{-1}}_*N_*C={M^{-1}}_*D1_nC={M^{-1}}_*DC={M^{-1}}_*M_*B=1_mB=B$.}
					\item \textbf{Yield the tuple $\langle E\rangle$.}
				\end{enumerate}
		\declaration{4.13}
			The notation \gls{matrixleftdiv} will be used to refer to the result yielded by executing \procedurehr{4.31} on $\langle A,B\rangle$.
		\procedure{4.32}
			\objective
				Choose a $m\times n$ rational matrix, $A$, and a $p\times n$ rational matrix, $B$. Execute \procedurehr{4.25} on $A$ and let $\langle M,D,,N\rangle$ receive the result. If the indices of the columns of $D$ that are entirely zero are also the indices of the columns of $BN_*$ that are entirely zero, then the objective of the following instructions is to construct a $p\times m$ rational matrix $E$ such that $EA=B$.
			\implementation
				Instructions are analogous to those of \procedurehr{4.31}.
		\declaration{4.14}
			The notation \gls{matrixrightdiv} will be used to refer to the result yielded by executing \procedurehr{4.32} on $\langle A,B\rangle$.
		\procedure{4.33}
			\objective
				Choose a $m\times n$ rational matrix, $A$, a $n\times p$ rational matrix, $E$, and a $m\times p$ rational matrix, $B$ such that $AE=B$. Execute \procedurehr{4.25} on $A$ and let $\langle M,D,,N\rangle$ receive the result. If the indices of the rows of $D$ that are entirely zero are not also the indices of the rows of $M_*B$ that are entirely zero, then the objective of the following instructions is to show that $0\ne 0$.
			\implementation
				\begin{enumerate}
					\item Verify that ${M^{-1}}_*D{N^{-1}}_*E=AE=B$.
					\item Therefore verify that $D{N^{-1}}_*E=M_*B$.
					\item Let $i$ be an integer such that $D_{i,*}$ is zero and yet $(M_*B)_{i,*}$ is not zero.
					\item Verify that $D_{i,*}=D_{i,*}{N^{-1}}_*E=(D{N^{-1}}_*E)_{i,*}=(M_*B)_{i,*}$.
					\item Let $j$ be an integer such that $(M_*B)_{i,j}\ne 0$.
					\item \textbf{Now verify that $0=D_{i,j}=(M_*B)_{i,j}\ne 0$.}
				\end{enumerate}
		\procedure{4.34}
			\objective
				Choose a $p\times m$ rational matrix, $E$, a $m\times n$ rational matrix, $A$, and a $p\times n$ rational matrix, $B$ such that $EA=B$. Execute \procedurehr{4.25} on $A$ and let $\langle M,D,,N\rangle$ receive the result. If the indices of the columns of $D$ that are entirely zero are not also the indices of the columns of $BN_*$ that are entirely zero, then the objective of the following instructions is to show that $0\ne 0$.
			\implementation
				Instructions are analogous to those of \procedurehr{4.33}.
		\procedure{4.35}
			\objective
				Choose two $m\times m$ rational matrices, $A$ and $B$, such that $AB=1_m$. The objective of the following instructions is to show that either $0=1$ or $BA=1_m$.
			\implementation
				\begin{enumerate}
					\item Execute \procedurehr{4.25} on $B$ and let $\langle M,D,,N\rangle$ receive the result.
					\item Verify that $B={M^{-1}}_*D{N^{-1}}_*$.
					\item If $D$ has a zero on its diagonal, then do the following:
					\begin{enumerate}
						\item Using \procedurehr{4.27}, verify that $\det(1_m)=\det(AB)=\det(A)\det(B)=\det(A)\det(D)=\det(A)*0=0$.
						\item Also verify that $\det(1_m)=1^m=1$.
						\item Therefore verify that $0=1$.
						\item \textbf{Abort procedure.}
					\end{enumerate}
					\item Otherwise do the following:
					\begin{enumerate}
						\item Verify that $D$ does not have a zero on its diagonal.
						\item Verify that $B\backslash 1_m=1_m(B\backslash 1_m)=AB(B\backslash 1_m)=A(B(B\backslash 1_m))=A1_m=A$.
						\item \textbf{Therefore verify that $BA=B(B\backslash 1_m)=1_m$.}
					\end{enumerate}
				\end{enumerate}
		\procedure{4.36}
			\objective
				Choose an $m\times m$ matrix, $M$, and an $m\times m$ rational matrix, $B$. The objective of the following instructions is to construct a $m\times m$ matrix, $Q$, and a $m\times m$ rational matrix, $R$, such that $M=(\lambda 1_m-B)Q+R$.
			\implementation
				\begin{enumerate}
					\item Let $M_0\lambda^b+M_1\lambda^{b-1}+\cdots+M_b\lambda^0=M$, where the $M_i$ are $m\times m$ rational matrices.
					\item Now let $R=B^bM_0+B^{b-1}M_1+\cdots+B^0M_b$.
					\item Let $Q=\sum_k^{[1:b]} (\lambda^{k-1}1_mB^0+\lambda^{k-2}1_mB^1+\cdots+\lambda^01_mB^{k-1})M_k$.
					\item Verify that $M-R=(\lambda 1_m-B)\sum_k^{[1:b]} (\lambda^{k-1}1_mB^0+\lambda^{k-2}1_mB^1+\cdots+\lambda^01_mB^{k-1})M_k=(\lambda 1_m-B)Q$.
					\item \textbf{Verify that $M=(\lambda 1_m-B)Q+R$.}
					\item \textbf{Yield the tuple $\langle Q,R\rangle$.}
				\end{enumerate}
		\procedure{4.37}
			\objective
				Choose an $m\times m$ matrix, $M$, and an $m\times m$ rational matrix, $B$. The objective of the following instructions is to construct a $m\times m$ matrix, $Q$, and a $m\times m$ rational matrix, $R$, such that $M=Q(\lambda 1_m-B)+R$.
			\implementation
				The instructions are analogous to those of \procedurehr{4.36}.
		\procedure{4.38}
			\objective
				Choose two $m\times m$ rational matrices, $B,A$, and two lists of $m\times m$ tilts such that $\lambda 1_m-B=M(\lambda 1_m-A)N$. The objective of the following instructions is to either show that $0=1$ or to construct $m\times m$ rational matrices $R_1$ and $R_3$ such that $1_m=R_1R_3$ and $B=R_1AR_3$.
			\implementation
				\begin{enumerate}
					\item Verify that $(\lambda 1_m-B)N^{-1}=M(\lambda 1_m-A)NN^{-1}=M(\lambda 1_m-A)1_m=M(\lambda 1_m-A)$.
					\item Execute \procedurehr{4.37} on $\langle M,B\rangle$ and let $\langle Q_1,R_1\rangle$ receive.
					\item Verify that $M=(\lambda 1_m-B)Q_1+R_1$.
					\item Execute \procedurehr{4.37} on $\langle N^{-1},A\rangle$ and let $\langle Q_2,R_2\rangle$ receive.
					\item Verify that $N^{-1}=Q_2(\lambda 1_m-A)+R_2$.
					\item By substituting $M$ and $N^{-1}$ into (2), verify that $(\lambda 1_m-B)(Q_2(\lambda 1_m-A)+R_2)=((\lambda 1_m-B)Q_1+R_1)(\lambda 1_m-A)$.
					\item By rearranging both sides, verify that $(\lambda 1_m-B)(Q_2-Q_1)(\lambda 1_m-A)=R_1(\lambda 1_m-A)-(\lambda 1_m-B)R_2$.
					\item By equating the coefficients of different powers of $\lambda$ both sides, verify that $Q_2-Q_1=0_{m\times m}$.
					\item Verify that $R_1(\lambda 1_m-A)-(\lambda 1_m-B)R_2=(\lambda 1_m-B)(Q_2-Q_1)(\lambda 1_m-A)=(\lambda 1_m-B)0_{m\times m}(\lambda 1_m-A)=0_{m\times m}$.
					\item Therefore by adding $(\lambda 1_m-B)R_2$ to both sides, verify that $\lambda R_1-R_1A=R_1(\lambda 1_m-A)=(\lambda 1_m-B)R_2=\lambda R_2-BR_2$.
					\item By equating the coefficients of $\lambda$ on both sides, verify that $R_1=R_2$.
					\item Therefore verify that $R_1A=BR_1$.
					\item Execute \procedurehr{4.37} on $\langle M^{-1},A\rangle$ and let $\langle Q_3,R_3\rangle$ receive.
					\item Verify that $M^{-1}=(\lambda 1_m-A)Q_3+R_3$.
					\item Verify that $1_m=MM^{-1}=((\lambda 1_m-B)Q_1+R_1)M^{-1}=(\lambda 1_m-B)Q_1M^{-1}+R_1M^{-1}=(\lambda 1_m-B)Q_1M^{-1}+R_1(\lambda I-A)Q_3+R_1R_3=(\lambda 1_m-B)Q_1M^{-1}+(\lambda I-B)R_1Q_3+R_1R_3=(\lambda 1_m-B)(Q_1M^{-1}+R_1Q_3)+R_1R_3$.
					\item By equating the powers of $\lambda$ on both sides, verify that $Q_1M^{-1}+R_1Q_3=0$.
					\item By substituting zero for $Q_1M^{-1}+R_1Q_3$, \textbf{verify that $1_m=(\lambda 1_m-B)0_{m\times m}+R_1R_3=R_1R_3$.}
					\item \textbf{Therefore using \procedurehr{4.35}, verify that $R_3R_1=1_m$.}
					\item \textbf{Also, verify that $B=B1_m=BR_1R_3=R_1AR_3$.}
					\item \textbf{Yield the pair $(R_1,R_3)$.}
				\end{enumerate}
		\procedure{4.39}
			\objective
				Choose a $m\times n$ matrix, $A$. Choose two integers $0\le i,j<m$ such that $i\ne j$. The objective of the following instructions is to negate row $i$ and swap it with row $j$ using only elementary row operations.
			\implementation
				\begin{enumerate}
					\item Let $A$ be our working matrix.
					\item Subtract row $j$ from row $i$.
					\item Add row $i$ to row $j$.
					\item Subtract row $j$ from row $i$.
					\item \textbf{Verify that the $i^{th}$ row has been negated and swapped with the $j^{th}$ row.}
				\end{enumerate}
		\procedure{4.40}
			\objective
				Choose a $m\times n$ matrix, $A$. Choose two integers $0\le i,j<n$ such that $i\ne j$. The objective of the following instructions is to negate column $i$ and swap it with row $j$ using only elementary column operations.
			\implementation
				The instructions are analogous to those of \procedurehr{4.39}.
		\procedure{4.41}
			\objective
				Choose an $m\times n$ diagonal matrix, $A$. Choose two integers $0\le i,j<\min(m,n)$ such that $i\ne j$. The objective of the following instructions is to swap $B_{i,i}$ and $B_{j,j}$ using only elementary row and column operations.
			\implementation
				\begin{enumerate}
					\item Let $A$ be our working matrix.
					\item Use \procedurehr{4.40} to negate the $i^{th}$ row and swap it with the $j^{th}$ row.
					\item Use \procedurehr{4.40} to negate the $i^{th}$ column and swap it with the $j^{th}$ column.
					\item \textbf{Therefore, overall verify that $B_{i,i}$ and $B_{j,j}$ have been swapped.}
				\end{enumerate}
		\procedure{4.42}
			\objective
				Choose an $m\times n$ diagonal matrix, $A$. Choose two integers $0\le i,j<\min(m,n)$ such that $i\ne j$. Choose a rational $k\ne 0$. The objective of the following instructions is to multiply $B_{i,i}$ by $k$ and $B_{j,j}$ by $\frac{1}{k}$ using only elementary row and column operations.
			\implementation
				\begin{enumerate}
					\item Let $A$ be our working matrix.
					\item Add $k$ times row $i$ to row $j$.
					\item Subtract $\frac{1}{k}$ times row $j$ from row $i$.
					\item Add $k$ times row $i$ to row $j$.
					\item Verify that the $i^{th}$ row has been scaled by $k$, the $j^{th}$ row by $-\frac{1}{k}$, and that both these rows are swapped.
					\item Use \procedurehr{4.40} to negate the $i^{th}$ row and swap it with the $j^{th}$ row.
					\item \textbf{Therefore, overall verify that $B_{i,i}$ has been multiplied by $k$, and $B_{j,j}$ by $\frac{1}{k}$.}
				\end{enumerate}
		\procedure{4.43}
			\objective
				Choose a $m\times m$ rational matrix, $A$. Execute \procedurehr{4.10} on the polynomial matrix $\lambda I-A$ and let $\langle B\rangle$ be the result. The objective of the following instructions is to show that either none of the diagonal entries of $B$ are equal to zero, or $1=0$.
			\implementation
				\begin{enumerate}
					\item Verify that $\det(\lambda I-A)$ is a monic polynomial of degree $m$.
					\item Therefore using \procedurehr{4.27}, verify that $\det(B)=\det(\lambda I-A)$.
					\item Therefore verify that $\det(B)$ is a monic polynomial of degree $m$.
					\item If any of the diagonal entries of $B$ equal zero, then do the following:
					\begin{enumerate}
						\item Verify that $\det(B)=B_{0,0}B_{1,1}\cdots B_{m-1,m-1}=0$.
						\item Therefore using (3) and (4a), verify that $1=0$.
						\item \textbf{Abort procedure.}
					\end{enumerate}
					\item Otherwise do the following:
					\begin{enumerate}
						\item \textbf{Verify that none of the diagonal entries of $B$ equal zero.}
					\end{enumerate}
				\end{enumerate}
		\procedure{4.44}
			\objective
				Choose a positive integer $m$ and an $m\times m$ rational matrix, $A$. Execute \procedurehr{4.25} on the polynomial matrix $\lambda 1_m-A$ and let $\langle ,B,v,\rangle$ be the result. The objective of the following instructions is to either show that $0<0$ or to construct an integer $a$ such that $\sum_i^{[a:m]}\deg(B_{i,i})=m$, $\deg(B_{i,i})>0$ for $i$ in $[a:m]$, and $\deg(B_{i,i})=0$ for $i$ in $[0:a]$.
			\implementation
				\begin{enumerate}
					\item Execute \procedurehr{4.43} on $A$.
					\item If $\deg(B_{i,i})=0$ for $i$ in $[0:m]$, then do the following:
					\begin{enumerate}
						\item Verify that $\det(\lambda 1_m-A)=\det(B)=B_{0,0}B_{1,1}\cdots B_{m-1,m-1}$.
						\item \textbf{Therefore verify that $0<m=\deg(\det(\lambda 1_m-A))=\deg(B_{0,0}B_{1,1}\cdots B_{m-1,m-1})=0+0+\cdots+0=0$.}
						\item \textbf{Abort procedure.}
					\end{enumerate}
					\item Otherwise do the following:
					\begin{enumerate}
						\item Let $0\le a<m$ be the least integer such that $\deg(B_{a,a})>0$.
						\item \textbf{Verify that $\deg(B_{i,i})=0$ for $i$ in $[0:a]$.}
						\item \textbf{Verify that $\sum_i^{[a:m]}\deg(B_{i,i})=\sum_i^{[0:m]}\deg(B_{i,i})=\deg(B_{0,0}B_{1,1}\cdots B_{m-1,m-1})=\deg(\det(B))=\deg(\lambda 1_m-A)=m$.}
						\item For $i$ in $[a+1:m]$, do the following:
						\begin{enumerate}
							\item Verify that $B_{i,i}=u_iB_{i-1,i-1}$.
							\item Verify that $B_{i,i}\ne 0$.
							\item Therefore verify that $u_i\ne 0$.
							\item \textbf{Therefore verify that $\deg(B_{i,i})=\deg(u_iB_{i-1,i-1})\ge\deg(B_{i-1,i-1})>0$.}
						\end{enumerate}
						\item \textbf{Yield the tuple $\langle a\rangle$.}
					\end{enumerate}
				\end{enumerate}
		\declaration{4.19}
			The notation \gls{matrixe} will be used to refer to the $k\times 1$ rational matrix such that its $i^{th}$ entry, $1$, is the only non-zero entry.
		\declaration{4.22}
			The notation \gls{matrixmattp} will be used as a shorthand for $\sum_j^{[0:t]}p_je_j$.
		\declaration{4.16}
			The notation \gls{matrixcompmatrix}, where $p\ne 0$ is a monic polynomial such that $\deg(p)>0$, will be used as a shorthand for the $\deg(p)\times\deg(p)$ rational matrix of the following constitution:
			\begin{enumerate}
				\item Its first $\deg(p)-1$ columns equal the last $\deg(p)-1$ columns of $1_k$.
				\item Its last column is $-\mat_{\deg(p)}(p)$.
			\end{enumerate}
		\procedure{4.45}
			\objective
				Choose a monic polynomial, $p$ such that $\deg(p)>0$. Let $k=\deg(p)$. Choose a $k\times k$ matrix, $D$, such that $D=\lambda 1_k-\comp(p)$. The objective of the following instructions is to transform $D$ into $\diag(1,\cdots,1,p)$ by a sequence of elementary operations.
			\implementation
				\begin{enumerate}
					\item Let the matrix $D$ be our working matrix.
					\item For $i$ in $[k:1]$, add $\lambda$ times row $i$ to row $i-1$.
					\item Verify that $D$'s first $k-1$ columns are now the last $k-1$ columns of $-1_k$.
					\item Verify that $D$'s last column is $p$ followed by some other polynomials.
					\item For $i$ in $[1:k]$, subtract $D_{i,k-1}$ times column $i-1$ from column $k-1$.
					\item Verify that $D$'s last column is now $p$ followed by zeros.
					\item For $i$ in $[1:k]$, negate row $i-1$ and exchange it with row $i$ using \procedurehr{4.40}.
					\item \textbf{Therefore verify that $D=\diag(1,\cdots,1,p)$.}
				\end{enumerate}
		\procedure{4.46}
			\objective
				Choose a positive integer $m$ and an $m\times m$ rational matrix, $A$. Execute \procedurehr{4.03} on the polynomial matrix $\lambda 1_m-A$ and let $\langle,B,,\rangle$ receive the result. Execute \procedurehr{4.44} on $A$ and let $\langle a\rangle$ receive the result. Let $E_i=\comp(\mon(B_{a+i,a+i}))$ for $i$ in $[0:m-a]$. The objective of the following instructions is to first show that $\cols(\diag(E))=m$, and second to apply a sequence of elementary operations on $\lambda 1_m-\diag(E)$ to obtain the matrix $B$.
			\implementation
				\begin{enumerate}
					\item Verify that the diagonal of $B$ comprises $a$ rationals followed by $B_{a,a},B_{a+1,a+1},\cdots,B_{m-1,m-1}$.
					\item \textbf{Using \procedurehr{4.45}, verify that $\cols(\diag(E))=\sum_i^{[0:{\lvert E\rvert}]}\cols(E_i)=\sum_i^{[0:{\lvert E\rvert}]}\cols(\comp(\mon(B_{a+i,a+i})))=\sum_i^{[0:{\lvert E\rvert}]}\deg(\mon(B_{a+i,a+i}))=\sum_i^{[0:{m-a}]}\deg(B_{a+i,a+i})=\sum_i^{[a:m]}\deg(B_{i,i})=m$.}
					\item Let $F=\lambda 1_m-\diag(E)$.
					\item Now for $i$ in $[0:\lvert E\rvert]$:
					\begin{enumerate}
						\item Let $j=\sum_r^{[0:{i}]}\cols(E_r)$.
						\item Let $k=j+\cols(E_i)$.
						\item Apply \procedurehr{4.45} on the tuple $\langle\mon(B_{a+i,a+i}),F_{[j:k],[j:k]}\rangle$.
					\end{enumerate}
					\item Now verify that $F$ is an $m\times m$ diagonal rational matrix.
					\item Also verify that the diagonal of $F$ comprises $\mon(B_{a,a}),\mon(B_{a+1,a+1}),\cdots,\mon(B_{m-1,m-1})$ and $a$ $1$s.
					\item Rearrange the diagonal of $F$ so that $\mon(B_{i,i})$ is at the $i^{th}$ position on the diagonal for $i$ in $[a:m]$ by doing pairwise swaps. In general, swap the $i^{th}$ and $j^{th}$ diagonal entries using \procedurehr{4.41}.
					\item For $i$ in $[0:m-1]$, do the following:
					\begin{enumerate}
						\item Let $k=\frac{(B_{i,i})_{\deg(B_{i,i})}}{(F_{i,i})_{\deg(F_{i,i})}}$.
						\item Scale $B_{i,i}$ by $k$ and $B_{i+1,i+1}$ by $\frac{1}{k}$ using \procedurehr{4.42}.
						\item Now verify that $F_{i,i}=B_{i,i}$.
					\end{enumerate}
					\item Now verify that $\det(F)_{m}=\det(\lambda 1_m-\diag(E))_m=1=\det(\lambda 1_m-A)_m=\det(B)_m$.
					\item Therefore verify that $(F_{m,m})_{\deg(F_{m,m})}=\frac{\det(F)_m}{(\det(F_{[1:m],[1:m]}))_{m-\deg(F_{m,m})}}=\frac{\det(B)_m}{(\det(B_{[1:m],[1:m]}))_{m-\deg(B_{m,m})}}=(B_{m,m})_{\deg(B_{m,m})}$.
					\item Therefore verify that $F_{m,m}=B_{m,m}$.
					\item \textbf{Therefore verify that $F=B$.}
				\end{enumerate}
		\procedure{4.47}
			\objective
				Choose a $m\times m$ rational matrix, $A$. Execute \procedurehr{4.44} on $A$ and let $\langle a\rangle$ receive the result. Let $E_i=\comp(\mon(B_{a+i,a+i}))$ for $i$ in $[0:m-a]$. The objective of the following instructions is to either show that $0=1$ or to construct $m\times m$ rational matrices $R,T$ such that $A=R\diag(E)T$, $RT=1_m$, and $TR=1_m$.
			\implementation
				\begin{enumerate}
					\item Execute \procedurehr{4.25} on the polynomial matrix $\lambda 1_m-A$ and let $\langle P,B,,Q\rangle$ be the result.
					\item Verify that $P_*(\lambda 1_m-A)Q_*=B$.
					\item Verify that $\lambda 1_m-A={P^{-1}}_*B{Q^{-1}}_*$.
					\item Let $Z$ be a variant of \procedurehr{4.25} where every occurence of \procedurehr{4.10} in its instructions is replaced with \procedurehr{4.46}, and where every mention of $v$ is ignored.
					\item Execute procedure $Z$ on the matrix $\lambda 1_m-\diag(E)$ and let $\langle M,,,N\rangle$ receive the result.
					\item Verify that $M_*(\lambda 1_m-\diag(E))N_*=B$.
					\item Verify that $\lambda 1_m-A={P^{-1}}_*B{Q^{-1}}_*={P^{-1}}_*M(\lambda 1_m-\diag(E))N{Q^{-1}}_*$.
					\item Execute \procedurehr{4.38} on the matrices $\langle A,{P}^{-1}M,\diag(E),N{Q}^{-1}\rangle$. Let the tuple $\langle R,T\rangle$ be the result.
					\item \textbf{Verify that $A=R\diag(E)T$.}
					\item \textbf{Verify that $RT=1_m$.}
					\item \textbf{Verify that $TR=1_m$.}
					\item \textbf{Yield the tuple $\langle R,E,T\rangle$.}
				\end{enumerate}
		\procedure{4.86}
			\objective
				Choose two polynomials $a,b$ and an $m\times m$ matrix $C$ such that $a=b$. The objective of the following instructions is to show that $\Lambda(a,C)=\Lambda(b,C)$.
			\implementation
				Implementation is analogous to that of \procedurehr{2.51}.
		\procedure{4.87}
			\objective
				Choose two polynomials $a,b$ and an $m\times n$ matrix $C$. The objective of the following instructions is to show that $\Lambda(a+b,C)=\Lambda(a,C)+\Lambda(b,C)$.
			\implementation
				Implementation is analogous to that of \procedurehr{2.56}.
		\procedure{4.88}
			\objective
				Choose a polynomial $a$ and an $m\times m$ matrix $B$. The objective of the following instructions is to show that $\Lambda(-a,B)=-\Lambda(a,B)$.
			\implementation
				Implementation is analogous to that of \procedurehr{2.00}.
		\procedure{4.89}
			\objective
				Choose two polynomials $a,b$ and an $m\times m$ matrix $C$. The objective of the following instructions is to show that $\Lambda(ab,C)=\Lambda(a,C)\Lambda(b,C)$.
			\implementation
				Implementation is analogous to that of \procedurehr{2.64}.
		\procedure{4.48}
			\objective
				Choose a polynomial, $r$, and $m\times m$ rational matrices, $R,A,S$ such that $SR=1_m$. The objective of the following instructions is to show that $\Lambda(r,RAS)=R\Lambda(r,A)S$.
			\implementation
				\begin{enumerate}
					\item Verify that $\Lambda(r,RAS)$
					\begin{enumerate}
						\item $=\sum_j^{[0:\lvert r\rvert]}r_j(RAS)^j$
						\item $=\sum_j^{[0:\lvert r\rvert]}r_jRA^jS$
						\item $=R(\sum_j^{[0:\lvert r\rvert]}r_jA^j)S$
						\item $=R\Lambda(r,A)S$.
					\end{enumerate}
				\end{enumerate}
		\procedure{4.49}
			\objective
				Choose a list of $m\times m$ rational matrices, $A$, and a polynomial, $r$. The objective of the following instructions is to show that $\Lambda(r,\diag(A))=\diag(\Lambda(r,A))$.
			\implementation
				\begin{enumerate}
					\item For $i=0$ up to $i=t$, by repeated applications of \procedurehr{4.09}, verify that $\diag(A)^i$ evaluates to $\diag(A^i)$.
					\item Therefore verify that $\Lambda(r,\diag(A))$
					\begin{enumerate}
						\item $=\sum_j^{[0:\lvert r\rvert]}r_j\diag(A)^j$
						\item $=\sum_j^{[0:\lvert r\rvert]}r_j\diag(A^j)$
						\item $=\sum_j^{[0:\lvert r\rvert]}\diag(r_jA^j)$
						\item $=\diag(\sum_j^{[0:\lvert r\rvert]}r_jA^j)$
						\item \textbf{$=\diag(\Lambda(r,A))$.}
					\end{enumerate}
				\end{enumerate}
		\procedure{4.50}
			\objective
				Choose a $m\times m$ rational matrix, $A$, and a polynomial, $r$. Execute \procedurehr{4.47} on the matrix $A$ and let the tuple $\langle R_1,E,R_3\rangle$ receive the result. The objective of the following instructions is to show that $\Lambda(r,A)=R_1\diag(\Lambda(r,E))R_3$.
			\implementation
				\begin{enumerate}
					\item Verify that $R_3R_1=1_m$.
					\item Using \procedurehr{4.48}, verify that $\Lambda(r,A)=\Lambda(r,R_1\diag(E)R_3)=R_1\Lambda(r,\diag(E))R_3$.
					\item Using \procedurehr{4.49}, verify that $\Lambda(r,\diag(E))=\diag(\Lambda(r,E))$.
					\item \textbf{Therefore verify that $\Lambda(r,A)=R_1\diag(\Lambda(r,E))R_3$.}
				\end{enumerate}
		\procedure{4.51}
			\objective
				Choose a monic polynomial $p\ne 0$ such that $\deg(p)>0$. The objective of the following instructions is to show that $\Lambda(p,\comp(p))=0_{\deg(p)\times\deg(p)}$.
			\implementation
				\begin{enumerate}
					\item Let $G=\comp(p)$.
					\item For $i$ in $[0:\deg(p)]$, verify that $G^{i}e_0=G^{i-1}e_1=\cdots=G^{0}e_i=e_i$.
					\item Therefore, for $i\in[0:\deg(p)]$, do the following:
					\begin{enumerate}
						\item Using (1), verify that $\Lambda(p,G)e_i$
						\begin{enumerate}
							\item $=(\sum_j^{[0:\lvert p\rvert]}p_{j}G^{j})e_i$
							\item $=(\sum_j^{[0:\lvert p\rvert]}p_{j}G^{j})G^{i}e_0$
							\item $=G^{i}(GG^{\deg(p)-1}+\sum_j^{[0:\deg(p)]}p_{j}G^{j})e_0$
							\item $=G^{i}(Ge_{\deg(p)-1}+\sum_j^{[0:\deg(p)]}p_{j}e_j)$
							\item $=G^{i}0_{\deg(p)\times 1}$
							\item $=0_{\deg(p)\times 1}$.
						\end{enumerate}
					\end{enumerate}
					\item \textbf{Therefore verify that $\Lambda(p,\comp(p))=\Lambda(p,G)=0_{\deg(p)\times\deg(p)}$.}
				\end{enumerate}
		\declaration{4.20}
			The notation \gls{matrixlast}, where $A$ is an $m\times m$ rational matrix, will be used as a shorthand for the polynomial yielded by executing the following instructions:
			\begin{enumerate}
				\item Execute \procedurehr{4.25} on the polynomial matrix $\lambda 1_m-A$ and let the tuple $\langle,B,,\rangle$ receive the result.
				\item Yield $\langle B_{m-1,m-1}\rangle$.
			\end{enumerate}
		\procedure{4.52}
			\objective
				Choose a $m\times m$ rational matrix, $A$. The objective of the following instructions is to show that either $1=0$ or $\last_A\ne 0$.
			\implementation
				\begin{enumerate}
					\item Execute \procedurehr{4.43} on $A$.
					\item \textbf{Therefore verify that $\last_A\ne 0$.}
				\end{enumerate}
		\procedure{4.53}
			\objective
				Choose a $m\times m$ rational matrix, $A$. The objective of the following instructions is to either show that $0<0$ or to show that $\Lambda(\last_A,A)=0_{m\times m}$.
			\implementation
				\begin{enumerate}
					\item Execute \procedurehr{4.25} on the matrix $A$ and let the tuple $\langle M,B,v,N\rangle$ receive the result.
					\item Execute \procedurehr{4.44} on $A$ and let $\langle a\rangle$ receive.
					\item Execute \procedurehr{4.47} on $A$ and let $\langle R,E,T\rangle$ receive.
					\item For $j$ in $[0:\lvert E\rvert]$:
					\begin{enumerate}
						\item Verify that $E_j=\comp(\mon(B_{a+j,a+j}))$.
						\item Verify that $\last_A=B_{m-1,m-1}=B_{a+j,a+j}\prod_r^{[a+j+1:m]} v_r$.
						\item Let $k=\deg(\mon(B_{a+j,a+j}))$.
						\item Therefore using \procedurehr{4.51} verify that $\Lambda(\last_A,E_j)=\Lambda(B_{m-1,m-1},E_j)=\Lambda(B_{a+j,a+j},\comp(\mon(B_{a+j,a+j})))\prod_r^{[a+j+1:m]}\Lambda(v_r,E_j)=0_{k\times k}\prod_r^{[a+j+1:m]}\Lambda(v_r,E_j)=0_{k\times k}$.
					\end{enumerate}
					\item \textbf{Therefore using \procedurehr{4.50} verify that $\Lambda(\last_A,A)=R\diag(\Lambda(\last_A,E))T=R\diag(\Lambda(B_{m-1,m-1},E))T=R0_{m\times m}T=0_{m\times m}$.}
				\end{enumerate}
		\procedure{4.54}
			\objective
				Choose a monic polynomial $p$ such that $\deg(p)>0$. Choose a polynomial $g\ne 0$ such that $\deg(g)<\deg(p)$. The objective of the following instructions is to show that $\Lambda(g,\comp(p))\ne 0_{\deg(p)\times\deg(p)}$.
			\implementation
				\begin{enumerate}
					\item Let $G=\comp(p)$.
					\item Therefore using \declarationhr{4.16}, verify that $\Lambda(g,G)e_0=(\sum_j^{[0:\deg(g)+1]}g_jG^j)e_0=\sum_j^{[0:\deg(g)+1]}g_je_j\ne 0_{\deg(p)\times 1}$.
					\item \textbf{Therefore verify that $\Lambda(g,G)\ne 0_{\deg(p)\times\deg(p)}$.}
				\end{enumerate}
		\procedure{4.55}
			\objective
				Choose a polynomial $g$ and a monic polynomial $p$ such that $\deg(p)=\deg(g)>0$ and $\Lambda(g,\comp(p))=0_{\deg(g)\times\deg(g)}$. The objective of the following instructions is to show that $g=g_{\deg(g)}p$.
			\implementation
				\begin{enumerate}
					\item Let $G=\comp(p)$.
					\item Using \declarationhr{4.16}, verify that $0_{\deg(g)\times 1}=\Lambda(g,G)e_0=(\sum_j^{[0:\lvert g\rvert]}g_jG^j)e_0=g_{\deg(g)}Ge_{\deg(g)-1}+\sum_j^{[0:\deg(g)]}g_je_j$.
					\item Therefore for $i$ in $[0:\deg(g)]$, do the following:
					\begin{enumerate}
						\item Verify that $0=(g_{\deg(g)}Ge_{\deg(g)-1}+\sum_j^{[0:\deg(g)]}g_je_j)_{i,0}$.
						\item Therefore using \declarationhr{4.16}, verify that $-g_{\deg(g)}p_{i}+g_{i}=0$.
						\item Therefore verify that $g_{i}=g_{\deg(g)}p_{i}$.
					\end{enumerate}
					\item \textbf{Therefore verify that $g=g_{\deg(g)}p$.}
				\end{enumerate}
		\procedure{4.56}
			\objective
				Choose a $m\times m$ rational matrix, $A$. Choose a polynomial $p\ne 0$, such that $\Lambda(p,A)=0_{m\times m}$. The objective of the following instructions is to either show that $0\ne 0$ or to construct a polynomial $f$ such that $p=f\last_A$.
			\implementation
				\begin{enumerate}
					\item Let $F$ be the $1\times 2$ matrix $\langle\langle p,\last_A\rangle\rangle$.
					\item Execute \procedurehr{4.25} on $F$ and let $\langle M,D,,N\rangle$ receive the result.
					\item Verify that $D_{0,0}\ne 0$.
					\item Let $g=D_{0,0}$.
					\item Verify that $F={M^{-1}}_*D{N^{-1}}_*=D{N^{-1}}_*$.
					\item Verify that $\last_A=F_{0,1}=D_{0,0}{{N^{-1}}_*}_{0,1}+D_{0,1}{{N^{-1}}_*}_{1,1}=D_{0,0}{{N^{-1}}_*}_{0,1}=g{{N^{-1}}_*}_{0,1}$.
					\item Therefore verify that ${{N^{-1}}_*}_{0,1}\ne 0$.
					\item Let $u=\deg(\last_A)$.
					\item Now verify that $u=\deg(\last_A)=\deg(D_{0,0}{{N^{-1}}_*}_{0,1})\ge\deg(D_{0,0})=\deg(g)$.
					\item Verify that $D=M_*FN_*=FN_*$.
					\item Therefore verify that $g=D_{0,0}={N_*}_{0,0}p+{N_*}_{1,0}\last_A$.
					\item Therefore using \procedurehr{4.51}, verify that $\Lambda(g,A)=\Lambda({N_*}_{0,0},A)\Lambda(p,A)+\Lambda({N_*}_{1,0},A)\Lambda(\last_A,A)=\Lambda({N_*}_{0,0},A)0_{m\times m}+\Lambda({N_*}_{1,0},A)0_{m\times m}=0_{m\times m}$.
					\item Execute \procedurehr{4.47} on the matrix $A$ and let the tuple $\langle R_1,E,R_3\rangle$ receive the result.
					\item Using \procedurehr{4.50}, and \procedurehr{4.47}, verify that $\diag(\Lambda(g,E))=1_m\diag(\Lambda(g,E))1_m=R_3R_1\diag(\Lambda(g,E))R_3R_1=R_3\Lambda(g,A)R_1=R_30_{m\times m}R_1=0_{m\times m}$.
					\item Let $G=\comp(\mon(\last_A))$.
					\item Verify that $\Lambda(g,G)=\Lambda(g,E_{\lvert E\rvert-1})=\diag(\Lambda(g,E))_{[m-u:m],[m-u:m]}=0_{u\times u}$.
					\item If $\deg(g)<u$, then:
					\begin{enumerate}
						\item Using \procedurehr{4.54}, verify that $\Lambda(g,G)\ne 0_{u\times u}$.
						\item \textbf{Therefore using (16), verify that $0_{u\times u}=\Lambda(g,G)\ne 0_{u\times u}$.}
						\item \textbf{Abort procedure.}
					\end{enumerate}
					\item Otherwise, do the following:
					\begin{enumerate}
						\item Verify that $\deg(g)=u$.
						\item Using \procedurehr{4.55}, verify that $g=g_{\deg(g)}\last_A$.
						\item \textbf{Therefore verify that $p=F_{0,0}=D_{0,0}{{N^{-1}}_*}_{0,0}+D_{0,1}{{N^{-1}}_*}_{1,0}={{N^{-1}}_*}_{0,0}g+{{N^{-1}}_*}_{1,0}*0={{N^{-1}}_*}_{0,0}g={{N^{-1}}_*}_{0,0}g_{\deg(g)}\last_A$.}
						\item \textbf{Yield the tuple $\langle {{N^{-1}}_*}_{0,0}g_{\deg(g)}\rangle$.}
					\end{enumerate}
				\end{enumerate}
		\procedure{4.57}
			\objective
				Choose an $m\times n$ rational matrix, $A$, and an $n\times m$ rational matrix, $B$, such that $AB=1_m$. The objective of the following instructions is to show that either $0=1$ or every column of $B$ is non-zero.
			\implementation
				\begin{enumerate}
					\item If any column $i$ of $B$, $Be_i$, is equal to zero, then:
					\begin{enumerate}
						\item Verify that $0_{n\times 1}=A0_{n\times 1}=A(Be_i)=(AB)e_i=1_me_i=e_i$.
						\item \textbf{Therefore verify that 0=1.}
						\item \textbf{Abort procedure.}
					\end{enumerate}
				\end{enumerate}
		\procedure{4.58}
			\objective
				Choose a $m\times m$ rational matrix, $A$. Choose a polynomial $p$ such that $p\ne 0$, $\Lambda(p,A)=0$, and $\deg(p)<\deg(\last_A)$. The objective of the following instructions is to show that $0<0$.
			\implementation
				\begin{enumerate}
					\item Execute \procedurehr{4.56} on $A$ and $p$ and let $f$ receive.
					\item Now verify that $p=f\last_A$.
					\item Now using the precondition and (2), verify that $f\ne 0$ and $\last_A\ne 0$.
					\item \textbf{Therefore using the precondition, (2), and (3), verify that $\deg(\last_A)>\deg(p)=\deg(f\last_A)\ge\deg(\last_A)$.}
					\item \textbf{Abort procedure.}
				\end{enumerate}
		\declaration{4.21}
			The notation \gls{matrixpows}, where $A$ is a $m\times m$ rational matrix, will be used as a shorthand for the result yielded by executing the following instructions:
			\begin{enumerate}
				\item Let $t=\deg(\last_A)$.
				\item Make an $m^2\times t$ matrix, $B$, whose $i^{th}$ column is the sequential concatenation of the columns of $A^i$.
				\item \textbf{Yield $\langle B\rangle$.}
			\end{enumerate}
		\procedure{4.59}
			\objective
				Choose a $m\times m$ rational matrix, $A$. Execute \procedurehr{4.25} on $\pows(A)$ and let the tuple $\langle M,D,,N\rangle$ receive the result. Let $t=\cols(\pows(A))$. The objective of the following instructions is to show that either $0<0$ or to show that ${C_t(D)}={C_t(D)}_{0,0}e_0\ne 0$.
			\implementation
				\begin{enumerate}
					\item Execute \procedurehr{4.25} on $\pows(A)$ and let the tuple $\langle M,D,,N\rangle$ receive the result.
					\item Verify that $M_*\pows(A)N_*=D$.
					\item Using \procedurehr{4.05}, verify that ${M^{-1}}_*M_*\pows(A)N_*=1_{m^2}\pows(A)N_*=\pows(A)N_*={M^{-1}}_*D$.
					\item If ${C_t(D)}_{0,0}=0$, then:
					\begin{enumerate}
						\item Verify that for some $0\le i<t$, $D_{i,i}=0$.
						\item Therefore verify that $De_i=0_{m^2\times 1}$.
						\item Therefore verify that $\pows(A)(Ne_i)=(\pows(A)N)e_i=(M^{-1}D)e_i=M^{-1}(De_i)=0_{m^2\times 1}$.
						\item Let $p=N_{0,i}\lambda^0+N_{1,i}\lambda^1+\cdots+N_{t-1,i}\lambda^{t-1}$.
						\item Therefore verify that $\Lambda(p,A)=0_{m\times m}$.
						\item Execute \procedurehr{4.57} on ${N^{-1}}_*$ and $N_*$.
						\item Therefore verify that $p\ne 0$.
						\item Execute \procedurehr{4.58} on $A$ and $p$.
						\item \textbf{Abort procedure.}
					\end{enumerate}
					\item Otherwise, do the following:
					\begin{enumerate}
						\item Execute \procedurehr{4.21} on $\langle D,1_t,t\rangle$ and let $E$ receive.
						\item Verify that $C_t(D)=C_t(D1_t)=EC_t(1_t)=E*1=E$.
						\item Verify that $E$ is a $\binom{m^2}{t}\times\binom{t}{t}$ diagonal matrix.
						\item Therefore verify that $C_t(D)$ is a $\binom{m^2}{t}\times 1$ diagonal matrix.
						\item \textbf{Therefore verify that $C_t(D)={C_t(D)}_{0,0}e_0\ne 0$.}
					\end{enumerate}
				\end{enumerate}
		\procedure{4.60}
			\objective
				Choose a $m\times m$ rational matrix, $A$. Let $t=\cols(\pows(A))$. The objective of the following instructions is to show that either $0<0$ or to show that ${C_t(\pows(A))}\ne 0$.
			\implementation
				\begin{enumerate}
					\item Execute \procedurehr{4.25} on $\pows(A)$ and let the tuple $\langle M,D,,N\rangle$ receive the result.
					\item Verify that $\pows(A)={M^{-1}}_*D{N^{-1}}_*$.
					\item Execute \procedurehr{4.57} on $C_t(M_*),C_t({M^{-1}}_*)$.
					\item Hence verify that all columns of $C_t({M^{-1}}_*)$ are non-zero.
					\item Execute \procedurehr{4.59} on $A$.
					\item Verify that $C_t(D)={C_t(D)}_{0,0}e_0\ne 0$.
					\item Therefore verify that ${C_t(D)}_{0,0}\ne 0$.
					\item Execute \procedurehr{4.57} on $C_t(N_*),C_t({N^{-1}}_*)$.
					\item Hence verify that $C_t(N^{-1})\ne 0$.
					\item \textbf{Verify that $C_t(\pows(A))=C_t({M^{-1}}_*D{N^{-1}}_*)=C_t({M^{-1}}_*)C_t(D)C_t({N^{-1}}_*)=C_t({M^{-1}}_*){C_t(D)}_{0,0}e_0C_t({N^{-1}}_*)={C_t(D)}_{0,0}C_t({N^{-1}}_*)C_t({M^{-1}}_*)e_0\ne 0_{\binom{m^2}{t}\times 1}$.}
				\end{enumerate}
		\declaration{4.26}
			The notation \gls{matrixtr}, where $A$ is a square matrix, will be used as a shorthand for the sum of its diagonal entries.
		\procedure{4.68}
			\objective
				Choose two $m\times m$ matrices $A,B$. The objective of the following instructions is to show that $\tr(A+B)=\tr(A)+\tr(B)$.
			\implementation
				\begin{enumerate}
					\item Verify that $\tr(A+B)$
					\begin{enumerate}
						\item $=\sum_r^{[0:m]}(A+B)_{r,r}$
						\item $=\sum_r^{[0:m]}(A_r+B_r)_{r,r}$
						\item $=\sum_r^{[0:m]}A_{r,r}+\sum_r^{[0:m]}B_{r,r}$
						\item $=\tr(A)+\tr(B)$.
					\end{enumerate}
				\end{enumerate}
		\procedure{4.69}
			\objective
				Choose a polynomial $b$ and an $m\times m$ matrix $A$. The objective of the following instructions is to show that $\tr(bA)=b\tr(A)$.
			\implementation
				\begin{enumerate}
					\item Verify that $\tr(bA)$
					\begin{enumerate}
						\item $=\tr(b_{m\times m}A)$
						\item $=\sum_r^{[0:m]}(b_{m\times m}A)_{r,r}$
						\item $=\sum_r^{[0:m]}\sum_t^{[0:m]}(b_{m\times m})_{r,t}A_{t,r}$
						\item $=\sum_r^{[0:m]}(b_{m\times m})_{r,r}A_{r,r}$
						\item $=\sum_r^{[0:m]}bA_{r,r}$
						\item $=b\sum_r^{[0:m]}A_{r,r}$
						\item $=b\tr(A)$.
					\end{enumerate}
				\end{enumerate}
		\procedure{4.70}
			\objective
				Choose an $m\times n$ matrix $A$ and an $n\times m$ matrix $B$. The objective of the following instructions is to show that $\tr(AB)=\tr(BA)$.
			\implementation
				\begin{enumerate}
					\item Verify that $\tr(AB)$
					\begin{enumerate}
						\item $=\sum_r^{[0:m]}(AB)_{r,r}$
						\item $=\sum_r^{[0:m]}\sum_t^{[0:n]}A_{r,t}B_{t,r}$
						\item $=\sum_t^{[0:n]}\sum_r^{[0:m]}B_{t,r}A_{r,t}$
						\item $=\sum_t^{[0:n]}(BA)_{t,t}$
						\item $=\tr(BA)$.
					\end{enumerate}
				\end{enumerate}
		\procedure{4.71}
			\objective
				Choose an $m\times n$ matrix $A$ such that $A\ne 0$. The objective of the following instructions is to show that $\tr(A^TA)>0$.
			\implementation
				\begin{enumerate}
					\item Verify that $\tr(A^TA)$
					\begin{enumerate}
						\item $=\sum_r^{[0:n]}(A^TA)_{r,r}$
						\item $=\sum_r^{[0:n]}\sum_t^{[0:m]}(A^T)_{r,t}A_{t,r}$
						\item $=\sum_r^{[0:n]}\sum_t^{[0:m]}A_{t,r}A_{t,r}$
						\item $=\sum_r^{[0:n]}\sum_t^{[0:m]}(A_{t,r})^2$
						\item $>0$.
					\end{enumerate}
				\end{enumerate}
		\declaration{4.27}
			The phrase "\gls{matrixsymmetric}" will be used to refer to matrices $A$ such that "$A^T=A$".
		\procedure{4.61}
			\objective
				Choose a symmetric $m\times m$ rational matrix, $A$. Let $t=\deg(\last_A)$. Choose two polynomials $u,w$ such that $\deg(u)<t$ and $\deg(w)<t$. The objective of the following instructions is to show that $\tr(\Lambda(uw,A))=\mat(u)^T\pows(A)^T\pows(A)\mat_t(w)$.
			\implementation
				\begin{enumerate}
					\item Verify that $\tr(\Lambda(uw,A))$
					\begin{enumerate}
						\item $=\tr(\Lambda(u,A)\Lambda(w,A))$
						\item $=\tr((\sum_p^{[0:t]} u_pA^{p})(\sum_q^{[0:t]} w_qA^{q}))$
						\item $=\tr(\sum_p^{[0:t]}\sum_q^{[0:t]} u_pw_qA^{p}A^{q})$
						\item $=\sum_p^{[0:t]}\sum_q^{[0:t]} u_pw_q\tr(A^{p}A^{q})$
						\item $=\sum_p^{[0:t]}\sum_q^{[0:t]} u_pw_q\sum_e^{[0:m]}\sum_f^{[0:m]}{A^{p}}_{e,f}\cdot{A^{q}}_{f,e}$
						\item $=\sum_p^{[0:t]}\sum_q^{[0:t]} u_pw_q\sum_e^{[0:m]}\sum_f^{[0:m]}{A^{p}}_{f,e}\cdot{A^{q}}_{f,e}$
						\item $=\sum_p^{[0:t]}\sum_q^{[0:t]} u_pw_q\sum_g^{[0:{m^2}]}{\pows(A)}_{g,p}{\pows(A)}_{g,q}$
						\item $=\sum_p^{[0:t]}\sum_q^{[0:t]} u_pw_q(\pows(A)^T\pows(A))_{p,q}$
						\item $=\sum_p^{[0:t]} u_p(\pows(A)^T\pows(A)\mat_t(w))_{p}$
						\item $=\mat_t(u)^T\pows(A)^T\pows(A)\mat_t(w)$
					\end{enumerate}
				\end{enumerate}
		\declaration{4.25}
			The notation \gls{matrixsel}, where $A$ is an $m\times m$ rational matrix, will be used as a shorthand for the result yielded by executing the following instructions:
			\begin{enumerate}
				\item Using \procedurehr{4.30}, \procedurehr{4.60}, and \procedurehr{4.71}, verify that $C_t(\pows(A)^T\pows(A))=C_t(\pows(A)^T)C_t(\pows(A))={C_t(\pows(A))}^TC_t(\pows(A))=\tr({C_t(\pows(A))}^TC_t(\pows(A)))>0$.
				\item Let $t=\deg(\last_A)$.
				\item Let $H=(\pows(A)^T\pows(A))\backslash e_{t-1}$.
				\item \textbf{Yield $\langle\frac{\sum_j^{[0:t]}H_{j,0}\lambda^j}{(\last_A)_t}\rangle$.}
			\end{enumerate}
		\procedure{4.62}
			\objective
				Choose a symmetric $m\times m$ rational matrix, $A$. Let $t=\deg(\last_A)$. Choose a polynomial $u$ such that $\deg(u)<t$. The objective of the following instructions is to show that $\tr(\Lambda(u\sel_A,A))=\frac{u_{t-1}}{(\last_A)_t}$.
			\implementation
				\begin{enumerate}
					\item Using \procedurehr{4.61}, verify that $\tr(\Lambda(u\sel_A,A))$
					\begin{enumerate}
						\item $=\mat(u)^T\pows(A)^T\pows(A)\mat_t(\sel_A)$
						\item $=\frac{\mat(u)^T\pows(A)^T\pows(A)((\pows(A)^T\pows(A))\backslash e_{t-1})}{(\last_A)_t}$
						\item $=\frac{\mat(u)^Te_{t-1}}{(\last_A)_t}$
						\item $=\frac{\mat(u)_{t-1,0}}{(\last_A)_t}$
						\item $=\frac{u_{t-1}}{(\last_A)_t}$.
					\end{enumerate}
				\end{enumerate}
		\procedure{4.63}
			\objective
				Choose a symmetric $m\times m$ rational matrix, $A$. The objective of the following instructions is to either show that $0\ne 0$ or construct polynomials $u,v$ such that $u\last_A+v\sel_A=1$.
			\implementation
				\begin{enumerate}
					\item Let $t=\deg(\last_A)$.
					\item Let $G$ be the $1\times 2$ matrix $\langle\langle\last_A,\sel_A\rangle\rangle$.
					\item Execute \procedurehr{4.25} on $G$ and let the tuple $\langle M,D,,N\rangle$ receive.
					\item Verify that $G={M^{-1}}_*D{N^{-1}}_*$.
					\item Verify that $\last_A\ne 0$.
					\item Therefore verify that $D_{0,0}\ne 0$.
					\item If $\deg(D_{0,0})>0$, then do the following:
					\begin{enumerate}
						\item Let $b={{N^{-1}}_*}_{0,0}$.
						\item Verify that $\last_A=bD_{0,0}$.
						\item Therefore verify that $b\ne 0$.
						\item Let $z=\deg(b)$.
						\item Verify that $t=\deg(\last_A)=\deg(bD_{0,0})=\deg(b)+\deg(D_{0,0})>\deg(b)=z$.
						\item Let $c={{N^{-1}}_*}_{0,1}$.
						\item Verify that $\sel_A=cD_{0,0}$.
						\item Let $u=\lambda^{t-z-1}b$.
						\item Execute \procedurehr{4.62} on $A$ and $u$.
						\item Hence verify that $(\last_A)_t\tr(\Lambda(u\sel_A,A))=u_{t-1}=b_z\ne 0$.
						\item Also verify that $\tr(\Lambda(u\sel_A,A))$
						\begin{enumerate}
							\item $=\tr(\Lambda(\lambda^{t-z-1}bcD_{0,0},A))$
							\item $=\tr(\Lambda(\lambda^{t-z-1}c\last_A,A))$
							\item $=\tr(\Lambda(\lambda^{t-z-1}c,A)\Lambda(\last_A,A))$
							\item $=\tr(\Lambda(\lambda^{t-z-1}c,A)0_{m\times m})$
							\item $=\tr(0_{m\times m})$
							\item $=0$.
						\end{enumerate}
						\item \textbf{Therefore verify that $0\ne 0$.}
						\item \textbf{Abort procedure.}
					\end{enumerate}
					\item Otherwise, do the following:
					\begin{enumerate}
						\item Verify that $\deg(D_{0,0})=0$.
						\item Let $u=\frac{N_{0,0}}{D_{0,0}}$.
						\item Let $v=\frac{N_{1,0}}{D_{0,0}}$.
						\item \textbf{Verify that $u\last_A+v\sel_A=1$.}
						\item \textbf{Yield the tuple $\langle u,v\rangle$.}
					\end{enumerate}
				\end{enumerate}
		\procedure{4.64}
			\objective
				Choose a symmetric $m\times m$ rational matrix $A$, where $m>0$. Let $t=\deg(\last_A)$. The objective of the following instructions is to either show that $0\ne 0$ or to construct lists of polynomials $s,q$ such that
				\begin{enumerate}
					\item For $i=0$ to $i=t$, $\deg(s_i)=i$.
					\item For $i=0$ to $i=t$, $\sgn((s_i)_i)=\sgn((s_t)_t)$.
					\item For $i=1$ to $i=t-1$, $s_{i-1}+s_{i+1}=q_is_i$.
					\item $s_t=\last_A$.
				\end{enumerate}
			\implementation
				\begin{enumerate}
					\item Let $s_t=\last_A$.
					\item Execute \procedurehr{4.63} on $A$ and let $\langle u,s_{t+1}\rangle$ receive the result.
					\item Hence verify that $us_t+s_{t+1}\sel_A=1$.
					\item Let $q_t=s_{t+1}\div s_t$.
					\item Let $s_{t-1}=s_{t+1}\mod s_t$.
					\item Verify that $s_{t+1}=q_ts_t+s_{t-1}$, where $\deg(s_{t-1})<\deg(s_t)=t$.
					\item Therefore verify that $us_t+(q_ts_t+s_{t-1})\sel_A=1$.
					\item Therefore verify that $\Lambda(s_{t-1}\sel_A,A)=\Lambda(us_t+(q_ts_t+s_{t-1})\sel_A,A)=\Lambda(1,A)=1_{m}$.
					\item Therefore using \procedurehr{4.62}, verify that $\frac{(s_{t-1})_{t-1}}{(s_t)_t}=\tr(\Lambda(s_{t-1}\sel_A,A)=\tr(1_{m})=m>0$.
					\item For $i\in[t:1]$, do the following:
					\begin{enumerate}
						\item Let $q_i=(-s_{i+1})\div(-s_i)$.
						\item Let $s_{i-1}=(-s_{i+1})\mod(-s_i)$.
						\item Verify that $\deg(q_i)=1$.
						\item Verify that $(q_i)_1=\frac{(s_{i+1})_{i+1}}{(s_i)_i}$.
						\item Also verify that $-s_{i+1}=-q_is_i+s_{i-1}$.
						\item Therefore verify that $q_is_i=s_{i+1}+s_{i-1}$.
						\item Therefore verify that $q_is_i-s_{i+1}=s_{i-1}$.
						\item Execute \procedurehr{2.26} on the tuple $\langle s,q,i-1\rangle$ and let $\langle p,j\rangle$ receive.
						\item Verify that $s_{i-1}=ps_{t-1}+js_t$.
						\item Verify that $\deg(p)=t-1-(i-1)=t-i$.
						\item Verify that $\deg(j)=t-2-(i-1)=t-1-i$
						\item Therefore verify that $\Lambda(s_{i-1},A)=\Lambda(ps_{t-1}+js_t,A)=\Lambda(ps_{t-1},A)+\Lambda(j,A)\Lambda(s_t,A)=\Lambda(ps_{t-1},A)+\Lambda(j,A)0_{m\times m}=\Lambda(ps_{t-1},A)$.
						\item If $\Lambda(p,A)=0$, then do the following:
						\begin{enumerate}
							\item Execute \procedurehr{4.58} on $A$ and $p$.
							\item \textbf{Abort procedure.}
						\end{enumerate}
						\item Otherwise, if $\Lambda(s_{i-1},A)=0_{m\times m}$, then do the following:
						\begin{enumerate}
							\item Verify that $\Lambda(ps_{t-1}\sel_A,A)=\Lambda(ps_{t-1},A)\Lambda(\sel_A,A)=\Lambda(s_{i-1},A)\Lambda(\sel_A,A)=0_{m\times m}\Lambda(\sel_A,A)=0_{m\times m}$.
							\item Verify that $\Lambda(ps_{t-1}\sel_A,A)=\Lambda(p,A)\Lambda(s_{t-1}\sel_A,A)=\Lambda(p,A)1_{m}=\Lambda(p,A)\ne0_{m\times m}$.
							\item Therefore verify that $0\ne 0$.
							\item \textbf{Abort procedure.}
						\end{enumerate}
						\item Otherwise if $\Lambda(s_{i-1}\sel_A,A)=0_{m\times m}$, then do the following:
						\begin{enumerate}
							\item Verify that $\Lambda(s_{i-1}\sel_As_{t-1},A)=\Lambda(s_{i-1}\sel_A,A)\Lambda(s_{t-1},A)=0_{m\times m}\Lambda(s_{t-1},A)=0_{m\times m}$.
							\item Verify that $\Lambda(s_{i-1}\sel_As_{t-1},A)=\Lambda(s_{i-1},A)\Lambda(\sel_As_{t-1},A)=\Lambda(s_{i-1},A)1_{m}=\Lambda(s_{i-1},A)\ne 0_{m\times m}$.
							\item Therefore verify that $0_{m\times m}\ne 0_{m\times m}$.
							\item \textbf{Abort procedure.}
						\end{enumerate}
						\item Otherwise, do the following:
						\begin{enumerate}
							\item Verify that $\deg(s_{i-1})<i$.
							\item Verify that $\Lambda(s_{i-1}\sel_A,A)\ne 0_{m\times m}$.
							\item Execute the \hyperref[sec:procedure 73 auxilliary procedure]{auxilliary procedure} on the tuple $(i-1, s_{i-1})$.
							\item Hence using \procedurehr{4.71}, verify that $\frac{(s_{i-1})_{i-1}}{(s_i)_i}=\tr(\Lambda({s_{i-1}}^2{\sel_A}^2,A))=\tr((\Lambda(s_{i-1}\sel_A,A))^2)=\tr((\Lambda(s_{i-1}\sel_A,A))^T(\Lambda(s_{i-1}\sel_A,A)))>0$.
							\item \textbf{Therefore verify that $\sgn((s_{i-1})_{i-1})=\sgn((s_i)_i)$.}
						\end{enumerate}
					\end{enumerate}
					\item Yield the tuple $\langle s_{[0:t+1]}, q_{[0:t]}\rangle$.
				\end{enumerate}
			\subsubsection*{Auxilliary procedure}\label{sec:procedure 73 auxilliary procedure}
				\paragraph{Objective}
					Choose an integer $0\le k\le t$ such that polynomial $s_k$ is defined. Choose a polynomial $g$ such that $\deg(g)\le\min(k,t-1)$. The objective of the following instructions is to show that $\tr(\Lambda(gs_k{\sel_A}^2,A))=\frac{g_k}{(s_{k+1})_{k+1}}$.
				\paragraph{Implementation}
					\begin{enumerate}
						\item If $k=t$, then verify that $\tr(\Lambda(gs_k{\sel_A}^2,A))$
						\begin{enumerate}
							\item $=\tr(\Lambda(gs_t{\sel_A}^2,A))$
							\item $=\tr(\Lambda(g{\sel_A}^2,A)\Lambda(s_t,A))$
							\item $=\tr(\Lambda(g{\sel_A}^2,A)0_{m\times m})$
							\item $=0$
							\item $=\frac{g_k}{(s_{k+1})_{k+1}}$.
						\end{enumerate}
						\item Otherwise if $k=t-1$, then verify that $\tr(\Lambda(gs_k{\sel_A}^2,A))$
						\begin{enumerate}
							\item $=\tr(\Lambda(gs_{t-1}{\sel_A}^2,A))$
							\item $=\tr(\Lambda(g{\sel_A},A)\Lambda(s_{t-1}{\sel_A},A))$
							\item $=\tr(\Lambda(g\sel_A,A)1_{m})$
							\item $=\tr(\Lambda(g\sel_A,A))$
							\item \textbf{=$\frac{g_k}{(s_{k+1})_{k+1}}$.}
						\end{enumerate}
						\item Otherwise if $k<t-1$, then do the following:
						\begin{enumerate}
							\item Verify that $\deg(gq_{k+1})=k+1\le t-1$.
							\item Execute the \hyperref[sec:procedure 73 auxilliary procedure]{auxilliary procedure} on the tuple $\langle k+1, gq_{k+1}\rangle$.
							\item Now verify that $\tr(\Lambda((gq_{k+1})s_{k+1}{\sel_A}^2,A))=\frac{\frac{(s_{k+2})_{k+2}}{(s_{k+1})_{k+1}}g_k}{(s_{k+2})_{k+2}}=\frac{g_k}{(s_{k+1})_{k+1}}$.
							\item Verify that $\deg(g)\le k\le t-2$.
							\item Execute the \hyperref[sec:procedure 73 auxilliary procedure]{auxilliary procedure} on the tuple $\langle k+2,g\rangle$.
							\item Now verify that $\tr(\Lambda(gs_{k+2}{\sel_A}^2,A))=\frac{g_{k+2}}{(s_{k+3})_{k+3}}=\frac{0}{(s_{k+3})_{k+3}}=0$.
							\item Therefore verify that $\tr(\Lambda(gs_k{\sel_A}^2,A))$
							\begin{enumerate}
								\item $=\tr(\Lambda(g(q_{k+1}s_{k+1}+s_{k+2}){\sel_A}^2,A))$
								\item $=\tr(\Lambda(gq_{k+1}s_{k+1}{\sel_A}^2+gs_{k+2}{\sel_A}^2,A))$
								\item $=\tr(\Lambda(gq_{k+1}s_{k+1}{\sel_A}^2,A)+\Lambda(gs_{k+2}{\sel_A}^2,A))$
								\item $=\tr(\Lambda(gq_{k+1}s_{k+1}{\sel_A}^2,A))+\tr(\Lambda(gs_{k+2}{\sel_A}^2,A))$
								\item $=\frac{g_k}{(s_{k+1})_{k+1}}+0$
								\item $=\frac{g_k}{(s_{k+1})_{k+1}}$.
							\end{enumerate}
						\end{enumerate}
					\end{enumerate}
		\procedure{4.65}
			\objective
				Choose a symmetric $m\times m$ rational matrix, $A$. Let $t=\deg(\last_A)$. The objective of the following instructions is to either show that $0<0$ or to construct two lists of rational numbers $c,d$ such that $c_0<d_0\le c_1<d_1\le\cdots\le c_{t-1}<d_{t-1}$ and $0\ne\sgn(\Lambda(\last_A,c_i))=-\sgn(\Lambda(\last_A,d_i))$ for $i$ in $[0:t]$.
			\implementation
				\begin{enumerate}
					\item Execute \procedurehr{4.64} on the matrix $A$ and let the tuple $\langle s,q\rangle$ receive the result.
					\item Execute \procedurehr{2.24} supplying the tuple $\langle s,q\rangle$. Let the tuple $\langle c,d\rangle$ receive the result.
					\item \textbf{Verify that $c_0<d_0\le c_1<d_1\le\cdots\le c_{t-1}<d_{t-1}$.}
					\item \textbf{Verify that $\sgn(\Lambda(\last_A,c_i))=-\sgn(\Lambda(\last_A,d_i))$ for $i$ in $[0:t]$.}
					\item \textbf{Yield $\langle c,d\rangle$.}
				\end{enumerate}
		\procedure{4.66}
			\objective
				Choose a symmetric $m\times m$ rational matrix, $A$. Let $t=\deg(\last_A)$. Execute \procedurehr{4.65} on $A$ and let the tuple $\langle c,d\rangle$ receive the result. Execute \procedurehr{4.25} on $A$ and let the tuple $\langle,,u,\rangle$ receive the result. The objective of the following instructions is to either show that $1=-1$ or to construct a list of non-negative integers $k$ such that $0\ne\sgn(\Lambda(u_{k_i},c_i))=-\sgn(\Lambda(u_{k_i},d_i))$ for $i$ in $[0:t]$.
			\implementation
				\begin{enumerate}
					\item Verify that $\last_A=u_0u_1\cdots u_{m-1}$.
					\item For $i$ in $[0:t]$, do the following:
					\begin{enumerate}
						\item Using the precondition, verify that $0\ne\sgn(\Lambda(\last_A,c_i))=-\sgn(\Lambda(\last_A,d_i))$.
						\item If $0\in\sgn(\Lambda(u,c_i))$, then do the following:
						\begin{enumerate}
							\item Verify that $0$
							\begin{enumerate}
								\item $=\sgn(\Lambda(u_0,c_i))\sgn(\Lambda(u_1,c_i))\cdots\sgn(\Lambda(u_{m-1},c_i))$
								\item $=\sgn(\Lambda(u_0,c_i)\Lambda(u_1,c_i)\cdots\Lambda(u_{m-1},c_i))$
								\item $=\sgn(\Lambda(u_0u_1\cdots u_{m-1},c_i))$
								\item $=\sgn(\Lambda(\last_A,c_i))$
								\item $\ne 0$.
							\end{enumerate}
						\end{enumerate}
						\item If $0\in\sgn(\Lambda(u,d_i))$, then do the following:
						\begin{enumerate}
							\item Verify that $0$
							\begin{enumerate}
								\item $=\sgn(\Lambda(u_0,d_i))\sgn(\Lambda(u_1,d_i))\cdots\sgn(\Lambda(u_{m-1},d_i))$
								\item $=\sgn(\Lambda(u_0,d_i)\Lambda(u_1,d_i)\cdots\Lambda(u_{m-1},d_i))$
								\item $=\sgn(\Lambda(u_0u_1\cdots u_{m-1},d_i))$
								\item $=\sgn(\Lambda(\last_A,d_i))$
								\item $\ne 0$.
							\end{enumerate}
						\end{enumerate}
						\item If $\sgn(\Lambda(u_j,c_i))=\sgn(\Lambda(u_j,d_i))\for j\in[0:m]$, then do the following:
						\begin{enumerate}
							\item Verify that $\sgn(\Lambda(\last_A,c_i))$
							\begin{enumerate}
								\item $=\sgn(\Lambda(u_0u_1\cdots u_{m-1},c_i))$
								\item $=\sgn(\Lambda(u_0,c_i))\sgn(\Lambda(u_1,c_i))\cdots\sgn(\Lambda(u_{m-1},c_i))$
								\item $=\sgn(\Lambda(u_0,d_i))\sgn(\Lambda(u_1,d_i))\cdots\sgn(\Lambda(u_{m-1},d_i))$
								\item $=\sgn(\Lambda(u_0u_1\cdots u_{m-1},d_i))$
								\item $=\sgn(\Lambda(\last_A,d_i))$.
							\end{enumerate}
							\item \textbf{Therefore verify that $1=-1$.}
							\item \textbf{Abort procedure.}
						\end{enumerate}
						\item Otherwise do the following:
						\begin{enumerate}
							\item \textbf{Let $k_i$ be the least integer such that $0\ne\sgn(\Lambda(u_{k_i},c_i))=-\sgn(\Lambda(u_{k_i},d_i))$.}
						\end{enumerate}
					\end{enumerate}
					\item \textbf{Yield $\langle k\rangle$.}
				\end{enumerate}
		\procedure{4.67}
			\objective
				Choose a symmetric $m\times m$ rational matrix, $A$. Execute \procedurehr{4.25} on $A$ and let the tuple $\langle,,u,\rangle$ receive the result. Execute \procedurehr{2.18} on $A$ and let $k$ receive. Let $t=\deg(\last_A)$. Let $n_j=\sum_i^{[0:t]} [k_i=j]$ for $j$ in $[0:m]$. The objective of the following instructions is to either show that $0<0$, or to show that $n_i=\deg(u_i)$ for $i$ in $[0:m]$.
			\implementation
				\begin{enumerate}
					\item Verify that $\sum_j^{[0:m]} n_j=\sum_j^{[0:m]}\sum_i^{[0:t]} [k_i=j]=\sum_i^{[0:t]}\sum_j^{[0:m]} [k_i=j]=\sum_i^{[0:t]} 1=t$.
					\item If for any $i$ in $[0:m]$, $n_i>\deg(u_i)$, then do the following:
					\begin{enumerate}
						\item Execute \procedurehr{2.18} on the polynomial $u_i$ along with $\deg(u_i)+1$ of the distinct pairs $\langle c_l,d_l\rangle$ such that $k_l=i$.
						\item \textbf{Abort procedure.}
					\end{enumerate}
					\item Otherwise if for any $i$ in $[0:m]$, $n_i<\deg(u_i)$, then do the following:
					\begin{enumerate}
						\item Verify that $\sum_i^{[0:m]} n_j<\sum_i^{[0:m]} \deg(u_j)=t$.
						\item Therefore using (1) and (a), verify that $\sum_i^{[0:m]} n_j<\sum_i^{[0:m]} n_j$.
						\item \textbf{Abort procedure.}
					\end{enumerate}
					\item Otherwise, do the following:
					\begin{enumerate}
						\item \textbf{For all $i$ in $[0:m]$, verify that $n_i=\deg(u_i)$.}
					\end{enumerate}
				\end{enumerate}
		\procedure{4.72}
			\objective
				Choose a symmetric $m\times m$ rational matrix, $A$. Let $t=\deg(\last_A)$. Execute \procedurehr{4.66} on the matrix $A$ and let the tuple $\langle k\rangle$ receive the result. The objective of the following instructions is to either show that $0<0$ or to show that $\sum_i^{[0:t]}(m-k_i)=m$.
			\implementation
				\begin{enumerate}
					\item Execute \procedurehr{4.25} on the matrix $A$ and let the tuple $\langle,D,u,\rangle$.
					\item Using \procedurehr{4.67}, verify that $\sum_i^{[0:t]}(m-k_i)$
					\begin{enumerate}
						\item $=\sum_i^{[0:t]}\sum_j^{[0:m]} [k_i\le j]$
						\item $=\sum_j^{[0:m]}\sum_i^{[0:t]} [k_i\le j]$
						\item $=\sum_j^{[0:m]}\sum_i^{[0:t]} [k_i\le j]\sum_l^{[0:m]} [k_i=l]$
						\item $=\sum_j^{[0:m]}\sum_l^{[0:m]}\sum_i^{[0:t]} [k_i\le j][k_i=l]$
						\item $=\sum_j^{[0:m]}\sum_l^{[0:m]}\sum_i^{[0:t]} [l\le j][k_i=l]$
						\item $=\sum_j^{[0:m]}\sum_l^{[0:m]} [l\le j]\sum_i^{[0:t]} [k_i=l]$
						\item $=\sum_j^{[0:m]}\sum_l^{[0:m]} [l\le j]\deg u_l$
						\item $=\sum_j^{[0:m]}\sum_l^{[0:{j+1}]} \deg u_l$
						\item $=\sum_j^{[0:m]} \deg D_{j,j}$
						\item $=m$
					\end{enumerate}
				\end{enumerate}
	\clearpage
	\begingroup
		\renewcommand{\addcontentsline}[3]{}
		\begin{thebibliography}{9}
			\bibitem{linearalgebra} 
				Harold Edwards.
				\textit{Linear Algebra}. 
				Springer Science+Business Media, 1995.
			\bibitem{introductiontheoryofnumbers}
				Hugh L. Montgomery, Ivan Niven, Herbert S. Zuckerman.
				\textit{An Introduction to the Theory of Numbers}.
				John Wiley \& Sons, 1991.
			\bibitem{philosophicalgrammar}
				Ludwig Wittgenstein.
				\textit{Philosophical Grammar}.
				Edited by Rush Rhees.
				Translated by Anthony Kenny.
				Basil Blackwell, Oxford, 1974.
		\end{thebibliography}
	\endgroup
\end{document}
\grid
\grid
