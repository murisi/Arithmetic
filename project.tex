\documentclass[twocolumn]{article}
\usepackage{fullpage}
\usepackage{parskip}
\usepackage{amsmath}
\usepackage{amssymb}
\usepackage{datetime}
\usepackage[nottoc,numbib]{tocbibind}
\usepackage{hyperref}
\usepackage{enumitem}
\hypersetup{
	colorlinks = true,
	linktoc = all,
}

\setcounter{page}{0}
\setcounter{section}{-1}
\setlist[enumerate,2,3,4]{leftmargin=0.3cm}
\setlength{\columnsep}{0.7cm}

\DeclareMathOperator{\sgn}{sgn}
\DeclareMathOperator{\mat}{mat}
\DeclareMathOperator{\pol}{pol}
\DeclareMathOperator{\tr}{tr}
\DeclareMathOperator{\disc}{disc}
\DeclareMathOperator{\bdiag}{bdiag}
\DeclareMathOperator{\rcan}{rcan}
\DeclareMathOperator{\mon}{mon}
\DeclareMathOperator{\pows}{pows}
\DeclareMathOperator{\cols}{cols}
\DeclareMathOperator{\rows}{rows}
\DeclareMathOperator{\sel}{sel}
\DeclareMathOperator{\last}{last}
\let\div\relax
\DeclareMathOperator{\div}{div}
\let\mod\relax
\DeclareMathOperator{\mod}{mod}

\newcommand{\ul}[1]{\underline{#1}}
\newcommand{\notation}[1]{\subsubsection*{Notation #1}}
\newcommand{\procedure}[2][]{\subsection*{Procedure #2 \ifthenelse{\equal{#1}{}}{}{(#1)}}\label{sec:procedure #2}}
\newcommand{\objective}{\subsubsection*{Objective}}
\newcommand{\implementation}{\subsubsection*{Implementation}}
\newcommand{\procedurehr}[2][]{\hyperref[sec:procedure #2]{\ifthenelse{\equal{#1}{}}{procedure #2}{#1}}}

\begin{document}
	\title{Arithmetic: A Programmatic Approach}
	\author{Murisi Tarusenga}
	\date{\today{} \currenttime}
	\maketitle
	\section{Preamble}
		\paragraph{Abstract}
			What follows is a reformulation of the elementary parts of number theory and linear algebra in terms of a system procedures for achieving particular objectives, objectives like solving a particular system of linear equations. That these procedures have the potential to achieve their respective objectives is shown by their syntax in the same way that the syntax of the code fragment, "if $a=b$ and $b=c$, then verify that $a=c$", shows the fragment's potential to work on different integer inputs.
		\paragraph{Usage Guide}
			The task of understanding the following procedures should be the same as that of understanding any codebase. Hence domain specific knowledge is required, which in this case comprises integer, rational, formal polynomial, and matrix arithmetic as well as inequalities. Otherwise, running a debugger, that is, executing the following procedures step by step on some chosen input(s) and observing their control flows and sequences of program states should be equally helpful in gaining intuition for their workings.
	\tableofcontents
	\section{Integer Arithmetic}
		\procedure{1.00}
			\objective
				Choose an integer $a$ and a positive integer $b$. The objective of the following instructions is to construct integers $a\div b$ and $a\mod b$ such that $a=(a\div b)b+a\mod b$ and $0\le a\mod b<b$.
			\implementation
				\begin{enumerate}
					\item Let $n=0$.
					\item While $(n+1)b\le a$, do the following:
					\begin{enumerate}
						\item Let $n$ receive $n+1$.
						\item Verify that $nb\le a$.
					\end{enumerate}
					\item While $nb>a$, do the following:
					\begin{enumerate}
						\item Let $n$ receive $n-1$.
						\item Verify that $(n+1)b>a$.
					\end{enumerate}
					\item Therefore verify that $nb\le a$.
					\item Also verify that $(n+1)b>a$.
					\item Let $a\div b=n$.
					\item Let $a\mod b=a-nb$.
					\item \textbf{Now verify that $b>a-nb=a\mod b\ge 0$.}
					\item \textbf{Also verify that $a=bn+a-nb=(a\div b)b+a\mod b$.}
					\item \textbf{Yield $\langle a\div b,a\mod b\rangle$.}
				\end{enumerate}
		\notation{1.00}
			Let us use the notation $a\div b$ as a shorthand for "the first part of the pair yielded by executing \procedurehr{1.00} on $\langle a,b\rangle$.
		\notation{1.01}
			Let us use the notation $a\mod b$ as a shorthand for "the second part of the pair yielded by executing \procedurehr{1.00} on $\langle a,b\rangle$.
		\notation{1.02}
			Let us use the notation $a\equiv b(\mod c)$ as a shorthand for "$a\mod c=b\mod c$".
		\procedure{1.01}
			\objective
				Choose four integers $a,b,c,d$ and a positive integer $e$ in such a way that $a\equiv c(\mod e)$ and $b\equiv d(\mod e)$. The objective of the following instructions is to show that $a+b\equiv c+d(\mod e)$.
			\implementation
				\begin{enumerate}
					\item Verify that $a+b$
					\begin{enumerate}
						\item $\equiv(a\div e)e+(a\mod e)+(b\div e)e+(b\mod e)$
						\item $\equiv(a\mod e)+(b\mod e)$
						\item $\equiv(c\mod e)+(d\mod e)$
						\item $\equiv(c\div e)e+(c\mod e)+(d\div e)e+(d\mod e)$
						\item \textbf{$\equiv c+d(\mod e)$.}
					\end{enumerate}
				\end{enumerate}
		\procedure{1.02}
			\objective
				Choose four integers $a,b,c,d$ and a positive integer $e$ in such a way that $a\equiv c(\mod e)$ and $b\equiv d(\mod e)$. The objective of the following instructions is to show that $ab\equiv cd(\mod e)$.
			\implementation
				\begin{enumerate}
					\item Verify that $ab$
					\begin{enumerate}
						\item $\equiv((a\div e)e+(a\mod e))((b\div e)e+(b\mod e))$
						\item $\equiv(a\div e)(b\div e)e^2+(a\div e)(b\mod e)e+(a\mod e)(b\div e)e+(a\mod e)(b\mod e)$
						\item $\equiv(a\mod e)(b\mod e)$
						\item $\equiv(c\mod e)(d\mod e)$
						\item $\equiv(c\div e)(d\div e)e^2+(c\div e)(d\mod e)e+(c\mod e)(d\div e)e+(c\mod e)(d\mod e)$
						\item \textbf{$\equiv cd(\mod e)$.}
					\end{enumerate}
				\end{enumerate}
		\procedure{1.03}
			\objective
				Choose an integer $a$ and two positive integers $b,c$. The objective of the following instructions is to show that $(a\mod bc)\mod b=a\mod b$.
			\implementation
				\begin{enumerate}
					\item \textbf{Verify that $(a\mod bc)\mod b=(a-(a\div bc)bc)\mod b=a\mod b$.}
				\end{enumerate}
		\procedure{1.04}
			\objective
				Choose a positive integer $a$ and four integers $b_1,b_0,c_1,c_0$ such that $0\le b_0<a$, $0\le c_0<a$, and $b_1a+b_0=c_1a+c_0$. The objective of the following instructions is to show that $b_1=c_1$ and $b_0=c_0$.
			\implementation
				\begin{enumerate}
					\item \textbf{Verify that $b_0=b_0\mod a=(b_1a+b_0)\mod a=(c_1a+c_0)\mod a=c_0\mod a=c_0$.}
					\item Therefore verify that $b_1a=c_1a$.
					\item \textbf{Therefore verify that $b_1=c_1$.}
				\end{enumerate}
		\procedure{1.05}
			\objective
				Choose an integer $a$ and two positive integers $b,c$. The objective of the following instructions is to show that $ca\mod cb=c(a\mod b)$ and that $ca\div cb=a\div b$.
			\implementation
				\begin{enumerate}
					\item Verify that $bc(a\div b)+c(a\mod b)=c(b(a\div b)+a\mod b)=ca=cb(ca\div cb)+ca\mod cb$.
					\item Now verify that $0\le a\mod b<b$.
					\item Therefore verify that $0\le c(a\mod b)<cb$.
					\item Now verify that $0\le ca\mod cb<cb$.
					\item Execute \procedurehr{1.04} on $\langle bc,a\div b,c(a\mod b),ca\div cb,ca\mod cb\rangle$.
					\item \textbf{Therefore verify that $c(a\mod b)=ca\mod cb$.}
					\item \textbf{Also verify that $a\div b=ca\div cb$.}
				\end{enumerate}
		\procedure{1.06}
			\objective
				Choose two integers $a,b$ and a positive integer $c$ such that $a\mod c+b\mod c<c$. The objective of the following instructions is to show that $a\div c+b\div c=(a+b)\div c$ and $a\mod c+b\mod c=(a+b)\mod c$.
			\implementation
				\begin{enumerate}
					\item Verify that $a=c(a\div c)+a\mod c$.
					\item Verify that $b=c(b\div c)+b\mod c$.
					\item Therefore verify that $a+b=c(a\div c+b\div c)+(a\mod c+b\mod c)$.
					\item Verify that $0\le a\mod c+b\mod c<c$.
					\item Also verify that $a+b=((a+b)\div c)c+(a+b)\mod c$.
					\item Verify that $0\le(a+b)\mod c<c$.
					\item Execute \procedurehr{1.04} on $\langle c,a\div c+b\div c,a\mod c+b\mod c,(a+b)\div c,(a+b)\mod c\rangle$.
					\item \textbf{Therefore verify that $a\div c+b\div c=(a+b)\div c$.}
					\item \textbf{Also verify that $a\mod c+b\mod c=(a+b)\mod c$.}
				\end{enumerate}
		\procedure{1.07}
			\objective
				Choose two integers $a,b$ and a positive integer $c$ such that $a\mod c+b\mod c\ge c$. The objective of the following instructions is to show that $1+a\div c+b\div c=(a+b)\div c$ and $a\mod c+b\mod c-c=(a+b)\mod c$.
			\implementation
				\begin{enumerate}
					\item Verify that $a=c(a\div c)+a\mod c$.
					\item Verify that $b=c(b\div c)+b\mod c$.
					\item Therefore verify that $a+b=c(a\div c+b\div c)+a\mod c+b\mod c=c(1+a\div c+b\div c)+(a\mod c+b\mod c-c)$.
					\item Verify that $c\le a\mod c+b\mod c<2c$.
					\item Therefore verify that $0\le a\mod c+b\mod c-c<c$.
					\item Also verify that $a+b=c((a+b)\div c)+(a+b)\mod c$.
					\item Verify that $0\le (a+b)\mod c<c$.
					\item Execute \procedurehr{1.04} on $\langle c,1+a\div c+b\div c,a\mod c+b\mod c-c,(a+b)\div c,(a+b)\mod c\rangle$.
					\item \textbf{Therefore verify that $1+a\div c+b\div c=(a+b)\div c$.}
					\item \textbf{Therefore verify that $a\mod c+b\mod c-c=(a+b)\mod c$.}
				\end{enumerate}
		\procedure{1.08}
			\objective
				Choose an integer $a$ and two positive integers $b,c$. The objective of the following instructions is to show that $a\div bc=(a\div b)\div c$ and $a\mod bc=((a\div b)\mod c)b+a\mod b$.
			\implementation
				\begin{enumerate}
					\item Verify that $a=(a\div b)b+a\mod b$.
					\item Verify that $a\div b=((a\div b)\div c)c+(a\div b)\mod c$.
					\item Therefore verify that $a=(((a\div b)\div c)c+(a\div b)\mod c)b+a\mod b=((a\div b)\div c)bc+((a\div b)\mod c)b+a\mod b$.
					\item Verify that $0\le(a\div b)\mod c\le c-1$.
					\item Therefore verify that $0\le((a\div b)\mod c)b\le cb-b$.
					\item Verify that $0\le a\mod b<b$.
					\item Therefore verify that $0\le ((a\div b)\mod c)b+a\mod b<cb$.
					\item Now verify that $a=(a\div bc)bc+a\mod bc$.
					\item Verify that $0\le a\mod bc<bc$.
					\item Execute \procedurehr{1.04} on $\langle bc,(a\div b)\div c,((a\div b)\mod c)b+a\mod b,a\div bc,a\mod bc\rangle$.
					\item \textbf{Therefore verify that $(a\div b)\div c=a\div bc$.}
					\item \textbf{Also verify that $((a\div b)\mod c)b+a\mod b=a\mod bc$.}
				\end{enumerate}
		\procedure{1.09}
			\objective
				Choose an integer $a$ and a non-negative integer $b$. The objective of the following instructions is to consruct integers $c,d,e,f,g$ such that $a=cd$, $b=ce$, $fa+gb=c$, and if $b=0$, then $c=\lvert a\rvert$, otherwise $0<c\le b$.
			\implementation
				\begin{enumerate}
					\item If $b=0$, then do the following:
					\begin{enumerate}
						\item \textbf{Verify that $a=\sgn(a)\lvert a\rvert$.}
						\item \textbf{Verify that $b=0\lvert a\rvert$.}
						\item \textbf{Verify that $\lvert a\rvert=\sgn(a)a+0b$.}
						\item \textbf{Yield $\langle\lvert a\rvert,\sgn(a),0,\sgn(a),0\rangle$.}
					\end{enumerate}
					\item Otherwise do the following:
					\begin{enumerate}
						\item Verify that $0\le a\mod b<b$.
						\item Execute \procedurehr{1.09} on $\langle b,a\mod b\rangle$ and let $\langle c,d,e,f,g\rangle$ receive.
						\item \textbf{Now verify that $b=cd$.}
						\item Also verify that $a\mod b=ce$.
						\item \textbf{Therefore verify that $a=(a\div b)b+(a\mod b)=c(d(a\div b)+e)$.}
						\item \textbf{Also verify that $(f-g(a\div b))b+ga=fb+g(a-(a\div b)b)=fb+g(a\mod b)=c$.}
						\item If $a\mod b=0$, then do the following:
						\begin{enumerate}
							\item \textbf{Using (O), (2) and (b), verify that $0<b=c\le b$.}
						\end{enumerate}
						\item Otherwise do the following:
						\begin{enumerate}
							\item \textbf{Using (b), verify that $0<c\le a\mod b<b$.}
						\end{enumerate}
						\item \textbf{Therefore yield $\langle c,d(a\div b)+e,d,g,f-g(a\div b)\rangle$.}
					\end{enumerate}
				\end{enumerate}
		\notation{1.03}
			Let us use the notation $(a,b)$ as a shorthand for "the first part of the quintuple yielded by executing \procedurehr{1.09} on the pair $\langle a,b\rangle$".
		\procedure{1.10}
			\objective
				Choose an integer $a$ and a positive integer $b$. Let $1\le c\le b$ be the largest integer such that $a\mod c=0$ and $b\mod c=0$. The objective of the following instructions is to either show that $0\ne 0$ or $(a,b)=c$.
			\implementation
				\begin{enumerate}
					\item Execute \procedurehr{1.09} on $\langle a,b\rangle$ and let $\langle d,e,f,g,h\rangle$ receive.
					\item Verify that $0<d\le b$.
					\item If $d>c$, then do the following:
					\begin{enumerate}
						\item Using (O), verify that $a\mod d\ne 0$ or $b\mod d\ne 0$.
						\item If $a\mod d\ne 0$, then do the following:
						\begin{enumerate}
							\item Using (1), verify that $a=ed$.
							\item Therefore verify that $a\mod d=0$.
							\item \textbf{Therefore using (3b) and (3bii), verify that $0\ne 0$.}
							\item \textbf{Abort procedure.}
						\end{enumerate}
						\item Otherwise if $b\mod d\ne 0$, then do the following:
						\begin{enumerate}
							\item Using (1), verify that $b=fd$.
							\item Therefore verify that $b\mod d=0$.
							\item \textbf{Therefore using (3c) and (3cii), verify that $0\ne 0$.}
							\item \textbf{Abort procedure.}
						\end{enumerate}
					\end{enumerate}
					\item Otherwise if $d<c$, then do the following:
					\begin{enumerate}
						\item Verify that $ga+hb=d$.
						\item Therefore verify that $0\equiv gc(a\div c)+hc(b\div c)=g(c(a\div c)+a\mod c)+h(c(b\div c)+b\mod c)=ga+hb=d\not\equiv 0(\mod c)$.
						\item \textbf{Therefore verify that $0\ne 0$.}
						\item \textbf{Abort procedure.}
					\end{enumerate}
					\item \textbf{Otherwise verify that $(a,b)=d=c$.}
				\end{enumerate}
		\procedure{1.11}
			\objective
				Choose integers $a,c,d,j$ and a non-negative integer $b$. Execute \procedurehr{1.09} on $\langle a,b\rangle$ and let $\langle e,f,g,h,i\rangle$ receive. The objective of the following instructions is to show that $ca+db=(c+gj)a+(d-fj)b$.
			\implementation
				\begin{enumerate}
					\item \textbf{Verify that $(c+gj)a+(d-fj)b=ca+db+gja-fjb=ca+db+gjef-fjeg=ca+db$.}
				\end{enumerate}
		\procedure{1.12}
			\objective
				Choose integers $a,c,d$ and a non-negative integer $b$ such that $ca+db=(a,b)$. Execute \procedurehr{1.09} on $\langle a,b\rangle$ and let $\langle e,f,g,h,i\rangle$ receive. The objective of the following instructions is to construct a $j$ such that $c=h+gj$ and $d=i-fj$.
			\implementation
				\begin{enumerate}
					\item Verify that $cef+deg=ca+db=(a,b)=e$.
					\item Therefore verify that $cf+dg=1$.
					\item Now verify that $hef+ieg=ha+ib=e$.
					\item Therefore verify that $hf+ig=1$.
					\item Let $j=ci-hd$.
					\item Now verify that $cf=1-dg$.
					\item Therefore verify that $c-cig=c(1-ig)=chf=h(1-dg)=h-hdg$.
					\item \textbf{Therefore verify that $c=h+cig-hdg=h+g(ci-hd)=h+gj$.}
					\item Now verify that $dg=1-cf$.
					\item Therefore verify that $d-dhf=d(1-hf)=dig=i(1-cf)=i-icf$.
					\item \textbf{Therefore verify that $d=i-icf+dhf=i-f(ic-dh)=i-fj$.}
					\item \textbf{Yield $\langle j\rangle$.}
				\end{enumerate}
		\procedure{1.13}
			\objective
				Choose an integer $a$ and a positive integer $b$ such that $0<(a,b)<b$. The objective of the following instructions is to show that $0\ne 0$ or $a\mod b\ne 0$.
			\implementation
				\begin{enumerate}
					\item If $a\mod b=0$, then do the following:
					\begin{enumerate}
						\item Using (1), verify that $af\equiv 0f\equiv 0(\mod b)$.
						\item Execute \procedurehr{1.09} on $\langle a,b\rangle$ and let $\langle c,d,e,f,g\rangle$ receive.
						\item Verify that $0<(a,b)=c=fa+gb<b$.
						\item Therefore verify $fa\equiv(a,b)\not\equiv 0(\mod b)$.
						\item Therefore using (1a) and (1d), verify that $0\ne 0$.
						\item \textbf{Abort ptocedure.}
					\end{enumerate}
					\item \textbf{Otherwise verify that $a\mod b\ne 0$.}
				\end{enumerate}
		\procedure{1.14}
			\objective
				Choose five integers $a,d,e,f,g$ and two non-negative integers $b,c$ such that $a=cd$, $b=ce$, and $fa+gb=c$. The objective of the following instructions is to show that $0<0$ or $(a,b)=c$.
			\implementation
				\begin{enumerate}
					\item Execute \procedurehr{1.09} on $\langle a,b\rangle$ and let $\langle u,v,x,y,z\rangle$ receive.
					\item Verify that $u\ge 0$.
					\item Verify that $a=uv$.
					\item Verify that $b=xu$.
					\item Therefore verify that $c=fa+gb=(fv+gx)u$.
					\item If $u=0$, then do the following:
					\begin{enumerate}
						\item \textbf{Verify that $c=(fv+gx)u=0=u=(a,b)$.}
						\item \textbf{Yield.}
					\end{enumerate}
					\item Also using (1) and (O), verify that $u=ya+zb=(yd+ze)c$.
					\item If $c=0$, then do the following:
					\begin{enumerate}
						\item \textbf{Verify that $(a,b)=u=(yd+ze)c=0=c$.}
						\item \textbf{Yield.}
					\end{enumerate}
					\item Verify that $c>0$.
					\item Now verify that $c=(fv+gx)u=(fv+gx)(yd+ze)c$.
					\item Therefore verify that $(fv+gx)(yd+ze)=1$.
					\item Therefore verify that $fv+gx=yd+ze=\pm 1$.
					\item If $fv+gx=yd+ze=-1$, then do the following:
					\begin{enumerate}
						\item Using (7) and (9), verify that $u=(yd+ze)c=-c<0$.
						\item \textbf{Therefore using (2) and (13a), verify that $0\le u<0$.}
						\item \textbf{Abort procedure.}
					\end{enumerate}
					\item Otherwise, do the following:
					\begin{enumerate}
						\item Verify that $fv+gx=yd+ze=1$.
						\item \textbf{Therefore verify that $c=(fv+gx)u=u=(a,b)$.}
					\end{enumerate}
				\end{enumerate}
		\procedure{1.15}
			\objective
				Choose an integer $a$ and a non-negative integer $b$. The objective of the following instructions is to show that $0<0$ or $(a,b)=(-a,b)$.
			\implementation
				\begin{enumerate}
					\item Execute \procedurehr{1.09} on $\langle a,b\rangle$ and let $\langle c,d,e,f,g\rangle$ receive.
					\item Verify that $a=dc$.
					\item Therefore verify that $-a=(-d)c$.
					\item Verify that $b=ec$.
					\item Verify that $fa+gb=c$.
					\item Therefore verify that $(-f)(-a)+gb=c$.
					\item Execute \procedurehr{1.14} on $\langle -a,b,c,-d,e,-f,g\rangle$.
					\item \textbf{Therefore verify that $(-a,b)=c=(a,b)$.}
				\end{enumerate}
		\procedure{1.16}
			\objective
				Choose two non-negative integers $a,b$. The objective of the following instructions is to show that $0<0$ or $(a,b)=(b,a)$.
			\implementation
				\begin{enumerate}
					\item Execute \procedurehr{1.09} on $\langle a,b\rangle$ and let $\langle c,d,e,f,g\rangle$ receive.
					\item Verify that $b=ec$.
					\item Verify that $a=dc$.
					\item Verify that $gb+fa=c$.
					\item Execute \procedurehr{1.14} on $\langle b,a,c,e,d,g,f\rangle$.
					\item \textbf{Therefore verify that $(b,a)=c=(a,b)$.}
				\end{enumerate}
		\procedure{1.17}
			\objective
				Choose two integers $a,b$ and a positive integer $c$ such that $a\equiv b(\mod c)$. The objective of the following instructions is to show that $0<0$ or $(a,c)=(b,c)$.
			\implementation
				\begin{enumerate}
					\item Execute \procedurehr{1.09} on $\langle a,c\rangle$ and let $\langle d,e,f,g,h\rangle$ receive.
					\item Verify that $a=ed$.
					\item Verify that $c=fd$.
					\item Let $j=b\div c-a\div c$.
					\item Therefore verify that $b=a+jc=ed+jfd=(e+jf)d$.
					\item Verify that $gb+(h-gj)c=g(a+jc)+(h-gj)c=ga+hc=d$.
					\item Now execute \procedurehr{1.14} on $\langle b,c,d,e+jf,f,g,h-gj\rangle$.
					\item \textbf{Therefore verify that $(b,c)=d=(a,c)$.}
				\end{enumerate}
		\procedure{1.18}
			\objective
				Choose an integer $a$ and two non-negative integers $b,c$. The objective of the following instructions is to show that either $0<0$ or $(ca,cb)=c(a,b)$.
			\implementation
				\begin{enumerate}
					\item Execute \procedurehr{1.09} on $\langle a,b\rangle$ and let $\langle d,e,f,g,h\rangle$ receive.
					\item Verify that $a=ed$.
					\item Therefore verify that $ca=e(cd)$.
					\item Verify that $b=df$.
					\item Therefore verify that $cb=f(cd)$.
					\item Verify that $ga+hb=d$.
					\item Therefore verify that $g(ca)+h(cb)=cd$.
					\item Now execute \procedurehr{1.14} on $\langle ca,cb,cd,e,f,g,h\rangle$.
					\item Therefore verify that $(ca,cb)=cd=c(a,b)$.
				\end{enumerate}
		\procedure{1.19}
			\objective
				Choose an integer $a$ and two non-negative integers $b,c$. The objective of the following instructions is to show that either $0<0$ or $(a,(b,c))=((a,b),c)$.
			\implementation
				\begin{enumerate}
					\item Execute \procedurehr{1.09} on $\langle a,b\rangle$ and let $\langle d_0,e_0,f_0,g_0,h_0\rangle$ receive.
					\item Execute \procedurehr{1.09} on $\langle b,c\rangle$ and let $\langle d_1,e_1,f_1,g_1,h_1\rangle$ receive.
					\item Execute \procedurehr{1.09} on $\langle (a,b),c\rangle$ and let $\langle d_2,e_2,f_2,g_2,h_2\rangle$ receive.
					\item Verify that $a=d_0e_0=e_0(a,b)=e_0d_2e_2=e_0e_2((a,b),c)$.
					\item Verify that $(b,c)$
					\begin{enumerate}
						\item $=g_1b+h_1c$
						\item $=g_1d_0f_0+h_1d_2f_2$
						\item $=g_1f_0(a,b)+h_1f_2((a,b),c)$
						\item $=g_1f_0d_2e_2+h_1f_2((a,b),c)$
						\item $=g_1f_0e_2((a,b),c)+h_1f_2((a,b),c)$
						\item $=(g_1f_0e_2+h_1f_2)((a,b),c)$.
					\end{enumerate}
					\item Verify that $((a,b),c)$
					\begin{enumerate}
						\item $=d_2$
						\item $=g_2(a,b)+h_2c$
						\item $=g_2d_0+h_2d_1f_1$
						\item $=g_2(g_0a+h_0b)+h_2f_1(b,c)$
						\item $=g_2g_0a+g_2h_0d_1e_1+h_2f_1(b,c)$
						\item $=g_2g_0a+g_2h_0e_1(b,c)+h_2f_1(b,c)$
						\item $=g_2g_0a+(g_2h_0e_1+h_2f_1)(b,c)$.
					\end{enumerate}
					\item Execute \procedurehr{1.14} on $\langle a,(b,c),((a,b),c),e_0e_2,g_1f_0e_2+h_1f_2,g_2g_0,g_2h_0e_1+h_2f_1\rangle$.
					\item \textbf{Therefore verify that $((a,b),c)=(a,(b,c))$.}
				\end{enumerate}
		\notation{1.04}
			Let us use the notation $(a_0,a_1,\cdots,a_{n-1})$ as a shorthand for "either $((a_0),(a_1,a_2,\cdots,a_{n-1}))$ or $((a_0,a_1),(a_2,a_3,\cdots,a_{n-1}))$ or $\cdots$ or $((a_0,a_1,\cdots,a_{n-2}),(a_{n-1}))$".
		\procedure{1.20}
			\objective
				Choose two integers $a,b$ and a non-negative integer $c$ such that $(a,c)=1$ and $(b,c)=1$. The objective of the following instructions is to show that either $0<0$ or $(ab,c)=1$.
			\implementation
				\begin{enumerate}
					\item Execute \procedurehr{1.09} on $\langle a,c\rangle$ and let $\langle d,e,f,g,h\rangle$ receive.
					\item Verify that $ga+hc=d=(a,c)=1$.
					\item Execute \procedurehr{1.09} on $\langle b,c\rangle$ and let $\langle t,u,v,w,x\rangle$ receive.
					\item Verify that $wb+xc=t=(b,c)=1$.
					\item Therefore verify that $(gw)(ab)+(gax+wbh+hxc)c=(ga+hc)(wb+xc)=1$.
					\item Now execute \procedurehr{1.14} on $\langle ab,c,1,ab,c,gw,gax+wbh+hxc\rangle$.
					\item \textbf{Therefore verify that $(ab,c)=1$.}
				\end{enumerate}
		\procedure{1.21}
			\objective
				Choose an integer $a$ and two non-negative integers $b,c$ such that $(a,bc)=1$. The objective of the following instructions is to show that either $0<0$ or $(a,b)=1$.
			\implementation
				\begin{enumerate}
					\item Execute \procedurehr{1.09} on $\langle a,bc\rangle$ and let $\langle d,e,f,g,h\rangle$ receive.
					\item Verify that $d=(a,bc)=1$.
					\item Verify that $ga+(hc)b=ga+h(bc)=d=1$.
					\item Now execute \procedurehr{1.14} on $\langle a,b,1,a,b,g,hc\rangle$.
					\item \textbf{Therefore verify that $(a,b)=1$.}
				\end{enumerate}
		\notation{1.05}
			Let us use the notation "$a$ is prime" as a shorthand for "$a>1$ and $a\mod k\ne 0$ for $1<k<a$".
		\procedure{1.22}
			\objective
				Choose an integer $a$ and a prime $b$ such that $a\mod b\ne 0$. The objective of the following instructions is to show that either $0\ne 0$ or $(a,b)=1$.
			\implementation
				\begin{enumerate}
					\item Execute \procedurehr{1.09} on $\langle a,b\rangle$ and let $\langle c,d,e,f,g\rangle$ receive.
					\item Verify that $0<c\le b$.
					\item If $c=b$, then do the following:
					\begin{enumerate}
						\item Verify that $a=cd=bd$.
						\item Therefore verify that $a\mod b=0$.
						\item \textbf{Therefore using (O) and (3b), verify that $0\ne 0$.}
						\item \textbf{Abort procedure.}
					\end{enumerate}
					\item Otherwise if $1<c<b$, then do the following:
					\begin{enumerate}
						\item Verify that $b=ce$.
						\item Therefore verify that $b\mod c=0$.
						\item \textbf{Therefore using (O) and (4b), verify that $0\ne 0$.}
						\item \textbf{Abort procedure.}
					\end{enumerate}
					\item Otherwise, do the following:
					\begin{enumerate}
						\item \textbf{Verify that $(a,b)=c=1$.}
					\end{enumerate}
				\end{enumerate}
		\procedure{1.23}
			\objective
				Choose two integers $a,b$ and a prime $c$ such that $a\mod c\ne 0$ and $b\mod c\ne 0$. The objective of the following instructions is to show that either $0\ne 0$ or $ab\mod c\ne 0$.
			\implementation
				\begin{enumerate}
					\item Execute \procedurehr{1.22} on $\langle a,c\rangle$.
					\item Verify that $(a,c)=1$.
					\item Execute \procedurehr{1.22} on $\langle b,c\rangle$.
					\item Verify that $(b,c)=1$.
					\item Execute \procedurehr{1.20} on $\langle a,b,c\rangle$.
					\item Now verify that $0<(ab,c)=1<c$.
					\item Execute \procedurehr{1.13} on $\langle ab,c\rangle$.
					\item \textbf{Now verify that $ab\mod c\ne 0$.}
				\end{enumerate}
		\notation{1.06}
			Let us use the notation $\lvert A\rvert$ as a shorthand for "the number of items in the list $A$".
		\notation{1.07}
			Let us use the notation $\prod_{r=a}^b c_r$ as a shorthand for "$1$ if $a=b$, otherwise $c_a\prod_{r=a+1}^b c_r$".
		\notation{1.07}
			Let us use the notation $a_*$ as a shorthand for "$\prod_{i=0}^{\lvert a\rvert}a_i$".
		\notation{1.08}
			Let us use the notation $A^\frown B$ as a shorthand for "the list formed by concatenating $B$ onto $A$".
		\procedure{1.24}
			\objective
				Choose a positive integer $a$. The objective of the following instructions is to construct a list of prime numbers $b$ such that $a=b_*$.
			\implementation
				\begin{enumerate}
					\item If $a=1$, then do the following:
					\begin{enumerate}
						\item Verify that $a=1=\langle\rangle_*$.
						\item Therefore yield $\langle\rangle$.
					\end{enumerate}
					\item Otherwsie, do the following:
					\begin{enumerate}
						\item Verify that $a>1$.
						\item For $c=2$ up to $c=a-1$, do the following:
						\begin{enumerate}
							\item If $a\mod c=0$, then do the following:
							\begin{enumerate}
								\item Verify that $a=(a\div c)c$.
								\item Therefore verify that $1<a\div c<a$.
								\item Execute \procedurehr{1.24} on $\langle a\div c\rangle$ and let $\langle d\rangle$ receive.
								\item Using (B) and (C), verify that $\lvert d\rvert>0$.
								\item Verify that every element of $d$ is prime.
								\item Verify that $a\div c=d_*$.
								\item Execute \procedurehr{1.24} on $\langle c\rangle$ and let $\langle e\rangle$ receive.
								\item Using (b) and (G), verify that $\lvert e\rvert>0$.
								\item Verify that every element of $e$ is prime.
								\item Verify that $c=e_*$.
								\item \textbf{Therefore verify that $\lvert d^\frown e\rvert>0$.}
								\item \textbf{Also verify that every element of $d^\frown e$ is prime.}
								\item \textbf{Also verify that $a=(a\div c)c=d_*e_*=(d^\frown e)_*$.}
								\item \textbf{Yield $\langle d^\frown e\rangle$.}
							\end{enumerate}
						\end{enumerate}
						\item Otherwise do the following:
						\begin{enumerate}
							\item \textbf{Verify that $a$ is prime.}
							\item \textbf{Yield $\langle a\rangle$.}
						\end{enumerate}
					\end{enumerate}
				\end{enumerate}
		\procedure{1.25}
			\objective
				Choose a prime $a$ and a list of primes $b$ such that $b_*\equiv 0(\mod a)$. The objective of the following instructions is to either show that $0=1$ or to construct a $k$ such that $a=b_k$.
			\implementation
				\begin{enumerate}
					\item Using (O), verify that $a>1$.
					\item If $\lvert b\rvert=0$, then do the following:
					\begin{enumerate}
						\item Verify that $1=b_*\equiv 0(\mod a)$.
						\item \textbf{Therefore using (1) and (a), verify that $0=1$.}
						\item \textbf{Abort procedure.}
					\end{enumerate}
					\item Otherwise if $0\not\in b\mod a$, then do the following:
					\begin{enumerate}
						\item Using \procedurehr{1.23}, verify that $b_*\not\equiv 0 (\mod a)$.
						\item \textbf{Therefore using (O) and (a), verify that $0\ne 0$.}
						\item \textbf{Abort procedure.}
					\end{enumerate}
					\item Otherwise do the following:
					\begin{enumerate}
						\item Let $k$ be such that $b_k\mod a=0$.
						\item Verify that $b_k=(b_k\div a)a$.
						\item Verify that $b_k\div a\ge 1$.
						\item If $b_k\div a>1$, then do the following:
						\begin{enumerate}
							\item Using (1),(b), and (d), verify that $1<a<b_k$.
							\item Now verify that $b_k\mod a=0$.
							\item \textbf{Hence using (O) and (ii), verify that $0\ne b_k\mod a=0$.}
							\item \textbf{Abort procedure.}
						\end{enumerate}
						\item Otherwise do the following:
						\begin{enumerate}
							\item Verify that $b_k\div a=1$.
							\item \textbf{Therefore verify that $b_k=a$.}
							\item \textbf{Yield $\langle k\rangle$.}
						\end{enumerate}
					\end{enumerate}
				\end{enumerate}
		\notation{1.09}
			Let us use the notation $[a:b]$ as a shorthand for "if $b>a$, the list $\langle a,a+1,\cdots,b-1\rangle$, if $b=a$, the list $\langle\rangle$, if $b<a$, the list $\langle a-1,a-2,\cdots,b\rangle$".
		\procedure{1.26}
			\objective
				Choose two lists of primes $a,b$ such that $a_*=b_*$. The objective of the following instructions is to show that either $1>1$ or $a$ is included in $b$.
			\implementation
				\begin{enumerate}
					\item If $\lvert a\rvert=0$, then do the following:
					\begin{enumerate}
						\item \textbf{Verify that $a$ is included in $b$.}
					\end{enumerate}
					\item Otherwise, do the following:
					\begin{enumerate}
						\item Verify that $\lvert a\rvert>0$.
						\item Verify that $b_*\equiv a_*\equiv 0 (\mod a_0)$.
						\item Execute \procedurehr{1.25} on $\langle a_0, b\rangle$ and let $\langle k\rangle$ receive.
						\item Therefore verify that $b_k=a_0$.
						\item Now verify $(a_{[1:\lvert a\rvert]})_*=(b_{[0:k]^\frown[k+1:\lvert b\rvert]})_*$.
						\item Now execute \procedurehr{1.26} on $\langle a_{[1:\lvert a\rvert]},b_{[0:k]^\frown[k+1:\lvert b\rvert]}\rangle$.
						\item Now verify that $a_{[1:\lvert a\rvert]}$ is included in $b_{[0:k]^\frown[k+1:\lvert b\rvert]}\rangle$.
						\item \textbf{Therefore verify that $a$ is included in $b$.}
					\end{enumerate}
				\end{enumerate}
		\procedure{1.27}
			\objective
				Choose two lists of primes $a,b$ such that $a_*=b_*$. The objective of the following instructions is to show that either $1>1$ or $a$ is a rearrangement of $b$.
			\implementation
				\begin{enumerate}
					\item Execute \procedurehr{1.26} on $\langle a,b\rangle$.
					\item Verify that $a$ is included in $b$.
					\item Execute \procedurehr{1.26} on $\langle b,a\rangle$.
					\item Verify that $b$ is included in $a$.
					\item \textbf{Therefore verify that $a$ is a rearrangement of $b$.}
				\end{enumerate}
		\procedure{1.28}
			\objective
				Choose a positive integer $a$. The objective of the following instructions is to either show that $0=1$ or to construct a prime $b$ such that $b>a$ and $[a+1:b]$ does not contain a prime.
			\implementation
				\begin{enumerate}
					\item Verify that $a!+1>1$.
					\item Execute \procedurehr{1.24} on $\langle a!+1\rangle$ and let $\langle d\rangle$ receive.
					\item Therefore using (1) and (2), verify that $\lvert d\rvert>0$.
					\item Now verify that $(a!+1)\mod d_0=0$.
					\item For $e=2$ up to $e=a$, do the following:
					\begin{enumerate}
						\item Verify that $a!+1\equiv 1 (\mod e)$.
						\item If $e=d_0$, then do the following:
						\begin{enumerate}
							\item Using (4) and (a), verify that $0\equiv a!+1\equiv 1(\mod e=d_0)$.
							\item \textbf{Therefore verify that $0=1$.}
							\item \textbf{Abort procedure.}
						\end{enumerate}
					\end{enumerate}
					\item Otherwise do the following:
					\begin{enumerate}
						\item \textbf{Using (2), verify that $d_0$ is prime.}
						\item Using (a), verify that $d_0>1$.
						\item \textbf{Using (a) and (5), verify that $d_0>a$.}
						\item \textbf{Let $b$ be the least prime between $a+1$ and $d_0$.}
						\item \textbf{Yield $\langle b\rangle$.}
					\end{enumerate}
				\end{enumerate}
		\procedure{1.29}
			\objective
				Choose a positive integer $a$. The objective of the following instructions is to construct a positive integer $b$ such that $[b+1:b+a]$ does not contain a prime.
			\implementation
				\begin{enumerate}
					\item Let $b=a!+1$.
					\item For $i=1$ up to $i=a-1$, do the following:
					\begin{enumerate}
						\item Verify that $b+i=a!+1+i=i!(i+1)(i+2)\cdots(a)+1+i=(1+i)(i!(i+2)(i+3)\cdots(a) +1)$.
						\item Therefore verify that $b+i\equiv 0 (\mod i+1)$.
						\item Also verify that $b+i=a!+1+i>a!\ge a\ge i+1>1$.
						\item \textbf{Therefore verify that $b+i$ is not prime.}
					\end{enumerate}
					\item \textbf{Yield $\langle b\rangle$.}
				\end{enumerate}
		\procedure{1.30}
			\objective
				Choose two lists of primes $a,b$ in such a way that their intersection is empty. The objective of the following instructions is to show that $0=1$ or $(a_*,b_*)=1$.
			\implementation
				\begin{enumerate}
					\item Execute \procedurehr{1.09} on $\langle a_*,b_*\rangle$ and let $\langle c,d,e,f,g\rangle$.
					\item Verify that $0<c\le b\rangle$.
					\item If $c>1$, then do the following:
					\begin{enumerate}
						\item Execute \procedurehr{1.24} on $\langle c\rangle$ and let $\langle h\rangle$ receive.
						\item Using (3) and (a), verify that $\lvert h\rvert>0$.
						\item Now verify that $a_*=dc=dh_*=dh_0(h_{[1:\vert h\rvert]})_*\equiv 0(\mod h_0)$.
						\item Execute \procedurehr{1.25} on $\langle h_0,a\rangle$ and let $\langle k\rangle$ receive.
						\item Now verify that $b_*=ec=eh_*=eh_0(h_{[1:\vert h\rvert]})_*\equiv 0(\mod h_0)$.
						\item Execute \procedurehr{1.25} on $\langle h_0,b\rangle$ and let $\langle m\rangle$ receive.
						\item \textbf{Therefore verify that $a_k=h_0=b_m$.}
						\item \textbf{Abort procedure.}
					\end{enumerate}
					\item Otherwise do the following:
					\begin{enumerate}
						\item \textbf{Verify that $(a_*,b_*)=c=1$.}
					\end{enumerate}
				\end{enumerate}
		\procedure{1.31}
			\objective
				Choose two lists of primes $a,b$. Let $c$ be the common sublist with multiplicity of $a$ and $b$. The objective of the following instructions is to show that either $0<0$ or $(a_*,b_*)=c_*$.
			\implementation
				\begin{enumerate}
					\item Let $d$ be the result of removing with multiplicity elements of $c$ from $a$.
					\item Verify that $a_*=c_*d_*$.
					\item Let $e$ be the result of removing with multiplicity elements of $c$ from $b$.
					\item Verify that $b_*=c_*e_*$.
					\item Verify that $d$ and $e$ share no common elements.
					\item \textbf{Therefore using \procedurehr{1.18} and \procedurehr{1.30}, verify that $(a_*,b_*)=(c_*d_*,c_*e_*)=c_*(d_*,e_*)=c_*$.}
				\end{enumerate}
		\procedure{1.32}
			\objective
				Choose an integer $a$ and a positive integer $b$. The objective of the following instructions is to construct integers $c,f,e$ such that $c=af$, $c=be$, $c(a,b)=ab$, and $\lvert a\rvert\le c\sgn(a)\le\lvert a\rvert b$.
			\implementation
				\begin{enumerate}
					\item Execute \procedurehr{1.09} on $\langle a,b\rangle$ and let $\langle d,e,f,g,h\rangle$ receive.
					\item \textbf{Let $c=af$.}
					\item \textbf{Verify that $c(a,b)=cd=afd=ab$.}
					\item Verify that $d>0$.
					\item Verify that $b=fd$.
					\item Therefore verify that $1\le f\le b$.
					\item Therefore verify that $\lvert a\rvert\le\lvert a\rvert f\le\lvert a\rvert b$.
					\item \textbf{Therefore verify that $\lvert a\rvert\le c\sgn(a)\le\lvert a\rvert b$.}
					\item \textbf{Verify that $c=af=def=be$.}
					\item \textbf{Yield the tuple $\langle c,f,e\rangle$.}
				\end{enumerate}
		\notation{1.10}
			Let us use the notation $[a,b]$ as a shorthand for "the first part of the triple yielded by executing \procedurehr{1.32} on $\langle a,b\rangle$".
		\procedure{1.33}
			\objective
				Choose two positive integers $a,b$. The objective of the following instructions is to show that either $0<0$ or $[a,b]=[b,a]$.
			\implementation
				\begin{enumerate}
					\item Verify that $(a,b)>0$.
					\item Using \procedurehr{1.16}, verify that $[a,b](a,b)=ab=ba=[b,a](b,a)=[b,a](a,b)$.
					\item \textbf{Therefore verify that $[a,b]=[b,a]$.}
				\end{enumerate}
		\procedure{1.34}
			\objective
				Choose an integer $a$ and two positive integers $b,c$. The objective of the following instructions is to show that either $0<0$ or $[ca,cb]=c[a,b]$.
			\implementation
				\begin{enumerate}
					\item Verify that $(ca,cb)>0$.
					\item Using \procedurehr{1.18}, verify that $[ca,cb](ca,cb)=cacb=c^2ab=c^2[a,b](a,b)=c[a,b](ca,cb)$.
					\item \textbf{Therefore verify that $[ca,cb]=c[a,b]$.}
				\end{enumerate}
		\procedure{1.35}
			\objective
				Choose an integer $a$ and two positive integers $b,c$. The objective of the following instructions is to show that either $0<0$ or $[[a,b],c]=[a,[b,c]]$.
			\implementation
				\begin{enumerate}
					\item Using \procedurehr{1.19}, verify that $(a,b)(ab,(ac,bc))(b,c)[[a,b],c]$
					\begin{enumerate}
						\item $=(ab,(ac,bc))(b,c)[(a,b)[a,b],(a,b)c]$
						\item $=(ab,(ac,bc))(b,c)[ab,(ac,bc)]$
						\item $=ab(ac,bc)(b,c)$
						\item $=abc(a,b)(b,c)$
						\item $=bc(a,b)(ab,ac)$
						\item $=(a,b)((ab,ac),bc)[(ab,ac),bc]$
						\item $=(a,b)(ab,(ac,bc))[(ab,ac),bc]$
						\item $=(a,b)(ab,(ac,bc))[a(b,c),[b,c](b,c)]$
						\item $=(a,b)(ab,(ac,bc))(b,c)[a,[b,c]]$.
					\end{enumerate}
					\item Verify that $(a,b)(ab,(ac,bc))(b,c)>0$.
					\item \textbf{Therefore verify that $[[a,b],c]=[a,[b,c]]$.}
				\end{enumerate}
		\notation{1.11}
			Let us use the notation $[a_0,a_1,\cdots,a_{n-1}]$ as a shorthand for "either $[[a_0],[a_1,a_2,\cdots,a_{n-1}]]$ or $[[a_0,a_1],[a_2,a_3,\cdots,a_{n-1}]]$ or $\cdots$ or $[[a_0,a_1,\cdots,a_{n-2}],[a_{n-1}]]$".
		\procedure{1.36}
			\objective
				Choose three positive integers $a,b,c$. The objective of the following instructions is to show that either $0<0$ or $([a,b],c)=[(a,c),(b,c)]$.
			\implementation
				\begin{enumerate}
					\item Using \procedurehr{1.32}, \procedurehr{1.18}, \procedurehr{1.19}, \procedurehr{1.16}, and \procedurehr{1.10}, verify that $(a,b)((a,c),(b,c))([a,b],c)$
					\begin{enumerate}
						\item $=((a,c),(b,c))((a,b)[a,b],(a,b)c)$
						\item $=((a,c),(b,c))(ab,(ac,bc))$
						\item $=(a^2b,a^2c,c^2a,c^2b,b^2a,bac,b^2c)$
						\item $=(a,b)(ab,ac,bc,c^2)$
						\item $=(a,b)(a,c)(b,c)$
						\item $=(a,b)((a,c),(b,c))[(a,c),(b,c)]$.
					\end{enumerate}
					\item Verify that $(a,b)((a,c),(b,c))>0$.
					\item \textbf{Therefore verify that $([a,b],c)=[(a,c),(b,c)]$.}
				\end{enumerate}
		\procedure{1.37}
			\objective
				Choose three positive integers $a,b,c$. The objective of the following instructions is to show that either $0<0$ or $[(a,b),c]=([a,c],[b,c])$.
			\implementation
				\begin{enumerate}
					\item Using \procedurehr{1.32}, \procedurehr{1.18}, \procedurehr{1.19}, \procedurehr{1.16}, and \procedurehr{1.10}, verify that $((a,b),c)(a,c)(b,c)[(a,b),c]$
					\begin{enumerate}
						\item $=(a,c)(b,c)(a,b)c$
						\item $=(ab,ac,cb,c^2)(a,b)c$
						\item $=(a^2b,a^2c,ac^2,ab^2,abc,cb^2,bc^2)c$
						\item $=(a,b,c)(ab,ac,bc)c$
						\item $=((a,b),c)(ac(b,c),bc(a,c))$
						\item $=((a,b),c)(a,c)(b,c)([a,c],[b,c])$.
					\end{enumerate}
					\item Verify that $((a,b),c)(a,c)(b,c)>0$.
					\item \textbf{Therefore verify that $[(a,b),c]=([a,c],[b,c])$.}
				\end{enumerate}
		\procedure{1.38}
			\objective
				Choose two integers $a,c$ and two positive integers $b,d$ in such a way that $a\equiv c(\mod(b,d))$. The objective of the following instructions is to construct an integer $0\le e<[b,d]$.
			\implementation
				\begin{enumerate}
					\item Execute \procedurehr{1.09} on $\langle b,d\rangle$ and let $\langle f,g,h,i,j\rangle$ receive.
					\item \textbf{Yield the tuple $\langle (a+((c-a)\div(b,d))ib)\mod[b,d]\rangle$.}
				\end{enumerate}
		\notation{1.12}
			Let us use the notation $\chi_{b,d}(a,c)$ as a shorthand for "the result yielded by executing \procedurehr{1.38} on $\langle a,c,b,d\rangle$.
		\procedure{1.39}
			\objective
				Choose three integers $x,a,c$ and two positive integers $b,d$ such that $x\equiv a(\mod b)$ and $x\equiv c(\mod d)$. The objective of the following instructions is to show that $0\ne 0$ if $a\not\equiv d(\mod(b,d))$, otherwise $x\equiv\chi_{b,d}(a,c)(\mod[b,d])$.
			\implementation
				\begin{enumerate}
					\item Execute \procedurehr{1.09} on $\langle b,d\rangle$ and let $\langle e,f,g,h,i\rangle$ receive.
					\item Let $j=x\div b-a\div b$.
					\item Verify that $x=a+jb$.
					\item Let $k=x\div d-c\div d$.
					\item Verify that $x=c+kd$.
					\item Therefore verify that $c-a=jb-kd$.
					\item If $a\not\equiv c(\mod(b,d))$, then do the following:
					\begin{enumerate}
						\item Verify that $0\not\equiv d-a=jb-kd=jef-keg\equiv 0(\mod e)$.
						\item \textbf{Therefore verify that $0\ne 0$.}
						\item \textbf{Abort procedure.}
					\end{enumerate}
					\item Otherwise do the following:
					\begin{enumerate}
						\item Verify that $c-a\equiv 0(\mod(b,d))$.
						\item Let $l=(c-a)\div(b,d)$.
						\item Verify that $l(b,d)=le=c-a=jb-kd=jef-keg$.
						\item Therefore verify that $l=jf-kg$.
						\item Therefore verify that $l\equiv jf(\mod g)$.
						\item Also, using (1) verify that $efh+egi=bh+di=e$.
						\item Therefore verify that $fh+gi=1$.
						\item Therefore verify that $fh\equiv 1(\mod g)$.
						\item Therefore verify that $lh\equiv jfh\equiv j(\mod g)$.
						\item Therefore using \procedurehr{1.05}, verify that $lhb\equiv jb(\mod bg=[b,d])$.
						\item \textbf{Therefore verify that $x\equiv a+jb\equiv a+lhb\equiv\chi_{b,d}(a,c)(\mod[b,d])$.}
					\end{enumerate}
				\end{enumerate}
		\procedure{1.40}
			\objective
				Choose two integers $a,c$ and two positive integers $b,d$ in such a way that $a\equiv c(\mod(b,d))$. The objective of the following instructions is to show that either $0<0$ or $\chi_{b,d}(a,c)=\chi_{d,b}(c,a)$.
			\implementation
				\begin{enumerate}
					\item Execute \procedurehr{1.09} on $\langle b,d\rangle$ and let $\langle f,g,h,i,j\rangle$ receive.
					\item Verify that $ib+jd=f=(b,d)$.
					\item Execute \procedurehr{1.09} on $\langle d,b\rangle$ and let $\langle k,l,m,n,p\rangle$ receive.
					\item Verify that $pb+nd=k=(d,b)=(b,d)$.
					\item Execute \procedurehr{1.12} on $\langle b,p,n,d\rangle$ and let $\langle q\rangle$ receive.
					\item Therefore verify that $n=j-qg$.
					\item Now using \procedurehr{1.33}, verify that $\chi_{b,d}(a,c)$
					\begin{enumerate}
						\item $=(a+((c-a)\div(b,d))ib)\mod[b,d]$
						\item $=(a+((c-a)\div(b,d))(f-jd))\mod[b,d]$
						\item $=(a+((c-a)\div(b,d))f+((a-c)\div(b,d))jd)\mod[b,d]$
						\item $=(a+(c-a)+((a-c)\div(b,d))jd)\mod[b,d]$
						\item $=(c+((a-c)\div(d,b))(n+qg)d)\mod[b,d]$
						\item $=(c+((a-c)\div(d,b))dn+((a-c)\div(d,b))q[b,d])\mod[b,d]$
						\item $=(c+((a-c)\div(d,b))dn)\mod[b,d]$
						\item $=(c+((a-c)\div(d,b))dn)\mod[d,b]$
						\item $=\chi_{d,b}(c,a)$.
					\end{enumerate}
				\end{enumerate}
		\procedure{1.41}
			\objective
				Choose three integers $x,a,c$ and two positive integers $b,d$ such that $a\equiv c(\mod(b,d))$ and $x\equiv\chi_{b,d}(a,c)(\mod[b,d])$. The objective of the following instructions is to show that $x\equiv a(\mod b)$.
			\implementation
				\begin{enumerate}
					\item Execute \procedurehr{1.09} on $\langle b,d\rangle$ and let $\langle e,f,g,h,i\rangle$.
					\item Verify that $x\mod[b,d]=\chi_{b,d}(a,c)\mod[b,d]$.
					\item Therefore verify that $x\mod(bg)=\chi_{b,d}(a,c)\mod(bg)$.
					\item Therefore verify that $(x\mod(bg))\mod b=(\chi_{b,d}(a,c)\mod(bg))\mod b$.
					\item \textbf{Therefore using \procedurehr{1.03}, verify that $x\mod b=\chi_{b,d}(a,c)\mod b=(a+((c-a)\div(b,d))hb)\mod b=a\mod b$.}
				\end{enumerate}
		\procedure{1.42}
			\objective
				Choose three integers $x,a,c$ and two positive integers $b,d$ such that $a\equiv c(\mod(b,d))$ and $x\equiv\chi_{b,d}(a,c)(\mod[b,d])$. The objective of the following instructions is to either show that $0<0$ or to show that $x\equiv a(\mod b)$ and $x\equiv c(\mod d)$.
			\implementation
				\begin{enumerate}
					\item Execute \procedurehr{1.41} on $\langle x,a,c,b,d\rangle$.
					\item \textbf{Therefore verify that $x\equiv a(\mod b)$.}
					\item Now using \procedurehr{1.40}, verify that $x\equiv\chi_{b,d}(a,c)\equiv\chi_{d,b}(c,a)(\mod[d,b])$
					\item Execute \procedurehr{1.41} on $\langle x,c,a,d,b\rangle$.
					\item \textbf{Therefore verify that $x\equiv c(\mod d)$.}
				\end{enumerate}
		\procedure{1.43}
			\objective
				Choose two integers $a,c$ and three positive integers $b,d,e$ such that $a\equiv c(\mod(b,d))$. The objective of the following instructions is to show that $\chi_{b,d}(ea,ec)=e\chi_{b,d}(a,c)$.
			\implementation
				\begin{enumerate}
					\item Verify that $\chi_{b,d}(a,c)\equiv a(\mod b)$.
					\item Therefore using \procedurehr{1.05}, verify that $e\chi_{b,d}(a,c)\equiv ea(\mod b)$.
					\item Verify that $\chi_{b,d}(a,c)\equiv c(\mod d)$.
					\item Therefore using \procedurehr{1.05}, verify that $e\chi_{b,d}(a,c)\equiv ec(\mod d)$.
					\item Also using \procedurehr{1.02} and (O), verify that $ea\equiv ec(\mod(b,d))$.
					\item Therefore using \procedurehr{1.39}, verify that $e\chi_{b,d}(a,c)\equiv\chi_{b,d}(ea,ec)(\mod[b,d])$.
					\item \textbf{Therefore verify that $e\chi_{b,d}(a,c)=\chi_{b,d}(ea,ec)$.}
				\end{enumerate}
		\procedure{1.44}
			\objective
				Choose two integers $a,c$ and three positive integers $b,d,e$ such that $a\equiv c(\mod(eb,ed))$. The objective of the following instructions is to show that $\chi_{eb,ed}(a,c)\mod[b,d]=\chi_{b,d}(a,c)$.
			\implementation
				\begin{enumerate}
					\item Verify that $\chi_{eb,ed}(a,c)\equiv a(\mod eb)$.
					\item Therefore using \procedurehr{1.03}, verify that $\chi_{eb,ed}(a,c)\equiv a(\mod b)$.
					\item Verify that $\chi_{eb,ed}(a,c)\equiv c(\mod ed)$.
					\item Therefore using \procedurehr{1.03}, verify that $\chi_{eb,ed}(a,c)\equiv c(\mod d)$.
					\item Now verify that $a\equiv c(\mod e(b,d))$.
					\item Therefore using \procedurehr{1.03}, verify that $a\equiv c(\mod (b,d))$.
					\item Therefore using \procedurehr{1.39}, verify that $\chi_{eb,ed}(a,c)\equiv\chi_{b,d}(a,c)(\mod[b,d])$.
					\item \textbf{Therefore verify that $\chi_{eb,ed}(a,c)\mod[b,d]=\chi_{b,d}(a,c)$.}
				\end{enumerate}
		\procedure{1.45}
			\objective
				Choose three integers $a,c,e$ and three positive integers $b,d,f$ such that $a\equiv e(\mod(b,f))$, and $c\equiv e(\mod(d,f))$. The objective of the following instructions is to show that either $0<0$ or $\chi_{b,d}(a,c)\equiv e(\mod([b,d],f))$.
			\implementation
				\begin{enumerate}
					\item Execute \procedurehr{1.09} on $\langle b,f\rangle$ and let $\langle g_0,h_0,i_0,j_0,k_0\rangle$ receive.
					\item Execute \procedurehr{1.09} on $\langle d,f\rangle$ and let $\langle g_1,h_1,i_1,j_1,k_1\rangle$ receive.
					\item Verify that $e\equiv a(\mod(b,f))$.
					\item Verify that $e\equiv c(\mod(d,f))$.
					\item Therefore using \procedurehr{1.39} and \procedurehr{1.44}, verify that $e$
					\begin{enumerate}
						\item $\equiv\chi_{(b,f),(d,f)}(a,c)$
						\item $\equiv\chi_{(b,f)h_1,(d,f)h_2}(a,c)$
						\item $=\chi_{b,d}(a,c)(\mod[(b,f),(d,f)])$.
					\end{enumerate}
					\item \textbf{Therefore using \procedurehr{1.36}, verify that $e\equiv\chi_{b,d}(a,c)(\mod([b,d],f))$.}
				\end{enumerate}
		\procedure{1.46}
			\objective
				Choose three integers $a,c,e$ and three positive integers $b,d,f$ such that $a\equiv c(\mod(b,d))$, $a\equiv e(\mod(b,f))$, and $c\equiv e(\mod(d,f))$. Execute \procedurehr{1.45} on $\langle a,c,e,b,d,f\rangle$. Execute \procedurehr{1.45} on $\langle c,e,a,d,f,b\rangle$. The objective of the following instructions is to show that $\chi_{[b,d],f}(\chi_{b,d}(a,c),e)=\chi_{b,[d,f]}(a,\chi_{d,f}(c,e))$.
			\implementation
				\begin{enumerate}
					\item Verify that $\chi_{[b,d],f}(\chi_{b,d}(a,c),e)\equiv e(\mod f)$.
					\item Verify that $\chi_{[b,d],f}(\chi_{b,d}(a,c),e)\equiv \chi_{b,d}(a,c)(\mod [b,d]=gb=hd)$.
					\item Therefore using \procedurehr{1.03}, verify that $\chi_{[b,d],f}(\chi_{b,d}(a,c),e)\equiv \chi_{b,d}(a,c)\equiv a(\mod b)$.
					\item Also using \procedurehr{1.03}, verify that $\chi_{[b,d],f}(\chi_{b,d}(a,c),e)\equiv \chi_{b,d}(a,c)\equiv c(\mod d)$.
					\item Therefore using (1), (4), and \procedurehr{1.39}, verify that $\chi_{[b,d],f}(\chi_{b,d}(a,c),e)\equiv\chi_{d,f}(c,e)(\mod[d,f])$.
					\item Therefore using (3), (5), and \procedurehr{1.39}, verify that $\chi_{[b,d],f}(\chi_{b,d}(a,c),e)\equiv\chi_{b,[d,f]}(a,\chi_{d,f}(c,e))(\mod [b,[d,f]]=[[b,d],f])$.
					\item \textbf{Therefore verify that $\chi_{[b,d],f}(\chi_{b,d}(a,c),e)=\chi_{b,[d,f]}(a,\chi_{d,f}(c,e))$.}
				\end{enumerate}
		\notation{1.13}
			Let us use the notation $\chi_{b_0,b_1,\cdots,b_{n-1}}(a_0,a_1,\cdots,a_{n-1})$ as a shorthand for "one of
			\begin{enumerate}
				\item $\chi_{b_0,[b_1,b_2,\cdots,b_{n-1}]}(a_0,\allowbreak\chi_{b_1,b_2,\cdots,b_{n-1}}(a_1,a_2,\cdots,a_{n-1}))$
				\item $\chi_{[b_0,b_1],[b_2,b_3,\cdots,b_{n-1}]}(\chi_{b_0,b_1}(a_0,a_1),\allowbreak\chi_{b_2,b_3,\cdots,b_{n-1}}(a_2,a_3,\cdots,a_{n-1})$
				\item $\vdots$
				\item $\chi_{[b_0,b_1,\cdots,b_{n-2}],b_{n-1}}(\chi_{b_0,b_1,\cdots,b_{n-2}}(a_0,a_1,\cdots,a_{n-2}),\allowbreak a_{n-1})$".
			\end{enumerate}
		\notation{1.14}
			Let us use the notation $\phi(n)$ as a shorthand for "the sublist of $[0:n]$ where each integer $x$ is such that $(x,n)=1$".
		\procedure{1.47}
			\objective
				Choose an integer $a$ and a positive integer $b$ such that $(a,b)=1$. The objective of the following instructions is to either show that $0<0$ or to show that each element of $a\phi(b)\mod b$ is in $\phi(b)$.
			\implementation
				\begin{enumerate}
					\item Verify that $(a,b)=1$.
					\item For $i$ in $[0:\lvert\phi(b)\rvert]$, do the following:
					\begin{enumerate}
						\item Using (O), verify that $(\phi(b)_i,b)=1$.
						\item Execute \procedurehr{1.20} on $\langle a,\phi(b)_i,b\rangle$.
						\item Therefore verify that $(a\phi(b)_i,b)=1$.
						\item Execute \procedurehr{1.17} on $\langle a\phi(b)_i\mod b,a\phi(b)_i,b\rangle$.
						\item Therefore verify that $(a\phi(b)_i\mod b,b)=(a\phi(b)_i,b)=1$.
						\item Also verify that $0\le a\phi(b)_i\mod b<b$.
						\item Therefore verify that $a\phi(b)_i\mod b$ is contained in the list $\phi(b)$.
					\end{enumerate}
					\item \textbf{Therefore verify that each element of $a\phi(b)\mod b$ is in $\phi(b)$.}
				\end{enumerate}
		\procedure{1.48}
			\objective
				Choose an integer $a$ and a positive integer $b$ such that $(a,b)=1$. The objective of the following instructions is to either show that $0\ne 0$ or to show that each element of $a\phi(b)\mod b$ is distinct.
			\implementation
				\begin{enumerate}
					\item Execute \procedurehr{1.09} on $\langle a,b\rangle$ and let $\langle r,t,u,v,w\rangle$ receive.
					\item Verify that $va+wb=r=(a,b)=1$.
					\item Therefore verify that $va\equiv 1(\mod b)$.
					\item Now for $i$ in $[0:\lvert\phi(b)\rvert]$, do the following:
					\begin{enumerate}
						\item For $j$ in $[i+1:\lvert\phi(b)\rvert]$, do the following:
						\begin{enumerate}
							\item If $a\phi(b)_i\equiv a\phi(b)_j(\mod b)$, then do the following:
							\begin{enumerate}
								\item Verify that $\phi(b)_i\equiv va\phi(b)_i\equiv va\phi(b)_j\equiv\phi(b)_j(\mod b)$.
								\item Therefore verify that $\phi(b)_i=\phi(b)_j$.
								\item Also verify that $i\ne j$.
								\item Therefore using (O), verify that $\phi(b)_i\ne\phi(b)_j$.
								\item \textbf{Therefore using (B) and (D), verify that $\phi(b)_i\ne\phi(b)_i$.}
								\item \textbf{Abort procedure.}
							\end{enumerate}
							\item Otherwise, do the following:
							\begin{enumerate}
								\item Verify that $a\phi(b)_i\not\equiv a\phi(b)_j(\mod b)$.
							\end{enumerate}
						\end{enumerate}
					\end{enumerate}
					\item \textbf{Therefore verify that $a\phi(b)\mod b$ is composed of distinct elements.}
				\end{enumerate}
		\procedure{1.49}
			\objective
				Choose an integer $a$ and a positive integer $b$ such that $(a,b)=1$. The objective of the following instructions is to either show that $0<0$ or to show that $a\phi(b)\mod b$ is a rearrangement of $\phi(b)$.
			\implementation
				\begin{enumerate}
					\item Execute \procedurehr{1.47} on $\langle a,b\rangle$.
					\item Therefore verify that each element of $a\phi(b)\mod b$ is in $\phi(b)$.
					\item Verify that $\lvert a\phi(b)\mod b\rvert=\lvert \phi(b)\rvert$.
					\item Execute \procedurehr{1.48} on $\langle a,b\rangle$.
					\item Therefore verify that $a\phi(b)\mod b$ is composed of distinct elements.
					\item \textbf{Therefore verify that $a\phi(b)\mod b$ is a rearrangement of $\phi(b)$.}
				\end{enumerate}
		\procedure{1.50}
			\objective
				Choose an integer $a$ and a positive integer $b$ such that $(a,b)=1$. The objective of the following instructions is to show that either $0<0$ or $a^{\lvert\phi(b)\rvert}\equiv 1(\mod b)$.
			\implementation
				\begin{enumerate}
					\item For $i$ in $[0:\lvert\phi(b)\rvert]$, do the following:
					\begin{enumerate}
						\item Execute \procedurehr{1.09} on $\langle\phi(b)_i,b\rangle$ and let $\langle r_i,t_i,u_i,v_i,w_i\rangle$.
						\item Using (O), verify that $v_i\phi(b)_i+w_ib=r_i=(\phi(b)_i,b)=1$.
						\item Therefore verify that $v_i\phi(b)_i\equiv 1(\mod b)$.
					\end{enumerate}
					\item Therefore using \procedurehr{1.49}, verify that $\prod_{i=0}^{\lvert\phi(b)\rvert}\phi(b)_i\equiv\prod_{i=0}^{\lvert\phi(b)\rvert}a\phi(b)_i\equiv a^{\lvert\phi(b)\rvert}\prod_{i=0}^{\lvert\phi(b)\rvert}\phi(b)_i(\mod b)$.
					\item \textbf{Therefore verify that $1\equiv\prod_{i=0}^{\lvert\phi(b)\rvert}(v_i\phi(b)_i)=\prod_{i=0}^{\lvert\phi(b)\rvert}v_i\prod_{i=0}^{\lvert\phi(b)\rvert}\phi(b)_i\equiv a^{\lvert\phi(b)\rvert}\prod_{i=0}^{\lvert\phi(b)\rvert}\phi(b)_i\prod_{i=0}^{\lvert\phi(b)\rvert}v_i\equiv a^{\lvert\phi(b)\rvert}(\mod b)$.}
				\end{enumerate}
		\notation{1.15}
			Let us use the notation $a\times b$ as a shorthand for "the $\lvert a\rvert\times\lvert b\rvert$ matrix such that for $i$ in $[0:\lvert a\rvert]$, for $j$ in $[0:\lvert b\rvert]$, $(a\times b)_{i,j}=\langle a_i,b_j\rangle$".
		\procedure{1.52}
			\objective
				Choose two positive integers $a,b$ such that $(a,b)=1$. The objective of the following instructions is to show that each entry of $\chi_{a,b}([0:a]\times[0:b])$ is in $[0:ab]$.
			\implementation
				\begin{enumerate}
					\item Let $h=\chi_{a,b}([0:a]\times[0:b])$.
					\item Therefore verify that $0\le h_{i,j}<[a,b]=[a,b](a,b)=ab$ for $i$ in $[0:a]$, for $j$ in $[0:b]$.
					\item \textbf{Therefore verify that each entry of $h$ is in $[0:ab]$.}
				\end{enumerate}
		\procedure{1.53}
			\objective
				Choose two positive integers $a,b$ such that $(a,b)=1$. The objective of the following instructions is to either show that $0<0$ or to show that each entry of $\chi_{a,b}([0:a]\times[0:b])$ is distinct.
			\implementation
				\begin{enumerate}
					\item Let $h=\chi_{a,b}([0:a]\times[0:b])$.
					\item For each distinct unordered pair of index pairs $\langle i,j\rangle$ and $\langle k,l\rangle$ of $h$, do the following:
					\begin{enumerate}
						\item If $h_{i,j}=h_{k,l}$, then do the following:
						\begin{enumerate}
							\item Verify that $\chi_{a,b}([0:a]_i,[0:b]_j)=h_{i,j}=h_{k,l}=\chi_{a,b}([0:a]_k,[0:b]_l)$.
							\item Verify that $\chi_{a,b}(i,j)=\chi_{a,b}(k,l)$.
							\item Therefore using \procedurehr{1.42}, verify that $i\equiv\chi_{a,b}(i,j)=\chi_{a,b}(k,l)\equiv k(\mod a)$.
							\item Therefore verify that $i=k$.
							\item Also using \procedurehr{1.42}, verify that $j\equiv\chi_{a,b}(i,j)=\chi_{a,b}(k,l)\equiv l(\mod b)$.
							\item Therefore verify that $j=l$.
							\item Therefore verify that $\langle i,j\rangle=\langle k,l\rangle$.
							\item Using (2), verify that $\langle i,j\rangle\ne\langle k,l\rangle$.
							\item \textbf{Therefore verify that $\langle i,j\rangle\ne\langle i,j\rangle$.}
							\item \textbf{Abort procedure.}
						\end{enumerate}
						\item Otherwise do the following:
						\begin{enumerate}
							\item Verify that $h_{i,j}\ne h_{k,l}$.
						\end{enumerate}
					\end{enumerate}
					\item \textbf{Therefore verify that each entry of $h$ is distinct.}
				\end{enumerate}
		\procedure{1.54}
			\objective
				Choose two positive integers $a,b$ such that $(a,b)=1$. The objective of the following instructions is to show that either $0<0$ or $\chi_{a,b}([0:a]\times[0:b])$ is a rearrangement $[0:ab]$.
			\implementation
				\begin{enumerate}
					\item Let $h=\chi_{a,b}([0:a]\times[0:b])$.
					\item Execute \procedurehr{1.52} on $\langle a,b\rangle$.
					\item Therefore verify that each element of $h$ is in $[0:ab]$.
					\item Also verify that $h$ has the same number of entries as $[0:ab]$.
					\item Execute \procedurehr{1.53} on $\langle a,b\rangle$.
					\item Therefore verify that $h$ is composed of distinct elements.
					\item \textbf{Therefore verify that $h$ is a rearrangement of $[0:ab]$.}
				\end{enumerate}
		\procedure{1.55}
			\objective
				Choose two positive integers $a,b$ such that $(a,b)=1$. The objective of the following instructions is to either show that $0<0$ or to show that each entry of $\chi_{a,b}(\phi(a)\times\phi(b))$ is in $\phi(ab)$.
			\implementation
				\begin{enumerate}
					\item Let $h=\chi_{a,b}(\phi(a)\times\phi(b))$.
					\item Now, for each index pair $\langle i,j\rangle$ of $h$, do the following:
					\begin{enumerate}
						\item Verify that $0\le h_{i,j}<[a,b]=[a,b](a,b)=ab$.
						\item Verify that $h_{i,j}=\chi_{a,b}(\phi(a)_i,\phi(b)_j)\equiv\phi(a)_i(\mod a)$.
						\item Execute \procedurehr{1.17} on $\langle h_{i,j},\phi(a)_i,a\rangle$.
						\item Therefore verify that $(a,h_{i,j})=(h_{i,j},a)=(\phi(a)_i,a)=1$.
						\item Verify that $h_{i,j}=\chi_{a,b}(\phi(a)_i,\phi(b)_j)\equiv\phi(b)_j(\mod b)$.
						\item Execute \procedurehr{1.17} on $\langle h_{i,j},\phi(b)_j,b\rangle$.
						\item Therefore verify that $(b,h_{i,j})=(h_{i,j},b)=(\phi(b)_j,b)=1$.
						\item Therefore verify that $(h_{i,j},ab)=(ab,h_{i,j})=1$.
						\item Therefore verify that $h_{i,j}$ is in $\phi(ab)$.
					\end{enumerate}
					\item \textbf{Therefore verify that each entry of $\chi_{a,b}(\phi(a)\times\phi(b))$ is in $\phi(ab)$.}
				\end{enumerate}
		\procedure{1.56}
			\objective
				Choose two positive integers $a,b$ such that $(a,b)=1$. The objective of the following instructions is to either show that $0<0$ or to show that each entry of $\phi(ab)$ is in $\chi_{a,b}(\phi(a)\times\phi(b))$.
			\implementation
				\begin{enumerate}
					\item For $i$ in $[0:\lvert\phi(ab)\rvert]$, do the following:
					\begin{enumerate}
						\item Verify that $(\phi(ab)_i,ab)=1$.
						\item Verify that $\phi(ab)_i\equiv\phi(ab)_i\mod a(\mod a)$.
						\item Therefore using \procedurehr{1.17}, verify that $(\phi(ab)_i\mod a,a)=(\phi(ab)_i,a)=1$.
						\item Also verify that $0\le \phi(ab)_i\mod a<a$.
						\item Therefore verify that $\phi(ab)_i\mod a$ is amongst $\phi(a)$.
						\item Verify that $\phi(ab)_i\equiv\phi(ab)_i\mod b(\mod b)$.
						\item Also using \procedurehr{1.17}, verify that $(\phi(ab)_i\mod b,b)=(\phi(ab)_i,b)=1$.
						\item Also verify that $0\le \phi(ab)_i\mod b<b$.
						\item Therefore verify that $\phi(ab)_i\mod b$ is amongst $\phi(b)$.
						\item Therefore verify that $\langle \phi(ab)_i\mod a,\phi(ab)_i\mod b\rangle$ is amongst $\phi(a)\times\phi(b)$.
						\item Also using (b) and (f) and \procedurehr{1.39}, verify that $\phi(ab)_i\equiv\chi_{a,b}(\phi(ab)_i\mod a,\phi(ab)_i\mod b)(\mod[a,b]=[a,b](a,b)=ab)$.
						\item Therefore verify that $\phi(ab)_i=\chi_{a,b}(\phi(ab)_i\mod a,\phi(ab)_i\mod b)$.
						\item Therefore using (j) and (l), verify that $\phi(ab)_i$ is amongst $\chi_{a,b}(\phi(a)\times\phi(b))$.
					\end{enumerate}
					\item \textbf{Therefore verify that each entry of $\phi(ab)$ is in $\chi_{a,b}(\phi(a)\times\phi(b))$.}
				\end{enumerate}
		\procedure{1.57}
			\objective
				Choose two positive integers $a,b$ such that $(a,b)=1$. The objective of the following instructions is to either show that $0<0$ or to show that $\phi(ab)$ is a rearrangement of $\chi_{a,b}(\phi(a)\times\phi(b))$ and that $\lvert\phi(ab)\rvert=\lvert\phi(a)\rvert\lvert\phi(b)\rvert$.
			\implementation
				\begin{enumerate}
					\item Execute \procedurehr{1.54} on $\langle a,b\rangle$.
					\item Therefore verify that $\chi_{a,b}([0:a]\times[0:b])$ are a rearrangement of $[0:ab]$.
					\item Verify that $\chi_{a,b}(\phi(a)\times\phi(b))$ is a submatrix of $\chi_{a,b}([0:a]\times[0:b])$.
					\item Therefore verify that the entries of $\chi_{a,b}(\phi(a)\times\phi(b))$ are distinct.
					\item Execute \procedurehr{1.55} on $\langle a,b\rangle$.
					\item Therefore verify that the entries of $\chi_{a,b}(\phi(a)\times\phi(b))$ are in $\phi(ab)$.
					\item Verify that that the entries of $\phi(ab)$ are distinct.
					\item Execute \procedurehr{1.56} on $\langle a,b\rangle$.
					\item Therefore verify that the entries of $\phi(ab)$ are in $\chi_{a,b}(\phi(a)\times\phi(b))$.
					\item \textbf{Therefore verify that $\phi(ab)$ is a rearrangement of $\chi_{a,b}(\phi(a)\times\phi(b))$.}
					\item \textbf{Therefore verify that $\lvert\phi(ab)\rvert=\lvert\chi_{a,b}(\phi(a)\times\phi(b))\rvert=\lvert\phi(a)\times\phi(b)\rvert=\lvert\phi(a)\rvert\lvert\phi(b)\rvert$.}
				\end{enumerate}
		\notation{1.16}
			Let us use the notation $[P]$ as a shorthand for "$1$ if $P$, $0$ if otherwise".
		\notation{1.07}
			Let us use the notation $\sum_{r=a}^b c_r$ as a shorthand for "$0$ if $a=b$, otherwise $c_a+\sum_{r=a+1}^b c_r$".
		\procedure{1.58}
			\objective
				Choose a positive integer $a$ and a prime $b$. The objective of the following instructions is to show that either $0<0$ or $\lvert\phi(b^a)\rvert=b^a-b^{a-1}$.
			\implementation
				\begin{enumerate}
					\item Using \procedurehr{1.21}, verify that $\sum_{r=0}^{b^a}[(r,b^a)=1]\le\sum_{r=0}^{b^a}[(r,b)=1]$.
					\item Using \procedurehr{1.20}, verify that $\sum_{r=0}^{b^a}[(r,b)=1]\le\sum_{r=0}^{b^a}[(r,b^a)=1]$.
					\item Therefore verify that $\sum_{r=0}^{b^a}[(r,b^a)=1]=\sum_{r=0}^{b^a}[(r,b)=1]$.
					\item Using \procedurehr{1.13}, verify that $\sum_{r=0}^{b^a}[(r,b)=1]\le\sum_{r=0}^{b^a}[r\mod b\ne 0]$.
					\item Using \procedurehr{1.22}, verify that $\sum_{r=0}^{b^a}[r\mod b\ne 0]\le\sum_{r=0}^{b^a}[(r,b)=1]$.
					\item Therefore verify that $\sum_{r=0}^{b^a}[(r,b)=1]=\sum_{r=0}^{b^a}[r\mod b\ne 0]$.
					\item \textbf{Therefore using (3) and (6), verify that $\lvert\phi(b^a)\rvert=\sum_{r=0}^{b^a}[(r,b^a)=1]=\sum_{r=0}^{b^a}[(r,b)=1]=\sum_{r=0}^{b^a}[r\mod b\ne 0]=\sum_{r=0}^{b^a}(1-[r\mod b=0])=b^a-b^{a-1}$.}
				\end{enumerate}
		\procedure{1.59}
			\objective
				Choose a list of primes $a$. Let $b$ be the list of distinct primes in $a$. Let $c$ be a list such that $c_i$ is the multiplicity of $b_i$ in $a$ for $i=1$ to $i=\lvert b\rvert$. The objective of the following instructions is to show that either $0<0$ or $\lvert\phi(a_*)\rvert=\prod_{i=0}^{\lvert b\rvert}({b_i}^{c_i}-{b_i}^{c_i-1})$.
			\implementation
				\begin{enumerate}
					\item If $a=\langle\rangle$, then do the following:
					\begin{enumerate}
						\item Verify that $\lvert b\rvert=\lvert a\rvert=0$.
						\item \textbf{Therefore verify that $\phi(a_*)=\phi(1)=1=\prod_{i=0}^{\lvert b\rvert}({b_i}^{c_i}-{b_i}^{c_i-1})$.}
					\end{enumerate}
					\item Otherwise, do the following:
					\begin{enumerate}
						\item Verify that $a_*=\prod_{i=0}^{\lvert b\rvert} {b_i}^{c_i}$.
						\item Verify that $\lvert a\rvert>0$.
						\item Therefore verify that $\lvert c\rvert=\lvert b\rvert>0$.
						\item Therefore using \procedurehr{1.30}, verify that $({b_0}^{c_0},\prod_{i=1}^{\lvert b\rvert}{b_i}^{c_i})=1$.
						\item Let $d$ be the list $a$ with all instances of $a_0$ removed.
						\item Verify that $\lvert d\rvert<\lvert a\rvert$.
						\item Now execute \procedurehr{1.59} on $\langle d\rangle$.
						\item Hence verify that $\phi(d_*)=\phi(\prod_{i=1}^{\lvert b\rvert}{b_i}^{c_i})=\prod_{i=1}^{\lvert b\rvert}({b_i}^{c_i}-{b_i}^{c_i-1})$.
						\item \textbf{Therefore using (d), (h), \procedurehr{1.57} and \procedurehr{1.58}, verify that $\lvert\phi(a_*)\rvert=\lvert\phi(\prod_{i=0}^{\lvert b\rvert} {b_i}^{c_i})\rvert=\lvert\phi({b_0}^{c_0}\prod_{i=1}^{\lvert b\rvert}{b_i}^{c_i})\rvert=\lvert\phi({b_0}^{c_0})\rvert\lvert\phi(\prod_{i=1}^{\lvert b\rvert}{b_i}^{c_i})\rvert=({b_0}^{c_0}-{b_0}^{c_0-1})\lvert\phi(\prod_{i=1}^{\lvert b\rvert}{b_i}^{c_i})\rvert=({b_0}^{c_0}-{b_0}^{c_0-1})\prod_{i=1}^{\lvert b\rvert}({b_i}^{c_i}-{b_i}^{c_i-1})=\prod_{i=0}^{\lvert b\rvert}({b_i}^{c_i}-{b_i}^{c_i-1})$.}
					\end{enumerate}
				\end{enumerate}
		\notation{1.17}
			Let us use the notation $a^{\ul{b}}$ as a shorthand for "$\prod_{i=0}^b(a-i)$".
		\procedure{1.60}
			\objective
				Choose a list of distinct elements $a$ and a non-negative integer $b$ such that $b\le\lvert a\rvert$. Let $c$ be a list of length-$b$ permutations of $a$. The objective of the following instructions is to show that $\lvert c\rvert=\lvert a\rvert^{\ul{b}}$.
			\implementation
				\begin{enumerate}
					\item If $\lvert b\rvert>0$, then do the following:
					\begin{enumerate}
						\item For each entry $d$ in $a$, do the following:
						\begin{enumerate}
							\item Let $e$ be the list formed by removing $d$ from $a$.
							\item Verify that the entries of $e$ are distinct.
							\item Verify that $\lvert e\rvert=\lvert a\rvert-1$.
							\item Now execute \procedurehr{1.60} on $\langle e,b-1\rangle$.
							\item Therefore verify that the number of length-$b-1$ permutations of $e$ is $\lvert e\rvert^{\ul{b-1}}$.
							\item Therefore verify that the number of length-$b$ permutations of $a$ beginning with $d$ is $\lvert e\rvert^{\ul{b-1}}=(\lvert a\rvert-1)^{\ul{b-1}}$.
						\end{enumerate}
						\item Therefore verify that the number of length-$b$ permutations of $a$ beginning with any entry of $a$ is $\lvert a\rvert(\lvert a\rvert-1)^{\ul{b-1}}=\lvert a\rvert^{\ul{b}}$.
						\item Therefore verify that the number of length-$b$ permutations of $a$ are $\lvert a\rvert^{\ul{b}}$.
						\item \textbf{Therefore verify that $\lvert c\rvert=\lvert a\rvert^{\ul{b}}$.}
					\end{enumerate}
					\item Otherwise do the following:
					\begin{enumerate}
						\item Verify that $b=0$.
						\item Verify that the number of length-$0$ permutations of $a$ is $1$.
						\item \textbf{Therefore verify that $\lvert c\rvert=1=\lvert a\rvert^{\ul{0}}=\lvert a\rvert^{\ul{b}}$.}
					\end{enumerate}
				\end{enumerate}
		\notation{1.18}
			Let us use the notation $\binom{n}{r}$ as a shorthand for "$n^{\ul{r}}\div(r!)$".
		\procedure{1.61}
			\objective
				Choose a list of distinct elements $n$ and a non-negative integer $r$ such that $r\le\lvert n\rvert$. Let $b$ be the largest list of length-$r$ sublists of $n$ such that no two of them are permutations of each other. The objective of the following instructions is to either show that $b$ contains at least two permutations of the same list, construct a list larger than $b$ that is also a list of length-$r$ sublists of $n$ such that no two of them are permutations of each other, or to show that $\lvert b\rvert=\binom{\lvert n\rvert}{r}$ and that $\lvert n\rvert^{\ul{r}}\mod r!=0$.
			\implementation
				\begin{enumerate}
					\item Let $a$ and $f$ be a list of all the permutations of $n$.
					\item Using \procedurehr{1.60}, verify that $\lvert a\rvert=\lvert n\rvert^{\ul{\lvert n\rvert}}$.
					\item For each list $c$ in $b$, do the following:
					\begin{enumerate}
						\item Using \procedurehr{1.60}, verify that the number of permutations of $c$ is $r!$.
						\item Let $d$ be the list obtained by removing the elements of $c$ from $n$.
						\item Using \procedurehr{1.60}, verify that the number of permutations of $d$ is $(n-r)!$.
						\item Let $e$ be the list of permutations of $n$ beginning with a permutations of $c$.
						\item Verify that $\lvert e\rvert=\lvert c\rvert\lvert d\rvert=r!(\lvert n\rvert-r)!$.
						\item If $e$ is not a sublist of $a$, then do the following:
						\begin{enumerate}
							\item Let $g$ be a list in $e$ that is not in $a$.
							\item Verify that $e$ is a sublist of $f$.
							\item Therefore verify that $g$ was in $a$ but then was removed.
							\item Therefore verify that the variable $c$ was formerly equal to a permutation of the current $c$.
							\item \textbf{Therefore verify that $b$ contains at least two permutations of $c$.}
							\item \textbf{Abort procedure.}
						\end{enumerate}
						\item Otherwise, do the following:
						\begin{enumerate}
							\item Verify that $e$ is a sublist of $a$.
							\item Remove the lists in $e$ from $a$.
						\end{enumerate}
					\end{enumerate}
					\item If $a\ne\langle\rangle$, then do the following:
					\begin{enumerate}
						\item Let $g$ be a list in $a$.
						\item Let $h$ be the sublist of $g$ corresponding to its first $r$ elements.
						\item Therefore verify that the permutations of $n$ beginning with a permutation of $h$ were never removed from $a$.
						\item Therefore verify that the variable $c$ was never equal to a permutation of $h$.
						\item Therefore verify that no permutation of $h$ is in $b$.
						\item \textbf{Therefore verify that $b^{\frown}\langle h\rangle$ is larger than $b$ and is also a list of length-$r$ sublists of $n$ such that no two of them are permutations of each other.}
						\item \textbf{Abort procedure.}
					\end{enumerate}
					\item Otherwise do the following:
					\begin{enumerate}
						\item Verify that $\lvert n\rvert!\mod(r!(\lvert n\rvert-r)!)=0$.
						\item Therefore verify that $\lvert n\rvert!=(\lvert n\rvert!\div(r!(\lvert n\rvert-r)!))r!(\lvert n\rvert-r)!$.
						\item Therefore verify that $\lvert n\rvert!\div(\lvert n\rvert-r)!=(\lvert n\rvert!\div(r!(\lvert n\rvert-r)!))r!$.
						\item \textbf{Therefore verify that $n^{\ul{r}}\mod r!=(\lvert n\rvert!\div(\lvert n\rvert-r)!)\mod r!=0$.}
						\item Also verify that (3) iterated $\lvert n\rvert!\div(r!(\lvert n\rvert-r)!)$ times.
						\item \textbf{Therefore using \procedurehr{1.08}, verify that $\lvert b\rvert=\lvert n\rvert!\div(r!(\lvert n\rvert-r)!)=(\lvert n\rvert!\div(\lvert n\rvert-r)!)\div(r!)=n^{\ul{r}}\div(r!)=\binom{n}{r}$.}
					\end{enumerate}
				\end{enumerate}
		\procedure{1.62}
			\objective
				Choose two positive integers $a,b$. The objective of the following instructions is to show that $\binom{a}{b}=\binom{a-1}{b-1}+\binom{a-1}{b}$.
			\implementation
				\begin{enumerate}
					\item Using \procedurehr{1.05} and \procedurehr{1.06}, verify that $\binom{a-1}{b-1}+\binom{a-1}{b}$
					\begin{enumerate}
						\item $=(a-1)^{\ul{b-1}}\div(b-1)!+(a-1)^{\ul{b}}\div b!$
						\item $=((a-1)^{\ul{b-1}}b)\div b!+(a-1)^{\ul{b}}\div b!$
						\item $=((a-1)^{\ul{b-1}}b+(a-1)^{\ul{b}})\div b!$
						\item $=((a-1)^{\ul{b-1}}b+(a-1)^{\ul{b-1}}(a-b))\div b!$
						\item $=((a-1)^{\ul{b-1}}a)\div b!$
						\item $=a^{\ul{b}}\div b!$
						\item $=\binom{a}{b}$.
					\end{enumerate}
				\end{enumerate}
		\notation{1.19}
			Let us use the notation $\mathbb{Z}$ as a shorthand for "integer".
		\notation{1.20}
			Let us use the notation $A[x_1,x_2,\cdots,x_n]$ as a shorthand for "formal polynomial with $A$ coefficients in the indeterminates $x_1,x_2,\cdots,x_n$".
		\procedure{1.63}
			\objective
				Choose a non-negative integer $a$. The objective of the following instructions is to show that the $\mathbb{Z}[x]$ $(1+x)^a=\sum_{r=0}^{a+1}\binom{a}{r}x^r$.
			\implementation
				\begin{enumerate}
					\item If $a=0$, then do the following:
					\begin{enumerate}
						\item \textbf{Verify that $(1+x)^a=(1+x)^0=1=\sum_{r=0}^1\binom{0}{r}x^r=\sum_{r=0}^{a+1}\binom{a}{r}x^r$.}
					\end{enumerate}
					\item Otherwise, do the following:
					\begin{enumerate}
						\item Verify that $a>0$.
						\item Therefore verify that $a-1\ge 0$.
						\item Execute \procedurehr{1.63} on $\langle a-1\rangle$.
						\item Therefore verify that $(1+x)^{a-1}=\sum_{r=0}^{a}\binom{a-1}{r}x^r$.
						\item Therefore using \procedurehr{1.62}, verify that $(1+x)^a$
						\begin{enumerate}
							\item $=(1+x)(1+x)^{a-1}$
							\item $=(1+x)\sum_{r=0}^{a}\binom{a-1}{r}x^r$
							\item $=\sum_{r=0}^{a}\binom{a-1}{r}x^r+\sum_{r=0}^{a}\binom{a-1}{r}x^{r+1}$
							\item $=\sum_{r=0}^{a+1}\binom{a-1}{r}x^r+\sum_{r=1}^{a+1}\binom{a-1}{r-1}x^{r}$
							\item $=1+\sum_{r=1}^{a+1}(\binom{a-1}{r}+\binom{a-1}{r-1})x^{r}$
							\item $=1+\sum_{r=1}^{a+1}\binom{a}{r}x^{r}$
							\item $=\sum_{r=0}^{a+1}\binom{a}{r}x^{r}$.
						\end{enumerate}
					\end{enumerate}
				\end{enumerate}
		\procedure{1.64}
			\objective
				Choose an integer $r$ and a prime $n$ such that $0<r<n$. The objective of the following instructions is to show that either $0\ne 0$ or $\binom{n}{r}\mod n=0$.
			\implementation
				\begin{enumerate}
					\item Using \procedurehr{1.61}, verify that $\binom{n}{r}r!=n^{\ul{r}}\equiv 0(\mod n)$.
					\item If $\binom{n}{r}\mod n\ne 0$, then do the following:
					\begin{enumerate}
						\item Verify that $i\mod n\ne 0$ for $i=1$ to $i=r$.
						\item Therefore using \procedurehr{1.23}, verify that $r!\mod n\ne 0$.
						\item Therefore using (2) and (b), verify that $\binom{n}{r}r!\mod n\ne 0$.
						\item \textbf{Therefore using (1) and (c), verify that $0\ne 0$.}
						\item \textbf{Abort procedure.}
					\end{enumerate}
					\item Otherwise, do the following:
					\begin{enumerate}
						\item \textbf{Verify that $\binom{n}{r}\mod n=0$.}
					\end{enumerate}
				\end{enumerate}
		\procedure{1.65}
			\objective
				Choose a non-negative integer $a$ and a prime $b$. The objective of the following instructions is to show that either $0\ne 0$ or the $\mathbb{Z}[x]$ $\sum_{r=0}^{a+1}\binom{a}{r}x^r\equiv\sum_{r=0}^{a+1}\binom{a\div b}{r\div b}\binom{a\mod b}{r\mod b}x^r(\mod b)$.
			\implementation
				\begin{enumerate}
					\item Using \procedurehr{1.02}, \procedurehr{1.63}, and \procedurehr{1.64}, verify that $\sum_{r=0}^{a+1}\binom{a}{r}x^r$
					\begin{enumerate}
						\item $=(1+x)^a$
						\item $=(1+x)^{(a\div b)b+a\mod b}$
						\item $=(1+x)^{(a\div b)b}(1+x)^{a\mod b}$
						\item $=((1+x)^b)^{a\div b}(1+x)^{a\mod b}$
						\item $=(\sum_{u=0}^{b+1}\binom{b}{u}x^u)^{a\div b}(\sum_{t=0}^{(a\mod b)+1}\binom{a\mod b}{t}x^t)$
						\item $\equiv(1+x^b)^{a\div b}(\sum_{t=0}^{(a\mod b)+1}\binom{a\mod b}{t}x^t)$
						\item $=(\sum_{u=0}^{(a\div b)+1}(x^b)^u\binom{a\div b}{u})\cdot(\sum_{t=0}^{(a\mod b)+1}\binom{a\mod b}{t}x^t)$
						\item $=\sum_{u=0}^{(a\div b)+1}\sum_{t=0}^{(a\mod b)+1}\binom{a\div b}{u}\binom{a\mod b}{t}x^{ub+t}$
						\item $=\sum_{u=0}^{(a\div b)+1}\sum_{t=0}^{(a\mod b)+1}\binom{a\div b}{(ub+t)\div b}\cdot\binom{a\mod b}{(ub+t)\mod b}x^{ub+t}$
						\item $=\sum_{r=0}^{a+1}\binom{a\div b}{r\div b}\binom{a\mod b}{r\mod b}x^{r}(\mod b)$.
					\end{enumerate}
				\end{enumerate}
	\section{Rational Arithmetic}
		\notation{2.00}
			Let us use the notation $\mathbb{Q}$ as a shorthand for "rational".
		\procedure{2.00}
			\objective
				Choose an integer $n\ge 0$ and a $\mathbb{Q}[x]$ $p=p_0x^n+p_1x^{n-1}+\cdots+p_n$. Let $y,z$ be indeterminates. The objective of the following instructions is to construct a $\mathbb{Q}[y,z]$ $G$ such that $p(z)-p(y)=(z-y)G(y,z)$.
			\implementation
				\begin{enumerate}
					\item Let the $\mathbb{Q}[y,z]$ $G=\sum_{r=1}^{n+1} p_{n-r}(z^{r-1}+z^{r-2}y+\cdots+zy^{r-2}+y^{r-1})$.
					\item Verify that $p(z)-p(y)$
					\begin{enumerate}
						\item $=(p_0z^n+p_1z^{n-1}+\cdots+p_n)-(p_0y^n+p_1y^{n-1}+\cdots+p_n)$
						\item $=(\sum_{r=0}^{n+1} p_{n-r}z^r)-(\sum_{r=0}^{n+1} p_{n-r}y^r)$
						\item $=\sum_{r=1}^{n+1} p_{n-r}(z^r-y^r)$
						\item $=\sum_{r=1}^{n+1} p_{n-r}(z-y)(z^{r-1}+z^{r-2}y+\cdots+zy^{r-2}+y^{r-1})$
						\item $=(z-y)\sum_{r=1}^{n+1} p_{n-r}(z^{r-1}+z^{r-2}y+\cdots+zy^{r-2}+y^{r-1})$
						\item $=(z-y)G(y,z)$.
					\end{enumerate}
					\item \textbf{Yield the tuple $\langle G\rangle$.}
				\end{enumerate}
		\procedure{2.01}
			\objective
				Choose a $\mathbb{Q}[x]$ $p=x^n+p_1x^{n-1}+\cdots+p_n$ and $\mathbb{Q}$s $a_0<a_1<\cdots<a_{n-1}<a_{n}$ in such a way that for $i=0$ to $i=n$, $p(a_i)=0$. The objective of the following instructions is to show that $0\ne 0$.
			\implementation
				\begin{enumerate}
					\item Write $p$ as $1*p$, so that it has two factors.
					\item For $i$ in $[0:n]$, do the following:
					\begin{enumerate}
						\item Let $g$ be the rightmost factor of $p$.
						\item If $g(a_i)\ne 0$, do the following:
						\begin{enumerate}
							\item For $k$ in $[0:i]$, verify that $(a_i-a_k)\ne 0$.
							\item Therefore verify that $p(a_i)\ne 0$.
							\item Therefore using (O) and (ii), verify that $0\ne 0$.
							\item \textbf{Abort procedure.}
						\end{enumerate}
						\item Otherwise $g(a_i)=0$. Now do the following:
						\begin{enumerate}
							\item Execute \procedurehr{2.00} on $g$ and let the tuple $\langle G\rangle$ receive the result.
							\item Let $x$ be an indeterminate.
							\item Let the $\mathbb{Q}[x]$ $q=q(x)=G(a_i,x)$.
							\item Verify that the $\mathbb{Q}[x]$ $g=g(x)=g(x)-g(a_i)=(x-a_i)G(a_i,x)=(x-a_i)q(x)=(x-a_i)q$.
							\item Verify that $p=\prod_{j=0}^{i+1}(x-a_j)q$.
						\end{enumerate}
					\end{enumerate}
					\item Now verify that $p=\prod_{j=0}^{n}(x-a_j)$.
					\item Using (3), verify that $p(a_{n})\ne 0$.
					\item Therefore using (O) and (4), verify that $0\ne 0$.
					\item \textbf{Abort procedure.}
				\end{enumerate}
		\procedure{2.02}
			\objective
				Choose a $\mathbb{Q}[x]$ $f$. Choose $\mathbb{Q}$s $a<b$ such that $\sgn(f(a))=-\sgn(f(b))$. Choose a rational number target $B>0$. The objective of the following instructions is to construct a $\mathbb{Q}$ $d$ such that $a\le d\le b$ and $\lvert f(d)\rvert<B$.
			\implementation
				\begin{enumerate}
					\item Execute \procedurehr{2.00} on $f$ and let the tuple $\langle G\rangle$ receive the result.
					\item Let $x,y$ be indeterminates.
					\item Verify that the $\mathbb{Q}[x,y]$ $f(y)-f(x)=(y-x)G(x,y)$.
					\item Let $c=a$ and $d=b$.
					\item Until $\lvert d-c\rvert \lvert G\rvert(\lvert a\rvert,\lvert b\rvert)<B$
					\begin{enumerate}
						\item Let $e=\frac{c+d}{2}$.
						\item If $\sgn(f(c))=-\sgn(f(e))$, then:
						\begin{enumerate}
							\item Let $d=e$.
						\end{enumerate}
						\item Otherwise if $\sgn(f(e))=-\sgn(f(d))$, then:
						\begin{enumerate}
							\item Let $c=e$.
						\end{enumerate}
						\item Otherwise if $f(e)=0$, then do the following:
						\begin{enumerate}
							\item \textbf{Verify that $\lvert f(e)\rvert=0<B$.}
							\item Yield the tuple $\langle e\rangle$.
						\end{enumerate}
					\end{enumerate}
					\item \textbf{Verify that $\lvert f(c)\rvert,\lvert f(d)\rvert<\lvert f(d)-f(c)\rvert=\lvert(d-c)G(c,d)\rvert=\lvert d-c\rvert\lvert G(c,d)\rvert\le\lvert d-c\rvert\lvert G\rvert(\lvert c\rvert,\lvert d\rvert)\le\lvert d-c\rvert\lvert G\rvert(\lvert a\rvert,\lvert b\rvert)<B$.}
					\item \textbf{Yield the tuple $\langle c\rangle$.}
				\end{enumerate}
		\notation{2.07}
			Let us use the notation $\min_{r=a}^b c_r$ as a shorthand for "$\infty$ if $a=b$, otherwise $\min(c_a,\min_{r=a+1}^b c_r)$".
		\procedure{2.03}
			\objective
				Choose a $\mathbb{Q}[x]$ $f=x^n+p_1x^{n-1}+\cdots+p_n$ and pairs of $\mathbb{Q}$s $(a_n,b_n),(a_{n-1},b_{n-1}),\cdots,(a_0,b_0)$ in such a way that:
				\begin{enumerate}
					\item $a_n<b_n\le a_{n-1}<b_{n-1}\le\cdots\le a_1<b_1\le a_0<b_0$.
					\item $\sgn(f(a_i))=-\sgn(f(b_i))$ for $i=0$ to $i=n$.
				\end{enumerate}
				The objective of the following instructions is to show that $1=-1$.
			\implementation
				\begin{enumerate}
					\item If $n>0$:
					\begin{enumerate}
						\item Let $B=\min_{k=0}^{n-1}\min(\lvert f(a_k)\rvert,\lvert f(b_k)\rvert)$.
						\item For $k=0$ to $k=n-1$, verify that $\lvert f(a_k)\rvert\ge B$.
						\item Execute \procedurehr{2.02} on the formal polynomial $f$, interval $(a_n, b_n)$, and target of $B$. Let the tuple $\langle d\rangle$ receive the result.
						\item Verify that $\lvert f(d)\rvert<B$.
						\item Execute \procedurehr{2.00} on the formal polynomial $f$ and let the tuple $\langle G\rangle$ receive the result.
						\item Let $x$ be an indeterminate.
						\item Let the formal polynomial $h=G(d,x)$.
						\item Verify that $h$ is a monic $(n-1)^{th}$ degree formal polynomial.
						\item Verify that the formal polynomial $f=f(x)=f(x)-f(d)+f(d)=(x-d)G(d,x)+f(d)=(x-d)h(x)+f(d)=(x-d)h+f(d)$.
						\item For $k=0$ to $k=n-1$, do the following:
						\begin{enumerate}
							\item If $f(a_k)\ge B$, in-order verify that:
							\begin{enumerate}
								\item $f(a_k)\ge B>\lvert f(d)\rvert\ge f(d)$.
								\item $f(a_k)-f(d)>0$.
								\item $(a_k-d)h(a_k)>0$.
								\item \textbf{$h(a_k)>0$.}
								\item $f(b_k)\le-B<-\lvert f(d)\rvert\le f(d)$.
								\item $f(b_k)-f(d)<0$.
								\item $(b_k-d)h(b_k)<0$.
								\item \textbf{$h(b_k)<0$.}
							\end{enumerate}
							\item Otherwise, if $f(a_k)\le -B$, do the following:
							\begin{enumerate}
								\item \textbf{Using steps analogous to (ji), verify that $h(a_k)<0$.}
								\item \textbf{Using steps analogous to (ji), verify that $h(b_k)>0$.}
							\end{enumerate}
						\end{enumerate}
						\item Execute \procedurehr{2.03} on $h$ and $a_{n-1}<b_{n-1}\le a_{n-2}<b_{n-2}\le\cdots\le a_1<b_1\le a_0<b_0$.
					\end{enumerate}
					\item Otherwise, do the following:
					\begin{enumerate}
						\item Verify that $n=0$.
						\item Therefore verify that $h=1$.
						\item \textbf{Therefore verify that $1=\sgn(1)=\sgn(f_0(a_0))=-\sgn(f_0(b_0))=-\sgn(1)=-1$.}
						\item \textbf{Abort procedure.}
					\end{enumerate}
				\end{enumerate}
		\notation{2.01}
			Let us use the notation $p\circ q$ as a shorthand for "the sum of products where each product is the coefficient of a monomial in $p$ times the coefficient of the same monomial in $q$".
		\procedure{2.04}
			\objective
				Choose two lists of $\mathbb{Q}[x]$s $s,q$ in such a way that:
				\begin{enumerate}
					\item $\lvert s\rvert>1$.
					\item For $i$ in $[0:\lvert s\rvert]$, $\deg(s_i)=i$.
					\item For $i$ in $[0:\lvert s\rvert]$, $\sgn(x^i\circ s_i)=\sgn(x^m\circ s_m)$.
					\item For $i$ in $[1:\lvert s\rvert-1]$, $s_{i-1}+s_{i+1}=q_is_i$.
				\end{enumerate}
				Let $x,y$ be indeterminates. The objective of the following instructions is to construct lists of $\mathbb{Q}[x]$s $g,h$ such that $g_is_{i+1}+h_is_i=1$ for $i$ in $[0:\lvert s\rvert-1]$.
			\implementation
				\begin{enumerate}
					\item Let $m=\lvert s\rvert-1$
					\item Let $g=h=\langle\rangle$.
					\item If $m>1$, do the following:
					\begin{enumerate}
						\item Verify that $q_{m-1}s_{m-1}-s_{m}=s_{m-2}$.
						\item Execute \procedurehr{2.04} on $s_{[0:m]}$ and $q_{[1:m-1]}$ and let the tuple $\langle,,g,h\rangle$ receive.
						\item Verify that $g_{m-2}s_{m-1}+h_{m-2}s_{m-2}=1$.
						\item Let $g_{m-1}=-h_{m-2}$.
						\item Let $h_{m-1}=g_{m-2}+h_{m-2}q_{m-1}$.
						\item \textbf{Therefore verify that $g_{m-1}s_{m}+h_{m-1}s_{m-1}=g_{m-2}s_{m-1}+h_{m-2}(q_{m-1}s_{m-1}-s_{m})=g_{m-2}s_{m-1}+h_{m-2}s_{m-2}=1$.}
					\end{enumerate}
					\item Otherwise, if $m=1$ do the following:
					\begin{enumerate}
						\item Let $g_0=0$.
						\item Let $h_0=\frac{1}{s_0}$.
						\item \textbf{Therefore verify that $g_0s_1+h_0s_0=1$.}
					\end{enumerate}
					\item \textbf{Yield the tuple $\langle s,q,g,h\rangle$.}
				\end{enumerate}
		\notation{2.02}
			Let us use the notation $J_{s}(x)$ as a shorthand for "the number of changes in the list $\sgn(s(x))$".
		\notation{2.07}
			Let us use the notation $\max_{r=a}^b c_r$ as a shorthand for "$-\infty$ if $a=b$, otherwise $\max(c_a,\max_{r=a+1}^b c_r)$".
		\procedure{2.05}
			\objective
				Execute \procedurehr{2.04} and let $\langle s,q,g,h\rangle$ receive. Execute \procedurehr{2.00} on $s$ and let $\langle G\rangle$ receive the result. Choose $\mathbb{Q}$s $c$ and $d$ in such a way that:
				\begin{enumerate}
					\item $0\not\in s(c)$ and $0\not\in s(d)$.
					\item Letting $B=\max_{i=0}^{\lvert s\rvert} \lvert G_i(c,d)\rvert$.
					\item Letting $C=\max_{i=0}^{\lvert s\rvert-1}\max(\lvert g_i(c)\rvert,\lvert h_i(c)\rvert,\lvert g_i(d)\rvert,\lvert h_i(d))\rvert$.
					\item Letting $D=\max_{i=1}^{\lvert s\rvert-1}\max(\lvert q_i(c)\rvert,\lvert q_i(d)\rvert,2)$.
					\item $\lvert d-c\rvert\le\frac{1}{BCD}$.
				\end{enumerate}
				The objective of the following instructions is to show that either $0<0$ or $\lvert J_{s}(d)-J_{s}(c)\rvert=[\sgn(s_{\lvert s\rvert-1}(c))\ne\sgn(s_{\lvert s\rvert-1}(d))]$.
			\implementation
				\begin{enumerate}
					\item Let $i=0$.
					\item If $i+1<\lvert s\rvert$, do the following:
					\begin{enumerate}
						\item Using (O), (2c), or (2divA), verify that $\sgn(s_i(c))=\sgn(s_i(d))$.
						\item Using (O), (2ci), or (2divC), verify that $J_{s_{[0:i+1]}}(c)=J_{s_{[0:i+1]}}(d)$.
						\item If $\sgn(s_{i+1}(c))=\sgn(s_{i+1}(d))$, do the following:
						\begin{enumerate}
							\item Verify that $J_{s_{[0:i+2]}}(c)=J_{s_{[0:i+2]}}(d)$.
							\item Set $i$ to $i+1$ and go to (2).
						\end{enumerate}
						\item Otherwise, if $\sgn(s_{i+1}(c))\ne\sgn(s_{i+1}(d))$ and $i+2<\lvert s\rvert$, do the following:
						\begin{enumerate}
							\item Execute \hyperref[sec:procedure 2.5 auxilliary procedure]{procedure 2.5 auxilliary procedure} on $i$.
							\item If $\sgn(s_{i+2}(c))\ne\sgn(s_{i+2}(d))$, do the following:
							\begin{enumerate}
								\item Verify that $\lvert s_{i+2}(c)\rvert<\lvert s_{i+2}(d)-s_{i+2}(c)\rvert=\lvert (d-c)G_{i+2}(c,d)\rvert\le\frac{1}{BCD}\cdot B=\frac{1}{CD}=\frac{1}{C}\cdot\frac{1}{D}\le\frac{1}{C}(1-\frac{1}{D})$.
								\item Using (A) and (i), verify that $\frac{1}{C}(1-\frac{1}{D})<\lvert s_{i+2}(c)\rvert<\frac{1}{C}(1-\frac{1}{D})$.
								\item \textbf{Abort procedure.}
							\end{enumerate}
							\item Otherwise if $\sgn(s_i(c))=\sgn(s_{i+2}(c))$, do the following:
							\begin{enumerate}
								\item Verify that $2\frac{1}{C}(1-\frac{1}{D})<\lvert s_i(c)\rvert+\lvert s_{i+2}(c)\rvert=\lvert s_i(c)+s_{i+2}(c)\rvert=\lvert q_{i+1}(c)s_{i+1}(c)\rvert<D\frac{1}{CD}$.
								\item Verify that $2(1-\frac{1}{D})<1$.
								\item Using (B) and the construction of $D$, verify that $2\le D<2$.
								\item \textbf{Abort procedure.}
							\end{enumerate}
							\item Otherwise, do the following:
							\begin{enumerate}
								\item Verify that $\sgn(s_i(d))=\sgn(s_i(c))\ne\sgn(s_{i+2}(c))=\sgn(s_{i+2}(d))$.
								\item Therefore verify that $1=J_{s_{[0:i+3]}}(c)-J_{s_{[0:i+1]}}(c)=J_{s_{[0:i+3]}}(d)-J_{s_{[0:i+1]}}(d)$.
								\item Therefore verify that $J_{s_{[0:i+1]}}(c)+1=J_{s_{[0:i+3]}}(c)=J_{s_{[0:i+3]}}(d)=J_{s_{[0:i+1]}}(d)+1$.
								\item Set $i$ to $i+2$ and go to (2).
							\end{enumerate}
						\end{enumerate}
						\item Otherwise, verify the following:
						\begin{enumerate}
							\item $\sgn(s_{i+1}(c))\ne\sgn(s_{i+1}(d))$.
							\item $\lvert J_{s_{[0:i+2]}}(c)-J_{s_{[0:i+2]}}(d)\rvert=1$.
							\item $i+2=\lvert s\rvert$.
						\end{enumerate}
					\end{enumerate}
					\textbf{
						\item If $\sgn(s_{\lvert s\rvert-1}(c))=\sgn(s_{\lvert s\rvert-1}(d))$, then do the following:
						\begin{enumerate}
							\item Using (O), (2ci), or (2divC), verify that $J_s(c)=J_s(d)$.
						\end{enumerate}
						\item Otherwise do the following:
						\begin{enumerate}
							\item Using (2eii), verify that $\lvert J_s(d)-J_s(c)\rvert=1$.
						\end{enumerate}
					}
				\end{enumerate}
			\subsubsection*{Auxilliary Procedure}\label{sec:procedure 2.5 auxilliary procedure}
				\paragraph{Objective}
					Choose a non-negative integer $i<m$ such that $\sgn(s_{i+1}(c))\ne\sgn(s_{i+1}(d))$ and $i+2\le m$. The objective of the following instructions is to show that $\lvert s_{i+1}(c)\rvert<\frac{1}{CD}$, $\lvert s_{i+1}(d)\rvert<\frac{1}{CD}$, $\frac{1}{C}(1-\frac{1}{D})<\lvert s_{i}(c)\rvert$, $\frac{1}{C}(1-\frac{1}{D})<\lvert s_{i}(d)\rvert$, $\frac{1}{C}(1-\frac{1}{D})<\lvert s_{i+2}(c)\rvert$, and $\frac{1}{C}(1-\frac{1}{D})<\lvert s_{i+2}(d)\rvert$.
				\paragraph{Implementation}
					\begin{enumerate}
						\item Verify the following in order:
						\begin{enumerate}
							\item $\lvert s_{i+1}(c)\rvert<\lvert s_{i+1}(c)-s_{i+1}(d)\rvert=\lvert c-d\rvert\lvert G_{i+1}(c,d)\rvert\le\lvert c-d\rvert B\le lB=\frac{1}{CD}$
							\item $\lvert s_{i+1}(d)\rvert<\lvert s_{i+1}(c)-s_{i+1}(d)\rvert\le\frac{1}{CD}$
							\item $1=g_{i}(c)s_{i+1}(c)+h_{i}(c)s_{i}(c)=\lvert g_{i}(c)s_{i+1}(c)+h_{i}(c)s_{i}(c)\rvert\le\lvert g_{i}(c)\rvert\lvert s_{i+1}(c)\rvert+\lvert h_{i}(c)\rvert\lvert s_{i}(c)\rvert<C(\frac{1}{CD}+\lvert s_{i}(c)\rvert)$
							\item $\frac{1}{C}(1-\frac{1}{D})<\lvert s_{i}(c)\rvert$
							\item $1<C(\frac{1}{CD}+\lvert s_{i}(d)\rvert)$
							\item $\frac{1}{C}(1-\frac{1}{D})<\lvert s_{i}(d)\rvert$
							\item $1=g_{i+1}(c)s_{i+2}(c)+h_{i+1}(c)s_{i+1}(c)=\lvert g_{i+1}(c)s_{i+2}(c)+h_{i+1}(c)s_{i+1}(c)\rvert\le\lvert g_{i+1}(c)\rvert\lvert s_{i+2}(c)\rvert+\lvert h_{i+1}(c)\rvert\lvert s_{i+1}(c)\rvert<C(\lvert s_{i+2}(c)\rvert+\frac{1}{CD})$
							\item $\frac{1}{C}(1-\frac{1}{D})<\lvert s_{i+2}(c)\rvert$
							\item $1<C(\lvert s_{i+2}(d)\rvert+\frac{1}{CD})$
							\item $\frac{1}{C}(1-\frac{1}{D})<\lvert s_{i+2}(d)\rvert$
						\end{enumerate}
					\end{enumerate}
		\procedure{2.06}
			\objective
				Choose a $\mathbb{Q}[x]$ $p=p_0x^n+p_1x^{n-1}+p_2x^{n-2}+\cdots+p_nx^0$, where $p_0\ne 0$. Choose a $\mathbb{Q}$ $k>1+\max_{i=1}^{n+1}\lvert\frac{p_i}{p_0}\rvert$. The objective of the following instructions is to show that $\sgn(p(k))=\sgn(p_0)$.
			\implementation
				\begin{enumerate}
					\item In reverse order verify the following:
					\begin{enumerate}
						\item \textbf{$\sgn(p_0k^n+p_1k^{n-1}+\cdots+p_nk^0)=\sgn(p_0)$}
						\item $\sgn(k^n+\frac{p_1}{p_0}k^{n-1}+\cdots+\frac{p_n}{p_0}k^0)=1$
						\item $k^n+\frac{p_1}{p_0}k^{n-1}+\cdots+\frac{p_n}{p_0}k^0>0$
						\item $k^n>-(\frac{p_1}{p_0}k^{n-1}+\cdots+\frac{p_n}{p_0}k^0)$
						\item $k^n>\lvert \frac{p_1}{p_0}k^{n-1}+\cdots+\frac{p_n}{p_0}k^0\rvert$
						\item $k^n>\lvert\max_{i=1}^{n+1}\lvert \frac{p_i}{p_0}\rvert(k^{n-1}+\cdots+k^0)\rvert$
						\item $k^n>\max_{i=1}^{n+1}\lvert \frac{p_i}{p_0}\rvert\frac{k^n-1}{k-1}$
						\item $k^{n+1}-k^n>\max_{i=1}^{n+1}\lvert \frac{p_i}{p_0}\rvert(k^n-1)$
						\item $k^{n+1}-(1+\max_{i=1}^{n+1}\lvert \frac{p_i}{p_0}\rvert)k^n+\max_{i=1}^{n+1}\lvert \frac{p_i}{p_0}\rvert>0$
						\item $k>1+\max_{i=1}^{n+1}\lvert \frac{p_i}{p_0}\rvert$
					\end{enumerate}
				\end{enumerate}
		\procedure{2.07}
			\objective
				Choose a $\mathbb{Q}[x]$ $p=p_0x^t+p_1x^{t-1}+p_2x^{t-2}+\cdots+p_tx^0$, where $p_0\ne 0$. Choose a $\mathbb{Q}$ $k<-(1+\max_{i=1}^{t+1}\lvert\frac{p_i}{p_0}\rvert)$. The objective of the following instructions is to show that $\sgn(p(k))=(-1)^t\sgn(p_0)$.
			\implementation
				\begin{enumerate}
					\item Let $q=q_0x^t+q_1x^{t-1}+q_2x^{t-2}+\cdots+q_tx^0$, where $q_i=(-1)^ip_i$.
					\item Verify that $k<-(1+\max_{i=1}^{t+1}\lvert\frac{q_i}{q_0}\rvert)$.
					\item Therefore verify that $-k>1+\max_{i=1}^{t+1}\lvert\frac{q_i}{q_0}\rvert$.
					\item Execute \procedurehr{2.06} on $q$ and $-k$.
					\item Hence verify that $(-1)^t\sgn(p(k))$
					\begin{enumerate}
						\item $=\sgn((-1)^tp(k))$
						\item $=\sgn((-1)^t\sum_{i=0}^{t+1} p_ik^{t-i})$
						\item $=\sgn(\sum_{i=0}^{t+1} (-1)^i(-1)^{t-i}p_ik^{t-i})$
						\item $=\sgn(\sum_{i=0}^{t+1} q_i(-k)^{t-i})$
						\item $=\sgn(q(-k))$
						\item $=\sgn(q_0)$
						\item $=\sgn(p_0)$.
					\end{enumerate}
					\item \textbf{Therefore verify that $\sgn(p(k))=(-1)^t(-1)^t\sgn(p(k))=(-1)^t\sgn(p_0)$.}
				\end{enumerate}
		\procedure{2.08}
			\objective
				Choose a list of $\mathbb{Q}[x]$s, $s$, and $\mathbb{Q}$s $a,l,c$ such that $a<c$ and $l>0$. The objective of the following instructions is to either show that $0<0$ or to construct a list of $\mathbb{Q}$s, $b$, such that $a=b_0<b_1<\cdots<b_{\lvert b\rvert-1}=c$, $b_{i}-b_{i-1}\le l$ for $i$ in $[1:\lvert b\rvert]$, and $0\not\in s(b_i)$ for $i$ in $[1:\lvert b\rvert-1]$.
			\implementation
				\begin{enumerate}
					\item Let $e=\langle\langle\rangle,\langle\rangle,\cdots,\langle\rangle\rangle$.
					\item Let $f=\sum_{r=0}^{\lvert s\rvert}\deg(s_r)$.
					\item Let $b=\langle a\rangle$.
					\item Let $d=b_1$.
					\item While $d+l<c$, do the following:
					\begin{enumerate}
						\item Let $m=l$.
						\item While $0\in s(d+m)$ and $\sum\lvert e\rvert\le f$, do the following:
						\begin{enumerate}
							\item Let $0\le i<\lvert s\rvert$ be an integer such that $s_i(d+m)=0$.
							\item Append $d+m$ onto $e_i$.
							\item Set $m=\frac{m}{2}$
						\end{enumerate}
						\item If $\sum\lvert e\rvert>f$, then do the following:
						\begin{enumerate}
							\item If $\lvert e_i\rvert\le\deg(s_i)$ for $0\le i<\lvert s\rvert$, then do the following:
							\begin{enumerate}
								\item Verify that $\sum\lvert e\rvert\le f$.
								\item Therefore using (c), verify that $\sum\lvert e\rvert\le f<\sum\lvert e\rvert$.
								\item \textbf{Abort procedure.}
							\end{enumerate}
							\item Otherwise, do the following:
							\begin{enumerate}
								\item Let $0\le i<\lvert s\rvert$ be an integer such that $\lvert e_i\rvert>\deg(s_i)$.
								\item Execute \procedurehr{2.01} on $s_i$ and a sorted $e_i$.
								\item \textbf{Abort procedure.}
							\end{enumerate}
						\end{enumerate}
						\item Otherwise, do the following:
						\begin{enumerate}
							\item \textbf{Verify that $0\not\in s(d+m)$.}
							\item Append $d+m$ onto $b$.
							\item \textbf{Verify that $0<b_{\lvert b\rvert-1}-b_{\lvert b\rvert-2}=m\le l$.}
							\item Set $d$ to $d+m$.
							\item Using (5), verify that $d<c$.
						\end{enumerate}
					\end{enumerate}
					\item Verify that $d<c$.
					\item Verify that $d+l\ge c$.
					\item \textbf{Therefore verify that $0<c-d\le l$.}
					\item Append $c$ onto $b$.
					\item \textbf{Yield $\langle b\rangle$.}
				\end{enumerate}
		\procedure{2.09}
			\objective
				Execute \procedurehr{2.04} and let $\langle s,q,g,h\rangle$ receive. Let $m=\lvert s\rvert-1$. The objective of the following instructions is to either show that $0<0$ or to construct two lists of rational numbers $c,d$ such that $c_0<d_0\le c_1<d_1\le\cdots\le c_{m-1}<d_{m-1}$ and $\sgn(s_m(c_i))=-\sgn(s_m(d_i))$ for $i$ in $[0:m]$.
			\implementation
				\begin{enumerate}
					\item Let $U=1+\max_{i=0}^{\lvert s\rvert}\left(1+\max_{j=1}^{i+1}\lvert\frac{x^{i-j}\circ s_i}{x^i\circ s_i}\rvert\right)$
					\item Using \procedurehr{2.06}, verify that $J(U)=0$.
					\item Using \procedurehr{2.07}, verify that $J(-U)=m$.
					\item Execute \procedurehr{2.00} on $s$ and let $\langle G\rangle$ receive the result.
					\item Let the rational $B=\max_{i=0}^{\lvert s\rvert}\lvert G_i\rvert(U,U)$.
					\item Let $C=\max_{i=0}^{\lvert s\rvert-1} \max(\lvert g_i\rvert(U),\lvert h_i\rvert(U))$.
					\item Let $D=\max(2, \max_{i=1}^{\lvert s\rvert-1} \lvert q_i\rvert(U))$.
					\item Let $l=\frac{1}{BCD}$.
					\item Execute \procedurehr{2.08} on $s$ with endpoints $-U,U$ and a step size of $l$ and let $\langle e\rangle$ receive the result.
					\item Let $c=d=\langle\rangle$.
					\item For $i=1$ to $i=\lvert e\rvert-1$:
					\begin{enumerate}
						\item Execute \procedurehr{2.05} on the tuple $\langle e_{i-1},e_{i}\rangle$.
						\item If $J_m(c)\ne J_m(d)$, then do the following:
						\begin{enumerate}
							\item Append $e_{i-1}$ to $c$.
							\item Append $e_{i}$ to $d$.
							\item Using (a) and (b), verify that $\lvert J_m(d)-J_m(c)\rvert=1$.
							\item Therefore verify that $\sgn(s_m(c_{\lvert c\rvert-1}))=-\sgn(s_m(d_{\lvert d\rvert-1}))$.
							\item Also verify that $d_{\lvert d\rvert-2}\le c_{\lvert c\rvert-1}<d_{\lvert d\rvert-1}$.
						\end{enumerate}
					\end{enumerate}
					\item If less than $m$ pairs of rational numbers were recorded, then do the following:
					\begin{enumerate}
						\item Verify that each change of $J_m(x)$ over the course of (12) was by $1$.
						\item Verify that $J_m(x)$ changed less than $m$ times over the course of (12).
						\item Therefore verify that $\lvert J_m(U)-J_m(-U)\rvert<m$.
						\item Therefore using (2) and (3), verify that $m=\lvert J_m(U)-J_m(-U)\rvert<m$.
						\item \textbf{Abort procedure.}
					\end{enumerate}
					\item Otherwise, do the following:
					\begin{enumerate}
						\item Verify that $m\le\lvert c\rvert=\lvert d\rvert$.
						\item \textbf{Yield the tuple $\langle c,d\rangle$.}
					\end{enumerate}
				\end{enumerate}
	\section{Matrix Arithmetic}
		\notation{3.00}
			Let us use the notation $\mathcal{M}_{m,n}(A)$ as a shorthand for "$m\times n$ matrix of $A$s".
		\procedure{3.00}
			\objective
				Choose a $\mathcal{M}_{m,2}(\mathbb{Q}[x])$, $A$. Let $\deg(0)=\infty$. Let $k=\min(\deg(A_{0,0}),\deg(A_{0,1}))$ and $q=\deg(A_{0,0})$. The objective of the following instructions is to make $A_{0,1}=0$, $\deg(A_{0,0})\le k$, and either leave $A_{*,0}$ unchanged or make $\deg(A_{0,0})<q$ by a sequence of operations whereby, in each step a $\mathbb{Q}[x]$ times either of the columns is added to the other.
			\implementation
				\begin{enumerate}
					\item Let $A$ be our working matrix.
					\item While $A_{0,1}\ne 0$, do the following:
					\begin{enumerate}
						\item If $\deg(A_{0,0})\le\deg(A_{0,1})$, then:
						\begin{enumerate}
							\item Subtract $\frac{x^{\deg(A_{0,1})}\circ A_{0,1}}{x^{\deg(A_{0,0})}\circ A_{0,0}}x^{\deg(A_{0,1})-\deg(A_{0,0})}$ times $A_{0,0}$ from $A_{0,1}$.
							\item Now verify that either $A_{0,1}$'s degree has decreased or $A_{0,1}=0$.
						\end{enumerate}
						\item Otherwise, do the following:
						\begin{enumerate}
							\item Let $p=\frac{x^{\deg(A_{0,0})}\circ A_{0,0}}{x^{\deg(A_{0,1})}\circ A_{0,1}}x^{\deg(A_{0,0})-\deg(A_{0,1})}$.
							\item If $A_{0,0}=pA_{0,1}$, then do the following:
							\begin{enumerate}
								\item Add $1-p$ times $A_{0,1}$ to $A_{0,0}$.
								\item Verify that now $A_{0,0}=A_{0,1}$.
							\end{enumerate}
							\item Otherwise, do the following:
							\begin{enumerate}
								\item Verify that $A_{0,0}\ne pA_{0,1}$.
								\item Add $-p$ times $A_{0,1}$ to $A_{0,0}$.
							\end{enumerate}
							\item Therefore verify that $A_{0,0}\ne 0$.
							\item Also verify that $A_{0,0}$'s degree has decreased.
						\end{enumerate}
					\end{enumerate}
					\item \textbf{Verify that $A_{0,1}=0$.}
					\item Verify that the changes to $A_{0,0}$, if any, have decreased its degree.
					\item If sensical, do the following:
					\begin{enumerate}
						\item Verify that all changes to $A_{0,1}$ but the last have decreased its degree.
						\item Verify that $\deg(A_{0,0})\le$ the degree of the penultimate value of $A_{0,1}$.
					\end{enumerate}
					\item \textbf{Therefore verify that $\deg(A_{0,0})\le k$.}
					\item If $A_{*,0}$ was changed, then do the following:
					\begin{enumerate}
						\item Verify that $A_{0,0}$ was also changed.
						\item \textbf{Therefore verify that $\deg(A_{0,0})<q$.}
					\end{enumerate}
					\item \textbf{Yield the tuple $\langle A\rangle$.}
				\end{enumerate}
		\notation{3.01}
			Let us use the notation "diagonal" as a shorthand for "matrix positions such that the row index equals the column index".
		\notation{3.02}
			Let us use the notation $\mathcal{D}_{m,n}(A)$ as a shorthand for "$\mathcal{M}_{m,n}(A)$ with $0$s in all the off-diagonal positions".
		\procedure{3.01}
			\objective
				Choose a $\mathcal{M}_{m,n}(\mathbb{Q}[x])$, $A$. The objective of the following instructions is to transform $A$ into a $\mathcal{D}_{m,n}(\mathbb{Q}[x])$ by a sequence of operations whereby either a $\mathbb{Q}[x]$ times any of the columns is added to a different column, or a $\mathbb{Q}[x]$ times any of the rows is added to a different row.
			\implementation
				\begin{enumerate}
					\item If $m=0$ or $n=0$, then do the following:
					\begin{enumerate}
						\item \textbf{Verify that $A$ is a $\mathcal{D}_{m,n}(\mathbb{Q}[x])$.}
						\item \textbf{Yield the tuple $\langle A\rangle$.}
					\end{enumerate}
					\item Otherwise do the following:
					\item Verify that $m>0$ and $n>0$.
					\item Let $A$ be our working matrix.
					\item Now do the following:
						\begin{enumerate}
						\item While $A_{0,[1:n]}\ne 0$, do the following:
						\begin{enumerate}
							\item Select the $\mathcal{M}_{m,2}(\mathbb{Q}[x])$ whose top-right entry coincides with the last non-zero entry of the first row
							\item Apply \procedurehr{3.00} on this submatrix.
							\item Verify that the top-left and top-right entries of the submatrix are now non-zero and zero respectively.
							\item If $A_{*,0}$ was modified by (5aii), then do the following:
							\begin{enumerate}
								\item Verify that $\deg(A_{1,1})$ decreased.
								\item Go back to (5).
							\end{enumerate}
						\end{enumerate}
						\item Now do the same operations as in (a), but this time with the operations themselves reflected across the matrix's diagonal.
					\end{enumerate}
					\item Verify that $A_{0,[1:n]}=0$.
					\item Also verify that $A_{[1:m],0}=0$.
					\item Apply \procedurehr{3.01} on the submatrix $A_{[1:m],[1:n]}$.
					\item Verify that (8)'s execution leaves the first row and column unchanged.
					\item Also verify that $A_{[1:m],[1:n]}$ is now a $\mathcal{D}_{m-1,n-n}(\mathbb{Q}[x])$.
					\item \textbf{Therefore verify that $A$ is now a $\mathcal{D}_{m,n}(\mathbb{Q}[x])$.}
					\item \textbf{Yield the tuple $\langle A\rangle$.}
				\end{enumerate}
		\procedure{3.02}
			\objective
				Choose a $\mathcal{M}_{m,n}(\mathbb{Q}[x])$, A, a $\mathcal{M}_{n,p}(\mathbb{Q}[x])$, B, and a $\mathcal{M}_{p,q}(\mathbb{Q}[x])$, C. The objective of the following instructions is to show that $(AB)C=A(BC)$.
			\implementation
				\begin{enumerate}
					\item Verify that $(AB)_{i,l}=\sum_{k=0}^{n} \left(A_{i,k}*B_{k,l}\right)$ for $0\le i<m$, for $0\le l<p$.
					\item Verify that $((AB)C)_{i,r}=\sum_{l=0}^{p} \left((AB)_{i,l}*C_{l,r}\right)=\sum_{l=0}^{p} \left(\sum_{k=0}^{n} \left(A_{i,k}*B_{k,l}\right)*C_{l,r}\right)$ for $0\le i<m$, for $0\le r<q$.
					\item Verify that $(BC)_{k,r}=\sum_{l=0}^{p}\left(B_{k,l}*C_{l,r}\right)$ for $0\le k<n$, for $0\le r<q$.
					\item Verify that $(A(BC))_{i,r}=\sum_{k=0}^{n}\left(A_{i,k}*(BC)_{k,r}\right)=\sum_{k=0}^{n}\left(A_{i,k}*\sum_{l=0}^{p}\left(B_{k,l}*C_{l,r}\right)\right)$ for $0\le i<m$, for $0\le r<q$.
					\item Therefore verify that $(2)=\sum_{l=0}^{p} \left(\sum_{k=0}^{n} \left(A_{i,k}*B_{k,l}*C_{l,r}\right)\right)=\sum_{k=0}^{n} \left(\sum_{l=0}^{p} \left(A_{i,k}*B_{k,l}*C_{l,r}\right)\right)=\sum_{k=0}^{n}\left(A_{i,k}*\sum_{l=0}^{p}\left(B_{k,l}*C_{l,r}\right)\right)=(4)$ for $0\le i<m$, for $0\le r<q$.
					\item \textbf{Therefore verify that $(AB)C=A(BC)$.}
				\end{enumerate}
		\notation{3.03}
			Let us use the notation $I_n$ as a shorthand for "the $\mathcal{M}_{n,n}(\mathbb{Q})$ with only $1$s on the diagonal and $0$s everywhere else".
		\notation{3.04}
			Let us use the notation $\mathcal{T}_{m}(\mathbb{Q}[x])$ as a shorthand for "$\mathcal{M}_{m,m}(\mathbb{Q}[x])$ with only $1$s on the diagonal, a single $\mathbb{Q}[x]$ off the diagonal, and $0$s everywhere else".
		\procedure{3.03}
			\objective
				Choose a procedure, $A$, and two non-negative integers $m,n$. The objective of the following instructions is, once $A$ has been executed, to construct a list of $\mathcal{T}_{m}(\mathbb{Q}[x])$s, $M$, and a list of $\mathcal{T}_{n}(\mathbb{Q}[x])$s, $N$ such that $M_{\lvert M\rvert-1-i}$ equals $I_m$ after applying the $i^{th}$ row operation carried out by $A$ also on it, and $N_i$ equals $I_n$ after applying the $i^{th}$ row operation carried out by $A$ also on it.
			\implementation
				\begin{enumerate}
					\item Make an empty list, $N$.
					\item Augment procedure $A$ so that each time a polynomial $x$ times a column $i$ is added onto column $j$, an $n\times n$ matrix that only has $1$s on its diagonal, and such that the only non-zero entry off its diagonal is $x$ at position $(i,j)$, is appended onto $N$.
					\item Make an empty list, $M$.
					\item Also augment procedure $A$ so that each time a polynomial $x$ times a row $i$ is added onto row $j$, an $n\times n$ matrix that only has $1$s on its diagonal, and such that the only non-zero entry off its diagonal is $x$ at position $(j,i)$, is prepended onto $M$.
					\item Now run procedure $A$.
					\item \textbf{Yield the tuple $\langle M,N\rangle$.}
				\end{enumerate}
		\procedure{3.04}
			\objective
				Choose a $\mathcal{M}_{m,n}(\mathbb{Q}[x])$, $A$. The objective of the following instructions is to show that $I_mA=A=AI_n$.
			\implementation
				\begin{enumerate}
					\item For $0\le r<m$, do the following:
					\begin{enumerate}
						\item For $0\le t<n$, do the following:
						\begin{enumerate}
							\item Verify that $(I_mA)_{r,t}=\sum_{u=0}^m (I_m)_{r,u}A_{u,t}=(I_m)_{r,r}A_{r,t}=1*A_{r,t}=A_{r,t}$.
						\end{enumerate}
					\end{enumerate}
					\item \textbf{Therefore verify that $I_mA=A$.}
					\item For $0\le r<m$, do the following:
					\begin{enumerate}
						\item For $0\le t<n$, do the following:
						\begin{enumerate}
							\item Verify that $(AI_n)_{r,t}=\sum_{u=0}^m A_{r,u}(I_n)_{u,t}=A_{r,t}(I_n)_{t,t}=A_{r,t}*1=A_{r,t}$.
						\end{enumerate}
					\end{enumerate}
					\item \textbf{Therefore verify that $AI_n=A$.}
				\end{enumerate}
		\procedure{3.05}
			\objective
				Choose a list of $\mathcal{T}_{m}(\mathbb{Q}[x])$, $A$. The objective of the following instructions is to construct a list of $\mathcal{T}_{m}(\mathbb{Q}[x])$, $A^{-1}$, such that $A_*{A^{-1}}_*=I_m$.
			\implementation
				\begin{enumerate}
					\item Let $A^{-1}$ be $\langle\rangle$.
					\item For $i$ in $[0:\lvert A\rvert]$, do the following:
					\begin{enumerate}
						\item Let $(j,k)$ be the position of the off diagonal entry of $A_i$.
						\item Let $B$ equal $A_i$ but with entry $(j,k)$ negated.
						\item For $r$ in $[0:m]$ and $r\ne j$, do the following:
						\begin{enumerate}
							\item For $t$ in $[0:m]$, do the following:
							\begin{enumerate}
								\item Verify that $(A_iB)_{r,t}=\sum_{u=0}^m (A_i)_{r,u}B_{u,t}=(A_i)_{r,r}B_{r,t}=1*B_{r,t}=[r=t]$.
							\end{enumerate}
						\end{enumerate}
						\item For $t$ in $[0:m]$ and $t\ne k$, do the following:
						\begin{enumerate}
							\item Verify that $(A_iB)_{j,t}=\sum_{u=0}^m (A_i)_{j,u}B_{u,t}=(A_i)_{j,t}B_{t,t}=(A_i)_{j,t}*1=[j=t]$.
						\end{enumerate}
						\item Verify that $(A_iB)_{j,k}=\sum_{u=0}^m (A_i)_{j,u}B_{u,k}=(A_i)_{j,j}B_{j,k}+(A_i)_{j,k}B_{k,k}=1*B_{j,k}+(A_i)_{j,k}*1=B_{j,k}+(A_i)_{j,k}=0$.
						\item Therefore verify that $A_iB=I_m$.
						\item Now prepend $B$ onto $A^{-1}$.
					\end{enumerate}
					\item Verify that $\lvert A\rvert=\lvert A^{-1}\rvert$.
					\item Therefore using \procedurehr{3.02} and \procedurehr{3.04}, verify that $A_*{A^{-1}}_*$
					\begin{enumerate}
						\item $=A_0\cdots A_{\lvert A\rvert-2}A_{\lvert A\rvert-1}{A^{-1}}_0{A^{-1}}_1\cdots {A^{-1}}_{\lvert A\rvert-1}$
						\item $=A_0\cdots A_{\lvert A\rvert-3}A_{\lvert A\rvert-2}I_m{A^{-1}}_1{A^{-1}}_2\cdots {A^{-1}}_{\lvert A\rvert-1}$
						\item $=A_0\cdots A_{\lvert A\rvert-3}A_{\lvert A\rvert-2}{A^{-1}}_1{A^{-1}}_2\cdots {A^{-1}}_{\lvert A\rvert-1}$
						\item $\vdots$
						\item $=A_0I_m{A^{-1}}_{\lvert A\rvert-1}$
						\item $=A_0{A^{-1}}_{\lvert A\rvert-1}$
						\item $=I_m$.
					\end{enumerate}
				\end{enumerate}
		\notation{3.05}
			Let us use the notation $A^{-1}$ as a shorthand for the result yielded by executing \procedurehr{3.05} on $A$.
		\procedure{3.06}
			\objective
				Choose a list of $\mathcal{T}_{m}(\mathbb{Q}[x])$, $A$. The objective of the following instructions is to show that $(A^{-1})^{-1}=A$ and ${A^{-1}}_*A_*=I_m$.
			\implementation
				\begin{enumerate}
					\item \textbf{Verify that $(A^{-1})^{-1}=A$.}
					\item \textbf{Therefore using \procedurehr{3.05}, verify that ${A^{-1}}_*A_*={A^{-1}}_*{(A^{-1})^{-1}}_*=I_m$.}
				\end{enumerate}
		\procedure{3.07}
			\objective
				Choose a $\mathcal{D}_{2,2}(\mathbb{Q}[x])$, $A$. The objective of the following instructions is to construct polynomials $u,v$ and transform $A$ into a $\mathcal{D}_{2,2}(\mathbb{Q}[x])$, $A'$, such that $A'_{1,1}=uA'_{0,0}$ and $A_{0,0}=vA'_{0,0}$ by a sequence of operations whereby either a $\mathbb{Q}[x]$ times any of the columns is added to a different column, or a $\mathbb{Q}[x]$ times any of the rows is added to a different row.
			\implementation
				\begin{enumerate}
					\item Add row $1$ to row $0$.
					\item Now verify that $A_{0,1}=A_{1,1}$.
					\item Set $A'=A$ and let $A'$ be our working matrix.
					\item Let $\langle M,N\rangle$ receive the results of executing \procedurehr{3.03} on the pair $\langle 2,2\rangle$ and the following procedure:
					\begin{enumerate}
						\item Execute \procedurehr{3.00} on $A'$.
					\end{enumerate}
					
					\item Using (4), verify that $M$ is empty.
					\item Using (4) and (5), verify that $AN_*=M_*AN_*=A'$.
					\item Using (6), verify that $A=AI_n=AN_*{N^{-1}}_*=A'{N^{-1}}_*$.
					\item Using (4), verify that $A'_{0,1}=0$.
					\item \textbf{Using (4) and (7), verify that $A_{0,0}=A'_{0,0}{{N^{-1}}_*}_{0,0}+A'_{0,1}{{N^{-1}}_*}_{1,0}=A'_{0,0}{{N^{-1}}_*}_{0,0}$.}
					\item Using (4) and (7), verify that $A_{1,1}=A_{0,1}=A'_{0,0}{{N^{-1}}_*}_{0,1}+A'_{0,1}{{N^{-1}}_*}_{1,1}=A'_{0,0}{{N^{-1}}_*}_{0,1}$.
					\item Using (2), verify that $A_{1,0}=0$.
					\item Using (6) and (11), verify that $A'_{1,0}=A_{1,0}{N_*}_{0,0}+A_{1,1}{N_*}_{1,0}=A_{1,1}{N_*}_{1,0}=A'_{0,0}{{N^{-1}}_*}_{0,1}{N_*}_{1,0}$.
					\item \textbf{Using (6) and (11), verify that $A'_{1,1}=A_{1,0}{N_*}_{0,1}+A_{1,1}{N_*}_{1,1}=A_{1,1}{N_*}_{1,1}=A'_{0,0}{{N^{-1}}_*}_{0,1}{N_*}_{1,1}$.}
					\item Subtract ${{N^{-1}}_*}_{0,1}{N_*}_{1,0}$ times row $0$ from row $1$.
					\item Now using (14) and (12), verify that $A'_{1,0}=0$.
					\item \textbf{Therefore verify that $A'$ is a $\mathcal{D}_{2,2}(\mathbb{Q}[x])$.}
					\item \textbf{Let $A=A'$.}
					\item \textbf{Yield $\langle{{N^{-1}}_*}_{0,1}{N_*}_{1,1},{{N^{-1}}_*}_{0,0}\rangle$.}
				\end{enumerate}
		\procedure{3.08}
			\objective
				Choose a $\mathcal{M}_{m,n}(\mathbb{Q}[x])$, $A$ such that $\min(m,n)>0$. The objective of the following instructions is to define a list of polynomials $u$ and transform $A$ into a $\mathcal{D}_{m,n}(\mathbb{Q}[x])$ such that $A_{k,k}=u_kA_{0,0}$ for $k$ in $[0:\min(m,n)]$ by a sequence of operations whereby either a $\mathbb{Q}[x]$ times any of the columns is added to a different column, or a $\mathbb{Q}[x]$ times any of the rows is added to a different row.
			\implementation
				\begin{enumerate}
					\item Let $u=\langle 1\rangle$.
					\item Execute \procedurehr{3.01} on $A$.
					\item Verify that $A$ is a $\mathcal{D}_{m,n}(\mathbb{Q}[x])$.
					\item For $j$ in $[1:\min(m,n)]$, do the following:
					\begin{enumerate}
						\item Using (h), verify that $A_{k,k}=u_kA_{0,0}$ for $k$ in $[0:j]$.
						\item Set $A'=A$.
						\item Execute \procedurehr{3.07} on $A'_{\langle 0,j\rangle,\langle 0,j\rangle}$ and let $\langle u_j,v\rangle$ receive.
						\item Using (c), verify that $A$ and $A'$ are the same modulo positions $\langle 0,0\rangle$ and $\langle j,j\rangle$.
						\item Therefore verify that $A'$ is a $\mathcal{D}_{m,n}(\mathbb{Q}[x])$.
						\item Also, using (c), verify that $A'_{j,j}=u_jA'_{0,0}$.
						\item Also, for $k$ in $[1:j]$, do the following:
						\begin{enumerate}
							\item Using (a), (c), and (d), verify that $A'_{k,k}=A_{k,k}=u_kA_{0,0}=u_kA'_{0,0}v$.
							\item Set $u_k=u_kv$.
							\item Hence verify that $A'_{k,k}=u_kA'_{0,0}$.
						\end{enumerate}
						\item Therefore verify that $A_{k,k}=u_kA_{0,0}$ for $k$ in $[0:j+1]$.
						\item Now let $A=A'$.
					\end{enumerate}
					\item \textbf{Hence using (4h), verify that $A_{k,k}=u_kA_{0,0}$ for $k$ in $[0:\min(m,n)]$.}
					\item \textbf{Also, using (4e), verify that $A$ is a $\mathcal{D}_{m,n}(\mathbb{Q}[x])$.}
					\item \textbf{Yield $\langle u\rangle$.}
				\end{enumerate}
		\procedure{3.09}
			\objective
				Choose a $\mathcal{M}_{m,n}(\mathbb{Q}[x])$, $A$, and a $\mathcal{M}_{n,k}(\mathbb{Q}[x])$, $B$. Choose integers $0\le a<m$, $0\le b<n$, and $0\le c<k$. The objective of the following instructions is to show that
				\begin{enumerate}
					\item $(AB)_{[0:a],[0:c]}=A_{[0:a],[0:b]}B_{[0:b],[0:c]}+A_{[0:a],[b:n]}B_{[b:n],[0:c]}$
					\item $(AB)_{[0:a],[c:k]}=A_{[0:a],[0:b]}B_{[0:b],[c:k]}+A_{[0:a],[b:n]}B_{[b:n],[c:k]}$
					\item $(AB)_{[a:m],[0:c]}=A_{[a:m],[0:b]}B_{[0:b],[0:c]}+A_{[a:m],[b:n]}B_{[b:n],[0:c]}$
					\item $(AB)_{[a:m],[c:k]}=A_{[a:m],[0:b]}B_{[0:b],[c:k]}+A_{[a:m],[b:n]}B_{[b:n],[c:k]}$.
				\end{enumerate}
			\implementation
				\begin{enumerate}
					\item For each $0\le i<a$, do the following:
					\begin{enumerate}
						\item For each $0\le j<c$, do the following:
							\begin{enumerate}
								\item Verify that $(AB)_{i,j}=\sum_{p=0}^n A_{i,p}B_{p,j}=\sum_{p=0}^{b} A_{i,p}B_{p,j}+\sum_{p=b}^n A_{i,p}B_{p,j}=\sum_{p=0}^{b} (A_{[0:a],[0:b]})_{i,p}(B_{[0:b],[0:c]})_{p,j}+\sum_{p=0}^{n-b} (A_{[0:a],[b:n]})_{i,p}(B_{[b:n],[0:c]})_{p,j}=(A_{[0:a],[0:b]}B_{[0:b],[0:c]})_{i,j}+(A_{[0:a],[b:n]}B_{[b:n],[0:c]})_{i,j}$.
							\end{enumerate}
					\end{enumerate}
					\item \textbf{Therefore verify that $(AB)_{[0:a],[0:c]}=A_{[0:a],[0:b]}B_{[0:b],[0:c]}+A_{[0:a],[b:n]}B_{[b:n],[0:c]}$.}
					\item \textbf{Using computations analogous to (1) and (2), show items (2), (3), and (4) of the objective.}
				\end{enumerate}
		\notation{3.13}
			Let us use the notation $\cols(A)$ as a shorthand for "the number of columns of $A$".
		\notation{3.14}
			Let us use the notation $\rows(A)$ as a shorthand for "the number of rows of $A$".
		\procedure{3.10}
			\objective
				Choose a list of $\mathcal{M}_{*}(\mathbb{Q})$, $C$. Let $m=\sum_{i=0}^{\lvert C\rvert}\cols(C_i)$. The objective of the following instructions is to construct a $\mathcal{M}_{m,m}(\mathbb{Q})$, $\bdiag(C)$.
			\implementation
				\begin{enumerate}
					\item Let $E$ be a $0\times 0$ matrices.
					\item Now for $i$ in $[0:\lvert C\rvert]$:
					\begin{enumerate}
						\item Add $\cols(C_i)$ columns filled with zeros to the right end of $E$.
						\item Add $\rows(C_i)$ rows filled with zeros to the bottom end of $E$.
						\item Set the bottom-right corner of $E$ equal to $C_i$.
					\end{enumerate}
					\item Verify that $\cols(E)=\sum_{i=0}^{\lvert C\rvert}\cols(C_i)=m$.
					\item \textbf{Yield the tuple $\langle E\rangle$.}
				\end{enumerate}
		\notation{3.15}
			Let us use the notation $\bdiag(C)$ as a shorthand for the result yielded by executing \procedurehr{3.10} on $C$.
		\procedure{3.11}
			\objective
				Choose a $\mathcal{M}_{m,n}(\mathbb{Q}[x])$, $A$. Let $A_{-1,-1}=1$. The objective of the following instructions is to construct the list of polynomials $v$ and transform $A$ into a $\mathcal{D}_{m,n}(\mathbb{Q}[x])$ such that $A_{k,k}=v_kA_{k-1,k-1}$ for $k$ in $[0:\min(m,n)]$ by a sequence of operations whereby either a $\mathbb{Q}[x]$ times any of the columns is added to a different column, or a $\mathbb{Q}[x]$ times any of the rows is added to a different row.
			\implementation
				\begin{enumerate}
					\item If $\min(m,n)=0$, then do the following:
					\begin{enumerate}
						\item \textbf{Verify that $A$ is a $\mathcal{D}_{m,n}(\mathbb{Q}[x])$.}
						\item \textbf{Yield $\langle\rangle$.}
					\end{enumerate}
					\item Otherwise do the following:
					\begin{enumerate}
						\item Apply \procedurehr{3.08} on $A$, and let $\langle u\rangle$ receive.
						\item Verify that $A$ is a $\mathcal{D}_{m,n}(\mathbb{Q}[x])$.
						\item Verify that $A_{k,k}=u_{k}A_{0,0}$ for $k$ in $[0:\min(m,n)]$.
						\item Let $B,C$ be a $\mathcal{D}_{m-1,n-1}(\mathbb{Q}[x])$ with $u_{1:\lvert u\rvert}$ on the diagonal.
						\item Let $\langle M,N\rangle$ receive the results of executing \procedurehr{3.03} on the pair $\langle m-1,n-1\rangle$ and the following procedure:
						\begin{enumerate}
							\item Execute \procedurehr{3.11} on $C$ and let $\langle w\rangle$ receive.
						\end{enumerate}
						\item Therefore verify that $C$ is a $\mathcal{D}_{m-1,n-1}(\mathbb{Q}[x])$.
						\item Also verify that $C=M_*BN_*$.
						\item Let $C_{-1,-1}=1$.
						\item Now using (ei), verify that $C_{k,k}=w_kC_{k-1,k-1}$ for $k$ in $[0:\min(m,n)-1]$.
						\item Therefore using (c), verify that $A_{0,0}C=M_*(A_{0,0}B)N_*=M_*A_{[1:m],[1:n]}N_*$.
						\item Premultiply $A$ by $\bdiag(1,M_k)$ for $k$ in $[\lvert M\rvert:0]$.
						\item Postmultiply $A$ by $\bdiag(1,N_k)$ for $k$ in $[0:\lvert N\rvert]$.
						\item Now verify that $A_{[1:m],[1:n]}=A_{0,0}C$.
						\item Now let $u=\langle A_{0,0}\rangle^{\frown}w$.
						\item \textbf{Therefore verify that $A_{k,k}=u_kA_{k-1,k-1}$ for $k$ in $[0:\min(m,n)]$.}
						\item \textbf{Yield the tuple $\langle u\rangle$.}
					\end{enumerate}
				\end{enumerate}
		\notation{3.06}
			Let us use the notation $\det(A)$ as a shorthand for the result yielded by executing \procedurehr{3.12} on $A$.
		\procedure{3.12}
			\objective
				Choose a $\mathcal{M}_{m,m}(\mathbb{Q}[x])$, $A$. The objective of the following instructions is to construct a $\mathbb{Q}[x]$, $\det(A)$.
			\implementation
				\begin{enumerate}
					\item If $m=0$, then do the following:
					\begin{enumerate}
						\item \textbf{Yield the tuple $\langle 1\rangle$.}
					\end{enumerate}
					\item Otherwise, do the following:
					\begin{enumerate}
						\item Let $h_r=A_{[0:r]^\frown[r+1,m],[1:m]}$ for $r$ in $[0:m]$.
						\item Verify that $h_r$ is a $\mathcal{M}_{m-1,m-1}(\mathbb{Q}[x])$ for $r$ in $[0:m]$.
						\item \textbf{Yield the tuple $\langle\sum_{r=0}^m (-1)^{r}A_{r,0}\det(h_r)\rangle$.}
					\end{enumerate}
				\end{enumerate}
		\procedure{3.13}
			\objective
				Choose a $\mathbb{Q}[x]$ $p$. Choose two $\mathcal{M}_{1,m}(\mathbb{Q}[x])$s, $B$ and $C$. Choose an integer $0\le i<m$. Choose a $\mathcal{M}_{m,m}(\mathbb{Q}[x])$, $A$, such that its $i^{th}$ row is $B+pC$. Let $A'$ be $A$ but with the $i^{th}$ row replaced by $B$ and let $A''$ be $A$ but with the $i^{th}$ row replaced by $C$. The objective of the following instructions is to show that $\det(A)=\det(A')+p\det(A'')$.
			\implementation
				\begin{enumerate}
					\item If $m=1$, then do the following:
					\begin{enumerate}
						\item Verify that $i=0$.
						\item \textbf{Therefore verify that $\det(A)=A_{0,0}=B_{0,0}+pC_{0,0}=\det(A')+p\det(A'')$.}
					\end{enumerate}
					\item Otherwise, do the following:
					\begin{enumerate}
						\item For $r$ in $[0:i]$, do the following:
						\begin{enumerate}
							\item Verify that $(A_{[0:r]^\frown[r+1:m],[1:m]})_{i-1,*}=B+pC$.
							\item Verify that $A'_{[0:r]^\frown[r+1:m],[1:m]}$ is $A_{[0:r]^\frown[r+1:m],[1:m]}$ with row $i-1$ replaced by $B$.
							\item Verify that $A''_{[0:r]^\frown[r+1:m],[1:m]}$ is $A_{[0:r]^\frown[r+1:m],[1:m]}$ with row $i-1$ replaced by $C$.
							\item Execute \procedurehr{3.13} on $\langle p,B,C,i-1,A_{[0:r]^\frown[r+1:m],[1:m]}\rangle$.
							\item Therefore verify that $\det(A_{[0:r]^\frown[r+1:m],[1:m]})=\det(A'_{[0:r]^\frown[r+1:m],[1:m]})+p\det(A''_{[0:r]^\frown[r+1:m],[1:m]})$.
						\end{enumerate}
						\item For $r$ in $[i+1:m]$, do the following:
						\begin{enumerate}
							\item Verify that $(A_{[0:r]^\frown[r+1:m],[1:m]})_{i,*}=B+pC$.
							\item Verify that $A'_{[0:r]^\frown[r+1:m],[1:m]}$ is $A_{[0:r]^\frown[r+1:m],[1:m]}$ with row $i$ replaced by $B$.
							\item Verify that $A''_{[0:r]^\frown[r+1:m],[1:m]}$ is $A_{[0:r]^\frown[r+1:m],[1:m]}$ with row $i$ replaced by $C$.
							\item Execute \procedurehr{3.13} on $\langle p,B,C,i,A_{[0:r]^\frown[r+1:m],[1:m]}\rangle$.
							\item Therefore verify that $\det(A_{[0:r]^\frown[r+1:m],[1:m]})=\det(A'_{[0:r]^\frown[r+1:m],[1:m]})+p\det(A''_{[0:r]^\frown[r+1:m],[1:m]})$.
						\end{enumerate}
						\item Therefore using (av) and (bv), verify that $\det(A)$
						\begin{enumerate}
							\item $=\sum_{r=0}^m (-1)^rA_{r,0}\det(A_{[0:r]^\frown[r+1:m],[1:m]})$
							\item $=\sum_{r=0}^i (-1)^rA_{r,0}\det(A_{[0:r]^\frown[r+1:m],[1:m]})+(-1)^iA_{i,0}\det(A_{[0:i]^\frown[i+1:m],[1:m]})+\sum_{r=i+1}^m (-1)^rA_{r,0}\det(A_{[0:r]^\frown[r+1:m],[1:m]})$
							\item $=\sum_{r=0}^i (-1)^rA_{r,0}(\det(A'_{[0:r]^\frown[r+1:m],[1:m]})+p\det(A''_{[0:r]^\frown[r+1:m],[1:m]}))+(-1)^i(A'_{i,0}+pA''_{i,0})\det(A_{[0:i]^\frown[i+1:m],[1:m]})+\sum_{r=i+1}^m (-1)^rA_{r,0}(\det(A'_{[0:r]^\frown[r+1:m],[1:m]})+p\det(A''_{[0:r]^\frown[r+1:m],[1:m]}))$
							\item $=\sum_{r=0}^i (-1)^rA_{r,0}\det(A'_{[0:r]^\frown[r+1:m],[1:m]})+(-1)^iA'_{i,0}\det(A_{[0:i]^\frown[i+1:m],[1:m]})+\sum_{r=i+1}^m (-1)^rA_{r,0}\det(A'_{[0:r]^\frown[r+1:m],[1:m]})
								+\sum_{r=0}^i (-1)^rA_{r,0}p\det(A''_{[0:r]^\frown[r+1:m],[1:m]})+(-1)^ipA''_{i,0}\det(A_{[0:i]^\frown[i+1:m],[1:m]})+\sum_{r=i+1}^m (-1)^rA_{r,0}p\det(A''_{[0:r]^\frown[r+1:m],[1:m]})$
							\item $=\sum_{r=0}^m (-1)^rA'_{r,0}\det(A'_{[0:r]^\frown[r+1:m],[1:m]})+p\sum_{r=0}^m (-1)^rA''_{r,0}\det(A''_{[0:r]^\frown[r+1:m],[1:m]})$
							\item $=\det(A')+p\det(A'')$.
						\end{enumerate}
					\end{enumerate}
				\end{enumerate}
		\procedure{3.14}
			\objective
				Choose a $\mathbb{Q}[x]$ $p$. Choose two $\mathcal{M}_{m,1}(\mathbb{Q}[x])$s, $B$ and $C$. Choose an integer $0\le i<m$. Choose a $\mathcal{M}_{m,m}(\mathbb{Q}[x])$, $A$, such that its $i^{th}$ column is $B+pC$. Let $A'$ be $A$ but with the $i^{th}$ column replaced by $B$ and let $A''$ be $A$ but with the $i^{th}$ column replaced by $C$. The objective of the following instructions is to show that $\det(A)=\det(A')+p\det(A'')$.
			\implementation
				\begin{enumerate}
					\item If $i=0$, then verify that $\det(A)$
					\begin{enumerate}
						\item $=\sum_{r=0}^m (-1)^{r}A_{r,0}\det(A_{[0:r]^\frown[r+1:m],[1:m]})$
						\item $=\sum_{r=0}^m (-1)^{r}(B+pC)_{r,0}\det(A_{[0:r]^\frown[r+1:m],[1:m]})$
						\item $=\sum_{r=0}^m (-1)^{r}(B)_{r,0}\det(A_{[0:r]^\frown[r+1:m],[1:m]})+\sum_{r=0}^m (-1)^{r}(pC)_{r,0}\det(A_{[0:r]^\frown[r+1:m],[1:m]})$
						\item $=\sum_{r=0}^m (-1)^{r}(B)_{r,0}\det(A_{[0:r]^\frown[r+1:m],[1:m]})+p\sum_{r=0}^m (-1)^{r}(C)_{r,0}\det(A_{[0:r]^\frown[r+1:m],[1:m]})$
						\item $=\sum_{r=0}^m (-1)^{r}(A')_{r,0}\det(A'_{[0:r]^\frown[r+1:m],[1:m]})+p\sum_{r=0}^m (-1)^{r}(A'')_{r,0}\det(A''_{[0:r]^\frown[r+1:m],[1:m]})$
						\item $=\det(A')+p\det(A'')$
					\end{enumerate}
					\item Otherwise, do the following:
					\begin{enumerate}
						\item For $r$ in $[0:m]$, do the following:
						\begin{enumerate}
							\item Execute \procedurehr{3.14} on $\langle p,B_{[0:r]^\frown[r+1:m],0},C_{[0:r]^\frown[r+1:m],0},i-1,A_{[0:r]^\frown[r+1:m],[1:m]}\rangle$.
							\item Therefore verify that $\det(A_{[0:r]^\frown[r+1:m],[1:m]})=\det(A'_{[0:r]^\frown[r+1:m],[1:m]})+p\det(A''_{[0:r]^\frown[r+1:m],[1:m]})$.
						\end{enumerate}
						\item Therefore using (a), verify that $\det(A)$
						\begin{enumerate}
							\item $=\sum_{r=0}^m (-1)^{r}A_{r,0}\cdot\det(A_{[0:r]^\frown[r+1:m],[1:m]})$
							\item $=\sum_{r=0}^m (-1)^{r}A_{r,0}(\det(A'_{[0:r]^\frown[r+1:m],[1:m]})+p\det(A''_{[0:r]^\frown[r+1:m],[1:m]}))$
							\item $=\sum_{r=0}^m (-1)^{r}A'_{r,0}\det(A'_{[0:r]^\frown[r+1:m],[1:m]})+\sum_{r=0}^m (-1)^{r}A''_{r,0}p\det(A''_{[0:r]^\frown[r+1:m],[1:m]})$
							\item $=\det(A')+p\det(A'')$.
						\end{enumerate}
					\end{enumerate}
				\end{enumerate}
		\procedure{3.15}
			\objective
				Choose a $\mathcal{M}_{m,m}(\mathbb{Q}[x])$, $A$. Choose an integer $0<i<m$. Let $A'$ be $A$ with rows $i-1$ and $i$ swapped. The objective of the following instructions is to show that $\det(A')=-\det(A)$.
			\implementation
				\begin{enumerate}
					\item If $m=2$, then do the following:
					\begin{enumerate}
						\item Verify that $i=1$.
						\item Therefore verify that $\det(A')=A'_{0,0}A'_{1,1}-A'_{1,0}A'_{0,1}=A_{1,0}A_{0,1}-A_{0,0}A_{1,1}=-\det(A)$.
					\end{enumerate}
					\item Otherwise do the following:
					\begin{enumerate}
						\item For $r$ in $[0:i-1]$, do the following:
						\begin{enumerate}
							\item Verify that $A_{[0:r]^\frown[r+1:m],[1:m]}$ is the same as $A'_{[0:r]^\frown[r+1:m],[1:m]}$ but with rows $i-2$ and $i-1$ swapped.
							\item Execute \procedurehr{3.15} on $\langle A_{[0:r]^\frown[r+1:m],[1:m]},i-1\rangle$.
							\item Hence verify that $\det(A'_{[0:r]^\frown[r+1:m],[1:m]})=-\det(A_{[0:r]^\frown[r+1:m],[1:m]})$.
						\end{enumerate}
						\item For $r$ in $[i+1:m]$, do the following:
						\begin{enumerate}
							\item Verify that $A_{[0:r]^\frown[r+1:m],[1:m]}$ is the same as $A'_{[0:r]^\frown[r+1:m],[1:m]}$ but with rows $i-1$ and $i$ swapped.
							\item Execute \procedurehr{3.15} on $\langle A_{[0:r]^\frown[r+1:m],[1:m]},i\rangle$.
							\item Hence verify that $\det(A'_{[0:r]^\frown[r+1:m],[1:m]})=-\det(A_{[0:r]^\frown[r+1:m],[1:m]})$.
						\end{enumerate}
						\item Verify that $\det(A)$
						\begin{enumerate}
							\item $=\sum_{r=0}^m (-1)^rA_{r,0}\det(A_{[0:r]^\frown[r+1:m],[1:m]})$
							\item $=\sum_{r=0}^{i-1} (-1)^rA_{r,0}\det(A_{[0:r]^\frown[r+1:m],[1:m]})+(-1)^{i-1}A_{i-1,0}\det(A_{[0:i-1]^\frown[i:m],[1:m]})+(-1)^iA_{i,0}\det(A_{[0:i]^\frown[i+1:m],[1:m]})+\sum_{r=i+1}^m (-1)^rA_{r,0}\det(A_{[0:r]^\frown[r+1:m],[1:m]})$
							\item $=-\sum_{r=0}^{i-1} (-1)^rA'_{r,0}\det(A'_{[0:r]^\frown[r+1:m],[1:m]})-(-1)^{i}A'_{i,0}\det(A'_{[0:i]^\frown[i+1:m],[1:m]})-(-1)^{i-1}A'_{i-1,0}\det(A'_{[0:i-1]^\frown[i:m],[1:m]})-\sum_{r=i+1}^m (-1)^rA'_{r,0}\det(A'_{[0:r]^\frown[r+1:m],[1:m]})$
							\item $=-\sum_{r=0}^m (-1)^rA'_{r,0}\det(A'_{[0:r]^\frown[r+1:m],[1:m]})$
							\item $=-\det(A')$.
						\end{enumerate}
					\end{enumerate}
				\end{enumerate}
		\procedure{3.16}
			\objective
				Choose a $\mathcal{M}_{m,m}(\mathbb{Q}[x])$, $A$. Choose an integer $0<i<m$. Let $A'$ be $A$ with columns $i-1$ and $i$ swapped. The objective of the following instructions is to show that $\det(A')=-\det(A)$.
			\implementation
				\begin{enumerate}
					\item If $i=1$, then verify that $\det(A)$
					\begin{enumerate}
						\item $=\sum_{r=0}^m (-1)^{r}A_{r,0}\det(A_{[0:r]^\frown[r+1:m],[1:m]})$
						\item $=\sum_{r=0}^m (-1)^{r}A_{r,0}\sum_{t=r+1}^m (-1)^{t-1}A_{t,1}\cdot\det(A_{[0:r]^\frown[r+1:t]^\frown[t+1:m],[2:m]})+\sum_{t=0}^m (-1)^{t}A_{t,0}\sum_{r=0}^{t} (-1)^{r}A_{r,1}\cdot\det(A_{[0:r]^\frown[r+1:t]^\frown[t+1:m],[2:m+1]})$
						\item $=\sum_{t=0}^m (-1)^{t-1}A_{t,1}\sum_{r=0}^{t} (-1)^{r}A_{r,0}\cdot\det(A_{[0:r]^\frown[r+1:t]^\frown[t+1:m],[2:m+1]})+\sum_{r=0}^m (-1)^{r}A_{r,1}\sum_{t=r+1}^m (-1)^{t}A_{t,0}\cdot\det(A_{[0:r]^\frown[r+1:t]^\frown[t+1:m],[2:m+1]})$
						\item $=\sum_{t=0}^m (-1)^{t-1}A'_{t,0}\sum_{r=0}^{t} (-1)^{r}A'_{r,1}\cdot\det(A'_{[0:r]^\frown[r+1:t]^\frown[t+1:m],[2:m+1]})+\sum_{r=0}^m (-1)^{r}A'_{r,0}\sum_{t=r+1}^m (-1)^{t}A'_{t,1}\cdot\det(A'_{[0:r]^\frown[r+1:t]^\frown[t+1:m],[2:m]})$
						\item $=-(\sum_{r=0}^m (-1)^{r}A'_{r,0}\sum_{t=r+1}^m (-1)^{t-1}A'_{t,1}\cdot\det(A'_{[0:r]^\frown[r+1:t]^\frown[t+1:m],[2:m]})+\sum_{t=0}^m (-1)^{t}A'_{t,0}\sum_{r=0}^{t} (-1)^{r}A'_{r,1}\cdot\det(A'_{[0:r]^\frown[r+1:t]^\frown[t+1:m],[2:m]}))$
						\item $=-\det(A')$.
					\end{enumerate}
					\item Otherwise do the following:
					\begin{enumerate}
						\item Verify that $i>1$.
						\item For $r$ in $[0:m]$, do the following:
						\begin{enumerate}
							\item Execute \procedurehr{3.16} on $\langle i-1,A_{[0:r]^\frown[r+1:m],[1:m]}\rangle$.
							\item Therefore verify that $\det(A_{[0:r]^\frown[r+1:m],[1:m]})=-\det(A'_{[0:r]^\frown[r+1:m],[1:m]})$.
						\end{enumerate}
						\item Therefore using (bii), verify that $\det(A)=\sum_{r=0}^m (-1)^{r}A_{r,0}\cdot\det(A_{[0:r]^\frown[r+1:m],[1:m]})=\sum_{r=0}^m (-1)^{r}A'_{r,0}\cdot(-\det(A'_{[0:r]^\frown[r+1:m],[1:m]}))=-\det(A')$.
					\end{enumerate}
				\end{enumerate}
		\procedure{3.17}
			\objective
				Choose integers $0<i<m$. Choose a $\mathcal{M}_{m,m}(\mathbb{Q}[x])$, $A$, such that columns $i-1$ and $i$ are the same. The objective of the following instructions is to show that $\det(A)=0$.
			\implementation
				\begin{enumerate}
					\item Let $A'$ be $A$ with columns $i-1$ and $i$ swapped.
					\item Execute \procedurehr{3.16} on $\langle A, i\rangle$.
					\item Also, verify that $A'=A$.
					\item Therefore verify that $\det(A)=\det(A')=-\det(A)$.
					\item \textbf{Therefore verify that $\det(A)=0$.}
				\end{enumerate}
		\procedure{3.18}
			\objective
				Choose integers $0<i<m$. Choose a $\mathcal{M}_{m,m}(\mathbb{Q}[x])$, $A$, such that rows $i-1$ and $i$ are the same. The objective of the following instructions is to show that $\det(A)=0$.
			\implementation
				Instructions are analogous to those of \procedurehr{3.17}.
		\procedure{3.19}
			\objective
				Choose integers $0\le i<m$. Choose an integer $-i\le j<m-i$. Choose a $\mathcal{M}_{m,m}(\mathbb{Q}[x])$, $A$. Let $A'$ be $A$ but with column $i$ moved $j$ places. The objective of the following instructions is to show that $\det(A')=(-1)^j\det(A)$.
			\implementation
				\begin{enumerate}
					\item Let $B=\langle A\rangle$.
					\item For $k$ in $[i:i+j]$, do the following:
					\begin{enumerate}
						\item Let $B_{\lvert B\rvert}$ be the result of swapping columns $k$ and $k+1$ of $B_{\lvert B\rvert-1}$.
						\item Using \procedurehr{3.16}, verify that $\det(B_{\lvert B\rvert-1})=-\det(B_{\lvert B\rvert-2})$.
					\end{enumerate}
					\item Verify that $A'=B_{\lvert B\rvert-1}$.
					\item \textbf{Therefore verify that $\det(A')=\det(B_{\lvert B\rvert-1})=(-1)^1\det(B_{\lvert B\rvert-2})=\cdots=(-1)^j\det(B_{0})=(-1)^j\det(A)$.}
				\end{enumerate}
		\procedure{3.20}
			\objective
				Choose integers $0\le i<m$. Choose an integer $-i\le j<m-i$. Choose a $\mathcal{M}_{m,m}(\mathbb{Q}[x])$, $A$. Let $A'$ be $A$ but with row $i$ moved $j$ places. The objective of the following instructions is to show that $\det(A')=(-1)^j\det(A)$.
			\implementation
				Instructions are analogous to those of \procedurehr{3.19}.
		\procedure{3.21}
			\objective
				Choose a $\mathcal{M}_{m,n}(\mathbb{Q}[x])$, $A$, and choose an integer $0\le k\le\min(m,n)$. The objective of the following instructions is to construct a $\mathcal{M}_{\binom{m}{k},\binom{n}{k}}(\mathbb{Q}[x])$, $C_k(A)$.
			\implementation
				\begin{enumerate}
					\item Yield a tuple comprising the $\binom{m}{k}\times\binom{n}{k}$ matrix constructed as follows:
					\begin{enumerate}
						\item The rows are labeled by the colexicographically sorted list of increasing length-$k$ sequences whose elements are picked from $[0:m]$.
						\item The columns are labeled by the colexicographically sorted list of increasing length-$k$ sequences whose elements are picked from $[0:n]$.
						\item For each row label $I$: For each column label $J$: Let the entry at position $(I,J)$ be $\det(A_{I,J})$.
					\end{enumerate}
				\end{enumerate}
		\notation{3.07}
			We will use the notation $C_k(A)$ to refer to the result yielded by executing \procedurehr{3.21} on the matrix $A$ and integer $k$.
		\notation{3.08}
			We will use the notation $A_{\ul{I},\ul{J}}$ to refer to the entry of $A$ with row label $I$ and column label $J$.
		\procedure{3.22}
			\objective
				Choose two integers $0\le k\le m$. The objective of the following instructions is to show that $C_k(I_m)=I_{\binom{m}{k}}$.
			\implementation
				\begin{enumerate}
					\item For each row label $I$ of $C_k(I_m)$, for each column label $J$ of $C_k(I_m)$, do the following:
					\begin{enumerate}
						\item If $I=J$, then do the following:
						\begin{enumerate}
							\item Verify that $((I_m)_{I,J})_{i,j}=((I_m)_{J,J})_{i,j}=(I_m)_{J_i,J_j}=[J_i=J_j]=[i=j]$ for $0\le i<k$, for $0\le j<k$.
							\item Therefore verify that $(C_k(I_m))_{\ul{I},\ul{J}}=I_k$.
							\item \textbf{Therefore using \procedurehr{3.12}, verify that $(C_k(I_m))_{\ul{I},\ul{J}}=\det((I_m)_{I,J})=\det(I_k)=1$.}
						\end{enumerate}
						\item Otherwise, do the following:
						\begin{enumerate}
							\item Verify that $I\ne J$.
							\item Let $i$ be the index of an element of $I$ that is not an element of $J$.
							\item Now verify that $(I_m)_{I_i,j}=[I_i=j]=0$, for each $j$ in $J$.
							\item Therefore verify that $((I_m)_{I,J})_{i,*}=0_{1\times k}$.
							\item \textbf{Therefore using \procedurehr{3.12}, verify that $(C_k(I_m))_{\ul{I},\ul{J}}=\det((I_m)_{I,J})=0$.}
						\end{enumerate}
					\end{enumerate}
					\item \textbf{Therefore verify that $C_k(I_m)=I_{\binom{m}{k}}$.}	
				\end{enumerate}
		\procedure{3.23}
			\objective
				Choose an integer $0\le k\le\min(m,n)$. Choose a $\mathcal{T}_{m}(\mathbb{Q}[x])$, $A$, such that the off diagonal entry is the $\mathbb{Q}[x]$ $p$ at $(i,j)$. Also choose a $\mathcal{M}_{m,n}(\mathbb{Q}[x])$, $B$. The objective of the following instructions is to construct a $\mathcal{M}_{\binom{m}{k},\binom{m}{k}}(\mathbb{Q}[x])$ $D$ such that $C_k(AB)=DC_k(B)$.
			\implementation
				\begin{enumerate}
					\item Let $D=C_k(I_m)=I_{\binom{m}{k}}$.
					\item Verify that $AB$ equals $B$, but with its row $i$ having $p$ times $B$'s row $j$ added to it.
					\item Go through the row labels, $I$, of $C_k(AB)$ and do the following:
					\begin{enumerate}
						\item If $i\notin I$, then do the following:
						\begin{enumerate}
							\item Verify that $(AB)_{I,*}=B_{I,*}$.
							\item Therefore for each column label $J$, verify that ${C_k(AB)}_{\ul{I},\ul{J}}=\det((AB)_{I,J})=\det(B_{I,J})={C_k(B)}_{\ul{I},\ul{J}}$.
							\item \textbf{Therefore verify that $(C_k(AB))_{\ul{I},*}=(C_k(B))_{\ul{I},*}$.}
						\end{enumerate}
						\item Otherwise, if $i\in I$, then:
						\begin{enumerate}
							\item Let $I'$ be $I$ but with an in-place replacement of $i$ by $j$.
							\item For each column label $J$: Using \procedurehr{3.14}, verify that ${C_k(AB)}_{\ul{I},\ul{J}}=\det((AB)_{I,J})=\det(B_{I,J})+p*\det(B_{I',J})$.
							\item If $j\in I$, then do the following:
							\begin{enumerate}
								\item Verify that the sequence $I'$ contains two $j$s.
								\item For each column label $J$: Using \procedurehr{3.18} verify that $\det(B_{I',J})=0$.
								\item Therefore for each column label $J$: verify that ${C_k(AB)}_{\ul{I},\ul{J}}=\det(B_{I,J})={C_k(B)}_{\ul{I},\ul{J}}$.
								\item \textbf{Therefore verify that ${C_k(AB)}_{\ul{I},*}={C_k(B)}_{\ul{I},*}$.}
							\end{enumerate}
							\item Otherwise if $j\notin I$, do the following:
							\begin{enumerate}
								\item Let $l$ be the signed number of places that the $j$ introduced above needs to be moved in order to make $I'$ an increasing sequence.
								\item Let $I''$ be obtained from $I'$ by moving the integer $j$ in $I'$ by $l$ places.
								\item For each column label $J$: Using \procedurehr{3.20}, verify that $\det(B_{I',J})=(-1)^l\det(B_{I'',J})$.
								\item Therefore for each column label $J$: Verify that ${C_k(AB)}_{\ul{I},\ul{J}}=\det(B_{I,J})+p*\det(B_{I',J})=\det(B_{I,J})+(-1)^lp*\det(B_{I'',J})$.
								\item Verify that $I''$ is a row label of $C_k(B)$.
								\item Therefore for each column label $J$: Verify that ${C_k(AB)}_{\ul{I},\ul{J}}=\det(B_{I,J})+(-1)^lp*\det(B_{I'',J})={C_k(B)}_{\ul{I},\ul{J}}+(-1)^lp*{C_k(B)}_{\ul{I''},\ul{J}}$.
								\item \textbf{Therefore verify that $(C_k(AB))_{\ul{I},*}=(C_k(B))_{\ul{I},*}+(-1)^lp(C_k(B))_{\ul{I''},*}$.}
								\item \textbf{Set $D_{\ul{I},\ul{I''}}$ to $(-1)^lp$.}
							\end{enumerate}
						\end{enumerate}
						\item \textbf{Therefore verify that ${C_k(AB)}_{\ul{I},*}=D_{\ul{I},*}C_k(B)$.}
					\end{enumerate}
					\item \textbf{Therefore verify that $C_k(AB)=DC_k(B)$.}
					\item \textbf{Yield $\langle D\rangle$.}
				\end{enumerate}
		\procedure{3.24}
			\objective
				Choose a $\mathcal{D}_{m,n}(\mathbb{Q}[x])$, $A$. Also choose an $\mathcal{M}_{n,n}(\mathbb{Q}[x])$, $B$. Also choose an integer $0\le k\le\min(m,n)$. The objective of the following instructions is to construct a $\mathcal{D}_{\binom{m}{k},\binom{n}{k}}(\mathbb{Q}[x])$ $D$ such that $C_k(AB)=DC_k(B)$.
			\implementation
				\begin{enumerate}
					\item Let $D=C_k(0_{m\times n})=0_{\binom{m}{k}\times\binom{n}{k}}$.
					\item Verify that $AB$ equals $B_{[0:\min(m,n)],*}$ with each row $i$ multiplied by $A_{i,i}$.
					\item Go through the row labels, $I$, of $C_k(AB)$ and do the following:
					\begin{enumerate}
						\item If $I_k<\min(m,n)$, then do the following:
						\begin{enumerate}
							\item Verify that every element of $I$ is less than $\min(m,n)$.
							\item Let $A_0=A$.
							\item For $i$ in $[0:k]$: Let $A_{i+1}$ equal $A_{i}$ but with position $(I_i,I_i)$ set to $1$.
							\item For each column label $J$: Repeatedly using \procedurehr{3.14}, verify that ${C_k(AB)}_{I,J}$
							\begin{enumerate}
								\item $=\det((AB)_{I,J})$
								\item $=\det((A_0B)_{I,J})$
								\item $=A_{I_0,I_0}\det((A_1B)_{I,J})$
								\item $=A_{I_0,I_0}A_{I_1,I_1}\det((A_2B)_{I,J})$
								\item $\vdots$
								\item $=A_{I_0,I_0}A_{I_1,I_1}\cdots A_{I_{k-1},I_{k-1}}\det((A_kB)_{I,J})$
								\item $=A_{I_0,I_0}A_{I_1,I_1}\cdots A_{I_{k-1},I_{k-1}}\det(B_{I,J})$
								\item $=A_{I_0,I_0}A_{I_1,I_1}\cdots A_{I_{k-1},I_{k-1}}{C_k(B)}_{\ul{I},\ul{J}}$.
							\end{enumerate}
							\item \textbf{Therefore verify that $(C_k(AB))_{\ul{I},*}=A_{I_1,I_1}A_{I_1,I_1}\cdots A_{I_k,I_k}*(C_k(B))_{\ul{I},*}$.}
							\item \textbf{Set $D_{\ul{I},\ul{I}}$ to $A_{I_0,I_0}A_{I_1,I_1}\cdots A_{I_{k-1},I_{k-1}}$.}
						\end{enumerate}
						\item Otherwise if $I_k\ge\min(m,n)$, then do the following:
						\begin{enumerate}
							\item Using (O), verify that $A_{I_k,*}=0_{1\times n}$.
							\item Therefore verify that $(AB)_{I_k,*}=0_{1\times n}$.
							\item Therefore verify that $((AB)_{I,*})_{k,*}=0_{1\times n}$.
							\item Therefore using \procedurehr{3.12}, for each column label $J$: verify that ${C_k(AB)}_{\ul{I},\ul{J}}=\det((AB)_{I,J})=0$.
							\item \textbf{Therefore verify that $(C_k(AB))_{\ul{I},*}$ is zero.}
						\end{enumerate}
						\item \textbf{Therefore verify that ${C_k(AB)}_{\ul{I},*}=D_{\ul{I},*}C_k(B)$.}
					\end{enumerate}
					\item \textbf{Verify that $D$ is diagonal.}
					\item \textbf{Verify that $C_k(AB)=DC_k(B)$.}
					\item \textbf{Yield $\langle D\rangle$.}
				\end{enumerate}
		\procedure{3.25}
			\objective
				Choose an integer $0\le k\le\min(m,n)$. Choose a $\mathcal{T}_{m}(\mathbb{Q}[x])$, $A$. Also choose a $\mathcal{M}_{m,n}(\mathbb{Q}[x])$, $B$. The objective of the following instructions is to show that $C_k(AB)=C_k(A)C_k(B)$.
			\implementation
				\begin{enumerate}
					\item Execute \procedurehr{3.23} on matrices $A$ and $I_m$ and let $\langle D\rangle$ receive.
					\item Using \procedurehr{3.22}, verify that $C_k(A)=C_k(AI_m)=DC_k(I_m)=DI_{\binom{m}{k}}=D$.
					\item Execute \procedurehr{3.23} on $\langle A,B\rangle$ and let $\langle D'\rangle$ receive.
					\item Verify that $D'=D=C_k(A)$.
					\item \textbf{Therefore verify that $C_k(AB)=D'C_k(B)=C_k(A)C_k(B)$.}
				\end{enumerate}
		\procedure{3.26}
			\objective
				Choose an integer $0\le k\le\min(m,n)$. Choose a $\mathcal{T}_{n}(\mathbb{Q}[x])$, $A$. Also choose a $\mathcal{M}_{m,n}(\mathbb{Q}[x])$, $B$. The objective of the following instructions is to show that $C_k(BA)=C_k(B)C_k(A)$.
			\implementation
				Instructions are analogous to those of \procedurehr{3.25}.
		\procedure{3.27}
			\objective
				Choose an integer $0\le k\le\min(m,n)$. Choose a $\mathcal{D}_{m,n}(\mathbb{Q}[x])$, $A$. Also choose a $\mathcal{M}_{n}(\mathbb{Q}[x])$, $B$. The objective of the following instructions is to show that $C_k(AB)=C_k(A)C_k(B)$.
			\implementation
				Instructions are analogous to those of \procedurehr{3.25}.
		\procedure{3.28}
			\objective
				Choose a $\mathcal{M}_{m,n}(\mathbb{Q}[x])$, $A$. Let $D_{-1,-1}=1$. The objective of the following instructions is to construct a list of $\mathcal{T}_{m}(\mathbb{Q}[x])$s, $M$, a $\mathcal{D}_{m,n}(\mathbb{Q}[x])$, $D$, a list of $\mathbb{Q}[x]$s, $v$, and a list of $\mathcal{T}_{n}(\mathbb{Q}[x])$s, $N$, such that $M_*AN_*=D$, $A={M^{-1}}_*D{N^{-1}}_*$, and $D_{i,i}=v_iD_{i-1,i-1}$ for $i$ in $[0:\min(m,n)]$.
			\implementation
				\begin{enumerate}
					\item Let $D$ be a copy of $A$.
					\item Let $\langle M,N\rangle$ receive the results of executing \procedurehr{3.03} on the pair $\langle m,n\rangle$ and the following procedure:
						\begin{enumerate}
							\item Execute \procedurehr{3.11} on the matrix $D$ and let $\langle v\rangle$ receive.
						\end{enumerate}
					\item \textbf{Verify that $D_{i,i}=v_iD_{i-1,i-1}$ for $i$ in $[0:\min(m,n)]$.}
					\item \textbf{Verify that $M_*AN_*=D$.}
					\item Hence verify that $A=I_mAI_n={M^{-1}}_*M_*AN_*{N^{-1}}_*={M^{-1}}_*D{N^{-1}}_*$.
					\item \textbf{Yield the tuple $\langle M,D,v,N\rangle$.}
				\end{enumerate}
		\procedure{3.29}
			\objective
				Choose integers $0\le k\le\min(m,n,p)$. Choose a $\mathcal{M}_{m,n}(\mathbb{Q}[x])$, $A$. Also choose a $\mathcal{M}_{n,p}(\mathbb{Q}[x])$, $B$. The objective of the following instructions is to show that $C_k(AB)=C_k(A)C_k(B)$.
			\implementation
				\begin{enumerate}
					\item Execute \procedurehr{3.28} on $A$ and let $\langle M,D,,N\rangle$ receive.
					\item Using repeated applications of \procedurehr{3.27}, verify that $C_k(AB)$
					\begin{enumerate}
						\item $=C_k({M^{-1}}_0\cdots {M^{-1}}_{\lvert M\rvert-1}D{N^{-1}}_0\cdots {N^{-1}}_{\lvert N\rvert-1}B)$
						\item $=C_k({M^{-1}}_0)\cdots C_k({M^{-1}}_{\lvert M\rvert-1})*C_k(D)*C_k({N^{-1}}_0)\cdots C_k({N^{-1}}_{\lvert N\rvert-1})C_k(B)$
						\item $=C_k({M^{-1}}_0\cdots {M^{-1}}_{\lvert M\rvert-1}D{N^{-1}}_0\cdots {N^{-1}}_{\lvert N\rvert-1})C_k(B)$
						\item $=C_k(A)C_k(B)$.
					\end{enumerate}
				\end{enumerate}
		\procedure{3.30}
			\objective
				Choose a $\mathcal{M}_{m,m}(\mathbb{Q}[x])$, $A$. Let $D$ be a copy of $A$. Execute \procedurehr{3.11} on $D$. The objective of the following instructions is to show that $\det(A)$ is the product of the diagonal entries of $D$.
			\implementation
				\begin{enumerate}
					\item Execute \procedurehr{3.28} on $A$ and let $\langle M,D,,N\rangle$ receive.
					\item Using \procedurehr{3.12} and \procedurehr{3.29}, verify that $\det(A)$
					\begin{enumerate}
						\item $=C_m(A)$
						\item $=C_m({M^{-1}}_0\cdots {M^{-1}}_{\lvert M\rvert-1}D{N^{-1}}_0\cdots {N^{-1}}_{\lvert N\rvert-1})$
						\item $=C_m({M^{-1}}_0)\cdots C_m({M^{-1}}_{\lvert M\rvert-1})C_m(D)C_m({N^{-1}}_0)\allowbreak\cdots C_m({N^{-1}}_{\lvert N\rvert-1})$
						\item $=1\cdots 1C_m(D)1\cdots 1=C_m(D)$
						\item $=\det(D)$
						\item $=\prod_{r=0}^m D_{r,r}$.
					\end{enumerate}
				\end{enumerate}
		\procedure{3.31}
			\objective
				Choose a $\mathcal{M}_{m,n}(\mathbb{Q}[x])$, $A$. The objective of the following instructions is to construct a $\mathcal{M}_{n,m}(\mathbb{Q}[x])$, $A^T$.
			\implementation
				\begin{enumerate}
					\item Make an $n\times m$ matrix, $A^T$.
					\item For $i$ in $[0:n]$: For $j$ in $[0:m]$:
					\begin{enumerate}
						\item Set ${A^T}_{i,j}=A_{j,i}$.
					\end{enumerate}
					\item \textbf{Yield the tuple $\langle A^T\rangle$.}
				\end{enumerate}
		\notation{3.09}
			Let us use the notation $A^T$ for the result yielded by executing \procedurehr{3.31} on $A$.
		\procedure{3.32}
			\objective
				Choose a $\mathcal{M}_{m,n}(\mathbb{Q}[x])$, $A$, and a $\mathcal{M}_{n,k}(\mathbb{Q}[x])$, $B$. The objective of the following instructions is to show that $B^TA^T=(AB)^T$.
			\implementation
				\begin{enumerate}
					\item Verify that $B^TA^T$ and $(AB)^T$ have dimensions $k\times m$.
					\item For $i$ in $[0:k]$: For $j$ in $[0:m]$:
					\begin{enumerate}
						\item Using \procedurehr{3.31}, verify that $(B^TA^T)_{i,j}=\sum_{l=0}^n B_{l,i}A_{j,l}=\sum_{l=0}^n A_{j,l}B_{l,i}=(AB)_{j,i}=((AB)^T)_{i,j}$.
					\end{enumerate}
					\item \textbf{Therefore verify that $B^TA^T=(AB)^T$.}
				\end{enumerate}
		\procedure{3.33}
			\objective
				Choose a $\mathcal{M}_{m,m}(\mathbb{Q}[x])$, $A$. The objective of the following instructions is to show that $\det(A^T)=\det(A)$.
			\implementation
				\begin{enumerate}
					\item Execute \procedurehr{3.28} on $A$ and let $\langle M,D,,N\rangle$ receive.
					\item Therefore using procedures \procedurehr{3.30} and \procedurehr{3.32}, verify that $\det(A^T)$
					\begin{enumerate}
						\item $=\det(({M^{-1}}_0\cdots {M^{-1}}_{\lvert M\rvert-1}D{N^{-1}}_0\cdots {N^{-1}}_{\lvert N\rvert-1})^T)$
						\item $=\det(({N^{-1}}_{\lvert N\rvert-1})^T\cdots({N^{-1}}_0)^TD^T({M^{-1}}_{\lvert M\rvert-1})^T\allowbreak\cdots({M^{-1}}_0)^T)$
						\item $=\det(D^T)$
						\item $=\det(D)$
						\item $=\det({M^{-1}}_0\cdots {M^{-1}}_{\lvert M\rvert-1}D{N^{-1}}_0\cdots {N^{-1}}_{\lvert N\rvert-1})$
						\item $=\det(A)$.
					\end{enumerate}
				\end{enumerate}
		\procedure{3.34}
			\objective
				Choose a $\mathcal{M}_{m,n}(\mathbb{Q}[x])$, $A$, and an integer $0\le k\le\min(m,n)$. The objective of the following instructions is to show that $C_k(A)^T=C_k(A^T)$.
			\implementation
				\begin{enumerate}
					\item For each row label $I$ of $C_k(A^T)$, do the following:
					\begin{enumerate}
						\item For each column label $J$ of $C_k(A^T)$, do the following:
						\begin{enumerate}
							\item Using \procedurehr{3.33}, verify that $(C_k(A^T))_{\ul{I},\ul{J}}=\det((A^T)_{I,J})=\det(A_{J,I})=(C_k(A))_{\ul{J},\ul{I}}$.
						\end{enumerate}
					\end{enumerate}
					\item \textbf{Therefore verify that $(C_k(A))^T=(C_k(A^T))$.}
				\end{enumerate}
		\procedure{3.35}
			\objective
				Choose a $\mathcal{M}_{m,n}(\mathbb{Q})$, $A$, and a $\mathcal{M}_{m,p}(\mathbb{Q})$, $B$. Execute \procedurehr{3.28} on $A$ and let $\langle M,D,,N\rangle$ receive the result. If the indices of the rows of $D$ that are entirely zero are also the indices of the rows of $M_*B$ that are entirely zero, then the objective of the following instructions is to construct a $\mathcal{M}_{n,p}(\mathbb{Q})$ $E$ such that $AE=B$.
			\implementation
				\begin{enumerate}
					\item Verify that $A={M^{-1}}_*D{N^{-1}}_*$.
					\item Verify that ${M^{-1}}_*$, $D$, and ${N^{-1}}_*$ are $\mathcal{M}_{*,*}(\mathbb{Q})$s.
					\item Let $C$ be an $n\times p$ matrix with its $i^{th}$ row given as follows:
					\begin{enumerate}
						\item If $D_{i,i}\ne 0$, then do the following:
						\begin{enumerate}
							\item Let row $i$ be row $i$ of $M_*B$ divided by $D_{i,i}$.
						\end{enumerate}
						\item Otherwise, do the following:
						\begin{enumerate}
							\item \textbf{Choose $p$ rational numbers to fill up the row.}
						\end{enumerate}
					\end{enumerate}
					\item Verify that $DC=M_*B$.
					\item Let $E$ be $N_*C$.
					\item \textbf{Therefore using \procedurehr{3.05}, verify that $AE={M^{-1}}_*D{N^{-1}}_*E={M^{-1}}_*D{N^{-1}}_*N_*C={M^{-1}}_*DI_nC={M^{-1}}_*DC={M^{-1}}_*M_*B=I_mB=B$.}
					\item \textbf{Yield the tuple $\langle E\rangle$.}
				\end{enumerate}
		\notation{3.10}
			The notation $A\backslash B$ shall be used to refer to the result yielded by executing \procedurehr{3.35} on $\langle A,B\rangle$.
		\procedure{3.36}
			\objective
				Choose a $\mathcal{M}_{m,n}(\mathbb{Q})$, $A$, and a $\mathcal{M}_{p,n}(\mathbb{Q})$, $B$. Execute \procedurehr{3.28} on $A$ and let $\langle M,D,,N\rangle$ receive the result. If the indices of the columns of $D$ that are entirely zero are also the indices of the columns of $BN_*$ that are entirely zero, then the objective of the following instructions is to construct a $\mathcal{M}_{p,m}(\mathbb{Q})$ $E$ such that $EA=B$.
			\implementation
				Instructions are analogous to those of \procedurehr{3.35}.
		\notation{3.11}
			The notation $B/A$ shall be used to refer to the result yielded by executing \procedurehr{3.36} on $\langle A,B\rangle$.
		\procedure{3.37}
			\objective
				Choose a $\mathcal{M}_{m,n}(\mathbb{Q})$, $A$, a $\mathcal{M}_{n,p}(\mathbb{Q})$, $E$, and a $\mathcal{M}_{m,p}(\mathbb{Q})$, $B$ such that $AE=B$. Execute \procedurehr{3.28} on $A$ and let $\langle M,D,,N\rangle$ receive the result. If the indices of the rows of $D$ that are entirely zero are not also the indices of the rows of $M_*B$ that are entirely zero, then the objective of the following instructions is to show that $0\ne 0$.
			\implementation
				\begin{enumerate}
					\item Verify that ${M^{-1}}_*D{N^{-1}}_*E=AE=B$.
					\item Therefore verify that $D{N^{-1}}_*E=M_*B$.
					\item Let $i$ be an integer such that $D_{i,*}$ is zero and yet $(M_*B)_{i,*}$ is not zero.
					\item Verify that $D_{i,*}=D_{i,*}{N^{-1}}_*E=(D{N^{-1}}_*E)_{i,*}=(M_*B)_{i,*}$.
					\item Let $j$ be an integer such that $(M_*B)_{i,j}\ne 0$.
					\item \textbf{Now verify that $0=D_{i,j}=(M_*B)_{i,j}\ne 0$.}
				\end{enumerate}
		\procedure{3.38}
			\objective
				Choose a $\mathcal{M}_{p,m}(\mathbb{Q})$, $E$, a $\mathcal{M}_{m,n}(\mathbb{Q})$, $A$, and a $\mathcal{M}_{p,n}(\mathbb{Q})$, $B$ such that $EA=B$. Execute \procedurehr{3.28} on $A$ and let $\langle M,D,,N\rangle$ receive the result. If the indices of the columns of $D$ that are entirely zero are not also the indices of the columns of $BN_*$ that are entirely zero, then the objective of the following instructions is to show that $0\ne 0$.
			\implementation
				Instructions are analogous to those of \procedurehr{3.37}.
		\procedure{3.39}
			\objective
				Choose two $\mathcal{M}_{m,m}(\mathbb{Q})$s, $A$ and $B$, such that $AB=I_m$. The objective of the following instructions is to show that either $0=1$ or $BA=I_m$.
			\implementation
				\begin{enumerate}
					\item Execute \procedurehr{3.28} on $B$ and let $\langle M,D,,N\rangle$ receive the result.
					\item Verify that $B={M^{-1}}_*D{N^{-1}}_*$.
					\item If $D$ has a zero on its diagonal, then do the following:
					\begin{enumerate}
						\item Using \procedurehr{3.30}, verify that $\det(I_m)=\det(AB)=\det(A)\det(B)=\det(A)\det(D)=\det(A)*0=0$.
						\item Using \procedurehr{3.12}, verify that $\det(I_m)=1^m=1$.
						\item Therefore verify that $0=1$.
						\item \textbf{Abort procedure.}
					\end{enumerate}
					\item Otherwise do the following:
					\begin{enumerate}
						\item Verify that $D$ does not have a zero on its diagonal.
						\item Verify that $B\backslash I_m=I_m(B\backslash I_m)=AB(B\backslash I_m)=A(B(B\backslash I_m))=AI_m=A$.
						\item \textbf{Therefore verify that $BA=B(B\backslash I_m)=I_m$.}
					\end{enumerate}
				\end{enumerate}
		\procedure{3.40}
			\objective
				Choose an $\mathcal{M}_{m,m}(\mathbb{Q}[x])$, $M$, and an $\mathcal{M}_{m,m}(\mathbb{Q})$, $B$. The objective of the following instructions is to construct a $\mathcal{M}_{m,m}(\mathbb{Q}[x])$, $Q$, and a $\mathcal{M}_{m,m}(\mathbb{Q})$, $R$, such that $M=(xI_m-B)Q+R$.
			\implementation
				\begin{enumerate}
					\item Let $M_0x^b+M_1x^{b-1}+\cdots+M_bx^0=M$, where the $M_i$ are $\mathcal{M}_{m,m}(\mathbb{Q})$s.
					\item Now let $R=B^bM_0+B^{b-1}M_1+\cdots+B^0M_b$.
					\item Let $Q=\sum_{k=1}^b (x^{k-1}I_mB^0+x^{k-2}I_mB^1+\cdots+x^0I_mB^{k-1})M_k$.
					\item Verify that $M-R=(xI_m-B)\sum_{k=1}^b (x^{k-1}I_mB^0+x^{k-2}I_mB^1+\cdots+x^0I_mB^{k-1})M_k=(xI_m-B)Q$.
					\item \textbf{Verify that $M=(xI_m-B)Q+R$.}
					\item \textbf{Yield the tuple $\langle Q,R\rangle$.}
				\end{enumerate}
		\procedure{3.41}
			\objective
				Choose an $\mathcal{M}_{m,m}(\mathbb{Q}[x])$, $M$, and an $\mathcal{M}_{m,m}(\mathbb{Q})$, $B$. The objective of the following instructions is to construct a $\mathcal{M}_{m,m}(\mathbb{Q}[x])$, $Q$, and a $\mathcal{M}_{m,m}(\mathbb{Q})$, $R$, such that $M=Q(xI_m-B)+R$.
			\implementation
				The instructions are analogous to those of \procedurehr{3.40}.
		\procedure{3.42}
			\objective
				Choose two $\mathcal{M}_{m,m}(\mathbb{Q})$s, $B,A$, and two lists of $\mathcal{T}_{m}(\mathbb{Q}[x])$s such that $xI_m-B=M(xI_m-A)N$. The objective of the following instructions is to either show that $0=1$ or to construct $\mathcal{M}_{m,m}(\mathbb{Q})$s $R_1$ and $R_3$ such that $I_m=R_1R_3$ and $B=R_1AR_3$.
			\implementation
				\begin{enumerate}
					\item Verify that $(xI_m-B)N^{-1}=M(xI_m-A)NN^{-1}=M(xI_m-A)I_m=M(xI_m-A)$.
					\item Execute \procedurehr{3.41} on $\langle M,B\rangle$ and let $\langle Q_1,R_1\rangle$ receive.
					\item Verify that $M=(xI_m-B)Q_1+R_1$.
					\item Execute \procedurehr{3.41} on $\langle N^{-1},A\rangle$ and let $\langle Q_2,R_2\rangle$ receive.
					\item Verify that $N^{-1}=Q_2(xI_m-A)+R_2$.
					\item By substituting $M$ and $N^{-1}$ into (2), verify that $(xI_m-B)(Q_2(xI_m-A)+R_2)=((xI_m-B)Q_1+R_1)(xI_m-A)$.
					\item By rearranging both sides, verify that $(xI_m-B)(Q_2-Q_1)(xI_m-A)=R_1(xI_m-A)-(xI_m-B)R_2$.
					\item By equating the coefficients of different powers of $x$ both sides, verify that $Q_2-Q_1=0_{m\times m}$.
					\item Verify that $R_1(xI_m-A)-(xI_m-B)R_2=(xI_m-B)(Q_2-Q_1)(xI_m-A)=(xI_m-B)0_{m\times m}(xI_m-A)=0_{m\times m}$.
					\item Therefore by adding $(xI_m-B)R_2$ to both sides, verify that $xR_1-R_1A=R_1(xI_m-A)=(xI_m-B)R_2=xR_2-BR_2$.
					\item By equating the coefficients of $x$ on both sides, verify that $R_1=R_2$.
					\item Therefore verify that $R_1A=BR_1$.
					\item Execute \procedurehr{3.41} on $\langle M^{-1},A\rangle$ and let $\langle Q_3,R_3\rangle$ receive.
					\item Verify that $M^{-1}=(xI_m-A)Q_3+R_3$.
					\item Verify that $I_m=MM^{-1}=((xI_m-B)Q_1+R_1)M^{-1}=(xI_m-B)Q_1M^{-1}+R_1M^{-1}=(xI_m-B)Q_1M^{-1}+R_1(xI-A)Q_3+R_1R_3=(xI_m-B)Q_1M^{-1}+(xI-B)R_1Q_3+R_1R_3=(xI_m-B)(Q_1M^{-1}+R_1Q_3)+R_1R_3$.
					\item By equating the powers of $x$ on both sides, verify that $Q_1M^{-1}+R_1Q_3=0$.
					\item By substituting zero for $Q_1M^{-1}+R_1Q_3$, \textbf{verify that $I_m=(xI_m-B)0_{m\times m}+R_1R_3=R_1R_3$.}
					\item \textbf{Therefore using \procedurehr{3.39}, verify that $R_3R_1=I_m$.}
					\item \textbf{Also, verify that $B=BI_m=BR_1R_3=R_1AR_3$.}
					\item \textbf{Yield the pair $(R_1,R_3)$.}
				\end{enumerate}
		\procedure{3.43}
			\objective
				Choose a $\mathcal{M}_{m,n}(\mathbb{Q}[x])$, $A$. Choose two integers $0\le i,j<m$ such that $i\ne j$. The objective of the following instructions is to negate row $i$ and swap it with row $j$ using only elementary row operations.
			\implementation
				\begin{enumerate}
					\item Let $A$ be our working matrix.
					\item Subtract row $j$ from row $i$.
					\item Add row $i$ to row $j$.
					\item Subtract row $j$ from row $i$.
					\item \textbf{Verify that the $i^{th}$ row has been negated and swapped with the $j^{th}$ row.}
				\end{enumerate}
		\procedure{3.44}
			\objective
				Choose a $\mathcal{M}_{m,n}(\mathbb{Q}[x])$, $A$. Choose two integers $0\le i,j<n$ such that $i\ne j$. The objective of the following instructions is to negate column $i$ and swap it with row $j$ using only elementary column operations.
			\implementation
				The instructions are analogous to those of \procedurehr{3.43}.
		\procedure{3.45}
			\objective
				Choose a $\mathcal{D}_{m,n}(\mathbb{Q}[x])$, $A$. Choose two integers $0\le i,j<\min(m,n)$ such that $i\ne j$. The objective of the following instructions is to swap $B_{i,i}$ and $B_{j,j}$ using only elementary row and column operations.
			\implementation
				\begin{enumerate}
					\item Let $A$ be our working matrix.
					\item Use \procedurehr{3.44} to negate the $i^{th}$ row and swap it with the $j^{th}$ row.
					\item Use \procedurehr{3.44} to negate the $i^{th}$ column and swap it with the $j^{th}$ column.
					\item \textbf{Therefore, overall verify that $B_{i,i}$ and $B_{j,j}$ have been swapped.}
				\end{enumerate}
		\procedure{3.46}
			\objective
				Choose a $\mathcal{D}_{m,n}(\mathbb{Q}[x])$, $A$. Choose two integers $0\le i,j<\min(m,n)$ such that $i\ne j$. Choose a rational $k\ne 0$. The objective of the following instructions is to multiply $B_{i,i}$ by $k$ and $B_{j,j}$ by $\frac{1}{k}$ using only elementary row and column operations.
			\implementation
				\begin{enumerate}
					\item Let $A$ be our working matrix.
					\item Add $k$ times row $i$ to row $j$.
					\item Subtract $\frac{1}{k}$ times row $j$ from row $i$.
					\item Add $k$ times row $i$ to row $j$.
					\item Verify that the $i^{th}$ row has been scaled by $k$, the $j^{th}$ row by $-\frac{1}{k}$, and that both these rows are swapped.
					\item Use \procedurehr{3.44} to negate the $i^{th}$ row and swap it with the $j^{th}$ row.
					\item \textbf{Therefore, overall verify that $B_{i,i}$ has been multiplied by $k$, and $B_{j,j}$ by $\frac{1}{k}$.}
				\end{enumerate}
		\notation{3.12}
			Let us use the notation "$p$ is monic" as a shorthand for "$x^{\deg(p)}\circ p=1$".
		\procedure{3.47}
			\objective
				Choose a $\mathcal{M}_{m,m}(\mathbb{Q})$, $A$. Execute \procedurehr{3.11} on the polynomial matrix $xI-A$ and let $\langle B\rangle$ be the result. The objective of the following instructions is to show that either none of the diagonal entries of $B$ are equal to zero, or $1=0$.
			\implementation
				\begin{enumerate}
					\item Using \procedurehr{3.12}, verify that $\det(xI-A)$ is a monic polynomial of degree $m$.
					\item Therefore using \procedurehr{3.30}, verify that $\det(B)=\det(xI-A)$.
					\item Therefore verify that $\det(B)$ is a monic polynomial of degree $m$.
					\item If any of the diagonal entries of $B$ equal zero, then do the following:
					\begin{enumerate}
						\item Using \procedurehr{3.12}, verify that $\det(B)=B_{0,0}B_{1,1}\cdots B_{m-1,m-1}=0$.
						\item Therefore using (3) and (4a), verify that $1=0$.
						\item \textbf{Abort procedure.}
					\end{enumerate}
					\item Otherwise do the following:
					\begin{enumerate}
						\item \textbf{Verify that none of the diagonal entries of $B$ equal zero.}
					\end{enumerate}
				\end{enumerate}
		\procedure{3.48}
			\objective
				Choose a positive integer $m$ and an $\mathcal{M}_{m,m}(\mathbb{Q})$, $A$. Execute \procedurehr{3.28} on the polynomial matrix $xI_m-A$ and let $\langle ,B,v,\rangle$ be the result. The objective of the following instructions is to either show that $0<0$ or to construct an integer $a$ such that $\sum_{i=a}^m\deg(B_{i,i})=m$, $\deg(B_{i,i})>0$ for $i$ in $[a:m]$, and $\deg(B_{i,i})=0$ for $i$ in $[0:a]$.
			\implementation
				\begin{enumerate}
					\item Execute \procedurehr{3.47} on $A$.
					\item If $\deg(B_{i,i})=0$ for $i$ in $[0:m]$, then do the following:
					\begin{enumerate}
						\item Verify that $\det(xI_m-A)=\det(B)=B_{0,0}B_{1,1}\cdots B_{m-1,m-1}$.
						\item \textbf{Therefore verify that $0<m=\deg(\det(xI_m-A))=\deg(B_{0,0}B_{1,1}\cdots B_{m-1,m-1})=0+0+\cdots+0=0$.}
						\item \textbf{Abort procedure.}
					\end{enumerate}
					\item Otherwise do the following:
					\begin{enumerate}
						\item Let $0\le a<m$ be the least integer such that $\deg(B_{a,a})>0$.
						\item \textbf{Verify that $\deg(B_{i,i})=0$ for $i$ in $[0:a]$.}
						\item \textbf{Verify that $\sum_{i=a}^m\deg(B_{i,i})=\sum_{i=0}^m\deg(B_{i,i})=\deg(B_{0,0}B_{1,1}\cdots B_{m-1,m-1})=\deg(\det(B))=\deg(xI_m-A)=m$.}
						\item For $i$ in $[a+1:m]$, do the following:
						\begin{enumerate}
							\item Verify that $B_{i,i}=u_iB_{i-1,i-1}$.
							\item Verify that $B_{i,i}\ne 0$.
							\item Therefore verify that $u_i\ne 0$.
							\item \textbf{Therefore verify that $\deg(B_{i,i})=\deg(u_iB_{i-1,i-1})\ge\deg(B_{i-1,i-1})>0$.}
						\end{enumerate}
						\item \textbf{Yield the tuple $\langle a\rangle$.}
					\end{enumerate}
				\end{enumerate}
		\procedure{3.49}
			\objective
				Choose a $\mathbb{Q}[x]$, $p=x^k+p_1x^{k-1}+p_2x^{k-2}+\cdots+p_kx^0$ such that $k>0$. The objective of the following instructions is to construct a $\mathcal{M}_{k,k}(\mathbb{Q})$, $\rcan(p)$.
			\implementation
				\begin{enumerate}
					\item Make a $k\times k$ matrix $C$.
					\item Let $C$'s first $k-1$ columns be filled with the last $k-1$ columns of $I_k$.
					\item Let $C$'s last column from top to bottom be $-p_k, -p_{k-1},\cdots,-p_1$.
					\item \textbf{Yield the tuple $\langle C\rangle$.}
				\end{enumerate}
		\notation{3.16}
			Let us use $\rcan(p)$ as a shorthand for the result yielded by executing \procedurehr{3.49} on $p$.
		\procedure{3.50}
			\objective
				Choose a monic $\mathbb{Q}[x]$, $p$ such that $\deg(p)>0$. Let $k=\deg(p)$. Choose a $\mathcal{M}_{k,k}(\mathbb{Q}[x])$, $D$, such that $D=xI_k-\rcan(p)$. The objective of the following instructions is to transform $D$ into $\bdiag(1,\cdots,1,p)$ by a sequence of elementary operations.
			\implementation
				\begin{enumerate}
					\item Let the matrix $D$ be our working matrix.
					\item For $i$ in $[k:1]$, add $x$ times row $i$ to row $i-1$.
					\item Verify that $D$'s first $k-1$ columns are now the last $k-1$ columns of $-I_k$.
					\item Verify that $D$'s last column is $p$ followed by some other polynomials.
					\item For $i$ in $[1:k]$, subtract $D_{i,k-1}$ times column $i-1$ from column $k-1$.
					\item Verify that $D$'s last column is now $p$ followed by zeros.
					\item For $i$ in $[1:k]$, negate row $i-1$ and exchange it with row $i$ using \procedurehr{3.44}.
					\item \textbf{Therefore verify that $D=\bdiag(1,\cdots,1,p)$.}
				\end{enumerate}
		\notation{3.17}
			Let us use the notation $\mon(p)$ as a shorthand for "$\frac{p}{x^{\deg(p)}\circ p}$".
		\procedure{3.51}
			\objective
				Choose a positive integer $m$ and an $\mathcal{M}_{m,m}(\mathbb{Q})$, $A$. Execute \procedurehr{3.03} on the polynomial matrix $xI_m-A$ and let $\langle,B,,\rangle$ receive the result. Execute \procedurehr{3.48} on $A$ and let $\langle a\rangle$ receive the result. Let $E_i=\rcan(\mon(B_{a+i,a+i}))$ for $i$ in $[0:m-a]$. The objective of the following instructions is to first show that $\cols(\bdiag(E))=m$, and second to apply a sequence of elementary operations on $xI_m-\bdiag(E)$ to obtain the matrix $B$.
			\implementation
				\begin{enumerate}
					\item Verify that the diagonal of $B$ comprises $a$ rationals followed by $B_{a,a},B_{a+1,a+1},\cdots,B_{m-1,m-1}$.
					\item \textbf{Using \procedurehr{3.50}, verify that $\cols(\bdiag(E))=\sum_{i=0}^{\lvert E\rvert}\cols(E_i)=\sum_{i=0}^{\lvert E\rvert}\cols(\rcan(\mon(B_{a+i,a+i})))=\sum_{i=0}^{\lvert E\rvert}\deg(\mon(B_{a+i,a+i}))=\sum_{i=0}^{m-a}\deg(B_{a+i,a+i})=\sum_{i=a}^m\deg(B_{i,i})=m$.}
					\item Let $F=xI_m-\bdiag(E)$.
					\item Now for $i$ in $[0:\lvert E\rvert]$:
					\begin{enumerate}
						\item Let $j=\sum_{r=0}^{i}\cols(E_r)$.
						\item Let $k=j+\cols(E_i)$.
						\item Apply \procedurehr{3.50} on the tuple $\langle\mon(B_{a+i,a+i}),F_{[j:k],[j:k]}\rangle$.
					\end{enumerate}
					\item Now verify that $F$ is a $\mathcal{D}_{m,m}(\mathbb{Q})$.
					\item Also verify that the diagonal of $F$ comprises $\mon(B_{a,a}),\mon(B_{a+1,a+1}),\cdots,\mon(B_{m-1,m-1})$ and $a$ $1$s.
					\item Rearrange the diagonal of $F$ so that $\mon(B_{i,i})$ is at the $i^{th}$ position on the diagonal for $i$ in $[a:m]$ by doing pairwise swaps. In general, swap the $i^{th}$ and $j^{th}$ diagonal entries using \procedurehr{3.45}.
					\item For $i$ in $[0:m-1]$, do the following:
					\begin{enumerate}
						\item Let $k=\frac{x^{\deg(B_{i,i})}\circ B_{i,i}}{x^{\deg(F_{i,i})}\circ F_{i,i}}$.
						\item Scale $B_{i,i}$ by $k$ and $B_{i+1,i+1}$ by $\frac{1}{k}$ using \procedurehr{3.46}.
						\item Now verify that $F_{i,i}=B_{i,i}$.
					\end{enumerate}
					\item Now verify that $x^m\circ\det(F)=x^m\circ\det(xI_m-\bdiag(E))=1=x^m\circ\det(xI_m-A)=x^m\circ\det(B)$.
					\item Therefore verify that $x^{\deg(F_{m,m})}\circ F_{m,m}=\frac{x^m\circ\det(F)}{x^{m-\deg(F_{m,m})}\circ(\det(F_{[1:m],[1:m]}))}=\frac{x^m\circ\det(B)}{x^{m-\deg(B_{m,m})}\circ(\det(B_{[1:m],[1:m]}))}=x^{\deg(B_{m,m})}\circ B_{m,m}$.
					\item Therefore verify that $F_{m,m}=B_{m,m}$.
					\item \textbf{Therefore verify that $F=B$.}
				\end{enumerate}
		\procedure{3.52}
			\objective
				Choose a $\mathcal{M}_{m,m}(\mathbb{Q})$, $A$. Execute \procedurehr{3.48} on $A$ and let $\langle a\rangle$ receive the result. Let $E_i=\rcan(\mon(B_{a+i,a+i}))$ for $i$ in $[0:m-a]$. The objective of the following instructions is to either show that $0=1$ or to construct $\mathcal{M}_{m,m}(\mathbb{Q})$s $R,T$ such that $A=R\bdiag(E)T$, $RT=I_m$, and $TR=I_m$.
			\implementation
				\begin{enumerate}
					\item Execute \procedurehr{3.28} on the polynomial matrix $xI_m-A$ and let $\langle P,B,,Q\rangle$ be the result.
					\item Verify that $P_*(xI_m-A)Q_*=B$.
					\item Verify that $xI_m-A={P^{-1}}_*B{Q^{-1}}_*$.
					\item Let $Z$ be a variant of \procedurehr{3.28} where every occurence of \procedurehr{3.11} in its instructions is replaced with \procedurehr{3.51}, and where every mention of $v$ is ignored.
					\item Execute procedure $Z$ on the matrix $xI_m-\bdiag(E)$ and let $\langle M,,,N\rangle$ receive the result.
					\item Verify that $M_*(xI_m-\bdiag(E))N_*=B$.
					\item Verify that $xI_m-A={P^{-1}}_*B{Q^{-1}}_*={P^{-1}}_*M(xI_m-\bdiag(E))N{Q^{-1}}_*$.
					\item Execute \procedurehr{3.42} on the matrices $\langle A,{P}^{-1}M,\bdiag(E),N{Q}^{-1}\rangle$. Let the tuple $\langle R,T\rangle$ be the result.
					\item \textbf{Verify that $A=R\bdiag(E)T$.}
					\item \textbf{Verify that $RT=I_m$.}
					\item \textbf{Verify that $TR=I_m$.}
					\item \textbf{Yield the tuple $\langle R,E,T\rangle$.}
				\end{enumerate}
		\procedure{3.53}
			\objective
				Choose a $\mathbb{Q}[x]$, $r=r_0x^t+r_1x^{t-1}+\cdots+r_tx^0$, and $\mathcal{M}_{m,m}(\mathbb{Q})$s, $R,A,S$ such that $SR=I_m$. The objective of the following instructions is to show that $r(RAS)=Rr(A)S$.
			\implementation
				\begin{enumerate}
					\item \textbf{Verify that $r(RAS)=r_0(RAS)^t+r_1(RAS)^{t-1}+\cdots+r_t(RAS)^0=r_0RA^tS+r_1RA^{t-1}S+\cdots+r_tRA^0S=R(r_0A^t+r_1A^{t-1}+\cdots+r_tA^0)S=Rr(A)S$.}
				\end{enumerate}
		\procedure{3.54}
			\objective
				Choose a list of $\mathcal{M}_{m,m}(\mathbb{Q})$s, $A$, and a $\mathbb{Q}[x]$, $r=r_0x^t+r_1x^{t-1}+\cdots+r_tx^0$. The objective of the following instructions is to show that $r(\bdiag(A))=\bdiag(r(A))$.
			\implementation
				\begin{enumerate}
					\item For $i=0$ up to $i=t$, by repeated applications of \procedurehr{3.09}, verify that $\bdiag(A)^i$ evaluates to $\bdiag(A^i)$ (where the exponentiation is done element-wise).
					\item Therefore verify that $r(\bdiag(A))$
					\begin{enumerate}
						\item $=r_0\bdiag(A)^t+r_1\bdiag(A)^{t-1}+\cdots+r_t\bdiag(A)^0$
						\item $=r_0\bdiag(A^t)+r_1\bdiag(A^{t-1})+\cdots+r_t\bdiag(A^0)$
						\item $=\bdiag(r_0A^t)+\bdiag(r_1A^{t-1})+\cdots+\bdiag(r_tA^0)$
						\item \textbf{$=\bdiag(r(A))$ (where $r$ is applied element-wise).}
					\end{enumerate}
				\end{enumerate}
		\procedure{3.55}
			\objective
				Choose a $\mathcal{M}_{m,m}(\mathbb{Q})$, $A$, and a $\mathbb{Q}[x]$, $r$. Execute \procedurehr{3.52} on the matrix $A$ and let the tuple $\langle R_1,E,R_3\rangle$ receive the result. The objective of the following instructions is to show that $r(A)=R_1\bdiag(r(E))R_3$ (where $r$ is applied element-wise).
			\implementation
				\begin{enumerate}
					\item Verify that $R_3R_1=I_m$.
					\item Using \procedurehr{3.53}, verify that $r(A)=r(R_1\bdiag(E)R_3)=R_1r(\bdiag(E))R_3$.
					\item Using \procedurehr{3.54}, verify that $r(\bdiag(E))=\bdiag(r(E))$ (where $r$ is applied element-wise).
					\item \textbf{Therefore verify that $r(A)=R_1\bdiag(r(E))R_3$ (where $r$ is applied element-wise).}
				\end{enumerate}
		\notation{3.18}
			Let us use the notation $e_i$ as a shorthand for "the $\mathcal{M}_{k,1}(\mathbb{Q})$ that is $0$, except for its $i^{th}$ entry which is $1$".
		\notation{3.19}
			Let us use the notation $0_{m\times n}$ as a shorthand for "the $\mathcal{M}_{m,n}(\mathbb{Q})$ such that every entry is $0$".
		\procedure{3.56}
			\objective
				Choose a $\mathbb{Q}[x]$ $p=x^k+p_1x^{k-1}+p_2x^{k-2}+\cdots+p_kx^0$ such that $k>0$. The objective of the following instructions is to show that $p(\rcan(p))=0_{k\times k}$.
			\implementation
				\begin{enumerate}
					\item Let $G=\rcan(p)$.
					\item Then by $G$'s construction, for $i$ in $[0:k]$, verify that $G^{i}e_0=G^{i-1}e_1=\cdots=G^{0}e_i=e_i$.
					\item Therefore, for $i$ in $[0:k]$: Using (1), verify that $p(G)e_i$
					\begin{enumerate}
						\item $=(G^k+p_1G^{k-1}+p_2G^{k-2}+\cdots+p_kG^0)e_i$
						\item $=(G^k+p_1G^{k-1}+p_2G^{k-2}+\cdots+p_kG^0)G^{i}e_0$
						\item $=G^{i}(GG^{k-1}+p_1G^{k-1}+p_2G^{k-2}+\cdots+p_kG^0)e_0$
						\item $=G^{i}(Ge_{k-1}+p_1e_{k-1}+p_2e_{k-2}+\cdots+p_ke_0)$
						\item $=G^{i}0_{k\times 1}$
						\item $=0_{k\times 1}$.
					\end{enumerate}
					\item \textbf{Therefore verify that $p(\rcan(p))=p(G)=0_{k\times k}$.}
				\end{enumerate}
		\procedure{3.57}
			\objective
				Choose a $\mathcal{M}_{m,m}(\mathbb{Q})$, $A$. The objective of the following instructions is to define the $\mathbb{Q}[x]$ $\last_A$ and show that either $1=0$ or $\last_A\ne 0$.
			\implementation
				\begin{enumerate}
					\item Execute \procedurehr{3.28} on the polynomial matrix $xI_m-A$ and let the tuple $\langle,B,,\rangle$ receive the result.
					\item Execute \procedurehr{3.47} on $A$.
					\item Verify that $B_{m-1,m-1}\ne 0$.
					\item Yield $\langle B_{m-1,m-1}\rangle$.
				\end{enumerate}
		\notation{3.20}
			Let us use the notation $\last_A$ as a shorthand for the result of executing \procedurehr{3.57} on $A$.
		\procedure{3.58}
			\objective
				Choose a $\mathcal{M}_{m,m}(\mathbb{Q})$, $A$. The objective of the following instructions is to either show that $0<0$ or to show that $\last_A(A)=0_{m\times m}$.
			\implementation
				\begin{enumerate}
					\item Execute \procedurehr{3.28} on the matrix $A$ and let the tuple $\langle M,B,v,N\rangle$ receive the result.
					\item Execute \procedurehr{3.48} on $A$ and let $\langle a\rangle$ receive.
					\item Execute \procedurehr{3.52} on $A$ and let $\langle R,E,T\rangle$ receive.
					\item For $j$ in $[0:\lvert E\rvert]$:
					\begin{enumerate}
						\item Verify that $E_j=\rcan(\mon(B_{a+j,a+j}))$.
						\item Verify that $\last_A=B_{m-1,m-1}=B_{a+j,a+j}\prod_{r=a+j+1}^m v_r$.
						\item Let $k=\deg(\mon(B_{a+j,a+j}))$.
						\item Therefore using \procedurehr{3.56} verify that $\last_A(E_j)=B_{m-1,m-1}(E_j)=B_{a+j,a+j}(\rcan(\mon(B_{a+j,a+j})))\prod_{r=a+j+1}^m v_r(E_j)=0_{k\times k}\prod_{r=a+j+1}^m v_r(E_j)=0_{k\times k}$.
					\end{enumerate}
					\item \textbf{Therefore using \procedurehr{3.55} verify that $\last_A(A)=R\bdiag(\last_A(E))T=R\bdiag(B_{m-1,m-1}(E))T=R0_{m\times m}T=0_{m\times m}$.}
				\end{enumerate}
		\procedure{3.59}
			\objective
				Choose a monic $\mathbb{Q}[x]$ $p$ such that $\deg(p)>0$. Choose a $\mathbb{Q}[x]$ $g=g_0x^k+g_1x^{k-1}+\cdots+g_kx^0$ such that $g_0\ne 0$ and $k<\deg(p)$. The objective of the following instructions is to show that $g(\rcan(p))\ne 0_{\deg(p)\times \deg(p)}$.
			\implementation
				\begin{enumerate}
					\item Let $G=\rcan(p)$.
					\item Therefore cognizant of $G$'s construction, verify that $g(G)e_0=(g_0G^k+g_1G^{k-1}+\cdots+g_kG^0)e_0=g_0e_{k}+g_1e_{k-1}+\cdots+g_we_0\ne 0_{\deg(p)\times 1}$.
					\item \textbf{Therefore verify that $g(G)\ne 0_{\deg(p)\times\deg(p)}$.}
				\end{enumerate}
		\procedure{3.60}
			\objective
				Choose two $\mathbb{Q}[x]$s $g=g_0x^u+g_1x^{u-1}+\cdots+g_ux^0$, $p=x^u+p_1x^{u-1}+p_2x^{u-2}+\cdots+p_ux^0$ such that $u=\deg(g)>0$ and $g(\rcan(p))=0_{u\times u}$. The objective of the following instructions is to show that $g=g_0p$.
			\implementation
				\begin{enumerate}
					\item Let $G=\rcan(p)$.
					\item Cognizant of $G$'s construction, verify that $0_{u\times 1}=g(G)e_0=(g_0G^u+g_1G^{u-1}+g_2G^{u-2}+\cdots+g_uG^0)e_0=g_0Ge_{u-1}+g_1e_{u-1}+g_2e_{u-2}+\cdots+g_ue_0$.
					\item Therefore for $i$ in $[0:u]$, do the following:
					\begin{enumerate}
						\item Verify that $0=(g_0Ge_{u-1}+g_1e_{u-1}+g_2e_{u-2}+\cdots+g_ue_0)_{i,0}$.
						\item Therefore cognizant of $G$'s construction, verify that $-g_0p_{u-i}+g_{u-i}=0$.
						\item Therefore verify that $g_{u-i}=g_0p_{u-i}$.
					\end{enumerate}
					\item \textbf{Therefore verify that $g=g_0p$.}
				\end{enumerate}
		\procedure{3.61}
			\objective
				Choose a $\mathcal{M}_{m,m}(\mathbb{Q})$, $A$. Choose a $\mathbb{Q}[x]$ $p=p_0x^t+p_1x^{t-1}+p_2x^{t-2}+\cdots+p_tx^0$ where $p_0\ne 0$, such that $p(A)=0_{m\times m}$. The objective of the following instructions is to either show that $0\ne 0$ or to construct a $\mathbb{Q}[x]$ $f$ such that $p=f\last_A$.
			\implementation
				\begin{enumerate}
					\item Let $F$ be a $\mathcal{M}_{1,2}(\mathbb{Q}[x])$ matrix consisting in-order of $p$ and $\last_A$.
					\item Execute \procedurehr{3.28} on $F$ and let $\langle M,D,,N\rangle$ receive the result.
					\item Verify that $D_{0,0}\ne 0$.
					\item Let $g=g_0x^w+g_1x^{w-1}+g_2x^{w-2}+\cdots+g_wx^0=D_{0,0}$ in such a way that $g_0\ne 0$.
					\item Verify that $F={M^{-1}}_*D{N^{-1}}_*=D{N^{-1}}_*$.
					\item Verify that $\last_A=F_{0,1}=D_{0,0}{{N^{-1}}_*}_{0,1}+D_{0,1}{{N^{-1}}_*}_{1,1}=D_{0,0}{{N^{-1}}_*}_{0,1}=g{{N^{-1}}_*}_{0,1}$.
					\item Let $u=\last_A$.
					\item Therefore verify that ${{N^{-1}}_*}_{0,1}\ne 0$.
					\item Therefore verify that $u=\deg(\last_A)=\deg(D_{0,0}{{N^{-1}}_*}_{0,1})\ge\deg(D_{0,0})=\deg(g)=w$.
					\item Verify that $D=M_*FN_*=FN_*$.
					\item Therefore verify that $g=D_{0,0}={N_*}_{0,0}p+{N_*}_{1,0}\last_A$.
					\item Therefore using \procedurehr{3.56}, verify that $g(A)={N_*}_{0,0}(A)p(A)+{N_*}_{1,0}(A)\last_A(A)={N_*}_{0,0}(A)0_{m\times m}+{N_*}_{1,0}(A)0_{m\times m}=0_{m\times m}$.
					\item Execute \procedurehr{3.52} on the matrix $A$ and let the tuple $\langle R_1,E,R_3\rangle$ receive the result.
					\item Using \procedurehr{3.55}, and \procedurehr{3.52}, verify that $\bdiag(g(E))=I_m\bdiag(g(E))I_m=R_3R_1\bdiag(g(E))R_3R_1=R_3g(A)R_1=R_30_{m\times m}R_1=0_{m\times m}$.
					\item Let $G=\rcan(\mon(\last_A))$.
					\item Verify that $g(G)=g(E_{\lvert E\rvert-1})=\bdiag(g(E))_{[m-u:m],[m-u:m]}=0_{u\times u}$.
					\item If $w<u$, then:
					\begin{enumerate}
						\item Using \procedurehr{3.59}, verify that $g(G)\ne 0_{u\times u}$.
						\item \textbf{Therefore using (16), verify that $0_{u\times u}=g(G)\ne 0_{u\times u}$.}
						\item \textbf{Abort procedure.}
					\end{enumerate}
					\item Otherwise, do the following:
					\begin{enumerate}
						\item Verify that $w=u$.
						\item Using \procedurehr{3.60}, verify that $g=g_0\last_A$.
						\item \textbf{Therefore verify that $p=F_{0,0}=D_{0,0}{{N^{-1}}_*}_{0,0}+D_{0,1}{{N^{-1}}_*}_{1,0}={{N^{-1}}_*}_{0,0}g+{{N^{-1}}_*}_{1,0}*0={{N^{-1}}_*}_{0,0}g={{N^{-1}}_*}_{0,0}g_0\last_A$.}
						\item \textbf{Yield the tuple $\langle {{N^{-1}}_*}_{0,0}g_0\rangle$.}
					\end{enumerate}
				\end{enumerate}
		\procedure{3.62}
			\objective
				Choose a $\mathcal{M}_{m,m}(\mathbb{Q})$, $A$. The objective of the following instructions is to construct a $\mathcal{M}_{m^2,*}(\mathbb{Q})$, $\pows(A)$.
			\implementation
				\begin{enumerate}
					\item Let $t=\deg(\last_A)$.
					\item Make an $m^2\times t$ matrix, $\pows(A)$, whose $i^{th}$ column is the sequential concatenation of the columns of $A^{t-1-i}$.
					\item Yield $\langle\pows(A)\rangle$.
				\end{enumerate}
		\notation{3.21}
			Let us use the notation $\pows(A)$ as a shorthand for the result yielded by executing \procedurehr{3.62} on $A$.
		\procedure{3.63}
			\objective
				Choose an $\mathcal{M}_{m,n}(\mathbb{Q})$, $A$, and an $\mathcal{M}_{n,m}(\mathbb{Q})$, $B$, such that $AB=I_m$. The objective of the following instructions is to show that either $0=1$ or every column of $B$ is non-zero.
			\implementation
				\begin{enumerate}
					\item If any column $i$ of $B$, $Be_i$, is equal to zero, then:
					\begin{enumerate}
						\item Verify that $0_{n\times 1}=A0_{n\times 1}=A(Be_i)=(AB)e_i=I_me_i=e_i$.
						\item \textbf{Therefore verify that 0=1.}
						\item \textbf{Abort procedure.}
					\end{enumerate}
				\end{enumerate}
		\procedure{3.64}
			\objective
				Choose a $\mathcal{M}_{m,m}(\mathbb{Q})$, $A$. Choose a $\mathbb{Q}[x]$ $p$ such that $p\ne 0$, $p(A)=0$, and $\deg(p)<\deg(\last_A)$. The objective of the following instructions is to show that $0<0$.
			\implementation
				\begin{enumerate}
					\item Execute \procedurehr{3.61} on $A$ and $p$ and let $f$ receive.
					\item Now verify that $p=f\last_A$.
					\item Now using (O) and (2), verify that $f\ne 0$ and $\last_A\ne 0$.
					\item \textbf{Therefore using (O), (2), and (3), verify that $\deg(\last_A)>\deg(p)=\deg(f\last_A)\ge\deg(\last_A)$.}
					\item \textbf{Abort procedure.}
				\end{enumerate}
		\procedure{3.65}
			\objective
				Choose a $\mathcal{M}_{m,m}(\mathbb{Q})$, $A$. Execute \procedurehr{3.28} on $\pows(A)$ and let the tuple $\langle M,D,,N\rangle$ receive the result. Let $t=\cols(\pows(A))$. The objective of the following instructions is to show that either $0<0$ or to show that ${C_t(D)}={C_t(D)}_{0,0}e_0\ne 0$.
			\implementation
				\begin{enumerate}
					\item Execute \procedurehr{3.28} on $\pows(A)$ and let the tuple $\langle M,D,,N\rangle$ receive the result.
					\item Verify that $M_*\pows(A)N_*=D$.
					\item Using \procedurehr{3.05}, verify that ${M^{-1}}_*M_*FN_*=I_{m^2}FN_*=FN_*={M^{-1}}_*D$.
					\item If ${C_t(D)}_{0,0}=0$, then:
					\begin{enumerate}
						\item Verify that for some $0\le i<t$, $D_{i,i}=0$.
						\item Therefore verify that $De_i=0_{m^2\times 1}$.
						\item Therefore verify that $F(Ne_i)=(FN)e_i=(M^{-1}D)e_i=M^{-1}(De_i)=0_{m^2\times 1}$.
						\item Let $p=N_{0,i}x^{t-1}+N_{1,i}x^{t-2}+\cdots+N_{t-1,i}x^0$.
						\item Therefore verify that $p(A)=0_{m\times m}$.
						\item Execute \procedurehr{3.63} on ${N^{-1}}_*$ and $N_*$.
						\item Therefore verify that $p\ne 0$.
						\item Execute \procedurehr{3.64} on $A$ and $p$.
						\item \textbf{Abort procedure.}
					\end{enumerate}
					\item Otherwise, do the following:
					\begin{enumerate}
						\item Execute \procedurehr{3.24} on $\langle D,I_t,t\rangle$ and let $E$ receive.
						\item Verify that $C_t(D)=C_t(DI_t)=EC_t(I_t)=E*1=E$.
						\item Verify that $E$ is a $\mathcal{D}_{\binom{m^2}{t},\binom{t}{t}}(\mathbb{Q}[x])$.
						\item Therefore verify that $C_t(D)$ is a $\mathcal{D}_{\binom{m^2}{t},1}(\mathbb{Q}[x])$.
						\item \textbf{Therefore verify that $C_t(D)={C_t(D)}_{0,0}e_0\ne 0$.}
					\end{enumerate}
				\end{enumerate}
		\procedure{3.66}
			\objective
				Choose a $\mathcal{M}_{m,m}(\mathbb{Q})$, $A$. Let $t=\cols(\pows(A))$. The objective of the following instructions is to show that either $0<0$ or to show that ${C_t(\pows(A))}\ne 0$.
			\implementation
				\begin{enumerate}
					\item Execute \procedurehr{3.28} on $\pows(A)$ and let the tuple $\langle M,D,,N\rangle$ receive the result.
					\item Verify that $\pows(A)={M^{-1}}_*D{N^{-1}}_*$.
					\item Execute \procedurehr{3.63} on $C_t(M_*),C_t({M^{-1}}_*)$.
					\item Hence verify that all columns of $C_t({M^{-1}}_*)$ are non-zero.
					\item Let $t=\cols(\pows(A))$.
					\item Execute \procedurehr{3.65} on $A$.
					\item Verify that $C_t(D)={C_t(D)}_{0,0}e_0\ne 0$.
					\item Therefore verify that ${C_t(D)}_{0,0}\ne 0$.
					\item Execute \procedurehr{3.63} on $C_t(N_*),C_t({N^{-1}}_*)$.
					\item Hence verify that $C_t(N^{-1})\ne 0$.
					\item \textbf{Verify that $C_t(\pows(A))=C_t({M^{-1}}_*D{N^{-1}}_*)=C_t({M^{-1}}_*)C_t(D)C_t({N^{-1}}_*)=C_t({M^{-1}}_*){C_t(D)}_{0,0}e_0C_t({N^{-1}}_*)={C_t(D)}_{0,0}C_t({N^{-1}}_*)C_t({M^{-1}}_*)e_0\ne 0_{\binom{m^2}{t}\times 1}$.}
				\end{enumerate}
		\notation{3.22}
			Let us use the notation $\mat_t(p)$ as a shorthand for "$(x^{t-1}\circ p)e_0+(x^{t-2}\circ p)e_1+\cdots+(x^0\circ p)e_{t-1}$".
		\notation{3.23}
			Let us use the notation $\pol(P)$ as a shorthand for "$P_{0,0}x^{t-1}+P_{1,0}x^{t-2}+\cdots+P_{t-1,0}$ where $t=\rows(P)$".
		\notation{3.24}
			Let us use the notation $\lVert A\rVert^2$ as a shorthand for "$\sum_{i=0}^{\rows(A)}\sum_{j=0}^{\cols(A)}{A_{i,j}}^2$".
		\procedure{3.67}
			\objective
				Choose an $\mathcal{M}_{m,m}(\mathbb{Q})$, $A$. The objective of the following instructions is to either show that $0<0$ or to construct a $\mathbb{Q}[x]$, $\sel_A$.
			\implementation
				\begin{enumerate}
					\item Using \procedurehr{3.34} and \procedurehr{3.66}, verify that $C_t(\pows(A)^T\pows(A))=C_t(\pows(A)^T)C_t(\pows(A))={C_t(\pows(A))}^TC_t(\pows(A))=\lVert C_t(\pows(A))\rVert^2>0$.
					\item Let $H=(\pows(A)^T\pows(A))\backslash e_0$.
					\item Let $t=\deg(\last_A)$.
					\item Let $\sel_A=\frac{\pol(H)}{x^t\circ\last_A}$.
					\item Yield $\langle\sel_A\rangle$.
				\end{enumerate}
		\notation{3.25}
			Let us use the notation $\sel_A$ as a shorthand for the result yielded by executing \procedurehr{3.67} on $A$.
		\notation{3.26}
			Let us use the notation $\tr(X)$ as a shorthand for "the sum of the diagonal entries of the square matrix $X$".
		\notation{3.27}
			Let us use the notation "$A$ is symmetric" as a shorthand for "$A^T=A$".
		\procedure{3.68}
			\objective
				Choose a symmetric $\mathcal{M}_{m,m}(\mathbb{Q})$, $A$. Let $t=\deg(\last_A)$. Choose two $\mathbb{Q}[x]$s $u=u_0x^{t-1}+u_1x^{t-2}+\cdots+u_{t-1}x^0,w=w_0x^{t-1}+w_1x^{t-2}+\cdots+w_{t-1}x^0$. The objective of the following instructions is to show that $\tr(u(A)w(A))=\mat(u)^T\pows(A)^T\pows(A)\mat_t(w)$.
			\implementation
				\begin{enumerate}
					\item Verify that $\tr(u(A)w(A))$
					\begin{enumerate}
						\item $=\tr((\sum_{p=0}^t u_pA^{t-1-p})(\sum_{q=0}^t w_qA^{t-1-q}))$
						\item $=\tr(\sum_{p=0}^t\sum_{q=0}^t u_pw_qA^{t-1-p}A^{t-1-q})$
						\item $=\sum_{p=0}^t\sum_{q=0}^t u_pw_q\tr(A^{t-1-p}A^{t-1-q})$
						\item $=\sum_{p=0}^t\sum_{q=0}^t u_pw_q\sum_{e=0}^m\sum_{f=0}^m{A^{t-1-p}}_{e,f}\cdot{A^{t-1-q}}_{f,e}$
						\item $=\sum_{p=0}^t\sum_{q=0}^t u_pw_q\sum_{e=0}^m\sum_{f=0}^m{A^{t-1-p}}_{f,e}\cdot{A^{t-1-q}}_{f,e}$
						\item $=\sum_{p=0}^t\sum_{q=0}^t u_pw_q\sum_{g=0}^{m^2}{\pows(A)}_{g,p}{\pows(A)}_{g,q}$
						\item $=\sum_{p=0}^t\sum_{q=0}^t u_pw_q(\pows(A)^T\pows(A))_{p,q}$
						\item $=\sum_{p=0}^t u_p(\pows(A)^T\pows(A)\mat_t(w))_{p}$
						\item $=\mat_t(u)^T\pows(A)^T\pows(A)\mat_t(w)$
					\end{enumerate}
				\end{enumerate}
		\procedure{3.69}
			\objective
				Choose a symmetric $\mathcal{M}_{m,m}(\mathbb{Q})$, $A$. Let $t=\deg(\last_A)$. Choose a $\mathbb{Q}[x]$ $u$ such that $\deg(u)<t$. The objective of the following instructions is to show that $\tr(u(A)\sel_A(A))=\frac{x^{t-1}\circ u}{x^t\circ\last_A}$.
			\implementation
				\begin{enumerate}
					\item Using \procedurehr{3.68} and \procedurehr{3.67}, verify that $\tr(u(A)\sel_A(A))$
					\begin{enumerate}
						\item $=\mat(u)^T\pows(A)^T\pows(A)\mat_t(\sel_A)$
						\item $=\frac{\mat(u)^T\pows(A)^T\pows(A)((\pows(A)^T\pows(A))\backslash e_0)}{x^t\circ\last_A}$
						\item $=\frac{\mat(u)^Te_0}{x^t\circ\last_A}$
						\item $=\frac{\mat(u)_{0,0}}{x^t\circ\last_A}$
						\item $=\frac{x^{t-1}\circ u}{x^t\circ\last_A}$.
					\end{enumerate}
				\end{enumerate}
		\procedure{3.70}
			\objective
				Choose a symmetric $\mathcal{M}_{m,m}(\mathbb{Q})$, $A$. The objective of the following instructions is to either show that $0\ne 0$ or construct $\mathbb{Q}[x]$s $u,v$ such that $u\last_A+v\sel_A=1$.
			\implementation
				\begin{enumerate}
					\item Let $t=\deg(\last_A)$.
					\item Let $G$ be a $\mathcal{M}_{1,2}(\mathbb{Q}[x])$ where $G_{0,0}=\last_A$ and $G_{0,1}=\sel_A$.
					\item Execute \procedurehr{3.28} on $G$ and let the tuple $\langle M,D,,N\rangle$ receive.
					\item Verify that $G={M^{-1}}_*D{N^{-1}}_*$.
					\item Verify that $\last_A\ne 0$.
					\item Therefore verify that $D_{0,0}\ne 0$.
					\item If $\deg(D_{0,0})>0$, then do the following:
					\begin{enumerate}
						\item Let $b={{N^{-1}}_*}_{0,0}$.
						\item Verify that $\last_A=bD_{0,0}$.
						\item Let $z=\deg(b)$.
						\item Verify that $t=\deg(\last_A)=\deg(bD_{0,0})=\deg(b)+\deg(D_{0,0})>\deg(b)=z$.
						\item Let $c={{N^{-1}}_*}_{0,1}$.
						\item Verify that $\sel_A=cD_{0,0}$.
						\item Let $u=x^{t-z-1}b$.
						\item Execute \procedurehr{3.69} on $A$ and $u$.
						\item Hence verify that $\tr(u(A)\sel_A(A))=x^{t-1}\circ u=x^z\circ b\ne 0$.
						\item Also verify that $\tr(u(A)\sel_A(A))=\tr(A^{z-1}b(A)c(A)D_{0,0}(A))=\tr(A^{z-1}c(A)b(A)D_{0,0}(A))=\tr(A^{z-1}c(A)\last_A(A))=\tr(A^{z-1}c(A)0_{m\times m})=\tr(0_{m\times m})=0$.
						\item \textbf{Therefore verify that $0\ne 0$.}
						\item \textbf{Abort procedure.}
					\end{enumerate}
					\item Otherwise, do the following:
					\begin{enumerate}
						\item Verify that $\deg(D_{0,0})=0$.
						\item Let $u=\frac{N_{0,0}}{D_{0,0}}$.
						\item Let $v=\frac{N_{1,0}}{D_{0,0}}$.
						\item \textbf{Verify that $u\last_A+v\sel_A=1$.}
						\item \textbf{Yield the tuple $\langle u,v\rangle$.}
					\end{enumerate}
				\end{enumerate}
		\procedure{3.71}
			\objective
				Choose two $\mathbb{Q}[x]$s, $\langle a,b\rangle$. The objective of the following instructions is to construct two $\mathbb{Q}[x]$s $u,w$ such that $a=ub+w$ and $\deg(w)<\deg(b)$.
			\implementation
				\begin{enumerate}
					\item If $\deg(a)\ge\deg(b)$:
					\begin{enumerate}
						\item Let $y=\frac{x^{\deg(a)}\circ a}{x^{\deg(b)}\circ b}x^{\deg(a)-\deg(b)}$
						\item Let $e=a-yb$.
						\item Verify that $\deg(e)<\deg(a)$.
						\item Execute \procedurehr{3.71} on the tuple $\langle e,b\rangle$. Let the tuple $\langle c,d\rangle$ receive the result.
						\item Verify that $cb+d=e$.
						\item Verify that $\deg(d)<\deg(b)$.
						\item Therefore verify that $cb+d=a-yb$
						\item \textbf{Therefore verify that $(y+c)b+d=a$.}
						\item \textbf{Also verify that $\deg(d)<\deg(b)$.}
						\item \textbf{Now yield the tuple $\langle y+c, d\rangle$.}
					\end{enumerate}
					\item Otherwise:
					\begin{enumerate}
						\item \textbf{Verify that $0*b+a=a$.}
						\item \textbf{Verify that $\deg(a)<\deg(b)$.}
						\item \textbf{Yield the tuple $\langle 0,a\rangle$.}
					\end{enumerate}
				\end{enumerate}
		\procedure{3.72}
			\objective
				Choose two lists of $\mathbb{Q}[x]$s $s,q$ and a non-negative integer $k$ in such a way that, letting $m=\lvert s\rvert-1$,
				\begin{enumerate}
					\item $k<m$.
					\item For $k\le i\le m$, $\deg(s_i)=i$.
					\item For $k<i<m$, $s_{i-1}+s_{i+1}=q_is_i$.
				\end{enumerate}
				Let $\deg(0)=-1$. The objective of the following instructions is to construct $\mathbb{Q}[x]$s $g,h$ such that $s_k=gs_{m-1}+hs_m$, $\deg(g)=m-1-k$, and $\deg(h)=m-2-k$.
			\implementation
				\begin{enumerate}
					\item If $k<m-2$, do the following:
					\begin{enumerate}
						\item Verify that $s_k+s_{k+2}=q_{k+1}s_{k+1}$.
						\item Therefore verify that $s_k=q_{k+1}s_{k+1}-s_{k+2}$.
						\item Execute \procedurehr{3.72} on $s,q,k+1$ and let the tuple $\langle g_1,h_1\rangle$ receive.
						\item Verify that $s_{k+1}=g_1s_{m-1}+h_1s_m$.
						\item Verify that $\deg(g_1)=m-1-(k+1)=m-k-2$.
						\item Verify that $\deg(h_1)=m-2-(k+1)=m-k-3$.
						\item Execute \procedurehr{3.72} on $s,q,k+2$ and let the tuple $\langle g_2,h_2\rangle$ receive.
						\item Verify that $s_{k+2}=g_2s_{m-1}+h_2s_m$.
						\item Verify that $\deg(g_2)=m-1-(k+2)=m-k-3$.
						\item Verify that $\deg(h_2)=m-2-(k+2)=m-k-4$.
						\item Let $g=q_{k+1}g_1-g_2$.
						\item \textbf{Verify that $\deg(g)=\max(1+(m-k-2),m-k-3)=m-1-k$.}
						\item Let $h=q_{k+1}h_1-h_2$.
						\item \textbf{Verify that $\deg(h)=\max(1+(m-k-3),m-k-4)=m-2-k$.}
						\item \textbf{Verify that $s_k=q_{k+1}(g_1s_{m-1}+h_1s_m)-(g_2s_{m-1}+h_2s_m)=(q_{k+1}g_1-g_2)s_{m-1}+(q_{k+1}h_1-h_2)s_m=gs_{m-1}+hs_m$.}
					\end{enumerate}
					\item Otherwise, if $k=m-2$ do the following:
					\begin{enumerate}
						\item Verify that $s_{m-2}+s_m=q_{m-1}s_{m-1}$.
						\item Let $g=q_{m-1}$.
						\item \textbf{Verify that $\deg(g)=1=m-1-k$.}
						\item Let $h=-1$.
						\item \textbf{Verify that $\deg(h)=0=m-2-k$.}
						\item \textbf{Therefore verify that $s_k=s_{m-2}=q_{m-1}s_{m-1}-s_m=gs_{m-1}+hs_m$.}
					\end{enumerate}
					\item Otherwise, if $k=m-1$ do the following:
					\begin{enumerate}
						\item Let $g=1$.
						\item \textbf{Verify that $\deg(g)=0=m-1-k$.}
						\item Let $h=0$.
						\item \textbf{Verify that $\deg(h)=-1=m-2-k$.}
						\item \textbf{Verify that $s_k=s_{m-1}=gs_{m-1}+hs_m$.}
					\end{enumerate}
					\item \textbf{Yield the tuple $\langle g,h\rangle$.}
				\end{enumerate}
		\procedure{3.73}
			\objective
				Choose a symmetric $\mathcal{M}_{m,m}(\mathbb{Q})$, $A$. Let $t=\deg(\last_A)$. The objective of the following instructions is to either show that $0\ne 0$ or to construct lists of $\mathbb{Q}[x]$s $s,q$ such that
				\begin{enumerate}
					\item For $i=0$ to $i=t$, $\deg(s_i)=i$.
					\item For $i=0$ to $i=t$, $\sgn(x^i\circ s_i)=\sgn(x^t\circ s_t)$.
					\item For $i=1$ to $i=t-1$, $s_{i-1}+s_{i+1}=q_is_i$.
					\item $s_t=\last_A$.
				\end{enumerate}
			\implementation
				\begin{enumerate}
					\item Execute \procedurehr{3.70} on $A$ and let $\langle u,s_{t+1}\rangle$ receive the result.
					\item Verify that $us_t+s_{t+1}\sel_A=1$.
					\item Execute \procedurehr{3.71} on the tuple $\langle s_{t+1},s_t\rangle$. Let the tuple $\langle q_t,s_{t-1}\rangle$ receive the result.
					\item Verify that $s_{t+1}=q_ts_t+s_{t-1}$, where $\deg(s_{t-1})<\deg(s_t)=t$.
					\item Therefore verify that $us_t+(q_ts_t+s_{t-1})\sel_A=1$.
					\item Therefore verify that $s_{t-1}(A)\sel_A(A)=u(A)s_t(A)+(q_t(A)s_t(A)+s_{t-1}(A))\sel_A(A)=I_{m,m}$.
					\item Therefore using \procedurehr{3.69}, verify that $\frac{x^{t-1}\circ s_{t-1}}{x^t\circ s_t}=\tr(s_{t-1}(A)\sel_A(A))=\tr(I_{m,m})=m>0$.
					\item For $i=t-1$ down to $i=1$, do the following:
					\begin{enumerate}
						\item Execute \procedurehr{3.71} on the tuple $\langle -s_{i+1},-s_i\rangle$. Let the tuple $\langle q_i,s_{i-1}\rangle$ receive the result.
						\item Verify that $\deg(q_i)=1$.
						\item Verify that $x\circ q_i=\frac{x^{i+1}\circ s_{i+1}}{x^i\circ s_i}$.
						\item Also verify that $-s_{i+1}=-q_is_i+s_{i-1}$.
						\item Therefore verify that $q_is_i=s_{i+1}+s_{i-1}$.
						\item Therefore verify that $q_is_i-s_{i+1}=s_{i-1}$.
						\item Execute \procedurehr{3.72} on the tuple $\langle s,q,i-1\rangle$ and let $\langle p,j\rangle$ receive.
						\item Verify that $s_{i-1}=ps_{t-1}+q_3s_t$.
						\item Verify that $\deg(p)=t-1-(i-1)=t-i$.
						\item Verify that $\deg(q_3)=t-2-(i-1)=t-1-i$
						\item Therefore verify that $s_{i-1}(A)=p(A)s_{t-1}(A)+j(A)s_t(A)=p(A)s_{t-1}(A)+j(A)0_{m\times m}=p(A)s_{t-1}(A)$.
						\item If $p(A)=0$, then do the following:
						\begin{enumerate}
							\item Execute \procedurehr{3.64} on $A$ and $p$.
							\item \textbf{Abort procedure.}
						\end{enumerate}
						\item Otherwise, if $s_{i-1}(A)=0_{m\times m}$, then do the following:
						\begin{enumerate}
							\item Verify that $p(A)s_{t-1}(A)\sel_A(A)=s_{i-1}(A)\sel_A(A)=0_{m\times m}\sel_A(A)=0_{m\times m}$.
							\item Verify that $p(A)s_{t-1}(A)\sel_A(A)=p(A)I_{m,m}=p(A)\ne0_{m\times m}$.
							\item Therefore verify that $0\ne 0$.
							\item \textbf{Abort procedure.}
						\end{enumerate}
						\item Otherwise if $s_{i-1}(A)\sel_A(A)=0_{m\times m}$, then do the following:
						\begin{enumerate}
							\item Verify that $s_{i-1}(A)\sel_A(A)s_{t-1}(A)=0_{m\times m}s_{t-1}(A)=0_{m\times m}$.
							\item Verify that $s_{i-1}(A)\sel_A(A)s_{t-1}(A)=s_{i-1}(A)I_{m,m}=s_{i-1}(A)\ne 0$.
							\item Therefore verify that $0\ne 0$.
							\item \textbf{Abort procedure.}
						\end{enumerate}
						\item Otherwise, do the following:
						\begin{enumerate}
							\item Verify that $\deg(s_{i-1})<i$.
							\item Verify that $s_{i-1}(A)\sel_A(A)\ne 0_{m\times m}$.
							\item Execute the \hyperref[sec:procedure 73 auxilliary procedure]{auxilliary procedure} on the tuple $(i-1, s_{i-1})$.
							\item Hence verify that $\frac{x^{i-1}\circ s_{i-1}}{x^i\circ s_i}=\tr(s_{i-1}(A)^2\sel_A(A)^2)=\tr((s_{i-1}(A)\sel_A(A))^2)=\lVert s_{i-1}(A)\sel_A(A)\rVert^2>0$.
							\item \textbf{Therefore verify that $\sgn(x^{i-1}\circ s_{i-1})=\sgn(x^i\circ s_i)$.}
						\end{enumerate}
					\end{enumerate}
					\item Yield the tuple $\langle s_{[0:t+1]}, q_{[0:t]}\rangle$.
				\end{enumerate}
			\subsubsection*{Auxilliary procedure}\label{sec:procedure 73 auxilliary procedure}
				\paragraph{Objective}
					Choose an integer $0\le k\le t$ such that polynomial $s_k$ is defined. Choose a $\mathbb{Q}[x]$ $g$ such that $\deg(g)\le\min(k,t-1)$. The objective of the following instructions is to show that $\tr(g(A)s_k(A)\sel_A(A)^2)=\frac{x^k\circ g}{x^{k+1}\circ s_{k+1}}$.
				\paragraph{Implementation}
					\begin{enumerate}
						\item If $k=t$, then verify that $\tr(g(A)s_k(A)\sel_A(A)^2)$
						\begin{enumerate}
							\item $=\tr(g(A)s_t(A)\sel_A(A)^2)$
							\item $=\tr(g(A)0_{m\times m}\sel_A(A)^2)$
							\item $=0$
							\item $=\frac{x^k\circ g}{x^{k+1}\circ s_{k+1}}$.
						\end{enumerate}
						\item Otherwise if $k=t-1$, then verify that $\tr(g(A)s_k(A)\sel_A(A)^2)$
						\begin{enumerate}
							\item $=\tr(g(A)s_{t-1}(A)\sel_A(A)^2)$.
							\item $=\tr(g(A)I_{m,m}\sel_A(A))$.
							\item $=\tr(g(A)\sel_A(A))$.
							\item \textbf{=$\frac{x^k\circ g}{x^{k+1}\circ s_{k+1}}$.}
						\end{enumerate}
						\item Otherwise if $k<t-1$, then do the following:
						\begin{enumerate}
							\item Verify that $\deg(gq_{k+1})=k+1\le t-1$.
							\item Execute the \hyperref[sec:procedure 73 auxilliary procedure]{auxilliary procedure} on the tuple $\langle k+1, gq_{k+1}\rangle$.
							\item Now verify that $\tr((g(A)q_{k+1}(A))s_{k+1}(A)\sel_A(A)^2)=\frac{\frac{x^{k+2}\circ s_{k+2}}{x^{k+1}\circ s_{k+1}}x^k\circ g}{x^{k+2}\circ s_{k+2}}=\frac{x^k\circ g}{x^{k+1}\circ s_{k+1}}$.
							\item Verify that $\deg(g)\le k\le t-2$.
							\item Execute the \hyperref[sec:procedure 73 auxilliary procedure]{auxilliary procedure} on the tuple $\langle k+2,g\rangle$.
							\item Now verify that $\tr(g(A)s_{k+2}(A)\sel_A(A)^2)=\frac{x^{k+2}\circ g}{x^{k+3}\circ{s_{k+3}}}=\frac{0}{x^{k+3}\circ{s_{k+3}}}=0$.
							\item Therefore verify that $\tr(g(A)s_k(A)\sel_A(A)^2)$
							\begin{enumerate}
								\item $=\tr(g(A)(q_{k+1}(A)s_{k+1}(A)+s_{k+2}(A))\sel_A(A)^2)$
								\item $=\tr(g(A)q_{k+1}(A)s_{k+1}(A)\sel_A(A)^2)+\tr(g(A)s_{k+2}(A)\sel_A(A)^2)$
								\item $=\frac{x^k\circ g}{x^{k+1}\circ s_{k+1}}+0$
								\item $=\frac{x^k\circ g}{x^{k+1}\circ s_{k+1}}$.
							\end{enumerate}
						\end{enumerate}
					\end{enumerate}
		\procedure{3.74}
			\objective
				Choose a symmetric $\mathcal{M}_{m,m}(\mathbb{Q})$, $A$. Let $t=\deg(\last_A)$. The objective of the following instructions is to either show that $0<0$ or to construct two lists of rational numbers $c,d$ such that $c_0<d_0\le c_1<d_1\le\cdots\le c_{t-1}<d_{t-1}$ and $\sgn(\last_A(c_i))=-\sgn(\last_A(d_i))$ for $i$ in $[0:t]$.
			\implementation
				\begin{enumerate}
					\item Execute \procedurehr{3.73} on the matrix $A$ and let the tuple $\langle s,q\rangle$ receive the result.
					\item Execute \procedurehr{2.09} supplying the tuple $\langle s,q\rangle$. Let the tuple $\langle c,d\rangle$ receive the result.
					\item \textbf{Verify that $c_0<d_0\le c_1<d_1\le\cdots\le c_{t-1}<d_{t-1}$.}
					\item \textbf{Verify that $\sgn(\last_A(c_i))=-\sgn(\last_A(d_i))$ for $i$ in $[0:t]$.}
					\item \textbf{Yield $\langle c,d\rangle$.}
				\end{enumerate}
		\procedure{3.75}
			\objective
				Choose a symmetric $\mathcal{M}_{m,m}(\mathbb{Q})$, $A$. Let $t=\deg(\last_A)$. Execute \procedurehr{3.74} on $A$ and let the tuple $\langle c,d\rangle$ receive the result. Execute \procedurehr{3.28} on $A$ and let the tuple $\langle,,u,\rangle$ receive the result. The objective of the following instructions is to either show that $1=-1$ or to construct a list of non-negative integers $k$ such that $\sgn(u_{k_i}(c_i))=-\sgn(u_{k_i}(d_i))$ for $i$ in $[0:t]$.
			\implementation
				\begin{enumerate}
					\item Verify that $\last_A=u_0u_1\cdots u_{m-1}$.
					\item For $i$ in $[0:t]$, do the following:
					\begin{enumerate}
						\item If $\sgn(u_0(c_i))=\sgn(u_0(d_i)),\sgn(u_1(c_i))=\sgn(u_1(d_i)),\cdots,\sgn(u_{m-1}(c_i))=\sgn(u_{m-1}(d_i))$, then do the following:
						\begin{enumerate}
							\item Verify that $\sgn(u_0(c_i))\sgn(u_1(c_i))\cdots\allowbreak\sgn(u_{m-1}(c_i))=\sgn(u_0(d_i))\sgn(u_1(d_i))\cdots\sgn(u_{m-1}(d_i))$.
							\item Therefore verify that $\sgn(u_0(c_i)u_1(c_i)\cdots u_{m-1}(c_i))=\sgn(u_0(d_i)u_1(d_i)\cdots u_{m-1}(d_i))$.
							\item Therefore verify that $\sgn(\last_A(c_i))=\sgn(\last_A(d_i))$.
							\item Using (O), verify that $\sgn(\last_A(c_i))=-\sgn(\last_A(d_i))$.
							\item Therefore verify that $\sgn(\last_A(c_i))=-\sgn(\last_A(c_i))$.
							\item Therefore verify that $1=-1$.
							\item \textbf{Abort procedure.}
						\end{enumerate}
						\item Otherwise do the following:
						\begin{enumerate}
							\item \textbf{Let $j$ be the least integer such that $\sgn(u_j(c_i))=-\sgn(u_j(d_i))$.}
							\item \textbf{Let $k_i=j$.}
						\end{enumerate}
					\end{enumerate}
					\item \textbf{Yield $\langle k\rangle$.}
				\end{enumerate}
		\procedure{3.76}
			\objective
				Choose a symmetric $\mathcal{M}_{m,m}(\mathbb{Q})$, $A$. Execute \procedurehr{3.28} on $A$ and let the tuple $\langle,,u,\rangle$ receive the result. Execute \procedurehr{2.03} on $A$ and let $k$ receive. Let $t=\deg(\last_A)$. Let $n_j=\sum_{i=0}^t [k_i=j]$ for $j$ in $[0:m]$. The objective of the following instructions is to either show that $0<0$, or to show that $n_i=\deg(u_i)$ for $i$ in $[0:m]$.
			\implementation
				\begin{enumerate}
					\item Verify that $\sum_{j=0}^m n_j=\sum_{j=0}^m\sum_{i=0}^t [k_i=j]=\sum_{i=0}^t\sum_{j=0}^m [k_i=j]=\sum_{i=0}^t 1=t$.
					\item If for any $i$ in $[0:m]$, $n_i>\deg(u_i)$, then do the following:
					\begin{enumerate}
						\item Execute \procedurehr{2.03} on the polynomial $u_i$ along with $\deg(u_i)+1$ of the distinct pairs $\langle c_l,d_l\rangle$ such that $k_l=i$.
						\item \textbf{Abort procedure.}
					\end{enumerate}
					\item Otherwise if for any $i$ in $[0:m]$, $n_i<\deg(u_i)$, then do the following:
					\begin{enumerate}
						\item Verify that $\sum_{i=0}^m n_j<\sum_{i=0}^m \deg(u_j)=t$.
						\item Therefore using (1) and (a), verify that $\sum_{i=0}^m n_j<\sum_{i=0}^m n_j$.
						\item \textbf{Abort procedure.}
					\end{enumerate}
					\item Otherwise, do the following:
					\begin{enumerate}
						\item \textbf{For all $i$ in $[0:m]$, verify that $n_i=\deg(u_i)$.}
					\end{enumerate}
				\end{enumerate}
		\notation{3.28}
			Let us use the notation "$A$ is upper triangular" as a shorthand for "all the entries of $A$ below the diagonal are zero" in what follows. 
		\procedure{3.77}
			\objective
				Choose two upper triangular $\mathcal{M}_{m,m}(\mathbb{Q}[x])$s, $A$ and $B$. Let $C=AB$. The objective of the following instructions is to show that $C$ is an upper triangular matrix where $C_{i,i}=A_{i,i}B_{i,i}$ for $i$ in $[0:m]$.
			\implementation
				\begin{enumerate}
					\item For $i$ in $[0:m]$, do the following:
					\begin{enumerate}
						\item \textbf{Verify that $C_{i,i}=\sum_{k=0}^m (A_{i,k}B_{k,i})=\sum_{k=0}^{i} (A_{i,k}B_{k,i})+A_{i,i}B_{i,i}+\sum_{k=i+1}^m (A_{i,k}B_{k,i})=\sum_{k=0}^{i} (0*B_{k,i})+A_{i,i}B_{i,i}+\sum_{k=i+1}^m (A_{i,k}*0)=A_{i,i}B_{i,i}$.}
					\end{enumerate}
					\item For $i$ in $[1:m]$, do the following:
					\begin{enumerate}
						\item For $j$ in $[0:i]$, do the following:
						\begin{enumerate}
							\item Verify that $C_{i,j}=\sum_{k=0}^m A_{i,k}B_{k,j}=\sum_{k=0}^{i} A_{i,k}B_{k,j}+\sum_{k=i}^m A_{i,k}B_{k,j}=\sum_{k=0}^{i} 0*B_{k,j}+\sum_{k=i}^m A_{i,k}*0=0$.
						\end{enumerate}
					\end{enumerate}
					\item \textbf{Therefore verify that $C$ is upper triangular.}
				\end{enumerate}
		\procedure{3.78}
			\objective
				Choose integers $m\ge n\ge 0$. Choose a $\mathcal{M}_{n,m}(\mathbb{Q}[x])$, $M$, and a $\mathcal{M}_{m,n}(\mathbb{Q}[x])$, $B$, such that $MB=I_n$. The objective of the following instructions is to either show that $1=0$ or to construct a list of $\mathcal{M}_{m,n}(\mathbb{Q}[x])$s, $A$, such that $A_0=B$ and for $i=0$ to $i=n$:
				\begin{enumerate}
					\item $BMA_i=A_i$
					\item $({A_i}^TA_i)_{[0:i],[0:i]}$ is a $\mathcal{D}_{i,i}(\mathbb{Q}[x])$
					\item ${A_i}^TA_i=\bdiag(({A_i}^TA_i)_{[0:i],[0:i]},\allowbreak({A_i}^TA_i)_{[i:n],[i:n]})$
					\item $({e_j}^TM)(A_ie_j)=\prod_{r=0}^{\min(i,j)}\lVert A_re_{r}\rVert^2$ for $j$ in $[0:n]$.
				\end{enumerate}
			\implementation
				\begin{enumerate}
					\item Let $A=\langle B\rangle$.
					\item For $i=1$ to $i=n$, do the following:
					\begin{enumerate}
						\item Let $D_i$ be a $n\times n$ diagonal matrix comprising $i$ $1$s followed by $n-i$ $\lVert A_{i-1}e_{i-1}\rVert^2$s.
						\item Verify that $D_i$ is upper triangular.
						\item Let $N_i=I_n$ except that its $(i-1)^{th}$ row is $i-1$ $0$s followed by a $1$ followed by $-({A_{i-1}}^TA_{i-1})_{i-1,i}$, then $-({A_{i-1}}^TA_{i-1})_{i-1,i+1}$, all the way up to $-({A_{i-1}}^TA_{i-1})_{i-1,n-1}$.
						\item Verify that $N_i$ is upper triangular.
						\item Let $A_i=A_{i-1}D_iN_i$.
						\item Verify that ${A_i}^TA_i=(A_{i-1}D_iN_i)^T(A_{i-1}D_iN_i)={N_i}^T{D_i}^T({A_{i-1}}^TA_{i-1})D_iN_i$.
						\item Now using \procedurehr{3.09}, verify that ${A_i}^TA_i$ and ${A_{i-1}}^TA_{i-1}$ are the same modulo the bottom-right $(n-i+1)\times(n-i+1)$ block.
						\item \textbf{Therefore using (1g) and the previous instance of (1k), verify that $({A_i}^TA_i)_{[0:i],[0:i]}$ is a $\mathcal{D}_{i,i}(\mathbb{Q}[x])$.}
						\item Also verify that $({A_i}^TA_i)_{i-1,[i:n]}=0$.
						\item Also verify that $({A_i}^TA_i)_{[i:n],i-1}=0$.
						\item \textbf{Therefore using (1i), (1j), and the previous instance of (1k), verify that ${A_i}^TA_i=\bdiag(({A_i}^TA_i)_{[0:i],[0:i]},\allowbreak({A_i}^TA_i)_{[i:n],[i:n]})$.}
						\item Using (1e), verify that $A_i=A_0(D_1N_1)\cdots(D_iN_i)$.
						\item Therefore verify that $MA_i=(D_1N_1)\cdots(D_iN_i)$.
						\item \textbf{Therefore verify that $A_0MA_i=A_i$.}
						\item Using \procedurehr{3.77}, for $j$ in $[0:n]$, verify that $({e_j}^TM)(A_ie_j)$
						\begin{enumerate}
							\item $={e_j}^T(MA_i)e_j$
							\item $={e_j}^T((D_1N_1)\cdots (D_iN_i))e_j$
							\item $=({D_1}_{j,j}{N_1}_{j,j})\cdots ({D_i}_{j,j}{N_i}_{j,j})$
							\item $={D_1}_{j,j}\cdots {D_i}_{j,j}$
							\item $={D_1}_{j,j}\cdots {D_{\min(i,j)}}_{j,j}$
							\item $=\lVert A_0e_0\rVert^2\cdots\lVert A_{\min(i,j)-1}e_{\min(i,j)-1}\rVert^2$
							\item $=\prod_{r=0}^{\min(i,j)}\lVert A_re_{r}\rVert^2$.
						\end{enumerate}
					\end{enumerate}
					\item \textbf{Yield the tuple $\langle A\rangle$.}
				\end{enumerate}
		\procedure{3.79}
			\objective
				Choose a $\mathcal{M}_{1,m}(\mathbb{Q})$, $A$, and a $\mathcal{M}_{m,1}(\mathbb{Q})$, $B$. The objective of the following instructions is to show that $(AB)^2\le(AA^T)(B^TB)$.
			\implementation
				\begin{enumerate}
					\item Verify that $0$
					\begin{enumerate}
						\item $\le\frac{1}{2}\sum_{i=0}^m\sum_{j=0}^m (A_iB_j-A_jB_i)^2$
						\item $=\frac{1}{2}\sum_{i=0}^m\sum_{j=0}^m ({A_i}^2{B_j}^2-2A_iB_jA_jB_i+{A_j}^2{B_i}^2)$
						\item $=\frac{1}{2}\sum_{i=0}^m {A_i}^2\sum_{j=0}^m {B_j}^2+\frac{1}{2}\sum_{i=0}^m {B_i}^2\cdot\allowbreak\sum_{j=0}^m {A_j}^2-\sum_{i=0}^m A_iB_i\sum_{j=0}^m A_jB_j$
						\item $=\frac{1}{2}(AA^T)(B^TB)+\frac{1}{2}(AA^T)(B^TB)-(AB)^2$
						\item $=(AA^T)(B^TB)-(AB)^2$.
					\end{enumerate}
					\item \textbf{Therefore verify that $(AB)^2\le(AA^T)(B^TB)$.}
				\end{enumerate}
		\notation{3.29}
			Let us use the notation $(2k)!!$ as a shorthand for "$2^k(k!)$".
		\procedure{3.80}
			\objective
				Choose integers $m\ge n>0$. Choose a $\mathcal{M}_{n,m}(\mathbb{Q}[x])$, $M$, and a $\mathcal{M}_{m,n}(\mathbb{Q}[x])$, $B$, such that $MB=I_n$. Choose a $\mathbb{Q}$, $x$. Let $a=\max(\lVert M(x)\rVert^2,1)$. Execute \procedurehr{3.78} on $\langle M,B\rangle$ and let the tuple $\langle A\rangle$ receive the result. Choose a column index $0\le j<n$ such that $\lVert A_n(x)e_j\rVert^2<\frac{1}{a^{(2n+2)!!}}$. The objective of the following instructions is to show that $1<1$.
			\implementation
				\begin{enumerate}
					\item Let $i=n$.
					\item Verify that $\lVert A_i(x)e_j\rVert^2<\frac{1}{a^{(2i+2)!!}}$.
					\item Using \procedurehr{3.79}, verify that $({e_j}^TM(x)A_i(x)e_j)^2\le\lVert{e_j}^TM(x)\rVert^2\lVert A_i(x)e_j\rVert^2<\lVert M(x)\rVert^2\frac{1}{a^{(2i+2)!!}}\le a\frac{1}{a^{(2i+2)!!}}\le\frac{1}{a^{(2i)!!*2i}}\le 1$.
					\item If $i=0$, then do the following:
					\begin{enumerate}
						\item Verify that $({e_j}^TM(x)A_i(x)e_j)^2=({e_j}^TM(x)A_0(x)e_j)^2=({e_j}^TI_ne_j)^2=1$.
						\item Therefore using (4) and (a), verify that $1<1$.
						\item \textbf{Abort procedure.}
					\end{enumerate}
					\item Otherwise, do the following:
					\item Using (O), verify that $(\prod_{r=0}^{\min(i,j)}\lVert A_re_r\rVert^2)^2=({e_j}^TM(x)A_i(x)e_j)^2<\frac{1}{a^{(2i)!!*2i}}\le 1$.
					\item If $\min(i,j)=0$, then do the following:
					\begin{enumerate}
						\item Verify that $(\prod_{r=0}^{\min(i,j)}\lVert A_re_r\rVert^2)^2=1^2=1$.
						\item \textbf{Therefore using (7) and (a), verify that $1<1$.}
						\item \textbf{Abort procedure.}
					\end{enumerate}
					\item Otherwise do the following:
					\begin{enumerate}
						\item Verify that $\min(i,j)>0$.
						\item If for all $k=0$ to $k=\min(i,j)-1$, $\lVert A_k(x)e_{k}\rVert^2\ge\frac{1}{a^{(2i)!!}}$, then do the following:
						\begin{enumerate}
							\item Verify that $({e_j}^TM(x)A_i(x)e_j)^2=(\prod_{r=0}^{\min(i,j)}\lVert A_re_r\rVert^2)^2\ge(\frac{1}{a^{(2i)!!}})^{2\min(i,j)}\ge(\frac{1}{a^{(2i)!!}})^{2i}=\frac{1}{a^{(2i)!!*2i}}$.
							\item \textbf{Therefore using (4) and (i), verify that $({e_j}^TM(x)A_i(x)e_j)^2<\frac{1}{a^{(2i)!!*2i}}\le ({e_j}^TM(x)A_i(x)e_j)^2$.}
							\item \textbf{Abort procedure.}
						\end{enumerate}
						\item Otherwise, do the following:
						\begin{enumerate}
							\item Let $k$, where $0\le k<\min(i,j)\le i$, be one of the integers for which $\lVert A_k(x)e_{k}\rVert^2<\frac{1}{a^{(2i)!!}}$.
							\item Verify that $\lVert A_k(x)e_{k}\rVert^2<\frac{1}{a^{(2i)!!}}\le\frac{1}{a^{(2k+2)!!}}$.
							\item Let $i=j=k$.
							\item Go to (2).
						\end{enumerate}
					\end{enumerate}
				\end{enumerate}
		\procedure{3.81}
			\objective
				Choose a symmetric $\mathcal{M}_{m,m}(\mathbb{Q})$, $A$. Let $t=\deg(\last_A)$. Execute \procedurehr{3.75} on the matrix $A$ and let the tuple $\langle k\rangle$ receive the result. The objective of the following instructions is to either show that $0<0$ or to show that $\sum_{i=0}^t(m-k_i)=m$.
			\implementation
				\begin{enumerate}
					\item Execute \procedurehr{3.28} on the matrix $A$ and let the tuple $\langle,D,u,\rangle$.
					\item Using \procedurehr{3.76}, verify that $\sum_{i=0}^t(m-k_i)$
					\begin{enumerate}
						\item $=\sum_{i=0}^t\sum_{j=0}^m [k_i\le j]$
						\item $=\sum_{j=0}^m\sum_{i=0}^t [k_i\le j]$
						\item $=\sum_{j=0}^m\sum_{i=0}^t [k_i\le j]\sum_{l=0}^m [k_i=l]$
						\item $=\sum_{j=0}^m\sum_{l=0}^m\sum_{i=0}^t [k_i\le j][k_i=l]$
						\item $=\sum_{j=0}^m\sum_{l=0}^m\sum_{i=0}^t [l\le j][k_i=l]$
						\item $=\sum_{j=0}^m\sum_{l=0}^m [l\le j]\sum_{i=0}^t [k_i=l]$
						\item $=\sum_{j=0}^m\sum_{l=0}^m [l\le j]\deg u_l$
						\item $=\sum_{j=0}^m\sum_{l=0}^{j+1} \deg u_l$
						\item $=\sum_{j=0}^m \deg D_{j,j}$
						\item $=m$
					\end{enumerate}
				\end{enumerate}
		\procedure{3.82}
			\objective
				Choose a symmetric $\mathcal{M}_{m,m}(\mathbb{Q})$, $A$. The objective of the following instructions is to either show that $0<0$ or to construct a rational, $\disc(A)$, such that $\disc(A)>0$.
			\implementation
				\begin{enumerate}
					\item Execute \procedurehr{3.74} on the matrix $A$ and let the tuple $\langle c,d\rangle$ receive the result.
					\item Execute \procedurehr{3.04} with $xI_m-A$ as the choice matrix. Let the tuple $\langle M,D,,N\rangle$ receive the result.
					\item Let $L=\lvert(\lVert {N^{-1}}_*\rVert^2)^{(2m+2)!!}\rvert$.
					\item Let $\disc(A)=\frac{1}{\max(1,L(\lvert c_1\rvert),L(\lvert d_t\rvert))}$.
					\item \textbf{Verify that $\disc(A)>0$.}
					\item \textbf{Yield the tuple $\langle\disc(A)\rangle$.}
				\end{enumerate}
		\notation{3.30}
			Let us use the notation $\disc(A)$ to refer to the result yielded by executing \procedurehr{3.82} on the matrix $A$.
		\procedure{3.83}
			\objective
				Choose integers $0\le k\le m$ and a list of $\mathcal{T}_m(\mathbb{Q}[x])$, $N$. Let $Q=(I_m)_{*,[k:m]}$. The objective of the following instructions is to either show that $1=0$ or to construct an $\mathcal{M}_{m,m-k}(\mathbb{Q}[x])$, $K$, and an $\mathcal{M}_{m-k,m-k}(\mathbb{Q}[x])$, $E$, such that $K=N_*QE$ and $K^TK$ is a $\mathcal{D}_{m-k,m-k}(\mathbb{Q}[x])$.
			\implementation
				\begin{enumerate}
					\item Verify that $(Q^T{N^{-1}}_*)(N_*Q)=Q^T({N^{-1}}_*N_*)Q=Q^TI_mQ=Q^TQ=I_{m-k}$.
					\item Execute \procedurehr{3.78} on the matrices $Q^T{N^{-1}}_*$ and $N_*Q$ and let the tuple $\langle Z\rangle$ receive.
					\item Let $K=Z_{m-k}$.
					\item \textbf{Verify that $K$ is a $\mathcal{M}_{m,m-k}(\mathbb{Q}[x])$.}
					\item \textbf{Using (2), verify that $K^TK$ is a $\mathcal{D}_{m-k,m-k}(\mathbb{Q}[x])$.}
					\item Let $E=Q^T{N^{-1}}_*K$.
					\item \textbf{Verify that $E$ is a $\mathcal{M}_{m-k,m-k}(\mathbb{Q}[x])$.}
					\item \textbf{Now, using (2) verify that $K=N_*QE$.}
					\item \textbf{Yield $\langle K,E\rangle$.}
				\end{enumerate}
		\procedure{3.84}
			\objective
				Choose a symmetric $\mathcal{M}_{m,m}(\mathbb{Q})$, $A$. Choose a $\mathbb{Q}$ $\epsilon>0$. Execute \procedurehr{3.75} on the matrix $A$ and let the tuple $\langle k\rangle$ receive the result. The objective of the following instructions is to either show that $1<1$ or to construct $\mathbb{Q}$s, $0<\delta\le 1\le K'$, a list of $\mathcal{M}_{m,*}(\mathbb{Q})$s, $K$, and a list of $\mathbb{Q}$s, $g$, such that for $i$ in $[0:\lvert k\rvert]$:
				\begin{enumerate}
					\item ${K_i}^T{K_i}$ is a $\mathcal{D}_{m-k_i}(\mathbb{Q})$.
					\item $(K_i)_{p,q}\le K'm$, for $0\le p<m$, for $0\le q<\cols(K_i)$.
					\item $({K_i}^T{K_i})_{j,j}\ge\disc(A)$ for $0\le j<\cols(K_i)$.
					\item $\lvert(g_iK_i-AK_i)_{p,q}\rvert<\frac{\epsilon\delta}{K'm^2}$, for $0\le p<m$, for $0\le q<\cols(K_i)$.
					\item $\delta\le\min_{i=0}^{\lvert g\rvert}\min_{j=i+1}^{\lvert g\rvert}\lvert g_j-g_i\rvert$.
				\end{enumerate}
			\implementation
				\begin{enumerate}
					\item Execute \procedurehr{3.74} on the matrix $A$ and let the tuple $\langle c,d\rangle$ receive the result.
					\item Execute \procedurehr{3.28} with $xI_m-A$ as the choice matrix. Let the tuple $\langle M,D,u,N\rangle$ receive the result.
					\item Let $M'=1+\max_{i=0}^m\max_{j=0}^m\lvert {{M^{-1}}_*}_{i,j}\rvert(\max(\lvert c_0\rvert,\lvert d_{\lvert d\rvert-1}\rvert))$.
					\item Let $N'=1+\max_{i=0}^m\max_{j=0}^m\lvert {N_*}_{i,j}\rvert(\max(\lvert c_0\rvert,\lvert d_{\lvert d\rvert-1}\rvert))$.
					\item Let $\delta=\min(1,\min_{i=1}^{\lvert d\rvert}(d_{i}-c_{i-1}))$.
					\item Execute \procedurehr{3.83} on $\langle k,m,N\rangle$ and let the tuple $\langle\langle K_0,E_0\rangle,\langle K_1,E_1\rangle,\cdots,\langle K_{\lvert k\rvert-1},E_{\lvert k\rvert-1}\rangle\rangle$ receive.
					\item \textbf{Using \procedurehr{3.81}, verify that $\sum_{p=0}^{\lvert k\rvert}\cols(K_p)=\sum_{p=0}^{\lvert k\rvert} m-k_p=m$.}
					\item Let $E'=1+\max_{i=0}^t\max_{j=0}^{m-k_i}\max_{l=0}^{m-k_i}\lvert E_{j,l}\rvert(\max(\lvert c_0\rvert,\lvert d_{\lvert d\rvert-1}\rvert))$.
					\item Let $U=\prod_{r=0}^m(1+\lvert u_r\rvert)$.
					\item Let $U'=U(\max(\lvert c_0\rvert,\lvert d_{\lvert d\rvert-1}\rvert))$.
					\item Let $b=\frac{\epsilon\delta}{M'N'E'^2m^3}$.
					\item For $i$ in $[0:\lvert k\rvert]$, do the following:
					\begin{enumerate}
						\item Verify that $\sgn(u_{k_i}(c_i))\ne\sgn(u_{k_i}(d_i))$.
						\item Execute \procedurehr{2.02} on the formal polynomial $u_{k_i}$, interval $(c_i, d_i)$, and target of $\frac{b}{U'}$ and let $\langle g_i\rangle$ receive.
						\item Now verify that $\lvert u_{k_i}(g_i)\rvert<\frac{b}{U'}$.
						\item Also verify that $c_i\le g_i\le d_i$.
						\item For $j$ in $[k_i:m]$, do the following:
						\begin{enumerate}
							\item Verify that $\lvert D_{j,j}(g_i)\rvert=\prod_{r=0}^{j+1}\lvert u_r(g_i)\rvert\le\lvert u_{k_i}(g_i)\rvert\prod_{r=0}^{k_i}\lvert u_r\rvert(\lvert g_i\rvert)\cdot\prod_{r=k_i+1}^{j+1}\lvert u_{r}\rvert(\lvert g_i\rvert)<\frac{b}{U'}U(\lvert g_i\rvert)=\frac{b}{U'}U'=b$.
						\end{enumerate}
						\item Let $Q=(I_m)_{*,[k_i:m]}$.
						\item If a diagonal entry of ${K_i(g_i)}^TK_i(g_i)$ is less than $\disc(A)$, then do the following:
						\begin{enumerate}
							\item Let $z$ be the column index of the diagonal entry less than $\disc(A)$.
							\item Verify that $\disc(A)\le\frac{1}{\max(\lVert (Q^TN^{-1})(g_i)\rVert^2,1)^{(2(m-k_i)+2)!!}}$.
							\item Execute \procedurehr{3.80} with matrices $Q^TN^{-1}$ and $NQ$, rational number $g_i$, and column index $z$.
							\item \textbf{Abort procedure.}
						\end{enumerate}
						\item Otherwise, do the following:
						\begin{enumerate}
							\item \textbf{For $j$ in $[0:m-k_i]$, verify that $({K_i(g_i)}^TK_i(g_i))_{j,j}\ge \disc(A)>0$.}
							\item Verify that $xK_i-AK_i=(xI_m-A)K_i=M^{-1}DN^{-1}K_i=M^{-1}DN^{-1}NQE_i=M^{-1}DQE_i$.
							\item \textbf{Verify that $(g_iK_i(g_i)-AK_i(g_i))_{p,q}=(M^{-1}(g_i)D(g_i)QE_i(g_i))_{p,q}<M'b(m-k_i)E'=M'\frac{\epsilon\delta}{M'N'E'^2m^3}(m-k_i)E'\le\frac{\epsilon\delta}{N'E'm^2}$ for $p$ in $[0:m]$, for $q$ in $[0:m-k_i]$.}
							\item \textbf{Verify that $K_i(g_i)_{p,q}=(N(g_i)QE_i(g_i))_{p,q}=N'(m-k_i)E'\le N'E'm$ for $p$ in $[0:m]$, for $q$ in $[0:m-k_i]$.}
						\end{enumerate}
					\end{enumerate}
					\item \textbf{Yield the tuple $\langle\delta,N'E',\langle K_0(g_0),\cdots,K_{t-1}(g_{t-1})\rangle,g\rangle$.}
				\end{enumerate}
		\notation{3.00}
			Let us use the notation $J_{m\times n}$ as a shorthand for "the $\mathcal{M}_{m,n}(\mathbb{Q})$ such that every entry is $1$".
		\procedure{3.85}
			\objective
				Choose a symmetric $\mathcal{M}_{m,m}(\mathbb{Q})$, $A$. Choose a $\mathbb{Q}$ $\epsilon>0$. The objective of the following instructions is to either show that $1<1$ or to construct an $\mathcal{M}_{m,m}(\mathbb{Q})$, $K$, and a $\mathcal{D}_{m,m}(\mathbb{Q})$, $C$, such that:
				\begin{enumerate}
					\item $\sum_{p=0}^m\sum_{q=0}^m\lvert(KC-AK)_{p,q}\rvert<\epsilon$.
					\item $\lvert(K^TK)_{i,j}\rvert\le 2\epsilon$ for $0\le i,j<m$, $i\ne j$.
					\item $(K^TK)_{j,j}\ge\disc(A)>0$ for $0\le j<m$.
				\end{enumerate}
			\implementation
				\begin{enumerate}
					\item Execute \procedurehr{3.84} on matrix $A$ and rational $\epsilon$. Let the tuple $\langle\delta,K',K,g\rangle$ receive the result.
					\item Let $C$ be a diagonal matrix whose $i^{th}$, where $0\le i<t$, group of entries are $m-k_i$ $g_i$s.
					\item \textbf{Using \procedurehr{3.81}, verify that $C$ is $m\times m$.}
					\item Let $K$ be a matrix whose columns are the in-order concatenation of those of $K_0,K_1,\cdots,K_{t-1}$.
					\item \textbf{Using \procedurehr{3.81}, verify that $K$ is $m\times m$.}
					\item \textbf{Using (1), verify that $\sum_{p=0}^m\sum_{q=0}^m\lvert(KC-AK)_{p,q}\rvert<\sum_{p=0}^m\sum_{q=0}^m \frac{\epsilon\delta}{K'm^2}=\frac{\epsilon\delta}{K'}\le\epsilon$.}
					\item For $i$ in $[0:m]$, do the following: For $j$ in $[0:m]$, do the following:
					\begin{enumerate}
						\item Let $a,c$ be such that $Ke_i$ came from $K_ae_c$.
						\item Let $b,d$ be such that $Ke_j$ came from $K_be_d$.
						\item If $a\ne b$, then do the following:
						\begin{enumerate}
							\item Using (1), verify that $\lvert(g_b-g_a)(Ke_i)^T(Ke_j)\rvert$
							\item $=\lvert g_b(Ke_i)^T(Ke_j)-g_a(Ke_i)^T(Ke_j)\rvert$
							\item $=\lvert(Ke_i)^T(g_bKe_j)-(g_aKe_i)^T(Ke_j)\rvert$
							\item $=\lvert(Ke_i)^T(AKe_j+g_bKe_j-AKe_j)-(AKe_i+g_aKe_i-AKe_i)^T(Ke_j)\rvert$
							\item $\le\lvert(Ke_i)^T(AKe_j)-(AKe_i)^T(Ke_j)\rvert+\lvert(Ke_i)^T(g_bKe_j-AKe_j)\rvert+\lvert(g_aKe_i-AKe_i)^T(Ke_j)\rvert$
							\item $\le\lvert(Ke_i)^TA(Ke_j)-(Ke_i)^TA^T(Ke_j)\rvert+\lvert mK'J_{1\times m}\frac{\epsilon\delta}{K'm^2}J_{m\times 1}\rvert+\lvert\frac{\epsilon\delta}{K'm^2}J_{1\times m}mK'J_{m\times 1}\rvert$
							\item $=2\epsilon\delta$.
							\item \textbf{Therefore using (1) and (vii), verify that $\lvert {e_i}^T(K^TK)e_j\rvert=\lvert(Ke_i)^T(Ke_j)\rvert\le\frac{2\epsilon\delta}{\lvert g_b-g_a\rvert}\le 2\epsilon$.}
						\end{enumerate}
						\item Otherwise if $c\ne d$, do the following:
						\begin{enumerate}
							\item Using (1), verify that ${K_a}^TK_b={K_a}^TK_a$ is a $\mathcal{D}_{*,*}(\mathbb{Q})$.
							\item \textbf{Therefore verify that $(Ke_i)^T(Ke_j)=(K_ae_c)^T(K_be_d)={e_c}^T{K_a}^TK_be_d=0\le 2\epsilon$.}
						\end{enumerate}
					\end{enumerate}
					\item \textbf{Therefore using (7), verify that $\lvert(K^TK)_{i,j}\rvert\le 2\epsilon$ for $1\le i\ne j\le m$.}
					\item \textbf{Using (1), verify that $(K^TK)_{j,j}\ge\disc(A)>0$ for $1\le j\le m$.}
					\item \textbf{Yield the tuple $\langle K,C\rangle$.}
				\end{enumerate}
	\begin{thebibliography}{9}
		\bibitem{linearalgebra} 
			Harold Edwards.
			\textit{Linear Algebra}. 
			Springer Science+Business Media, 1995.
		\bibitem{philosophicalgrammar}
			Ludwig Wittgenstein.
			\textit{Philosophical Grammar}.
			Edited by Rush Rhees.
			Translated by Anthony Kenny.
			Basil Blackwell, Oxford, 1974.
		\bibitem{introductiontheoryofnumbers}
			Ivan Niven, Herbert S. Zuckerman, Hugh L. Montgomery.
			\textit{An Introduction to the Theory of Numbers}.
			John Wiley \& Sons, 1991.
	\end{thebibliography}
\end{document}
\grid
\grid
